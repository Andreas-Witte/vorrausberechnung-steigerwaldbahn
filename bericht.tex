% !TeX encoding = UTF-8
%chktex-file 1 chktex-file 27 chktex-file 21 chktex-file -2 chktex-file 41
\documentclass[fontsize=12pt,a4paper]{scrreprt}
% UTF-8
\usepackage[utf8]{luainputenc}
% Besser lesbare Schriftart
\usepackage[T1]{fontenc}
\usepackage[bitstream-charter]{mathdesign}
% deutsche Silbentrennung
\usepackage[ngerman]{babel}
% Links
\usepackage[luatex,pdfa,hidelinks]{hyperref}
% erster Absatz ebenfalls einrücken
\usepackage{indentfirst}
% Sourcecode
\usepackage[newfloat]{minted}
% Farben
\usepackage[table]{xcolor}
% Hübschere Anführungszeichen
\usepackage{csquotes}
% Zeilenabstand
\usepackage{setspace}
\setstretch{1.15}

\usepackage{endfloat}

% Unterstriche
\usepackage{underscore}

\DeclareDelayedFloat{listing}{Listings}


% Einstellungen für minted
\setminted{
    breaklines=true,
    linenos=true,
    firstnumber=1,
    numbersep=3pt,
    autogobble=true,
    escapeinside=\#\#,
}

%setzt die Queries ans Ende
\renewcommand{\floatplace}[1]{% #1 = float type (e.g. figure)
   \begin{center}
     \hyperlink{#1.\csname thepost#1\endcsname}%
        {siehe \csname #1name\endcsname~\csname thepost#1\endcsname\ --- SQL-Query um dieses Ergebnis nachzuvollziehen.}
      % {[\csname #1name\endcsname~\csname thepost#1\endcsname\ about here.]}
   \end{center}}

% verhindert, dass endfloat für jedes neue Query eine neue Seite beginnt.
\renewcommand{\efloatseparator}{\mbox{}}

% \documentclass[fontsize=12pt,a4paper]{scrreport}
% deutsche Silbentrennung
% \usepackage[ngerman]{babel}
% Links
% \usepackage{hyperref}
% Quellcode
% \usepackage{minted}

\begin{document}

    \title{Vorrausberechnung der Straßenverkehrsentlastung
    durch eine Reaktivierung der Steigerwaldbahn
    Schweinfurt-Gerolzhofen-Wiesentheid-Kitzingen}
    \date{im Mai, 2020}
    \author{vorgelegt von Andreas Witte, unterstützt durch Stephan Wohlfeil}
    \maketitle

    \renewcommand{\contentsname}{Inhaltsverzeichnis}
    \tableofcontents

    \chapter{verkehrsfachliche Methodik}

Diese Berechnung nimmt an, dass für jeden Fahrgast die korrespondierende Autofahrt entfällt und ermittelt somit die Verlagerung von der Straße auf die Schiene.

Die Berechnung schätzt an einzelnen Stellen eine Verkehrsneuinduktion, insbesondere dort, wo mit dem PKW der Bahnhof erreicht werden muss. Hierfür wird ein Hol- und Bringverkehr angenommen.

Dadurch kann man eine Veränderung des Straßenverkehrs vorraus berechnet werden.

        \section{Fahrgastaufkommen}
Dr.\ Konrad Schliephake legte im Dezember 2016 zusammen mit Dipl.-Geogr. Stefan Albrecht und cand. Geogr. Moritz Gerber die Studie \enquote{Die Nachfrage nach Personenverkehrsleistungen bei einem Regelbetrieb der Bahnstrecke Schweinfurt-Gerolzhofen-Kitzingen} vor. Gemeinhin ist dieses Werk als \enquote{Schliephake-Studie} in der Region bekannt.

Darin berechnet die Arbeitsgruppe auf Basis eines einwohnerbezogenen Verkehrserzeugungsmodells die regelmäßigen Nutzerzahlen an regulären Werktagen vorraus, sofern die Steigerwaldbahn gemäß den Infrastrukturkriterien der BEG mit einem stündlichen Zugpaar für den Personenverkehr reaktiviert würde. Unberücksichtigt blieben dabei Tourismusverkehre und Sonderverkehre wie zum Beispiel anläßlich von Weinfesten und der Schülerverkehr. Diese Verkehre wird auch diese Berechnung daher nicht erfassen.

Die Ergebnisse dieser Studie beinhalten bereits große Abschläge und werden daher ohne weitere Veränderung als gegeben angenommen. Das bayerische Saatsministerium hat diese Studie geprüft und schriftlich mitgeteilt, dass diese Studie als \enquote{belastbar} angesehen werden kann. Es ist somit die wohl reputativste Studie, die den gesamten Abschnitt der Steigerwaldbahn beleuchtet.

Der Studie können für jeden Ort Fahrgäste entnommen werden und die Studie nennt deren Fahrziele einzeln ortsgenau.

\section{Bezugsstraßenverkehr}
Als Bezug des Straßenverkehrs werden die Verkehrszählungen des \enquote{Bayerischen Staatsministeriums für Wohnen, Bau und Verkehr} aus dem Jahr 2015 herran gezogen. Dieses sind einerseits die neuesten Daten, korrespondieren diese Daten sehr gut mit dem Veröffentlichungszeitpunkt der \enquote{Schliephake-Studie}.
Die Verkehrszählungen und die Lage der Messpunkte können durch die \enquote{BAYSIS Datenabfrage (Straßenverkehrszählungen)} online von jedem abgerufen werden.
\url{https://www.baysis.bayern.de/web/content/verkehrsdaten/SVZ/strassenverkehrszaehlungen.aspx}

\subsection{Messstellen}
Viele der Messstellen, insbesondere auf nachrangigeren Straßen wie Staats- und Kreisstraßen werden oft für Abschnitte über mehrere Orte verwendet. Wo dies der Fall ist werden diese Zuordnungen, die durch das bay. Staatsministerium für Wohnen Bau und Verkehr vorgenommen wurden, nicht verändert. 
Vereinzelt lässt sich bestimmt diskutieren und anzweifeln, ob die Messstelle für den gesamten Abschnitt repräsentativ ist, diese Diskussion ist jedoch mit dem Staatsministerium für Wohnen, Bau und Verkehr zu führen. Eine Veränderung oder verkehrsfachliche Interpratation im Zuge dieser Berechnung wäre unnötig angreifbar.

\subsection{Fehlende Messungen}
Fehlende Messungen, zum Beispiel im Zuge des Umgehungsstraßenbaus von Rüdenhausen werden nicht aufgefüllt. 

Für betreffende Straßenabschnitte wird lediglich eine absolute Veränderung des MIV angegeben, aber kein Bezug zum IST-Verkehr (relative Veränderung, Neue geschätzte Anzahl Leichtskraftfahrzeuge, \ldots) hergestellt.
        \section{Ermittlung der Wege im Straßenverkehr}
Bevor man die Veränderungen des MIV aufsummieren kann, ist es nötig, die Veränderungen auf die Straßen zu übertragen.

Für jede dieser einzelnen Wege wurde ein Query an die Google Maps API versendet. Aktuell verlangt Google pro 1000 Requests 5,00 USD.\@ Die Kosten für die Nutzung der API sind bei der geringen Anzahl an Einzel-Verbindungen, die im Schritt davor ausfindig gemacht wurden, doch sehr überschaubar, verglichen damit, wie viel Zeit und Aufwand man hätte aufwenden müssen, um das gleiche Ergebnis ohne eine solche API zu erzielen.
        \section{Berechnung}
Die Berechnung der Veränderung des Straßenverkehrs erfolgt durch Abzug der Bewegungen (Hin und Rückfahrten) aus der \enquote{Schliephake-Studie} unter der Annahme, dass ¼ der entfallenden Fahrzeuge eine zweite Person transportiert haben.

Die entfallenden Fahrzeuge werden dem Straßenverkehr abgezogen, auf der Route, die Google Maps als Dominantestes Navigationssystem für PKWs bei normaler Verkehrslage empfielt.

Wo Potenziale angenommen wurden, die nicht direkt im Ort einen Bahnhof haben, wird konservativ rechnend angenommen, dass diese Personen mit dem MIV zum jeweils angegebenen nächsten Bahnhof gelangt. Dabei wird angenommen, dass ¼ der Personen für einen Weg zwei PKW-Fahrten verursachen, da sie geholt oder gebracht werden.

Konservativ rechnend werden entfallende Fahrten abgerundet und neue Fahrten aufgerundet. Ebenfalls werden halbe Netto-Potentiale konservativ rechnend abgerundet.


    \chapter{datentechnische Methodik}
Die Berechnung wird nur nachvollziehbar, wenn die verwendete Methodik zur Berechnung dazu dokumentiert ist. Aus diesem Grund erlätert dieses Kapitel die dahinter liegende, vorgenommene Datenverarbeitung.
        \section{reationale Datenbanken}
Eine relationale Datenbank ist eine digitale Datenbank, die zur elektronischen Datenverwaltung in Computersystemen dient und auf einem tabellenbasierten relationalen Datenbankmodell beruht. Grundlage des Konzeptes relationaler Datenbanken ist die Relation.

Hier wurde das relationale Datenbankmanagementsystem mariadb 10.4 verwendet. Dies verwendet die standartisierte Querysprache SQL um Abfragen aus den Daten oder Manipulationen an den Daten vorzunehmen.

Die Relationalisierung wurde so weit wie sinnvoll vorgenommen. Zur Eingabe wurde ein kleines Webinterface mit dem MVC-Framework Cakephp4 gebaut, das an dieser Stelle keine weitere Betrachtung findet.

Im folgenden werden immer die verwendeten SQL-Queries angegeben, mit denen sich ein Ergebnis nachvollziehen lässt.

\section{Modellierung der Potenziale}
In der Schliephake-Studie sind die als \enquote{Netto-Potential} genannten Bewegungen (Hin- und Rückfahrten) relevant. Diese finden sich in dem Dokument einzeln aufgelistet, teilweise im Text mit verkehrsfachlichen Begründungen, teilweise in Tabellenform.\newline
\newline
Deren gemeinsames Merkmal ist, dass das Potential von einem Ort ausgeht und zu einem Ort zielt. Daher werden diese Orte in einer Tabelle erfasst.\newline
\newline
Die Potentiale sind verbindungen zwischen zwei Strecken, aus deren Netto-Potential wir die Veränderung des MIV ableiten und denen wir auch eine Fahrstrecke auf der Straße bezüglich Länge und Zeit zuordnen können.\newline
\newline
Die meisten Orte lassen sich klar zuordnen. Bei manchen Angaben aus der \enquote{Schliephake-Studie}, wie zum Beispiel \enquote{Nürnberg, Erlangen} wurden Punkte gewählt, die beiden Orten aus MIV-Sicht einen optimalen Zugang gewähren (in diesem Beispiel das Autobahnkreuz Nürnberg-Erlangen), wo diese gemeinsamen Orte nicht gegeben waren, wurde das Ziel oder die Quelle jeweils auf das mutmaßlich überwiegende Ziel oder Quelle gelegt (zum Beispiel \enquote{Haßfurt, Bamberg} wurde Bamberg zugeordnet).

\subsection{Tabelle places}
In der Tabelle places speichern wir die Orte, also Quelle und Ziel, und deren Koordinaten.

In der Auswertung werden wir die Koordinaten brauchen um unsere Karten zu gernerieren. LAT und LONG sind Fließkommazahlen, die Notation in Minten und Sekunden wird in der Datenbank nicht angewendet.
Um auf der Karte entsprechend einen Ort auch gemessen an seiner Wichtigkeit und Größe darzustellen, Typisieren wir nach den nachfolgenden Kategorien.
Die Kategorien haben keinerlei Einfluss auf das rechnerische Ergebnis und dienen ausschließlich um später ein passendes Rendering in den Grafiken erzeugen zu können.

\begin{itemize}
    \item \enquote{city}: Größere Städte
    \item \enquote{smallcity}: Kleinere Städte
    \item \enquote{town}: Gemeinden
    \item \enquote{village}: Dörfer
    \item \enquote{traffic}: Orte die wir zur Modellierung der Straßen anlegen, zum Beispiel Autobahnausfahrten.
\end{itemize}

Die Tabelle hat also folgenden Aufbau:
\newline
\newline
\begin{tabular}[h]{|l|l|l|l|l|}
        \hline
        id & name & LAT & LONG & type\\
        \hline
        ID des Ortes & Name des Ortes & Breitengrad & Längengrad & Typisierung des Ortes \\
        \hline
\end{tabular}
\newline
\newline
\newline
Mit SQL kann diese folgenderweise erstellt werden:
\begin{minted}{sql}
CREATE TABLE `places` (
        `id` INT(10) UNSIGNED NOT NULL AUTO_INCREMENT,
        `name` VARCHAR(100) NOT NULL DEFAULT '' COLLATE 'utf8mb4_general_ci',
        `LAT` VARCHAR(11) NOT NULL DEFAULT '0.0000000' COLLATE 'utf8mb4_general_ci',
        `LONG` VARCHAR(11) NOT NULL DEFAULT '0.0000000' COLLATE 'utf8mb4_general_ci',
        `type` ENUM('city','smallcity','town','village','traffic') NULL DEFAULT NULL COLLATE 'utf8mb4_general_ci',
        PRIMARY KEY (`id`) USING BTREE,
        UNIQUE INDEX `name` (`name`) USING BTREE
)
COLLATE='utf8mb4_general_ci'
ENGINE=InnoDB;
\end{minted}

\subsection{Tabelle potentials}

In dieser Tabelle Tragen wir die \enquote{Netto-Potenziale} aus der \enquote{Schliephake-Studie} in der Spalte \enquote{netto} ein und Errechnen daraus die Veränderung der Fahrzeugbewegungen in die Spalte \enquote{miv-change}. Die Quelle wird über die \enquote{from_id} aus der Tabelle \enquote{places} zugeordnet, das Ziel des Potenzials wird mit der \enquote{to_id} aus der Tabelle \enquote{places} zugeordnet. Das \enquote{Netto-Potenzial} wird in die Spalte \enquote{netto} übernommen. Ebenfalls Werden die Wege zu den Bahnhöfen als \enquote{Potentiale} erhoben. Die Veränderung des motorisierten Individualverkehrs wird in der Spalte \enquote{miv-change} hinterlegt. Aus Google Maps wird die Länge der Strecke entnommen und in der Spalte \enquote{length} in Metern gespeichert. Die Fahrdauer wird genau so entnommen und in der Spalte \enquote{miv-duration} in vollen Minuten gespeichert.

Die Tabelle hat also folgenden Aufbau:
\newline
\newline
\begin{tabular}[h]{|l|l|l|l|l|l|l|}
        \hline
        id & from_id & to_id & netto & miv-change & length & miv-duration\\
        \hline
        ID des Potentials & Quelle (aus \enquote{places}) & Ziel (aus \enquote{places}) & Netto-Potential laut Berechnung Dr.\ Konrad Schliephake & Daraus Errechnete Veränderung der PKW-Fahrten & Länge der Strecke in Metern & Dauer der Fahrzeit in Minuten \\
        \hline
\end{tabular}
\newline
\newline
\newline
Mit SQL kann diese folgenderweise erstellt werden:
\begin{minted}{sql}
CREATE TABLE `potentials` (
        `id` INT(10) UNSIGNED NOT NULL AUTO_INCREMENT,
        `from_id` INT(10) UNSIGNED NOT NULL,
        `to_id` INT(10) UNSIGNED NOT NULL,
        `netto` INT(10) UNSIGNED NULL DEFAULT NULL COMMENT 'Netto-Potential laut Schliephake',
        `miv-change` INT(11) NOT NULL COMMENT 'Veränderung des MIV-Verkehrs durch Reaktivierung',
        `length` INT(11) NULL DEFAULT NULL COMMENT 'Länge des Pfades in metern',
        `miv-duration` INT(10) UNSIGNED NULL DEFAULT NULL COMMENT 'Fahrdauer ohne besonderen Verkehr im PKW',
        PRIMARY KEY (`id`) USING BTREE,
        UNIQUE INDEX `from_id_to_id` (`from_id`, `to_id`) USING BTREE,
        INDEX `FK_potentials_places` (`to_id`) USING BTREE,
        CONSTRAINT `FK_potentials_places` FOREIGN KEY (`to_id`) REFERENCES `schliephake-miv-berechnung3`.`places` (`id`) ON UPDATE RESTRICT ON DELETE RESTRICT,
        CONSTRAINT `FK_potentials_places_from` FOREIGN KEY (`from_id`) REFERENCES `schliephake-miv-berechnung3`.`places` (`id`) ON UPDATE RESTRICT ON DELETE RESTRICT
)
COLLATE='utf8mb4_general_ci'
ENGINE=InnoDB;        
\end{minted}

Anmerkung: Der Einsatz von Unique Indexes und Constraints stellt hierbei die Integrität der bei der Eingabe sicher. Die Verwendung dieser Funktionen ist nicht zwingend, aber gilt als Best-Practice in der Informationstechnologie.

\section{Modellierung der Straßeninfrastruktur}

Die Straßenverkehrsinfrastruktur wird als einzelne Straßen abgebildet, welche zwei Orte verbinden. Die Richtung der Verbindung ist unerheblich für die Benutzung der Straßen. Orte können auch \enquote{virtuell} erfundene Orte sein, zum Beispiel Autobahnausfahrten. Diese liegen selten am Ort, nach dem Sie benannt sind.\newline
\newline
Die Betrachtung erfolgt hier nur für\newline
- Autobahnen (Kürzel \enquote{A}); die höchste Straßenkategorie in Deutschland;\newline
- Bundesstraßen (Kürzel \enquote{B}); meist hochwertig ausgebaute Fernstraßen für den deutschlandweiten und internationalen Verkehr, deren Baulast bei der Bundesrepublik liegt;\newline
- Staatsstraßen (Kürzel \enquote{St}); Straßen, welche für den bayerischen Straßenverkehr vom Freistaat Bayern unterhalten werden;\newline
- Kreisstraßen (Kürzel \enquote{WÜ} für den Landkreis Würzburg, \enquote{KT} für den Landkreis Kitzingen, \enquote{SW} für den Landkreis Schweinfurt)\newline
\newline
Überörtliche Ortstraßen sind ohnehin kaum betroffen und werden hier nicht weiter berücksichtigt.\newline
\newline
Ebenfalls unberücksichtigt bleibt der innerörtliche Verkehr, wenn keine Durchgangsstraße durch den jeweiligen Ort verläuft, weil die Auflösung des \enquote{Netto-Potenzials} aus der Studie von Dr.\ Konrad Schliephake bereits nicht Straßengenau erfolgt ist und dadurch sich diese Studie eine Präzision anmuten würde, die sie defakto nicht besitzt und nicht besitzen kann. Die Abbildung erfolgt in der Regel bis zum Ortsrand.

\subsection{Tabelle streets}

Eine Straße (zum Beispiel die \enquote{B286}) wind in dieser Tabelle in sinnvollen Stücken (zum Beispiel Wiesentheid-Neuses; Neuses-Gerolzhofen) unterteilt, gespeichert. In dieser Tabelle tragen wir den Straßennamen in die Spalte \enquote{street} ein. Ein Straßenabschnitt ist definiert dur die beiden Punkte \enquote{from_id}, was den Beginn des Abschnitts darstellt und der \enquote{to_id}, welche das Ende des Abschnitts bildet. Beide Spalten referenzieren auf die Tabelle \enquote{places}. Weiterhin wird in der Spalte \enquote{measurement_id} die ID aus der \enquote{BAYSIS Datenabfrage (Straßenverkehrszählungen)} hinterlegt, um die Veränderung später mit dem gezählten IST-Verkehr zu einem relativem Rückgang (\enquote{in Prozent}) zu verrechnen.

Die Tabelle hat also folgenden Aufbau:
\newline
\newline
\begin{tabular}[h]{|l|l|l|l|l|}
        \hline
        id & from_id & to_id & street & measurement_id\\
        \hline
        ID der Straße & Beginn (aus \enquote{places}) & Ende (aus \enquote{places}) & Straßenname/Nummer & Messstellen-ID der \enquote{BAYSIS Datenabfrage (Straßenverkehrszählungen)}\\
        \hline
\end{tabular}
\newline
\newline
\newline
Mit SQL kann diese folgenderweise erstellt werden:
\begin{minted}{sql}
CREATE TABLE `streets` (
        `id` INT(10) UNSIGNED NOT NULL AUTO_INCREMENT,
        `from_id` INT(10) UNSIGNED NOT NULL DEFAULT '0',
        `to_id` INT(10) UNSIGNED NOT NULL DEFAULT '0',
        `street` VARCHAR(50) NOT NULL COLLATE 'utf8mb4_general_ci',
        `measuremt_id` INT(10) UNSIGNED NULL DEFAULT NULL,
        PRIMARY KEY (`id`) USING BTREE,
        UNIQUE INDEX `from_id_to_id_street` (`from_id`, `to_id`, `street`) USING BTREE,
        INDEX `FK_streets_to_places` (`to_id`) USING BTREE,
        INDEX `FK_streets_from_places` (`from_id`) USING BTREE,
        CONSTRAINT `FK_streets_from_places` FOREIGN KEY (`from_id`) REFERENCES `schliephake-miv-berechnung3`.`places` (`id`) ON UPDATE RESTRICT ON DELETE RESTRICT,
        CONSTRAINT `FK_streets_to_places` FOREIGN KEY (`to_id`) REFERENCES `schliephake-miv-berechnung3`.`places` (`id`) ON UPDATE RESTRICT ON DELETE RESTRICT
)
COLLATE='utf8mb4_general_ci'
ENGINE=InnoDB;             
\end{minted}

Anmerkung: Der Einsatz von Unique Indexes und Constraints stellt hierbei die Integrität der bei der Eingabe sicher. Die Verwendung dieser Funktionen ist nicht zwingend, aber gilt als Best-Practice in der Informationstechnologie.

\section{Modellierung der Wege aus den Potenzialen im Straßenverkehr}

Ein Potenzial verläuft mindestens entlang einer Straße, wenn Ziel und Quelle mit dem Beginn und Ende der Straße zusammen fallen. Benutzten die Fahrer einer Relation mehrere Straßen nacheinander um von Ihrer Quelle zum Ziel und zurück zu kommen, müssen mehrere Straßen dem Potenzial zugeordnet werden. Gleichzeitig kann aber auch eine Straße von den Fahrzeugen mehrerer Potentiale genutzt werden, um von der Quelle zum Ziel und wieder zurück zu gelangen.\newline
\newline
Diese Zuordnung nennt man in relationalen Datenbanken \enquote{n:m-Beziehungen}, wobei \enquote{n:m} die Kardinalität der Beziehung spezifiziert.\@ n:m-Beziehungen benötigen eine eigene Tabelle, welche auf die beiden Tabellen je eine Spalte mit einem Fremdschlüssel beinhaltet.\newline
\newline
Durch die Zuordnung von Straßen zu Potenzialen und Potenzialen zu Straßen, kann einerseits der Weg, der für die Fahrzeuge eines Potentials angenommen wurde, nachvollzogen werden; andererseits können die Potenziale und die Veränderungen im Straßenverkehr für jede Straße aus den Potenzialen aufaddiert werden. Dadurch entsteht am Ende die Fähigkeit, vorrauszusagen, wie viel Straßenverkehr auf jeder Straße durch die Reaktivierung der Steigerwaldbahn entfallen könnte.

\subsection{Tabelle routes}

In dieser Tabelle verbinden wir die Potentiale aus der Tabelle \enquote{potentials} in der Spalte \enquote{potential_id} mit den Straßen aus der Tabelle \enquote{streets} in der Spalte \enquote{street_id}. Damit bei einer Auswertung der Route für ein Potenzial die Straßen in der richtigen Reihenfolge von Quelle zu Ziel auftauchen, wird zusätzlich beginnend mit \enquote{0} in der Spalte \enquote{number_on_route} hochgezählt.

Die Tabelle hat also folgenden Aufbau:
\newline
\newline
\begin{tabular}[h]{|l|l|l|l|}
        \hline
        id & potential_id & street_id & number_on_route\\
        \hline
        ID der Route & Zuordnung eines Potentials (aus \enquote{potentials}) & Zuordnung einer Straße (aus \enquote{streets}) & Nummer der Route auf dem Weg von Quelle zu Ziel (beginnend mit \enquote{0})\\
        \hline
\end{tabular}
\newline
\newline
\newline
Mit SQL kann diese folgenderweise erstellt werden:
\begin{minted}{sql}
CREATE TABLE `routes` (
        `id` INT(10) UNSIGNED NOT NULL AUTO_INCREMENT,
        `potential_id` INT(10) UNSIGNED NOT NULL DEFAULT '0' COMMENT 'Referenz zur Potential-ID',
        `street_id` INT(10) UNSIGNED NOT NULL DEFAULT '0' COMMENT 'Referenz zur Straßen-ID',
        `number_on_route` INT(10) UNSIGNED NOT NULL DEFAULT '0' COMMENT 'beginnend mit 0, die Nummerierung in der Abfolge',
        PRIMARY KEY (`id`) USING BTREE,
        UNIQUE INDEX `potential_id_street_id_number_on_route` (`potential_id`, `street_id`, `number_on_route`) USING BTREE,
        UNIQUE INDEX `potential_id_street_id` (`potential_id`, `street_id`) USING BTREE,
        INDEX `FK_routes_streets` (`street_id`) USING BTREE,
        INDEX `FK_routes_potentials` (`potential_id`) USING BTREE,
        CONSTRAINT `FK_routes_potentials` FOREIGN KEY (`potential_id`) REFERENCES `schliephake-miv-berechnung3`.`potentials` (`id`) ON UPDATE RESTRICT ON DELETE RESTRICT,
        CONSTRAINT `FK_routes_streets` FOREIGN KEY (`street_id`) REFERENCES `schliephake-miv-berechnung3`.`streets` (`id`) ON UPDATE RESTRICT ON DELETE RESTRICT
)
COLLATE='utf8mb4_general_ci'
ENGINE=InnoDB;      
\end{minted}

Anmerkung: Der Einsatz von Unique Indexes und Constraints stellt hierbei die Integrität der bei der Eingabe sicher. Die Verwendung dieser Funktionen ist nicht zwingend, aber gilt als Best-Practice in der Informationstechnologie.


\section{vollständiger Datenbank-Dump}
Der vollständige Dump inklusive aller Daten findet sich hier:
    \begin{minted}{sql}
    Später einfügen...
    \end{minted}
    \chapter{Verarbeitung}

        \section{Ermittlung der Fahrzeugbewegungen}
In diesem Abschnitt werden zur Überprüfbarkeit und Nachvollziehbarkeit die aus der Schliephake-Studie übernommenen Netto-Potentiale und die daraus hervorgehende Veränderung für den MIV aufgelistet. Ebenfalls wird ein SQL-Query angegeben, mit dem dieser Vorgang aus der gegebenen Datenbank wiederholt werden könnte.
            \subsection{Schweinfurt}
            \begin{tabular}{ l  l  l  l  l }
Quelle & Ziel & NettoPotenzial & MIV-Veränderung & Potenzial-ID\\
Schweinfurt & Sennfeld & 0 & 0 & 1\\ 
Schweinfurt & Gochsheim & 0 & 0 & 2\\ 
Schweinfurt & Grettstatt & 36 & -57 & 3\\ 
Schweinfurt & Gerolzhofen & 139 & -222 & 4\\ 
Schweinfurt & Wiesentheid & 36 & -57 & 5\\ 
Schweinfurt & Kitzingen & 48 & -76 & 6\\ 
\end{tabular}
\newline
\newline
\begin{listing}[htbp]
    \begin{minted}{sql}
    SELECT
    `from_places`.`name` AS `Quelle`, 
    `to_places`.`name` AS `Ziel`, 
    `potentials`.`netto` AS `NettoPotenzial`, 
    `potentials`.`miv-change` AS `MIV-Veränderung`, 
    `potentials`.`id` AS `Potenzial-ID`
    FROM `potentials`
    LEFT JOIN `places` `from_places` ON `from_places`.`id` = `potentials`.`from_id`
    LEFT JOIN `places` `to_places` ON `to_places`.`id` = `potentials`.`to_id`
    WHERE `from_places`.`name` = "Schweinfurt";
    \end{minted}
    \caption{SQL-Abfrage der Netto-Potenziale und MIV-Veränderung mit der Quelle Schweinfurt}\label{lst-fz-schweinfurt}
\end{listing}
            
            \subsection{Sennfeld}
            \begin{tabular}{ l  l  l  l  l }
    Quelle & Ziel & NettoPotenzial & MIV-Veränderung & Potenzial-ID\\ 
    Sennfeld & Schweinfurt & 109 & -174 & 8\\ 
    Sennfeld & Gerolzhofen & 7 & -11 & 10\\ 
    Sennfeld & Kitzingen & 12 & -19 & 11\\ 
    Sennfeld & Würzburg & 23 & -36 & 7\\ 
    \end{tabular}    
\newline
\newline
\begin{listing}[htbp]
\begin{minted}{sql}
SELECT
`from_places`.`name` AS `Quelle`, 
`to_places`.`name` AS `Ziel`, 
`potentials`.`netto` AS `NettoPotenzial`, 
`potentials`.`miv-change` AS `MIV-Veränderung`, 
`potentials`.`id` AS `Potenzial-ID`
FROM `potentials`
LEFT JOIN `places` `from_places` ON `from_places`.`id` = `potentials`.`from_id`
LEFT JOIN `places` `to_places` ON `to_places`.`id` = `potentials`.`to_id`
WHERE `from_places`.`name` = "Sennfeld";
\end{minted}
\caption{SQL-Abfrage der Netto-Potenziale und MIV-Veränderung mit der Quelle Sennfeld}\label{lst-fz-sennfeld}
\end{listing}

            \subsection{Gochsheim}
            \begin{tabular}{ l  l  l  l  l }
    Quelle & Ziel & NettoPotenzial & MIV-Veränderung & Potenzial-ID\\ 
    Gochsheim & Schweinfurt & 333 & -532 & 15\\ 
    Gochsheim & Gerolzhofen & 16 & -25 & 16\\ 
    Gochsheim & Würzburg, Rottendorf & 36 & -57 & 12\\ 
    Gochsheim & Bamberg, Haßfurt & 20 & -32 & 13\\ 
    Gochsheim & Bad Kissingen & 14 & -22 & 14\\ 
    \end{tabular}    
\newline
\newline
\begin{listing}[htbp]
\begin{minted}{sql}
SELECT
`from_places`.`name` AS `Quelle`, 
`to_places`.`name` AS `Ziel`, 
`potentials`.`netto` AS `NettoPotenzial`, 
`potentials`.`miv-change` AS `MIV-Veränderung`, 
`potentials`.`id` AS `Potenzial-ID`
FROM `potentials`
LEFT JOIN `places` `from_places` ON `from_places`.`id` = `potentials`.`from_id`
LEFT JOIN `places` `to_places` ON `to_places`.`id` = `potentials`.`to_id`
WHERE `from_places`.`name` = "Gochsheim";
\end{minted}
\caption{SQL-Abfrage der Netto-Potenziale und MIV-Veränderung mit der Quelle Gochsheim}\label{lst-fz-gochsheim}
\end{listing}

            \subsection{Gochsheim OT Weyer}
            \begin{tabular}{ l  l  l  l  l }
Quelle & Ziel & NettoPotenzial & MIV-Veränderung & Potenzial-ID\\ 
Gochsheim OT Weyer & Würzburg, Rottendorf & 2 & -3 & 17\\ 
Gochsheim OT Weyer & Bamberg, Haßfurt & 1 & -1 & 18\\ 
Gochsheim OT Weyer & Bad Kissingen & 1 & -1 & 19\\ 
Gochsheim OT Weyer & Schweinfurt & 33 & -82 & 20\\ 
Gochsheim OT Weyer & Gerolzhofen & 1 & -1 & 21\\ 
Gochsheim OT Weyer & Gochsheim & * & 22 & 22\\ 
\end{tabular}
\newline
\newline
* Neue Verkehre um den Bahnhof zu erreichen.
\newline
\begin{listing}[htbp]
\begin{minted}{sql}
SELECT
`from_places`.`name` AS `Quelle`, 
`to_places`.`name` AS `Ziel`, 
`potentials`.`netto` AS `NettoPotenzial`, 
`potentials`.`miv-change` AS `MIV-Veränderung`, 
`potentials`.`id` AS `Potenzial-ID`
FROM `potentials`
LEFT JOIN `places` `from_places` ON `from_places`.`id` = `potentials`.`from_id`
LEFT JOIN `places` `to_places` ON `to_places`.`id` = `potentials`.`to_id`
WHERE `from_places`.`name` = "Gochsheim OT Weyer";
\end{minted}
\caption{SQL-Abfrage der Netto-Potenziale und MIV-Veränderung mit der Quelle Weyer}\label{lst-fz-weyer}
\end{listing}

            \subsection{Schwebheim}
            \begin{tabularx}{\textwidth}{*5{X}}
Quelle & Ziel & NettoPotenzial & MIV-Veränderung & Potenzial-ID\\ 
Schwebheim & Schweinfurt & 237 & -379 & 23\\ 
Schwebheim & Gochsheim &  & 592 & 24\\ 
Schwebheim & Gerolzhofen & 6 & -9 & 25\\ 
Schwebheim & Grettstatt &  & 15 & 26\\ 
\end{tabularx}     
\newline
\newline
* Neue Verkehre um den Bahnhof zu erreichen.
\newline
\begin{listing}[htbp]
\begin{minted}{sql}
SELECT
`from_places`.`name` AS `Quelle`, 
`to_places`.`name` AS `Ziel`, 
`potentials`.`netto` AS `NettoPotenzial`, 
`potentials`.`miv-change` AS `MIV-Veränderung`, 
`potentials`.`id` AS `Potenzial-ID`
FROM `potentials`
LEFT JOIN `places` `from_places` ON `from_places`.`id` = `potentials`.`from_id`
LEFT JOIN `places` `to_places` ON `to_places`.`id` = `potentials`.`to_id`
WHERE `from_places`.`name` = "Schwebheim";
\end{minted}
\caption{SQL-Abfrage der Netto-Potenziale und MIV-Veränderung mit der Quelle Schwebheim}\label{lst-fz-schwebheim}
\end{listing}

            \subsection{Grettstatt}
            \begin{tabularx}{\textwidth}{*5{X}}
Quelle & Ziel & NettoPotenzial & MIV-Veränderung & Potenzial-ID\\ 
Grettstatt & Würzburg, Rottendorf & 12 & -19 & 27\\ 
Grettstatt & Bamberg, Haßfurt & 4 & -6 & 28\\ 
Grettstatt & Schweinfurt & 215 & -344 & 29\\ 
Grettstatt & Sennfeld & 12 & -19 & 30\\ 
Grettstatt & Gochsheim & 131 & -209 & 31\\ 
Grettstatt & Gerolzhofen & 12 & -19 & 32\\ 
\end{tabularx}       
\newline
\newline
\begin{listing}[htbp]
\begin{minted}{sql}
SELECT
`from_places`.`name` AS `Quelle`, 
`to_places`.`name` AS `Ziel`, 
`potentials`.`netto` AS `NettoPotenzial`, 
`potentials`.`miv-change` AS `MIV-Veränderung`, 
`potentials`.`id` AS `Potenzial-ID`
FROM `potentials`
LEFT JOIN `places` `from_places` ON `from_places`.`id` = `potentials`.`from_id`
LEFT JOIN `places` `to_places` ON `to_places`.`id` = `potentials`.`to_id`
WHERE `from_places`.`name` = "Grettstatt";
\end{minted}
\caption{SQL-Abfrage der Netto-Potenziale und MIV-Veränderung mit der Quelle Grettstatt}\label{lst-fz-grettstatt}
\end{listing}

            \subsection{Grettstatt OT Dürrfeld}
            \begin{tabularx}{\textwidth}{*5{X}}
Quelle & Ziel & NettoPotenzial & MIV-Veränderung & Potenzial-ID\\ 
Grettstatt OT Dürrfeld & Würzburg, Rottendorf & 2 & -3 & 33\\ 
Grettstatt OT Dürrfeld & Schweinfurt & 22 & -35 & 35\\ 
Grettstatt OT Dürrfeld & Gochsheim & 27 & -43 & 37\\ 
Grettstatt OT Dürrfeld & Gerolzhofen & 2 & -3 & 38\\ 
Grettstatt OT Dürrfeld & Grettstatt & * & 85 & 39\\ 
\end{tabularx}       
\newline
\newline
* Neue Verkehre um den Bahnhof zu erreichen.
\newline
\begin{listing}[htbp]
\begin{minted}{sql}
SELECT
`from_places`.`name` AS `Quelle`, 
`to_places`.`name` AS `Ziel`, 
`potentials`.`netto` AS `NettoPotenzial`, 
`potentials`.`miv-change` AS `MIV-Veränderung`, 
`potentials`.`id` AS `Potenzial-ID`
FROM `potentials`
LEFT JOIN `places` `from_places` ON `from_places`.`id` = `potentials`.`from_id`
LEFT JOIN `places` `to_places` ON `to_places`.`id` = `potentials`.`to_id`
WHERE `from_places`.`name` = "Grettstatt OT Dürrfeld";
\end{minted}
\caption{SQL-Abfrage der Netto-Potenziale und MIV-Veränderung mit der Quelle Dürrfeld}\label{lst-fz-duerrfeld}
\end{listing}

            \subsection{Donnersdorf}
            \begin{tabular}{ l  l  l  l  l }
Quelle & Ziel & NettoPotenzial & MIV-Veränderung & Potenzial-ID\\ 
Donnersdorf & Schweinfurt & 11 & -17 & 40\\ 
Donnersdorf & Grettstatt & * & 28 & 42\\ 
\end{tabular}    
\newline
\newline
* Neue Verkehre um den Bahnhof zu erreichen.
\newline
\begin{listing}[htbp]
\begin{minted}{sql}
SELECT
`from_places`.`name` AS `Quelle`, 
`to_places`.`name` AS `Ziel`, 
`potentials`.`netto` AS `NettoPotenzial`, 
`potentials`.`miv-change` AS `MIV-Veränderung`, 
`potentials`.`id` AS `Potenzial-ID`
FROM `potentials`
LEFT JOIN `places` `from_places` ON `from_places`.`id` = `potentials`.`from_id`
LEFT JOIN `places` `to_places` ON `to_places`.`id` = `potentials`.`to_id`
WHERE `from_places`.`name` = "Donnersdorf";
\end{minted}
\caption{SQL-Abfrage der Netto-Potenziale und MIV-Veränderung mit der Quelle Donnersdorf}\label{lst-fz-donnersdorf}
\end{listing}

            \subsection{Sulzheim}
            \begin{tabularx}{\textwidth}{*5{X}}
Quelle & Ziel & NettoPotenzial & MIV-Veränderung & Potenzial-ID\\ 
Sulzheim & Schweinfurt & 54 & -86 & 43\\ 
Sulzheim & Gochsheim & 3 & -4 & 44\\ 
Sulzheim & Sennfeld & 2 & -3 & 45\\ 
Sulzheim & Gerolzhofen & 141 & -225 & 46\\ 
Sulzheim & Kitzingen & 2 & -3 & 47\\ 
Sulzheim & Alitzheim & * & 510 & 48\\ 
\end{tabularx}    
\newline
\newline
* Neue Verkehre um den Bahnhof zu erreichen.
\newline
\begin{listing}[htbp]
\begin{minted}{sql}
SELECT
`from_places`.`name` AS `Quelle`, 
`to_places`.`name` AS `Ziel`, 
`potentials`.`netto` AS `NettoPotenzial`, 
`potentials`.`miv-change` AS `MIV-Veränderung`, 
`potentials`.`id` AS `Potenzial-ID`
FROM `potentials`
LEFT JOIN `places` `from_places` ON `from_places`.`id` = `potentials`.`from_id`
LEFT JOIN `places` `to_places` ON `to_places`.`id` = `potentials`.`to_id`
WHERE `from_places`.`name` = "Sulzheim";
\end{minted}
\caption{SQL-Abfrage der Netto-Potenziale und MIV-Veränderung mit der Quelle Sulzheim}\label{lst-fz-sulzheim}
\end{listing}

            \subsection{Alitzheim}
            \begin{tabular}{ l  l  l  l  l }
Quelle & Ziel & NettoPotenzial & MIV-Veränderung & Potenzial-ID\\ 
Alitzheim & Schweinfurt & 39 & -62 & 49\\ 
Alitzheim & Gochsheim & 2 & -3 & 50\\ 
Alitzheim & Sennfeld & 2 & -3 & 51\\ 
Alitzheim & Gerolzhofen & 102 & -163 & 52\\ 
Alitzheim & Kitzingen & 2 & -3 & 53\\ 
\end{tabular}        
\newline
\newline
\begin{listing}[htbp]
\begin{minted}{sql}
SELECT
`from_places`.`name` AS `Quelle`, 
`to_places`.`name` AS `Ziel`, 
`potentials`.`netto` AS `NettoPotenzial`, 
`potentials`.`miv-change` AS `MIV-Veränderung`, 
`potentials`.`id` AS `Potenzial-ID`
FROM `potentials`
LEFT JOIN `places` `from_places` ON `from_places`.`id` = `potentials`.`from_id`
LEFT JOIN `places` `to_places` ON `to_places`.`id` = `potentials`.`to_id`
WHERE `from_places`.`name` = "Alitzheim";
\end{minted}
\caption{SQL-Abfrage der Netto-Potenziale und MIV-Veränderung mit der Quelle Alitzheim}\label{lst-fz-alitzheim}
\end{listing}

            \subsection{Mönchstockheim}
            \begin{tabularx}{\textwidth}{*5{X}}
Quelle & Ziel & NettoPotenzial & MIV-Veränderung & Potenzial-ID\\ 
Mönchstockheim & Schweinfurt & 15 & -24 & 54\\ 
Mönchstockheim & Gochsheim & 1 & -1 & 55\\ 
Mönchstockheim & Alitzheim & * & 40 & 56\\ 
\end{tabularx}    
\newline
\newline
* Neue Verkehre um den Bahnhof zu erreichen.
\newline
\begin{listing}[htbp]
\begin{minted}{sql}
SELECT
`from_places`.`name` AS `Quelle`, 
`to_places`.`name` AS `Ziel`, 
`potentials`.`netto` AS `NettoPotenzial`, 
`potentials`.`miv-change` AS `MIV-Veränderung`, 
`potentials`.`id` AS `Potenzial-ID`
FROM `potentials`
LEFT JOIN `places` `from_places` ON `from_places`.`id` = `potentials`.`from_id`
LEFT JOIN `places` `to_places` ON `to_places`.`id` = `potentials`.`to_id`
WHERE `from_places`.`name` = "Mönchstockheim";
\end{minted}
\caption{SQL-Abfrage der Netto-Potenziale und MIV-Veränderung mit der Quelle Mönchstockheim}\label{lst-fz-moenchstockheim}
\end{listing}

            \subsection{Vögnitz}
            \begin{tabular}{ l  l  l  l  l }
Quelle & Ziel & NettoPotenzial & MIV-Veränderung & Potenzial-ID\\ 
Vögnitz & Schweinfurt & 8 & -12 & 57\\ 
Vögnitz & Alitzheim &  & 20 & 58\\ 
\end{tabular}
\newline
\newline
* Neue Verkehre um den Bahnhof zu erreichen.
\newline
\begin{listing}[htbp]
\begin{minted}{sql}
SELECT
`from_places`.`name` AS `Quelle`, 
`to_places`.`name` AS `Ziel`, 
`potentials`.`netto` AS `NettoPotenzial`, 
`potentials`.`miv-change` AS `MIV-Veränderung`, 
`potentials`.`id` AS `Potenzial-ID`
FROM `potentials`
LEFT JOIN `places` `from_places` ON `from_places`.`id` = `potentials`.`from_id`
LEFT JOIN `places` `to_places` ON `to_places`.`id` = `potentials`.`to_id`
WHERE `from_places`.`name` = "Vögnitz";
\end{minted}
\caption{SQL-Abfrage der Netto-Potenziale und MIV-Veränderung mit der Quelle Vögnitz}\label{lst-fz-voegnitz}
\end{listing}

            \subsection{Kolitzheim}
            Berücksichtigt wurden nur die Ortsteile der Gemeinde, für die eine Benutzung der Schiene plausibel ist:
            - Herlheim\newline
            - Oberspießheim\newline
            - Unterspießheim\newline
            - Zeilitzheim

            \subsection{Kolitzheim OT Herlheim}
            \begin{tabular}{ l  l  l  l  l }
Quelle & Ziel & NettoPotenzial & MIV-Veränderung & Potenzial-ID\\ 
Herlheim & Schweinfurt & 26 & -41 & 59\\ 
Herlheim & Gochsheim & 1 & -1 & 60\\ 
Herlheim & Sennfeld & 1 & -1 & 61\\ 
Herlheim & Alitzheim & * & 70 & 62\\ 
\end{tabular}    
\newline
\newline
* Neue Verkehre um den Bahnhof zu erreichen.
\newline
\begin{listing}[htbp]
\begin{minted}{sql}
SELECT
`from_places`.`name` AS `Quelle`, 
`to_places`.`name` AS `Ziel`, 
`potentials`.`netto` AS `NettoPotenzial`, 
`potentials`.`miv-change` AS `MIV-Veränderung`, 
`potentials`.`id` AS `Potenzial-ID`
FROM `potentials`
LEFT JOIN `places` `from_places` ON `from_places`.`id` = `potentials`.`from_id`
LEFT JOIN `places` `to_places` ON `to_places`.`id` = `potentials`.`to_id`
WHERE `from_places`.`name` = "Herlheim";
\end{minted}
\caption{SQL-Abfrage der Netto-Potenziale und MIV-Veränderung mit der Quelle Herlheim}\label{lst-fz-herlheim}
\end{listing}

            \subsection{Kolitzheim OT Oberspießheim}
            \begin{tabular}{ c  c  c  c  c }
Quelle & Ziel & NettoPotenzial & MIV-Veränderung & Potenzial-ID\\ 
Oberspießheim & Kitzingen & 1 & -1 & 63\\ 
Oberspießheim & Alitzheim & * & 3 & 64\\ 
\end{tabular}
\newline
\newline
* Neue Verkehre um den Bahnhof zu erreichen.
\newline
\begin{listing}[htbp]
\begin{minted}{sql}
SELECT
`from_places`.`name` AS `Quelle`, 
`to_places`.`name` AS `Ziel`, 
`potentials`.`netto` AS `NettoPotenzial`, 
`potentials`.`miv-change` AS `MIV-Veränderung`, 
`potentials`.`id` AS `Potenzial-ID`
FROM `potentials`
LEFT JOIN `places` `from_places` ON `from_places`.`id` = `potentials`.`from_id`
LEFT JOIN `places` `to_places` ON `to_places`.`id` = `potentials`.`to_id`
WHERE `from_places`.`name` = "Oberspießheim";
\end{minted}
\caption{SQL-Abfrage der Netto-Potenziale und MIV-Veränderung mit der Quelle Oberspießheim}\label{lst-fz-oberspiessheim}
\end{listing}

            \subsection{Kolitzheim OT Unterspießheim}
            \begin{tabularx}{\textwidth}{*5{X}}
Quelle & Ziel & NettoPotenzial & MIV-Veränderung & Potenzial-ID\\ 
Unterspießheim & Lülsfeld & 1 & -1 & 65\\ 
Unterspießheim & Wiesentheid & 1 & -1 & 66\\ 
Unterspießheim & Kitzingen & 1 & -1 & 67\\ 
Unterspießheim & Alitzheim & * & 8 & 68\\ 
\end{tabularx}    
\newline
\newline
* Neue Verkehre um den Bahnhof zu erreichen.
\newline
\begin{listing}[htbp]
\begin{minted}{sql}
SELECT
`from_places`.`name` AS `Quelle`, 
`to_places`.`name` AS `Ziel`, 
`potentials`.`netto` AS `NettoPotenzial`, 
`potentials`.`miv-change` AS `MIV-Veränderung`, 
`potentials`.`id` AS `Potenzial-ID`
FROM `potentials`
LEFT JOIN `places` `from_places` ON `from_places`.`id` = `potentials`.`from_id`
LEFT JOIN `places` `to_places` ON `to_places`.`id` = `potentials`.`to_id`
WHERE `from_places`.`name` = "Unterspießheim";
\end{minted}
\caption{SQL-Abfrage der Netto-Potenziale und MIV-Veränderung mit der Quelle Unterspießheim}\label{lst-fz-unterspiessheim}
\end{listing}

            \subsection{Kolitzheim OT Zeilitzheim}
            \begin{tabularx}{\textwidth}{*5{X}}
Quelle & Ziel & NettoPotenzial & MIV-Veränderung & Potenzial-ID\\ 
Zeilitzheim & Wiesentheid & 1 & -1 & 69\\ 
Zeilitzheim & Kitzingen & 1 & -1 & 70\\ 
Zeilitzheim & Alitzheim & * & 5 & 71\\ 
\end{tabularx}    
\newline
\newline
* Neue Verkehre um den Bahnhof zu erreichen.
\newline
\begin{listing}[htbp]
\begin{minted}{sql}
SELECT
`from_places`.`name` AS `Quelle`, 
`to_places`.`name` AS `Ziel`, 
`potentials`.`netto` AS `NettoPotenzial`, 
`potentials`.`miv-change` AS `MIV-Veränderung`, 
`potentials`.`id` AS `Potenzial-ID`
FROM `potentials`
LEFT JOIN `places` `from_places` ON `from_places`.`id` = `potentials`.`from_id`
LEFT JOIN `places` `to_places` ON `to_places`.`id` = `potentials`.`to_id`
WHERE `from_places`.`name` = "Zeilitzheim";
\end{minted}
\caption{SQL-Abfrage der Netto-Potenziale und MIV-Veränderung mit der Quelle Zeilitzheim}\label{lst-fz-zeilitzheim}
\end{listing}

            \subsection{Gerolzhofen}
            \begin{tabular}{ l  l  l  l  l }
Quelle & Ziel & NettoPotenzial & MIV-Veränderung & Potenzial-ID\\ 
Gerolzhofen & Rottendorf & 4 & -6 & 72\\ 
Gerolzhofen & Würzburg & 41 & -65 & 73\\ 
Gerolzhofen & Haßfurt & 12 & -19 & 74\\ 
Gerolzhofen & Bad Kissingen & 7 & -11 & 75\\ 
Gerolzhofen & Schweinfurt & 411 & -657 & 76\\ 
Gerolzhofen & Sennfeld & 17 & -27 & 77\\ 
Gerolzhofen & Gochsheim & 22 & -35 & 78\\ 
Gerolzhofen & Lülsfeld & 6 & -9 & 79\\ 
Gerolzhofen & Prichsenstadt & 16 & -25 & 80\\ 
Gerolzhofen & Wiesentheid & 23 & -36 & 81\\ 
Gerolzhofen & Kitzingen & 108 & -172 & 82\\ 
\end{tabular}
\newline
\newline
\begin{listing}[htbp]
\begin{minted}{sql}
SELECT
`from_places`.`name` AS `Quelle`, 
`to_places`.`name` AS `Ziel`, 
`potentials`.`netto` AS `NettoPotenzial`, 
`potentials`.`miv-change` AS `MIV-Veränderung`, 
`potentials`.`id` AS `Potenzial-ID`
FROM `potentials`
LEFT JOIN `places` `from_places` ON `from_places`.`id` = `potentials`.`from_id`
LEFT JOIN `places` `to_places` ON `to_places`.`id` = `potentials`.`to_id`
WHERE `from_places`.`name` = "Gerolzhofen";
\end{minted}
\caption{SQL-Abfrage der Netto-Potenziale und MIV-Veränderung mit der Quelle Gerolzhofen}\label{lst-fz-gerolzhofen}
\end{listing}
    

            \subsection{Dingolshausen}
            \begin{tabular}{ l  l  l  l  l }
Quelle & Ziel & NettoPotenzial & MIV-Veränderung & Potenzial-ID\\ 
Dingolshausen & Würzburg & 3 & -4 & 83\\ 
Dingolshausen & Schweinfurt & 29 & -46 & 84\\ 
Dingolshausen & Lülsfeld & 1 & -1 & 85\\ 
Dingolshausen & Kitzingen & 8 & -12 & 86\\ 
Dingolshausen & Gerolzhofen & * & 102 & 87\\ 
\end{tabular}
\newline
\newline
* Neue Verkehre um den Bahnhof zu erreichen.
\newline
\begin{listing}[htbp]
\begin{minted}{sql}
SELECT
`from_places`.`name` AS `Quelle`, 
`to_places`.`name` AS `Ziel`, 
`potentials`.`netto` AS `NettoPotenzial`, 
`potentials`.`miv-change` AS `MIV-Veränderung`, 
`potentials`.`id` AS `Potenzial-ID`
FROM `potentials`
LEFT JOIN `places` `from_places` ON `from_places`.`id` = `potentials`.`from_id`
LEFT JOIN `places` `to_places` ON `to_places`.`id` = `potentials`.`to_id`
WHERE `from_places`.`name` = "Dingolshausen";
\end{minted}
\caption{SQL-Abfrage der Netto-Potenziale und MIV-Veränderung mit der Quelle Dingolshausen}\label{lst-fz-dingolhausen}
\end{listing}

            \subsection{Dingolshausen OT Bischwind}
            \begin{tabular}{ l  l  l  l  l }
Quelle & Ziel & NettoPotenzial & MIV-Veränderung & Potenzial-ID\\ 
Dingolshausen OT Bischwind & Würzburg & 1 & -1 & 88\\ 
Dingolshausen OT Bischwind & Schweinfurt & 6 & -9 & 89\\ 
Dingolshausen OT Bischwind & Kitzingen & 2 & -3 & 91\\ 
Dingolshausen OT Bischwind & Gerolzhofen & * & 23 & 92\\ 
\end{tabular}
\newline
\newline
* Neue Verkehre um den Bahnhof zu erreichen.
\newline
\begin{listing}[htbp]
\begin{minted}{sql}
SELECT
`from_places`.`name` AS `Quelle`, 
`to_places`.`name` AS `Ziel`, 
`potentials`.`netto` AS `NettoPotenzial`, 
`potentials`.`miv-change` AS `MIV-Veränderung`, 
`potentials`.`id` AS `Potenzial-ID`
FROM `potentials`
LEFT JOIN `places` `from_places` ON `from_places`.`id` = `potentials`.`from_id`
LEFT JOIN `places` `to_places` ON `to_places`.`id` = `potentials`.`to_id`
WHERE `from_places`.`name` = "Dingolshausen OT Bischwind";
\end{minted}
\caption{SQL-Abfrage der Netto-Potenziale und MIV-Veränderung mit der Quelle Bischwind}\label{lst-fz-bischwind}
\end{listing}

            \subsection{Michelau}
            \begin{tabular}{ l  l  l  l  l }
Quelle & Ziel & NettoPotenzial & MIV-Veränderung & Potenzial-ID\\ 
Michelau & Schweinfurt & 12 & -18 & 93\\ 
Michelau & Gerolzhofen & * & 30 & 94\\ 
\end{tabular}
\newline
\newline
* Neue Verkehre um den Bahnhof zu erreichen.
\newline
\begin{listing}[htbp]
\begin{minted}{sql}
SELECT
`from_places`.`name` AS `Quelle`, 
`to_places`.`name` AS `Ziel`, 
`potentials`.`netto` AS `NettoPotenzial`, 
`potentials`.`miv-change` AS `MIV-Veränderung`, 
`potentials`.`id` AS `Potenzial-ID`
FROM `potentials`
LEFT JOIN `places` `from_places` ON `from_places`.`id` = `potentials`.`from_id`
LEFT JOIN `places` `to_places` ON `to_places`.`id` = `potentials`.`to_id`
WHERE `from_places`.`name` = "Michelau";
\end{minted}
\caption{SQL-Abfrage der Netto-Potenziale und MIV-Veränderung mit der Quelle Michelau}\label{lst-fz-michelau}
\end{listing}
            
            \subsection{Frankenwinheim}
            \begin{tabular}{ l  l  l  l  l }
Quelle & Ziel & NettoPotenzial & MIV-Veränderung & Potenzial-ID\\ 
Frankenwinheim & Würzburg & 6 & -9 & 95\\ 
Frankenwinheim & Schweinfurt & 20 & -32 & 96\\ 
Frankenwinheim & Kitzingen & 3 & -2 & 97\\ 
Frankenwinheim & Gerolzhofen & * & 65 & 98\\ 
Frankenwinheim & Lülsfeld & * & 8 & 99\\ 
\end{tabular}    
\newline
\newline
* Neue Verkehre um den Bahnhof zu erreichen.
\newline
\begin{listing}[htbp]
\begin{minted}{sql}
SELECT
`from_places`.`name` AS `Quelle`, 
`to_places`.`name` AS `Ziel`, 
`potentials`.`netto` AS `NettoPotenzial`, 
`potentials`.`miv-change` AS `MIV-Veränderung`, 
`potentials`.`id` AS `Potenzial-ID`
FROM `potentials`
LEFT JOIN `places` `from_places` ON `from_places`.`id` = `potentials`.`from_id`
LEFT JOIN `places` `to_places` ON `to_places`.`id` = `potentials`.`to_id`
WHERE `from_places`.`name` = "Frankenwinheim";
\end{minted}
\caption{SQL-Abfrage der Netto-Potenziale und MIV-Veränderung mit der Quelle Frankenwinheim}\label{lst-fz-frankenwinheim}
\end{listing}

            \subsection{Oberschwarzach}
            \begin{tabular}{ l  l  l  l  l }
Quelle & Ziel & NettoPotenzial & MIV-Veränderung & Potenzial-ID\\ 
Oberschwarzach & Schweinfurt & 18 & -28 & 100\\ 
Oberschwarzach & Lülsfeld & 3 & -2 & 101\\ 
Oberschwarzach & Wiesentheid & 4 & -3 & 102\\ 
Oberschwarzach & Kitzingen & 3 & -2 & 103\\ 
Oberschwarzach & Järkendorf & * & 13 & 104\\ 
Oberschwarzach & Gerolzhofen & * & 45 & 105\\ 
\end{tabular}
\newline
\newline
* Neue Verkehre um den Bahnhof zu erreichen.
\newline
\begin{listing}[htbp]
\begin{minted}{sql}
SELECT
`from_places`.`name` AS `Quelle`, 
`to_places`.`name` AS `Ziel`, 
`potentials`.`netto` AS `NettoPotenzial`, 
`potentials`.`miv-change` AS `MIV-Veränderung`, 
`potentials`.`id` AS `Potenzial-ID`
FROM `potentials`
LEFT JOIN `places` `from_places` ON `from_places`.`id` = `potentials`.`from_id`
LEFT JOIN `places` `to_places` ON `to_places`.`id` = `potentials`.`to_id`
WHERE `from_places`.`name` = "Oberschwarzach";
\end{minted}
\caption{SQL-Abfrage der Netto-Potenziale und MIV-Veränderung mit der Quelle Oberschwarzach}\label{lst-fz-oberschwarzach}
\end{listing}

            \subsection{Volkach}
            \begin{tabular}{ l  l  l  l  l }
Quelle & Ziel & NettoPotenzial & MIV-Veränderung & Potenzial-ID\\ 
Volkach & Schweinfurt & 10 & -16 & 106\\ 
Volkach & Kitzingen & 18 & -28 & 107\\ 
Volkach & Lülsfeld & * & 45 & 108\\ 
Volkach & Gerolzhofen & * & 25 & 109\\ 
\end{tabular}
\newline
\newline
* Neue Verkehre um den Bahnhof zu erreichen.
\newline
\begin{listing}[htbp]
\begin{minted}{sql}
SELECT
`from_places`.`name` AS `Quelle`, 
`to_places`.`name` AS `Ziel`, 
`potentials`.`netto` AS `NettoPotenzial`, 
`potentials`.`miv-change` AS `MIV-Veränderung`, 
`potentials`.`id` AS `Potenzial-ID`
FROM `potentials`
LEFT JOIN `places` `from_places` ON `from_places`.`id` = `potentials`.`from_id`
LEFT JOIN `places` `to_places` ON `to_places`.`id` = `potentials`.`to_id`
WHERE `from_places`.`name` = "Volkach";
\end{minted}
\caption{SQL-Abfrage der Netto-Potenziale und MIV-Veränderung mit der Quelle Volkach}\label{lst-fz-volkach}
\end{listing}

            \subsection{Lülsfeld}
            \begin{tabularx}{\textwidth}{*5{X}}
Quelle & Ziel & NettoPotenzial & MIV-Veränderung & Potenzial-ID\\ 
Lülsfeld & Schweinfurt & 20 & -32 & 110\\ 
Lülsfeld & Gerolzhofen & 38 & -60 & 111\\ 
Lülsfeld & Wiesentheid & 2 & -3 & 112\\ 
Lülsfeld & Kitzingen & 4 & -6 & 113\\ 
\end{tabularx}
\newline
\newline
\begin{listing}[htbp]
\begin{minted}{sql}
SELECT
`from_places`.`name` AS `Quelle`, 
`to_places`.`name` AS `Ziel`, 
`potentials`.`netto` AS `NettoPotenzial`, 
`potentials`.`miv-change` AS `MIV-Veränderung`, 
`potentials`.`id` AS `Potenzial-ID`
FROM `potentials`
LEFT JOIN `places` `from_places` ON `from_places`.`id` = `potentials`.`from_id`
LEFT JOIN `places` `to_places` ON `to_places`.`id` = `potentials`.`to_id`
WHERE `from_places`.`name` = "Lülsfeld";
\end{minted}
\caption{SQL-Abfrage der Netto-Potenziale und MIV-Veränderung mit der Quelle Lülsfeld}\label{lst-fz-luelsfeld}
\end{listing}

            \subsection{Schallfeld}
            \begin{tabular}{ l  l  l  l  l }
Quelle & Ziel & NettoPotenzial & MIV-Veränderung & Potenzial-ID\\ 
Schallfeld & Schweinfurt & 9 & -14 & 114\\ 
Schallfeld & Wiesentheid & 1 & -1 & 115\\ 
Schallfeld & Kitzingen & 2 & -3 & 116\\ 
Schallfeld & Lülsfeld & * & 30 & 117\\ 
\end{tabular}    
\newline
\newline
* Neue Verkehre um den Bahnhof zu erreichen.
\newline
\begin{listing}[htbp]
\begin{minted}{sql}
SELECT
`from_places`.`name` AS `Quelle`, 
`to_places`.`name` AS `Ziel`, 
`potentials`.`netto` AS `NettoPotenzial`, 
`potentials`.`miv-change` AS `MIV-Veränderung`, 
`potentials`.`id` AS `Potenzial-ID`
FROM `potentials`
LEFT JOIN `places` `from_places` ON `from_places`.`id` = `potentials`.`from_id`
LEFT JOIN `places` `to_places` ON `to_places`.`id` = `potentials`.`to_id`
WHERE `from_places`.`name` = "Schallfeld";
\end{minted}
\caption{SQL-Abfrage der Netto-Potenziale und MIV-Veränderung mit der Quelle Schallfeld}\label{lst-fz-schallfeld}
\end{listing}

            \subsection{Prichsenstadt}
            \begin{tabular}{ l  l  l  l  l }
Quelle & Ziel & NettoPotenzial & MIV-Veränderung & Potenzial-ID\\ 
Prichsenstadt & Bamberg, Haßfurt & 3 & -4 & 118\\ 
Prichsenstadt & Schweinfurt & 9 & -14 & 119\\ 
Prichsenstadt & Gerolzhofen & 9 & -14 & 120\\ 
Prichsenstadt & Lülsfeld & 13 & -20 & 121\\ 
Prichsenstadt & Wiesentheid & 73 & -116 & 122\\ 
Prichsenstadt & Kitzingen & 47 & -75 & 123\\ 
Prichsenstadt & Würzburg, Rottendorf & 11 & -17 & 124\\ 
Prichsenstadt & Nürnberg, Erlangen & 3 & -4 & 125\\ 
\end{tabular}    
\newline
\newline
\begin{listing}[htbp]
\begin{minted}{sql}
SELECT
`from_places`.`name` AS `Quelle`, 
`to_places`.`name` AS `Ziel`, 
`potentials`.`netto` AS `NettoPotenzial`, 
`potentials`.`miv-change` AS `MIV-Veränderung`, 
`potentials`.`id` AS `Potenzial-ID`
FROM `potentials`
LEFT JOIN `places` `from_places` ON `from_places`.`id` = `potentials`.`from_id`
LEFT JOIN `places` `to_places` ON `to_places`.`id` = `potentials`.`to_id`
WHERE `from_places`.`name` = "Prichsenstadt";
\end{minted}
\caption{SQL-Abfrage der Netto-Potenziale und MIV-Veränderung mit der Quelle Prichsenstadt}\label{lst-fz-prichsenstadt}
\end{listing}

            \subsection{Prichsenstadt OT Altenschönbach}
            \begin{tabular}{ l  l  l  l  l }
Quelle & Ziel & NettoPotenzial & MIV-Veränderung & Potenzial-ID\\ 
Prichsenstadt OT Altenschönbach & Bamberg, Haßfurt & 1 & -1 & 126\\ 
Prichsenstadt OT Altenschönbach & Schweinfurt & 2 & -3 & 127\\ 
Prichsenstadt OT Altenschönbach & Gerolzhofen & 2 & -3 & 128\\ 
Prichsenstadt OT Altenschönbach & Lülsfeld & 3 & -4 & 129\\ 
Prichsenstadt OT Altenschönbach & Kitzingen & 10 & -16 & 130\\ 
Prichsenstadt OT Altenschönbach & Würzburg, Rottendorf & 4 & -6 & 131\\ 
Prichsenstadt OT Altenschönbach & Nürnberg, Erlangen & 1 & -1 & 132\\ 
Prichsenstadt OT Altenschönbach & Prichsenstadt & * & 57 & 133\\ 
\end{tabular}    
\newline
\newline
* Neue Verkehre um den Bahnhof zu erreichen.
\newline
\begin{listing}[htbp]
\begin{minted}{sql}
SELECT
`from_places`.`name` AS `Quelle`, 
`to_places`.`name` AS `Ziel`, 
`potentials`.`netto` AS `NettoPotenzial`, 
`potentials`.`miv-change` AS `MIV-Veränderung`, 
`potentials`.`id` AS `Potenzial-ID` 
FROM `potentials`
LEFT JOIN `places` `from_places` ON `from_places`.`id` = `potentials`.`from_id`
LEFT JOIN `places` `to_places` ON `to_places`.`id` = `potentials`.`to_id`
WHERE `from_places`.`name` = "Prichsenstadt OT Altenschönbach";
\end{minted}
\caption{SQL-Abfrage der Netto-Potenziale und MIV-Veränderung mit der Quelle Altenschönbach}\label{lst-fz-altenschoenbach}
\end{listing}

            \subsection{Prichsenstadt OT Bimbach} 
            \begin{tabularx}{\textwidth}{*5{X}}
Quelle & Ziel & NettoPotenzial & MIV-Veränderung & Potenzial-ID\\ 
Prichsenstadt OT Bimbach & Bamberg, Haßfurt & 1 & -1 & 134\\ 
Prichsenstadt OT Bimbach & Schweinfurt & 1 & -1 & 135\\ 
Prichsenstadt OT Bimbach & Gerolzhofen & 1 & -1 & 136\\ 
Prichsenstadt OT Bimbach & Lülsfeld & 1 & -1 & 137\\ 
Prichsenstadt OT Bimbach & Wiesentheid & 6 & -9 & 138\\ 
Prichsenstadt OT Bimbach & Kitzingen & 4 & -6 & 139\\ 
Prichsenstadt OT Bimbach & Würzburg, Rottendorf & 2 & -3 & 140\\ 
Prichsenstadt OT Bimbach & Nürnberg, Erlangen & 1 & -1 & 141\\ 
Prichsenstadt OT Bimbach & Järkendorf & * & 43 & 142\\ 
\end{tabularx}
\newline
\newline
* Neue Verkehre um den Bahnhof zu erreichen.
\newline
\begin{listing}[htbp]
\begin{minted}{sql}
SELECT
`from_places`.`name` AS `Quelle`, 
`to_places`.`name` AS `Ziel`, 
`potentials`.`netto` AS `NettoPotenzial`, 
`potentials`.`miv-change` AS `MIV-Veränderung`, 
`potentials`.`id` AS `Potenzial-ID`
FROM `potentials`
LEFT JOIN `places` `from_places` ON `from_places`.`id` = `potentials`.`from_id`
LEFT JOIN `places` `to_places` ON `to_places`.`id` = `potentials`.`to_id`
WHERE `from_places`.`name` = "Prichsenstadt OT Bimbach";
\end{minted}
\caption{SQL-Abfrage der Netto-Potenziale und MIV-Veränderung mit der Quelle Bimbach}\label{lst-fz-bimbach}
\end{listing}

            \subsection{Prichsenstadt OT Brünnau}
            \begin{tabular}{ l  l  l  l  l }
Quelle & Ziel & NettoPotenzial & MIV-Veränderung & Potenzial-ID\\ 
Prichsenstadt OT Brünnau & Schweinfurt & 1 & -1 & 143\\ 
Prichsenstadt OT Brünnau & Gerolzhofen & 1 & -1 & 144\\ 
Prichsenstadt OT Brünnau & Lülsfeld & 1 & -1 & 145\\ 
Prichsenstadt OT Brünnau & Wiesentheid & 7 & -11 & 146\\ 
Prichsenstadt OT Brünnau & Kitzingen & 5 & -8 & 147\\ 
Prichsenstadt OT Brünnau & Würzburg, Rottendorf & 2 & -3 & 148\\ 
Prichsenstadt OT Brünnau & Järkendorf &  & 43 & 149\\ 
\end{tabular}
\newline
\newline
* Neue Verkehre um den Bahnhof zu erreichen.
\newline
\begin{listing}[htbp]
\begin{minted}{sql}
SELECT
`from_places`.`name` AS `Quelle`, 
`to_places`.`name` AS `Ziel`, 
`potentials`.`netto` AS `NettoPotenzial`, 
`potentials`.`miv-change` AS `MIV-Veränderung`, 
`potentials`.`id` AS `Potenzial-ID`
FROM `potentials`
LEFT JOIN `places` `from_places` ON `from_places`.`id` = `potentials`.`from_id`
LEFT JOIN `places` `to_places` ON `to_places`.`id` = `potentials`.`to_id`
WHERE `from_places`.`name` = "Prichsenstadt OT Brünnau";
\end{minted}
\caption{SQL-Abfrage der Netto-Potenziale und MIV-Veränderung mit der Quelle Brünnau}\label{lst-fz-bruennau}
\end{listing}

            \subsection{Järkendorf}
            \begin{tabularx}{\textwidth}{*5{X}}
Quelle & Ziel & NettoPotenzial & MIV-Veränderung & Potenzial-ID\\ 
Järkendorf & Bamberg, Haßfurt & 1 & -1 & 150\\ 
Järkendorf & Schweinfurt & 1 & -1 & 151\\ 
Järkendorf & Gerolzhofen & 1 & -1 & 152\\ 
Järkendorf & Lülsfeld & 2 & -3 & 153\\ 
Järkendorf & Wiesentheid & 10 & -16 & 154\\ 
Järkendorf & Kitzingen & 6 & -9 & 155\\ 
Järkendorf & Würzburg, Rottendorf & 2 & -3 & 156\\ 
Järkendorf & Nürnberg, Erlangen & 1 & -1 & 157\\ 
\end{tabularx}
\newline
\newline
\begin{listing}[htbp]
\begin{minted}{sql}
SELECT
`from_places`.`name` AS `Quelle`, 
`to_places`.`name` AS `Ziel`, 
`potentials`.`netto` AS `NettoPotenzial`, 
`potentials`.`miv-change` AS `MIV-Veränderung`, 
`potentials`.`id` AS `Potenzial-ID`
FROM `potentials`
LEFT JOIN `places` `from_places` ON `from_places`.`id` = `potentials`.`from_id`
LEFT JOIN `places` `to_places` ON `to_places`.`id` = `potentials`.`to_id`
WHERE `from_places`.`name` = "Järkendorf";
\end{minted}
\caption{SQL-Abfrage der Netto-Potenziale und MIV-Veränderung mit der Quelle Järkendorf}\label{lst-fz-jaerkendorf}
\end{listing}

            \subsection{Prichsenstadt OT Kirchschönbach}
            \begin{tabularx}{\textwidth}{*5{X}}
Quelle & Ziel & NettoPotenzial & MIV-Veränderung & Potenzial-ID\\ 
Prichsenstadt OT Kirchschönbach & Bamberg, Haßfurt & 1 & -1 & 158\\ 
Prichsenstadt OT Kirchschönbach & Schweinfurt & 2 & -3 & 159\\ 
Prichsenstadt OT Kirchschönbach & Gerolzhofen & 2 & -3 & 160\\ 
Prichsenstadt OT Kirchschönbach & Lülsfeld & 3 & -4 & 161\\ 
Prichsenstadt OT Kirchschönbach & Kitzingen & 10 & -16 & 162\\ 
Prichsenstadt OT Kirchschönbach & Würzburg, Rottendorf & 5 & -8 & 163\\ 
Prichsenstadt OT Kirchschönbach & Nürnberg, Erlangen & 1 & -1 & 164\\ 
Prichsenstadt OT Kirchschönbach & Prichsenstadt & * & 60 & 165\\ 
\end{tabularx}
\newline
\newline
* Neue Verkehre um den Bahnhof zu erreichen.
\newline
\begin{listing}[htbp]
\begin{minted}{sql}
SELECT
`from_places`.`name` AS `Quelle`, 
`to_places`.`name` AS `Ziel`, 
`potentials`.`netto` AS `NettoPotenzial`, 
`potentials`.`miv-change` AS `MIV-Veränderung`, 
`potentials`.`id` AS `Potenzial-ID`
FROM `potentials`
LEFT JOIN `places` `from_places` ON `from_places`.`id` = `potentials`.`from_id`
LEFT JOIN `places` `to_places` ON `to_places`.`id` = `potentials`.`to_id`
WHERE `from_places`.`name` = "Prichsenstadt OT Kirchschönbach";
\end{minted}
\caption{SQL-Abfrage der Netto-Potenziale und MIV-Veränderung mit der Quelle Kirchschönbach}\label{lst-fz-kirchschoenbach}
\end{listing}

            \subsection{Prichsenstadt OT Laub}
            \begin{tabularx}{\textwidth}{*5{X}}
Quelle & Ziel & NettoPotenzial & MIV-Veränderung & Potenzial-ID\\ 
Prichsenstadt OT Laub & Bamberg, Haßfurt & 1 & -1 & 166\\ 
Prichsenstadt OT Laub & Schweinfurt & 1 & -1 & 167\\ 
Prichsenstadt OT Laub & Gerolzhofen & 1 & -1 & 168\\ 
Prichsenstadt OT Laub & Lülsfeld & 2 & -3 & 169\\ 
Prichsenstadt OT Laub & Kitzingen & 7 & -11 & 170\\ 
Prichsenstadt OT Laub & Würzburg, Rottendorf & 3 & -4 & 171\\ 
Prichsenstadt OT Laub & Nürnberg, Erlangen & 1 & -1 & 172\\ 
Prichsenstadt OT Laub & Prichsenstadt & * & 28 & 173\\ 
Prichsenstadt OT Laub & Stadelschwarzach & * & 13 & 174\\ 
\end{tabularx}    
\newline
\newline
* Neue Verkehre um den Bahnhof zu erreichen.
\newline
\begin{listing}[htbp]
\begin{minted}{sql}
SELECT
`from_places`.`name` AS `Quelle`, 
`to_places`.`name` AS `Ziel`, 
`potentials`.`netto` AS `NettoPotenzial`, 
`potentials`.`miv-change` AS `MIV-Veränderung`, 
`potentials`.`id` AS `Potenzial-ID`
FROM `potentials`
LEFT JOIN `places` `from_places` ON `from_places`.`id` = `potentials`.`from_id`
LEFT JOIN `places` `to_places` ON `to_places`.`id` = `potentials`.`to_id`
WHERE `from_places`.`name` = "Prichsenstadt OT Laub";
\end{minted}
\caption{SQL-Abfrage der Netto-Potenziale und MIV-Veränderung mit der Quelle Laub}\label{lst-fz-laub}
\end{listing}

            \subsection{Prichsenstadt OT Neudorf}
            \begin{tabular}{ l  l  l  l  l }
Quelle & Ziel & NettoPotenzial & MIV-Veränderung & Potenzial-ID\\ 
Prichsenstadt OT Neudorf & Schweinfurt & 1 & -1 & 175\\ 
Prichsenstadt OT Neudorf & Gerolzhofen & 1 & -1 & 176\\ 
Prichsenstadt OT Neudorf & Lülsfeld & 1 & -1 & 177\\ 
Prichsenstadt OT Neudorf & Wiesentheid & 5 & -8 & 178\\ 
Prichsenstadt OT Neudorf & Kitzingen & 3 & -4 & 179\\ 
Prichsenstadt OT Neudorf & Würzburg, Rottendorf & 1 & -1 & 180\\ 
Prichsenstadt OT Neudorf & Stadelschwarzach & * & 30 & 181\\ 
\end{tabular}
\newline
\newline
* Neue Verkehre um den Bahnhof zu erreichen.
\newline
\begin{listing}[htbp]
\begin{minted}{sql}
SELECT
`from_places`.`name` AS `Quelle`, 
`to_places`.`name` AS `Ziel`, 
`potentials`.`netto` AS `NettoPotenzial`, 
`potentials`.`miv-change` AS `MIV-Veränderung`, 
`potentials`.`id` AS `Potenzial-ID`
FROM `potentials`
LEFT JOIN `places` `from_places` ON `from_places`.`id` = `potentials`.`from_id`
LEFT JOIN `places` `to_places` ON `to_places`.`id` = `potentials`.`to_id`
WHERE `from_places`.`name` = "Prichsenstadt OT Neudorf";
\end{minted}
\caption{SQL-Abfrage der Netto-Potenziale und MIV-Veränderung mit der Quelle Neudorf}\label{lst-fz-neudorf}
\end{listing}
            
            \subsection{Prichsenstadt OT Neuses}
            \begin{tabularx}{\textwidth}{*5{X}}
Quelle & Ziel & NettoPotenzial & MIV-Veränderung & Potenzial-ID\\ 
Prichsenstadt OT Neuses & Lülsfeld & 1 & -1 & 182\\ 
Prichsenstadt OT Neuses & Wiesentheid & 5 & -8 & 183\\ 
Prichsenstadt OT Neuses & Kitzingen & 3 & -4 & 184\\ 
Prichsenstadt OT Neuses & Würzburg, Rottendorf & 1 & -1 & 185\\ 
Prichsenstadt OT Neuses & Stadelschwarzach & * & 25 & 186\\ 
\end{tabularx}
\newline
\newline
* Neue Verkehre um den Bahnhof zu erreichen.
\newline
\begin{listing}[htbp]
\begin{minted}{sql}
SELECT
`from_places`.`name` AS `Quelle`, 
`to_places`.`name` AS `Ziel`, 
`potentials`.`netto` AS `NettoPotenzial`, 
`potentials`.`miv-change` AS `MIV-Veränderung`, 
`potentials`.`id` AS `Potenzial-ID`
FROM `potentials`
LEFT JOIN `places` `from_places` ON `from_places`.`id` = `potentials`.`from_id`
LEFT JOIN `places` `to_places` ON `to_places`.`id` = `potentials`.`to_id`
WHERE `from_places`.`name` = "Prichsenstadt OT Neuses";
\end{minted}
\caption{SQL-Abfrage der Netto-Potenziale und MIV-Veränderung mit der Quelle Neuses}\label{lst-fz-neuses}
\end{listing}

            \subsection{Prichsenstadt OT Stadelschwarzach}
            \begin{tabularx}{\textwidth}{*5{X}}
Quelle & Ziel & NettoPotenzial & MIV-Veränderung & Potenzial-ID\\ 
Stadelschwarzach & Bamberg, Haßfurt & 2 & -3 & 187\\ 
Stadelschwarzach & Schweinfurt & 5 & -8 & 188\\ 
Stadelschwarzach & Gerolzhofen & 5 & -8 & 189\\ 
Stadelschwarzach & Lülsfeld & 7 & -11 & 190\\ 
Stadelschwarzach & Wiesentheid & 41 & -65 & 191\\ 
Stadelschwarzach & Kitzingen & 26 & -41 & 192\\ 
Stadelschwarzach & Würzburg, Rottendorf & 6 & -9 & 193\\ 
Stadelschwarzach & Nürnberg, Erlangen & 2 & -3 & 194\\ 
\end{tabularx}    
\newline
\newline
\begin{listing}[htbp]
\begin{minted}{sql}
SELECT
`from_places`.`name` AS `Quelle`, 
`to_places`.`name` AS `Ziel`, 
`potentials`.`netto` AS `NettoPotenzial`, 
`potentials`.`miv-change` AS `MIV-Veränderung`, 
`potentials`.`id` AS `Potenzial-ID`
FROM `potentials`
LEFT JOIN `places` `from_places` ON `from_places`.`id` = `potentials`.`from_id`
LEFT JOIN `places` `to_places` ON `to_places`.`id` = `potentials`.`to_id`
WHERE `from_places`.`name` = "Stadelschwarzach";
\end{minted}
\caption{SQL-Abfrage der Netto-Potenziale und MIV-Veränderung mit der Quelle Stadelschwarzach}\label{lst-fz-stadelschwarzach}
\end{listing}

            \subsection{Wiesentheid}
            \begin{tabular}{ l  l  l  l  l }
Quelle & Ziel & NettoPotenzial & MIV-Veränderung & Potenzial-ID\\ 
Wiesentheid & Schweinfurt & 22 & -35 & 195\\ 
Wiesentheid & Gerolzhofen & 19 & -30 & 196\\ 
Wiesentheid & Prichsenstadt & 35 & -56 & 197\\ 
Wiesentheid & Kleinlangheim & 7 & -11 & 198\\ 
Wiesentheid & Kitzingen & 172 & -275 & 199\\ 
Wiesentheid & Würzburg, Rottendorf & 52 & -83 & 200\\ 
Wiesentheid & Nürnberg, Erlangen & 3 & -4 & 201\\ 
\end{tabular}    
\newline
\newline
\begin{listing}[htbp]
\begin{minted}{sql}
SELECT
`from_places`.`name` AS `Quelle`, 
`to_places`.`name` AS `Ziel`, 
`potentials`.`netto` AS `NettoPotenzial`, 
`potentials`.`miv-change` AS `MIV-Veränderung`, 
`potentials`.`id` AS `Potenzial-ID`
FROM `potentials`
LEFT JOIN `places` `from_places` ON `from_places`.`id` = `potentials`.`from_id`
LEFT JOIN `places` `to_places` ON `to_places`.`id` = `potentials`.`to_id`
WHERE `from_places`.`name` = "Wiesentheid";
\end{minted}
\caption{SQL-Abfrage der Netto-Potenziale und MIV-Veränderung mit der Quelle Wiesentheid}\label{lst-fz-wiesentheid}
\end{listing}

            \subsection{Wiesentheid OT Feuerbach}
            \begin{tabular}{ l  l  l  l  l }
Quelle & Ziel & NettoPotenzial & MIV-Veränderung & Potenzial-ID\\ 
Wiesentheid OT Feuerbach & Schweinfurt & 2 & -3 & 202\\ 
Wiesentheid OT Feuerbach & Gerolzhofen & 2 & -3 & 203\\ 
Wiesentheid OT Feuerbach & Prichsenstadt & 3 & -4 & 204\\ 
Wiesentheid OT Feuerbach & Kleinlangheim & 1 & -1 & 205\\ 
Wiesentheid OT Feuerbach & Kitzingen & 13 & -20 & 206\\ 
Wiesentheid OT Feuerbach & Würzburg, Rottendorf & 4 & -6 & 207\\ 
Wiesentheid OT Feuerbach & Nürnberg, Erlangen & 1 & -1 & 208\\ 
\end{tabular}
\newline
\newline
\begin{listing}[htbp]
\begin{minted}{sql}
SELECT
`from_places`.`name` AS `Quelle`, 
`to_places`.`name` AS `Ziel`, 
`potentials`.`netto` AS `NettoPotenzial`, 
`potentials`.`miv-change` AS `MIV-Veränderung`, 
`potentials`.`id` AS `Potenzial-ID`
FROM `potentials`
LEFT JOIN `places` `from_places` ON `from_places`.`id` = `potentials`.`from_id`
LEFT JOIN `places` `to_places` ON `to_places`.`id` = `potentials`.`to_id`
WHERE `from_places`.`name` = "Wiesentheid OT Feuerbach";
\end{minted}
\caption{SQL-Abfrage der Netto-Potenziale und MIV-Veränderung mit der Quelle Feuerbach}\label{lst-fz-feuerbach}
\end{listing}

            \subsection{Wiesentheid OT Geesdorf}
            \begin{tabular}{ l  l  l  l  l }
Quelle & Ziel & NettoPotenzial & MIV-Veränderung & Potenzial-ID\\ 
Wiesentheid OT Geesdorf & Schweinfurt & 1 & -1 & 209\\ 
Wiesentheid OT Geesdorf & Gerolzhofen & 1 & -1 & 210\\ 
Wiesentheid OT Geesdorf & Kitzingen & 1 & -1 & 211\\ 
Wiesentheid OT Geesdorf & Würzburg, Rottendorf & 6 & -9 & 212\\ 
Wiesentheid OT Geesdorf & Wiesentheid & * & 43 & 213\\ 
\end{tabular}    
\newline
\newline
* Neue Verkehre um den Bahnhof zu erreichen.
\newline
\begin{listing}[htbp]
\begin{minted}{sql}
SELECT
`from_places`.`name` AS `Quelle`, 
`to_places`.`name` AS `Ziel`, 
`potentials`.`netto` AS `NettoPotenzial`, 
`potentials`.`miv-change` AS `MIV-Veränderung`, 
`potentials`.`id` AS `Potenzial-ID`
FROM `potentials`
LEFT JOIN `places` `from_places` ON `from_places`.`id` = `potentials`.`from_id`
LEFT JOIN `places` `to_places` ON `to_places`.`id` = `potentials`.`to_id`
WHERE `from_places`.`name` = "Wiesentheid OT Geesdorf";
\end{minted}
\caption{SQL-Abfrage der Netto-Potenziale und MIV-Veränderung mit der Quelle Geesdorf}\label{lst-fz-geesdorf}
\end{listing}

            \subsection{Wiesentheid OT Reupelsdorf}
            \begin{tabularx}{\textwidth}{*5{X}}
Quelle & Ziel & NettoPotenzial & MIV-Veränderung & Potenzial-ID\\ 
Wiesentheid OT Reupelsdorf & Schweinfurt & 1 & -1 & 214\\ 
Wiesentheid OT Reupelsdorf & Gerolzhofen & 1 & -1 & 215\\ 
Wiesentheid OT Reupelsdorf & Kleinlangheim & 1 & -1 & 216\\ 
Wiesentheid OT Reupelsdorf & Kitzingen & 9 & -14 & 217\\ 
Wiesentheid OT Reupelsdorf & Würzburg, Rottendorf & 5 & -8 & 218\\ 
Wiesentheid OT Reupelsdorf & Nürnberg, Erlangen & 1 & -1 & 219\\ 
Wiesentheid OT Reupelsdorf & Stadelschwarzach & * & 5 & 220\\ 
Wiesentheid OT Reupelsdorf & Wiesentheid & * & 40 & 221\\ 
\end{tabularx}
\newline
\newline
* Neue Verkehre um den Bahnhof zu erreichen.
\newline
\begin{listing}[htbp]
\begin{minted}{sql}
SELECT
`from_places`.`name` AS `Quelle`, 
`to_places`.`name` AS `Ziel`, 
`potentials`.`netto` AS `NettoPotenzial`, 
`potentials`.`miv-change` AS `MIV-Veränderung`, 
`potentials`.`id` AS `Potenzial-ID`
FROM `potentials`
LEFT JOIN `places` `from_places` ON `from_places`.`id` = `potentials`.`from_id`
LEFT JOIN `places` `to_places` ON `to_places`.`id` = `potentials`.`to_id`
WHERE `from_places`.`name` = "Wiesentheid OT Reupelsdorf";
\end{minted}
\caption{SQL-Abfrage der Netto-Potenziale und MIV-Veränderung mit der Quelle Reupelsdorf}\label{lst-fz-reupelsdorf}
\end{listing}

            \subsection{Wiesentheid OT Untersambach}
            \begin{tabular}{ l  l  l  l  l }
Quelle & Ziel & NettoPotenzial & MIV-Veränderung & Potenzial-ID\\ 
Wiesentheid OT Untersambach & Schweinfurt & 1 & -1 & 222\\ 
Wiesentheid OT Untersambach & Gerolzhofen & 1 & -1 & 223\\ 
Wiesentheid OT Untersambach & Kitzingen & 7 & -11 & 224\\ 
Wiesentheid OT Untersambach & Würzburg, Rottendorf & 4 & -6 & 225\\ 
Wiesentheid OT Untersambach & Wiesentheid & * & 33 & 226\\ 
\end{tabular}
\newline
\newline
* Neue Verkehre um den Bahnhof zu erreichen.
\newline
\begin{listing}[htbp]
\begin{minted}{sql}
SELECT
`from_places`.`name` AS `Quelle`, 
`to_places`.`name` AS `Ziel`, 
`potentials`.`netto` AS `NettoPotenzial`, 
`potentials`.`miv-change` AS `MIV-Veränderung`, 
`potentials`.`id` AS `Potenzial-ID`
FROM `potentials`
LEFT JOIN `places` `from_places` ON `from_places`.`id` = `potentials`.`from_id`
LEFT JOIN `places` `to_places` ON `to_places`.`id` = `potentials`.`to_id`
WHERE `from_places`.`name` = "Wiesentheid OT Untersambach";
\end{minted}
\caption{SQL-Abfrage der Netto-Potenziale und MIV-Veränderung mit der Quelle Untersambach}\label{lst-fz-untersambach}
\end{listing}

            \subsection{Rüdenhausen}
            \begin{tabular}{ l  l  l  l  l }
Quelle & Ziel & NettoPotenzial & MIV-Veränderung & Potenzial-ID\\ 
Rüdenhausen & Schweinfurt & 3 & -4 & 227\\ 
Rüdenhausen & Kitzingen & 21 & -57 & 228\\ 
Rüdenhausen & Würzburg, Rottendorf & 12 & -19 & 229\\ 
Rüdenhausen & Wiesentheid OT Feuerbach & * & 83 & 230\\ 
Rüdenhausen & Wiesentheid & * & 8 & 231\\ 
\end{tabular}
\newline
\newline
* Neue Verkehre um den Bahnhof zu erreichen.
\newline
\begin{listing}[htbp]
\begin{minted}{sql}
SELECT
`from_places`.`name` AS `Quelle`, 
`to_places`.`name` AS `Ziel`, 
`potentials`.`netto` AS `NettoPotenzial`, 
`potentials`.`miv-change` AS `MIV-Veränderung`, 
`potentials`.`id` AS `Potenzial-ID`
FROM `potentials`
LEFT JOIN `places` `from_places` ON `from_places`.`id` = `potentials`.`from_id`
LEFT JOIN `places` `to_places` ON `to_places`.`id` = `potentials`.`to_id`
WHERE `from_places`.`name` = "Rüdenhausen";
\end{minted}
\caption{SQL-Abfrage der Netto-Potenziale und MIV-Veränderung mit der Quelle Rüdenhausen}\label{lst-fz-ruedenhausen}
\end{listing}

            \subsection{Abtswind}
            \begin{tabular}{ l  l  l  l  l }
Quelle & Ziel & NettoPotenzial & MIV-Veränderung & Potenzial-ID\\ 
Abtswind & Kitzingen & 21 & -57 & 232\\ 
Abtswind & Würzburg, Rottendorf & 11 & -17 & 233\\ 
Abtswind & Wiesentheid OT Feuerbach & * & 80 & 234\\ 
\end{tabular}    
\newline
\newline
* Neue Verkehre um den Bahnhof zu erreichen.
\newline
\begin{listing}[htbp]
\begin{minted}{sql}
SELECT
`from_places`.`name` AS `Quelle`, 
`to_places`.`name` AS `Ziel`, 
`potentials`.`netto` AS `NettoPotenzial`, 
`potentials`.`miv-change` AS `MIV-Veränderung`, 
`potentials`.`id` AS `Potenzial-ID`
FROM `potentials`
LEFT JOIN `places` `from_places` ON `from_places`.`id` = `potentials`.`from_id`
LEFT JOIN `places` `to_places` ON `to_places`.`id` = `potentials`.`to_id`
WHERE `from_places`.`name` = "Abtswind";
\end{minted}
\caption{SQL-Abfrage der Netto-Potenziale und MIV-Veränderung mit der Quelle Abstwind}\label{lst-fz-abstwind}
\end{listing}

            \subsection{Kleinlangheim}
            \begin{tabular}{ l  l  l  l  l }
Quelle & Ziel & NettoPotenzial & MIV-Veränderung & Potenzial-ID\\ 
Kleinlangheim & Schweinfurt & 4 & -6 & 235\\ 
Kleinlangheim & Gerolzhofen & 5 & -8 & 236\\ 
Kleinlangheim & Wiesentheid & 25 & -40 & 237\\ 
Kleinlangheim & Kitzingen & 321 & -513 & 238\\ 
Kleinlangheim & Würzburg, Rottendorf & 5 & -8 & 239\\ 
\end{tabular}    
\newline
\newline
\begin{listing}[htbp]
\begin{minted}{sql}
SELECT
`from_places`.`name` AS `Quelle`, 
`to_places`.`name` AS `Ziel`, 
`potentials`.`netto` AS `NettoPotenzial`, 
`potentials`.`miv-change` AS `MIV-Veränderung`, 
`potentials`.`id` AS `Potenzial-ID`
FROM `potentials`
LEFT JOIN `places` `from_places` ON `from_places`.`id` = `potentials`.`from_id`
LEFT JOIN `places` `to_places` ON `to_places`.`id` = `potentials`.`to_id`
WHERE `from_places`.`name` = "Kleinlangheim";
\end{minted}
\caption{SQL-Abfrage der Netto-Potenziale und MIV-Veränderung mit der Quelle Kleinlangheim}\label{lst-fz-kleinlangheim}
\end{listing}

            \subsection{Wiesenbronn}
            \begin{tabular}{ l  l  l  l  l }
Quelle & Ziel & NettoPotenzial & MIV-Veränderung & Potenzial-ID\\ 
Wiesenbronn & Schweinfurt & 3 & -4 & 240\\ 
Wiesenbronn & Gerolzhofen & 2 & -3 & 241\\ 
Wiesenbronn & Würzburg, Rottendorf & 8 & -12 & 242\\ 
Wiesenbronn & Kleinlangheim & * & 13 & 243\\ 
Wiesenbronn & Großlangheim & * & 20 & 244\\ 
\end{tabular}
\newline
\newline
* Neue Verkehre um den Bahnhof zu erreichen.
\newline
\begin{listing}[htbp]
\begin{minted}{sql}
SELECT
`from_places`.`name` AS `Quelle`, 
`to_places`.`name` AS `Ziel`, 
`potentials`.`netto` AS `NettoPotenzial`, 
`potentials`.`miv-change` AS `MIV-Veränderung`, 
`potentials`.`id` AS `Potenzial-ID`
FROM `potentials`
LEFT JOIN `places` `from_places` ON `from_places`.`id` = `potentials`.`from_id`
LEFT JOIN `places` `to_places` ON `to_places`.`id` = `potentials`.`to_id`
WHERE `from_places`.`name` = "Wiesenbronn";
\end{minted}
\caption{SQL-Abfrage der Netto-Potenziale und MIV-Veränderung mit der Quelle Wiesenbronn}\label{lst-fz-wiesenbronn}
\end{listing}

            \subsection{Großlangheim}
            \begin{tabular}{ l  l  l  l  l }
Quelle & Ziel & NettoPotenzial & MIV-Veränderung & Potenzial-ID\\ 
Großlangheim & Schweinfurt & 5 & -8 & 245\\ 
Großlangheim & Wiesentheid & 7 & -11 & 246\\ 
Großlangheim & Kitzingen & 313 & -500 & 247\\ 
Großlangheim & Würzburg, Rottendorf & 25 & -40 & 248\\ 
\end{tabular}    
\newline
\newline
\begin{listing}[htbp]
\begin{minted}{sql}
SELECT
`from_places`.`name` AS `Quelle`, 
`to_places`.`name` AS `Ziel`, 
`potentials`.`netto` AS `NettoPotenzial`, 
`potentials`.`miv-change` AS `MIV-Veränderung`, 
`potentials`.`id` AS `Potenzial-ID`
FROM `potentials`
LEFT JOIN `places` `from_places` ON `from_places`.`id` = `potentials`.`from_id`
LEFT JOIN `places` `to_places` ON `to_places`.`id` = `potentials`.`to_id`
WHERE `from_places`.`name` = "Großlangheim";
\end{minted}
\caption{SQL-Abfrage der Netto-Potenziale und MIV-Veränderung mit der Quelle Großlangheim}\label{lst-fz-grosslangheim}
\end{listing}

            \subsection{Kitzingen}
            \begin{tabular}{ l  l  l  l  l }
Quelle & Ziel & NettoPotenzial & MIV-Veränderung & Potenzial-ID\\ 
Kitzingen & Schweinfurt & 64 & -102 & 249\\ 
Kitzingen & Gochsheim & 9 & -14 & 250\\ 
Kitzingen & Gerolzhofen & 11 & -17 & 251\\ 
Kitzingen & Prichsenstadt & 18 & -28 & 252\\ 
Kitzingen & Wiesentheid & 49 & -78 & 253\\ 
Kitzingen & Kleinlangheim & 10 & -16 & 254\\ 
Kitzingen & Großlangheim & 37 & -59 & 255\\ 
\end{tabular} 
\newline
\newline
\begin{listing}[htbp]
\begin{minted}{sql}
SELECT
`from_places`.`name` AS `Quelle`, 
`to_places`.`name` AS `Ziel`, 
`potentials`.`netto` AS `NettoPotenzial`, 
`potentials`.`miv-change` AS `MIV-Veränderung`, 
`potentials`.`id` AS `Potenzial-ID`
FROM `potentials`
LEFT JOIN `places` `from_places` ON `from_places`.`id` = `potentials`.`from_id`
LEFT JOIN `places` `to_places` ON `to_places`.`id` = `potentials`.`to_id`
WHERE `from_places`.`name` = "Kitzingen";
\end{minted}
\caption{SQL-Abfrage der Netto-Potenziale und MIV-Veränderung mit der Quelle Kitzingen}\label{lst-fz-kitzingen}
\end{listing}

            \subsection{Würzburg}
            \begin{tabular}{ l  l  l  l  l }
Quelle & Ziel & NettoPotenzial & MIV-Veränderung & Potenzial-ID\\ 
Würzburg & Sennfeld & 24 & -38 & 256\\ 
Würzburg & Gochsheim & 60 & -96 & 257\\ 
Würzburg & Gerolzhofen & 37 & -59 & 258\\ 
Würzburg & Prichsenstadt & 24 & -38 & 259\\ 
Würzburg & Wiesentheid & 46 & -73 & 260\\ 
\end{tabular}
\newline
\newline
\begin{listing}[htbp]
\begin{minted}{sql}
SELECT
`from_places`.`name` AS `Quelle`, 
`to_places`.`name` AS `Ziel`, 
`potentials`.`netto` AS `NettoPotenzial`, 
`potentials`.`miv-change` AS `MIV-Veränderung`, 
`potentials`.`id` AS `Potenzial-ID`
FROM `potentials`
LEFT JOIN `places` `from_places` ON `from_places`.`id` = `potentials`.`from_id`
LEFT JOIN `places` `to_places` ON `to_places`.`id` = `potentials`.`to_id`
WHERE `from_places`.`name` = "Würzburg";
\end{minted}
\caption{SQL-Abfrage der Netto-Potenziale und MIV-Veränderung mit der Quelle Würzburg}\label{lst-fz-wuerzburg}
\end{listing}

        \section{Übertragung der Potentiale auf Straßen}
                In diesem Abschnitt werden zur Überprüfbarkeit und Nachvollziehbarkeit die aus Google Maps entnommenen Routenentscheidungen für jedes Potenzial aufgelistet. Dazu wird ein Link zu Google Maps angegeben, mit dem diese Routenentscheidung seitens des Kartendienstleisters überprüft werden kann. Ebenfalls wird ein SQL-Query angegeben, mit dem diese Routenentscheidung in der Datenbank nachvollzogen werden kann.
                \subsection{Schweinfurt}
                Zugeordnete Routen:

\begin{longtabu}{|l|l|l|l|*2{X[l]|}}
    \hline
    id & Quelle & Ziel & Straße & Straßenbeginn & Straßenende\\ 
    \hline
    1 & Schweinfurt & Sennfeld &  &  & \\ 
    \hline
    2 & Schweinfurt & Gochsheim &  &  & \\ 
    \hline
    3 & Schweinfurt & Grettstatt & B286 & B286/B26 (Schweinfurt Nähe Mainbrücke) & B286/B303\\ 
    3 &  &  & B286 & B286 Schweinfurt Abfahrt Hans-Böckler-Straße & B286/B26 (Schweinfurt Nähe Mainbrücke)\\ 
    3 &  &  & B286 & Ausfahrt Schweinfurt-Zentrum & B286 Schweinfurt Abfahrt Hans-Böckler-Straße\\ 
    3 &  &  & B286 & B286/SW3/St2271 (nördlich Schwebheim) & Ausfahrt Schweinfurt-Zentrum\\ 
    3 &  &  & SW3 & St2272/SW3 Gochsheim Kreisel Industriestraße & B286/SW3/St2271 (nördlich Schwebheim)\\ 
    3 &  &  & St2272 & St2272/SW3 Gochsheim Kreisel Industriestraße & St2272/St2277 (Kreisel südlich Gochsheim)\\ 
    3 &  &  & St2272 & St2272/St2277 (Kreisel südlich Gochsheim) & Grettstatt\\ 
    \hline
    4 & Schweinfurt & Gerolzhofen & B286 & B286/B26 (Schweinfurt Nähe Mainbrücke) & B286/B303\\ 
    4 &  &  & B286 & B286 Schweinfurt Abfahrt Hans-Böckler-Straße & B286/B26 (Schweinfurt Nähe Mainbrücke)\\ 
    4 &  &  & B286 & Ausfahrt Schweinfurt-Zentrum & B286 Schweinfurt Abfahrt Hans-Böckler-Straße\\ 
    4 &  &  & B286 & B286/SW3/St2271 (nördlich Schwebheim) & Ausfahrt Schweinfurt-Zentrum\\ 
    4 &  &  & B286 & B286/St2277 (bei Schwebheim) & B286/SW3/St2271 (nördlich Schwebheim)\\ 
    4 &  &  & B286 & B286/St2271 (bei Unterspießheim) & B286/St2277 (bei Schwebheim)\\ 
    4 &  &  & B286 & B286/St2272 (bei Alitzheim) & B286/St2271 (bei Unterspießheim)\\ 
    4 &  &  & B286 & B286/St2275 (Gerolzhofen bei Rügshofen) & B286/St2272 (bei Alitzheim)\\ 
    4 &  &  & St2275 & Gerolzhofen & B286/St2275 (Gerolzhofen bei Rügshofen)\\ 
    \hline
    5 & Schweinfurt & Wiesentheid & B286 & B286/B26 (Schweinfurt Nähe Mainbrücke) & B286/B303\\ 
    5 &  &  & B286 & B286 Schweinfurt Abfahrt Hans-Böckler-Straße & B286/B26 (Schweinfurt Nähe Mainbrücke)\\ 
    5 &  &  & B286 & Ausfahrt Schweinfurt-Zentrum & B286 Schweinfurt Abfahrt Hans-Böckler-Straße\\ 
    5 &  &  & B286 & B286/SW3/St2271 (nördlich Schwebheim) & Ausfahrt Schweinfurt-Zentrum\\ 
    5 &  &  & B286 & B286/St2277 (bei Schwebheim) & B286/SW3/St2271 (nördlich Schwebheim)\\ 
    5 &  &  & B286 & B286/St2271 (bei Unterspießheim) & B286/St2277 (bei Schwebheim)\\ 
    5 &  &  & B286 & B286/St2272 (bei Alitzheim) & B286/St2271 (bei Unterspießheim)\\ 
    5 &  &  & B286 & B286/St2275 (Gerolzhofen bei Rügshofen) & B286/St2272 (bei Alitzheim)\\ 
    5 &  &  & B286 & B286/St2274 (Gerolzhofen bei Geomaris) & B286/St2275 (Gerolzhofen bei Rügshofen)\\ 
    5 &  &  & B286 & Prichsenstadt OT Neuses & B286/St2274 (Gerolzhofen bei Geomaris)\\ 
    5 &  &  & B286 & B286/St2272 (Wiesentheid bei Blutbank) & Prichsenstadt OT Neuses\\ 
    5 &  &  & St2272 & B286/St2272 (Wiesentheid bei Blutbank) & Wiesentheid\\ 
    \hline
    6 & Schweinfurt & Kitzingen & B286 & B286/B26 (Schweinfurt Nähe Mainbrücke) & B286/B303\\ 
    6 &  &  & B286 & B286 Schweinfurt Abfahrt Hans-Böckler-Straße & B286/B26 (Schweinfurt Nähe Mainbrücke)\\ 
    6 &  &  & B286 & Ausfahrt Schweinfurt-Zentrum & B286 Schweinfurt Abfahrt Hans-Böckler-Straße\\ 
    6 &  &  & A70 & Ausfahrt Schweinfurt-Hafen & Ausfahrt Schweinfurt-Zentrum\\ 
    6 &  &  & A70 & Ausfahrt Schweinfurt-Bergrheinfeld & Ausfahrt Schweinfurt-Hafen\\ 
    6 &  &  & A70 & Autobahnkreuz Werntal & Ausfahrt Schweinfurt-Bergrheinfeld\\ 
    6 &  &  & A70 & Ausfahrt Werneck & Autobahnkreuz Werntal\\ 
    6 &  &  & A70 & Autobahnkreuz Scheinfurt-Werneck & Ausfahrt Werneck\\ 
    6 &  &  & A7 & Autobahnkreuz Scheinfurt-Werneck & Ausfahrt Gramschatzer Wald\\ 
    6 &  &  & A7 & Ausfahrt Gramschatzer Wald & Ausfahrt Würzburg-Estenfeld\\ 
    6 &  &  & A7 & Ausfahrt Würzburg-Estenfeld & Autobahnkreuz Biebelried\\ 
    6 &  &  & A7 & Autobahnkreuz Biebelried & Ausfahrt Kitzingen\\ 
    6 &  &  & B8 & Ausfahrt Kitzingen & B8/KT27 (Abzweig bei GWF)\\ 
    6 &  &  & B8 & B8/KT27 (Abzweig bei GWF) & Kitzingen\\         
    \hline
\end{longtabu}

Anmerkung: Für die Abschnitte Schweinfurt-Sennfeld und Schweinfurt-Gochsheim wurde in der Schliephake-Studie angenommen, dass durch die Ortsbusse und die relative Nähe kein Bedarf und somit kein Potenzial besteht. Daher wurden hier keine Routen/Straßen zugeordnet.
\newline
\begin{listing}[htbp]
\begin{minted}{sql}
    SELECT 
	`potentials`.`id` AS `id`,
	`from_places`.`name` AS `Quelle`, 
	`to_places`.`name` AS `Ziel`,
	`streets`.`street` AS `Straße`,
	`from_street_places`.`name` AS `Straßenbeginn`,
   `to_street_places`.`name` AS `Straßenende`
FROM `potentials`
LEFT JOIN `places` AS `from_places` ON `potentials`.`from_id` = `from_places`.`id`
LEFT JOIN `places` AS `to_places` ON `potentials`.`to_id` = `to_places`.`id`
LEFT JOIN `routes` ON `routes`.`potential_id` = `potentials`.`id`
LEFT JOIN `streets` ON `streets`.`id` = `routes`.`street_id`
LEFT JOIN `places` AS `from_street_places` ON `streets`.`from_id` = `from_street_places`.`id`
LEFT JOIN `places` AS `to_street_places` ON `streets`.`to_id` = `to_street_places`.`id`
WHERE `from_places`.`name` = 'Schweinfurt'
ORDER BY `potentials`.`id`, `routes`.`number_on_route`;
\end{minted}
\caption{SQL-Abfrage der zugeordneten Straßen mit der Quelle Schweinfurt}\label{lst-rt-schweinfurt}
\end{listing}


Länge, Fahrzeiten und Google Maps:

\begin{longtabu}{| l | l | l |*3{X[l]|}}
    \hline
    id & Quelle & Ziel & Fahrtstrecke [m] & Fahrtdauer [min] & Google-Maps Link\\ 
    \hline
    1 & Schweinfurt & Sennfeld & 3400 & 6 & \href{https://www.google.com/maps/dir/50.0439484,10.2257843/50.0422146,10.2609081}{https://www.google.com/maps/dir/50.0439484,10.2257843/50.0422146,10.2609081}\\ 
    \hline
    2 & Schweinfurt & Gochsheim & 5700 & 9 & \href{https://www.google.com/maps/dir/50.0439484,10.2257843/50.019526,10.2822383}{https://www.google.com/maps/dir/50.0439484,10.2257843/50.019526,10.2822383}\\ 
    \hline
    3 & Schweinfurt & Grettstatt & 13200 & 15 & \href{https://www.google.com/maps/dir/50.0439484,10.2257843/49.9847108,10.3121683}{https://www.google.com/maps/dir/50.0439484,10.2257843/49.9847108,10.3121683}\\ 
    \hline
    4 & Schweinfurt & Gerolzhofen & 21800 & 22 & \href{https://www.google.com/maps/dir/50.0439484,10.2257843/49.9010511,10.3489622}{https://www.google.com/maps/dir/50.0439484,10.2257843/49.9010511,10.3489622}\\ 
    \hline
    5 & Schweinfurt & Wiesentheid & 34200 & 29 & \href{https://www.google.com/maps/dir/50.0439484,10.2257843/49.7942401,10.3426344}{https://www.google.com/maps/dir/50.0439484,10.2257843/49.7942401,10.3426344}\\ 
    \hline
    6 & Schweinfurt & Kitzingen & 55100 & 40 & \href{https://www.google.com/maps/dir/50.0439484,10.2257843/49.7355709,10.1617438}{https://www.google.com/maps/dir/50.0439484,10.2257843/49.7355709,10.1617438}\\ 
    \hline
\end{longtabu}

\begin{listing}[htbp]
    \begin{minted}{sql}
        SELECT 
        `potentials`.`id` AS `id`, 
        `from_places`.`name` AS `Quelle`,
        `to_places`.`name` AS `Ziel`, 
        `potentials`.`length` AS `Fahrtstrecke [m]`, 
        `potentials`.`miv-duration` AS `Fahrtdauer [min]`,
        CONCAT('https://www.google.com/maps/dir/', `from_places`.`LAT`, ",", `from_places`.`LONG`, '/', `to_places`.`LAT`, ',', `to_places`.`LONG`) AS `Google-Maps Link`
    FROM `potentials`
    LEFT JOIN `places` AS `from_places` ON `potentials`.`from_id` = `from_places`.`id`
    LEFT JOIN `places` AS `to_places` ON `potentials`.`to_id` = `to_places`.`id`
    WHERE `from_places`.`name` = 'Schweinfurt'
    ORDER BY `potentials`.`id`;
    \end{minted}
    \caption{SQL-Abfrage der Fahrtstrecke, Fahrtdauer und des Google-Maps-Link mit der Quelle Schweinfurt}\label{lst-f-schweinfurt}
\end{listing}

                \subsection{Sennfeld}
                Zougeordnete Routen:
\newline
\newline
\begin{tabular}{|l|l|l|l|l|l|l|}
    \hline
    id & Quelle & Ziel & Straße & Straßenbeginn & Straßenende\\ 
    7 & Sennfeld & Würzburg & St2272 & St2271/St2272 (bei Sennfeld) & Ausfahrt Gochsheim\\ 
    7 &  &  & A70 & Ausfahrt Schweinfurt-Zentrum & Ausfahrt Gochsheim\\ 
    7 &  &  & A70 & Ausfahrt Schweinfurt-Hafen & Ausfahrt Schweinfurt-Zentrum\\ 
    7 &  &  & A70 & Ausfahrt Schweinfurt-Bergrheinfeld & Ausfahrt Schweinfurt-Hafen\\ 
    7 &  &  & A70 & Autobahnkreuz Werntal & Ausfahrt Schweinfurt-Bergrheinfeld\\ 
    7 &  &  & A70 & Ausfahrt Werneck & Autobahnkreuz Werntal\\ 
    7 &  &  & A70 & Autobahnkreuz Scheinfurt-Werneck & Ausfahrt Werneck\\ 
    7 &  &  & A7 & Autobahnkreuz Scheinfurt-Werneck & Ausfahrt Gramschatzer Wald\\ 
    7 &  &  & A7 & Ausfahrt Gramschatzer Wald & Ausfahrt Würzburg-Estenfeld\\ 
    7 &  &  & B19 & B19 Ausfahrt Estenfeld Ost & Ausfahrt Würzburg-Estenfeld\\ 
    7 &  &  & B19 & B19/WÜ8 & B19 Ausfahrt Estenfeld Ost\\ 
    7 &  &  & B19 & Würzburg B19 Ikea & B19/WÜ8\\ 
    7 &  &  & B19 & Würzburg B19 Lengfeld & Würzburg B19 Ikea\\ 
    \hline
    8 & Sennfeld & Schweinfurt & St2272 & St2271/St2272 (bei Sennfeld) & Ausfahrt Gochsheim\\ 
    8 &  &  & St2272 & B26/St2272 & St2271/St2272 (bei Sennfeld)\\ 
    8 &  &  & B26 & B286/B26 (Schweinfurt Nördlicher) & B26/St2272\\ 
    \hline
    10 & Sennfeld & Gerolzhofen & St2272 & St2271/St2272 (bei Sennfeld) & Ausfahrt Gochsheim\\ 
    10 &  &  & A70 & Ausfahrt Schweinfurt-Zentrum & Ausfahrt Gochsheim\\ 
    10 &  &  & B286 & B286/St2277 (bei Schwebheim) & B286/SW3/St2271 (nördlich Schwebheim)\\ 
    10 &  &  & B286 & B286/St2271 (bei Unterspießheim) & B286/St2277 (bei Schwebheim)\\ 
    10 &  &  & B286 & B286/St2272 (bei Alitzheim) & B286/St2271 (bei Unterspießheim)\\ 
    10 &  &  & B286 & B286/St2275 (Gerolzhofen bei Rügshofen) & B286/St2272 (bei Alitzheim)\\ 
    10 &  &  & St2275 & Gerolzhofen & B286/St2275 (Gerolzhofen bei Rügshofen)\\ 
    \hline
    11 & Sennfeld & Kitzingen & St2272 & St2271/St2272 (bei Sennfeld) & Ausfahrt Gochsheim\\ 
    11 &  &  & A70 & Ausfahrt Schweinfurt-Zentrum & Ausfahrt Gochsheim\\ 
    11 &  &  & A70 & Ausfahrt Schweinfurt-Hafen & Ausfahrt Schweinfurt-Zentrum\\ 
    11 &  &  & A70 & Ausfahrt Schweinfurt-Bergrheinfeld & Ausfahrt Schweinfurt-Hafen\\ 
    11 &  &  & A70 & Autobahnkreuz Werntal & Ausfahrt Schweinfurt-Bergrheinfeld\\ 
    11 &  &  & A70 & Ausfahrt Werneck & Autobahnkreuz Werntal\\ 
    11 &  &  & A70 & Autobahnkreuz Scheinfurt-Werneck & Ausfahrt Werneck\\ 
    11 &  &  & A7 & Autobahnkreuz Scheinfurt-Werneck & Ausfahrt Gramschatzer Wald\\ 
    11 &  &  & A7 & Ausfahrt Gramschatzer Wald & Ausfahrt Würzburg-Estenfeld\\ 
    11 &  &  & A7 & Ausfahrt Würzburg-Estenfeld & Autobahnkreuz Biebelried\\ 
    11 &  &  & A7 & Autobahnkreuz Biebelried & Ausfahrt Kitzingen\\ 
    11 &  &  & B8 & Ausfahrt Kitzingen & B8/KT27 (Abzweig bei GWF)\\ 
    11 &  &  & B8 & B8/KT27 (Abzweig bei GWF) & Kitzingen\\ 
    \hline
\end{tabular}
\newline
\newline
Anmerkung: Für die Abschnitte Schweinfurt-Sennfeld und Schweinfurt-Gochsheim wurde in der Schliephake-Studie angenommen, dass durch die Ortsbusse und die relative Nähe kein Bedarf und somit kein Potenzial besteht. Daher wurden hier keine Routen/Straßen zugeordnet.
\newline
\begin{listing}[htbp]
\begin{minted}{sql}
    SELECT 
	`potentials`.`id` AS `id`,
	`from_places`.`name` AS `Quelle`, 
	`to_places`.`name` AS `Ziel`,
	`streets`.`street` AS `Straße`,
	`from_street_places`.`name` AS `Straßenbeginn`,
   `to_street_places`.`name` AS `Straßenende`
FROM `potentials`
LEFT JOIN `places` AS `from_places` ON `potentials`.`from_id` = `from_places`.`id`
LEFT JOIN `places` AS `to_places` ON `potentials`.`to_id` = `to_places`.`id`
LEFT JOIN `routes` ON `routes`.`potential_id` = `potentials`.`id`
LEFT JOIN `streets` ON `streets`.`id` = `routes`.`street_id`
LEFT JOIN `places` AS `from_street_places` ON `streets`.`from_id` = `from_street_places`.`id`
LEFT JOIN `places` AS `to_street_places` ON `streets`.`to_id` = `to_street_places`.`id`
WHERE `from_places`.`name` = 'Sennfeld'
ORDER BY `potentials`.`id`, `routes`.`number_on_route`;
\end{minted}
\caption{SQL-Abfrage der zugeordneten Straßen mit der Quelle Sennfeld}\label{lst-rt-sennfeld}
\end{listing}


Länge, Fahrzeiten und Google Maps:
\newline
\begin{tabular}{| l | l | l | l | l | l |}
    \hline
    7 & Sennfeld & Würzburg & 47200 & 36 & \href{https://www.google.com/maps/dir/50.0422146,10.2609081/49.7931,9.9280108}{https://www.google.com/maps/dir/50.0422146,10.2609081/49.7931,9.9280108}\\ 
    \hline
    8 & Sennfeld & Schweinfurt & 3900 & 7 & \href{https://www.google.com/maps/dir/50.0422146,10.2609081/50.0439484,10.2257843}{https://www.google.com/maps/dir/50.0422146,10.2609081/50.0439484,10.2257843}\\ 
    \hline
    10 & Sennfeld & Gerolzhofen & 22300 & 20 & \href{https://www.google.com/maps/dir/50.0422146,10.2609081/49.9010511,10.3489622}{https://www.google.com/maps/dir/50.0422146,10.2609081/49.9010511,10.3489622}\\ 
    \hline
    11 & Sennfeld & Kitzingen & 55500 & 38 & \href{https://www.google.com/maps/dir/50.0422146,10.2609081/49.7355709,10.1617438}{https://www.google.com/maps/dir/50.0422146,10.2609081/49.7355709,10.1617438}\\
    \hline
\end{tabular}    
\newline
\newline
\begin{listing}[htbp]
    \begin{minted}{sql}
        SELECT 
        `potentials`.`id` AS `id`, 
        `from_places`.`name` AS `Quelle`,
        `to_places`.`name` AS `Ziel`, 
        `potentials`.`length` AS `Fahrtstrecke [m]`, 
        `potentials`.`miv-duration` AS `Fahrtdauer [min]`,
        CONCAT('https://www.google.com/maps/dir/', `from_places`.`LAT`, ",", `from_places`.`LONG`, '/', `to_places`.`LAT`, ',', `to_places`.`LONG`) AS `Google-Maps Link`
    FROM `potentials`
    LEFT JOIN `places` AS `from_places` ON `potentials`.`from_id` = `from_places`.`id`
    LEFT JOIN `places` AS `to_places` ON `potentials`.`to_id` = `to_places`.`id`
    WHERE `from_places`.`name` = 'Sennfeld'
    ORDER BY `potentials`.`id`;
    \end{minted}
    \caption{SQL-Abfrage der Fahrtstrecke, Fahrtdauer und des Google-Maps-Link mit der Quelle Sennfeld}\label{lst-f-sennfeld}
\end{listing}
                
                \subsection{Gochsheim}
                Zugeordnete Routen:
\newline
\newline
\begin{tabular}{|l|l|l|l|l|l|l|}
    \hline
    id & Quelle & Ziel & Straße & Straßenbeginn & Straßenende\\ 
    \hline
    12 & Gochsheim & Würzburg, Rottendorf & St2272 & Ausfahrt Gochsheim & St2272/SW3 Gochsheim\\ 
    12 &  &  & A70 & Ausfahrt Schweinfurt-Zentrum & Ausfahrt Gochsheim\\ 
    12 &  &  & A70 & Ausfahrt Schweinfurt-Hafen & Ausfahrt Schweinfurt-Zentrum\\ 
    12 &  &  & A70 & Ausfahrt Schweinfurt-Bergrheinfeld & Ausfahrt Schweinfurt-Hafen\\ 
    12 &  &  & A70 & Autobahnkreuz Werntal & Ausfahrt Schweinfurt-Bergrheinfeld\\ 
    12 &  &  & A70 & Ausfahrt Werneck & Autobahnkreuz Werntal\\ 
    12 &  &  & A70 & Autobahnkreuz Scheinfurt-Werneck & Ausfahrt Werneck\\ 
    12 &  &  & A7 & Autobahnkreuz Scheinfurt-Werneck & Ausfahrt Gramschatzer Wald\\ 
    12 &  &  & A7 & Ausfahrt Gramschatzer Wald & Ausfahrt Würzburg-Estenfeld\\ 
    12 &  &  & B19 & B19 Ausfahrt Estenfeld Ost & Ausfahrt Würzburg-Estenfeld\\ 
    12 &  &  & B19 & B19/WÜ8 & B19 Ausfahrt Estenfeld Ost\\ 
    12 &  &  & B19 & Würzburg B8/B19 (Grainbergknoten) & Würzburg B19 Lengfeld\\ 
    12 &  &  & B19 & Würzburg B19 Ikea & B19/WÜ8\\ 
    12 &  &  & B19 & Würzburg B19 Lengfeld & Würzburg B19 Ikea\\ 
    \hline
    13 & Gochsheim & Bamberg, Haßfurt & St2272 & Ausfahrt Gochsheim & St2272/SW3 Gochsheim\\ 
    13 &  &  & A70 & Ausfahrt Gochsheim & Ausfahrt Schonungen\\ 
    13 &  &  & A70 & Ausfahrt Schonungen & Ausfahrt Haßfurt\\ 
    13 &  &  & A70 & Ausfahrt Haßfurt & Ausfahrt Knetzgau\\ 
    13 &  &  & A70 & Ausfahrt Knetzgau & Ausfahrt Eltmann\\ 
    13 &  &  & A70 & Ausfahrt Eltmann & Ausfahrt Viereth-Thunstadt\\ 
    13 &  &  & A70 & Ausfahrt Viereth-Thunstadt & Ausfahrt Bamberg-Hafen\\ 
    13 &  &  & A70 & Ausfahrt Bamberg-Hafen & Ausfahrt Hallstadt\\ 
    13 &  &  & A70 & Ausfahrt Hallstadt & Ausfahrt Bamberg\\ 
    \hline
    14 & Gochsheim & Bad Kissingen & St2272 & Ausfahrt Gochsheim & St2272/SW3 Gochsheim\\ 
    14 &  &  & St2272 & St2271/St2272 (bei Sennfeld) & Ausfahrt Gochsheim\\ 
    14 &  &  & St2272 & B26/St2272 & St2271/St2272 (bei Sennfeld)\\ 
    14 &  &  & B26 & B286/B26 (Schweinfurt Nördlicher) & B26/St2272\\ 
    14 &  &  & B286 & B286/B26 (Schweinfurt Nördlicher) & B286/St2280 (in Schweinfurt)\\ 
    14 &  &  & B286 & B286/St2280 (in Schweinfurt) & Maibach\\ 
    14 &  &  & B286 & Maibach & Ausfahrt Poppenhausen\\ 
    14 &  &  & B286 & Ausfahrt Poppenhausen & B286/B19 (bei Poppenhausen)\\ 
    14 &  &  & B286 & B286/B19 (bei Oerlenbach) & Oerlenbach\\ 
    14 &  &  & B286 & Oerlenbach & B286/KG46\\ 
    14 &  &  & B286 & B286/KG46 & Arnshausen\\ 
    14 &  &  & B286 & Arnshausen & Bad Kissingen\\ 
    \hline
    15 & Gochsheim & Schweinfurt & St2272 & Ausfahrt Gochsheim & St2272/SW3 Gochsheim\\ 
    15 &  &  & St2272 & St2271/St2272 (bei Sennfeld) & Ausfahrt Gochsheim\\ 
    15 &  &  & St2272 & B26/St2272 & St2271/St2272 (bei Sennfeld)\\ 
    15 &  &  & B26 & B286/B26 (Schweinfurt Nördlicher) & B26/St2272\\ 
    \hline
    16 & Gochsheim & Gerolzhofen & St2272 & St2272/St2277 (Kreisel südlich Gochsheim) & Grettstatt\\ 
    16 &  &  & St2272 & Grettstatt & Sulzheim\\ 
    16 &  &  & St2272 & Sulzheim & Alitzheim\\ 
    16 &  &  & St2272 & Alitzheim & B286/St2272 (bei Alitzheim)\\ 
    16 &  &  & B286 & B286/St2275 (Gerolzhofen bei Rügshofen) & B286/St2272 (bei Alitzheim)\\ 
    16 &  &  & St2275 & Gerolzhofen & B286/St2275 (Gerolzhofen bei Rügshofen)\\ 
    \hline
\end{tabular}
\newline
\newline
Anmerkung: Für die Abschnitte Schweinfurt-Sennfeld und Schweinfurt-Gochsheim wurde in der Schliephake-Studie angenommen, dass durch die Ortsbusse und die relative Nähe kein Bedarf und somit kein Potenzial besteht. Daher wurden hier keine Routen/Straßen zugeordnet.
\newline
\begin{listing}[htbp]
\begin{minted}{sql}
    SELECT 
	`potentials`.`id` AS `id`,
	`from_places`.`name` AS `Quelle`, 
	`to_places`.`name` AS `Ziel`,
	`streets`.`street` AS `Straße`,
	`from_street_places`.`name` AS `Straßenbeginn`,
   `to_street_places`.`name` AS `Straßenende`
FROM `potentials`
LEFT JOIN `places` AS `from_places` ON `potentials`.`from_id` = `from_places`.`id`
LEFT JOIN `places` AS `to_places` ON `potentials`.`to_id` = `to_places`.`id`
LEFT JOIN `routes` ON `routes`.`potential_id` = `potentials`.`id`
LEFT JOIN `streets` ON `streets`.`id` = `routes`.`street_id`
LEFT JOIN `places` AS `from_street_places` ON `streets`.`from_id` = `from_street_places`.`id`
LEFT JOIN `places` AS `to_street_places` ON `streets`.`to_id` = `to_street_places`.`id`
WHERE `from_places`.`name` = 'Gochsheim'
ORDER BY `potentials`.`id`, `routes`.`number_on_route`;
\end{minted}
\caption{SQL-Abfrage der zugeordneten Straßen mit der Quelle Gochsheim}\label{lst-rt-gochsheim}
\end{listing}


Länge, Fahrzeiten und Google Maps:
\newline
\begin{tabular}{| l | l | l | l | l | l |}
    \hline
    id & Quelle & Ziel & Fahrtstrecke [m] & Fahrtdauer [min] & Google-Maps Link\\ 
    \hline
    12 & Gochsheim & Würzburg, Rottendorf & 44300 & 31 & \href{https://www.google.com/maps/dir/50.019526,10.2822383/49.7931,9.9280108}{https://www.google.com/maps/dir/50.019526,10.2822383/49.7931,9.9280108}\\ 
    \hline
    13 & Gochsheim & Bamberg, Haßfurt & 52800 & 37 & \href{https://www.google.com/maps/dir/50.019526,10.2822383/49.8912678,10.8865984}{https://www.google.com/maps/dir/50.019526,10.2822383/49.8912678,10.8865984}\\ 
    \hline
    14 & Gochsheim & Bad Kissingen & 38200 & 35 & \href{https://www.google.com/maps/dir/50.019526,10.2822383/50.1990369,10.0762182}{https://www.google.com/maps/dir/50.019526,10.2822383/50.1990369,10.0762182}\\ 
    \hline
    15 & Gochsheim & Schweinfurt & 6300 & 11 & \href{https://www.google.com/maps/dir/50.019526,10.2822383/50.0439484,10.2257843}{https://www.google.com/maps/dir/50.019526,10.2822383/50.0439484,10.2257843}\\ 
    \hline
    16 & Gochsheim & Gerolzhofen & 15800 & 17 & \href{https://www.google.com/maps/dir/50.019526,10.2822383/49.9010511,10.3489622}{https://www.google.com/maps/dir/50.019526,10.2822383/49.9010511,10.3489622}\\ 
    \hline
\end{tabular}    
\newline
\newline
\begin{listing}[htbp]
    \begin{minted}{sql}
        SELECT 
        `potentials`.`id` AS `id`, 
        `from_places`.`name` AS `Quelle`,
        `to_places`.`name` AS `Ziel`, 
        `potentials`.`length` AS `Fahrtstrecke [m]`, 
        `potentials`.`miv-duration` AS `Fahrtdauer [min]`,
        CONCAT('https://www.google.com/maps/dir/', `from_places`.`LAT`, ",", `from_places`.`LONG`, '/', `to_places`.`LAT`, ',', `to_places`.`LONG`) AS `Google-Maps Link`
    FROM `potentials`
    LEFT JOIN `places` AS `from_places` ON `potentials`.`from_id` = `from_places`.`id`
    LEFT JOIN `places` AS `to_places` ON `potentials`.`to_id` = `to_places`.`id`
    WHERE `from_places`.`name` = 'Gochsheim'
    ORDER BY `potentials`.`id`;
    \end{minted}
    \caption{SQL-Abfrage der Fahrtstrecke, Fahrtdauer und des Google-Maps-Link mit der Quelle Gochsheim}\label{lst-f-gochsheim}
\end{listing}
                
                \subsection{Gochsheim OT Weyer}
                Zugeordnete Routen:
\newline
\newline
\begin{longtabu}{|l|l|l|l|*2{X[l]|}}
    \hline
    id & Quelle & Ziel & Straße & Straßenbeginn & Straßenende\\ 
    \hline
    17 & Gochsheim OT Weyer & Würzburg, Rottendorf & St2277 & Gochsheim OT Weyer & Ausfahrt Schonungen\\ 
    17 &  &  & A70 & Ausfahrt Gochsheim & Ausfahrt Schonungen\\ 
    17 &  &  & A70 & Ausfahrt Schweinfurt-Zentrum & Ausfahrt Gochsheim\\ 
    17 &  &  & A70 & Ausfahrt Schweinfurt-Hafen & Ausfahrt Schweinfurt-Zentrum\\ 
    17 &  &  & A70 & Ausfahrt Schweinfurt-Bergrheinfeld & Ausfahrt Schweinfurt-Hafen\\ 
    17 &  &  & A70 & Autobahnkreuz Werntal & Ausfahrt Schweinfurt-Bergrheinfeld\\ 
    17 &  &  & A70 & Ausfahrt Werneck & Autobahnkreuz Werntal\\ 
    17 &  &  & A70 & Autobahnkreuz Scheinfurt-Werneck & Ausfahrt Werneck\\ 
    17 &  &  & A7 & Autobahnkreuz Scheinfurt-Werneck & Ausfahrt Gramschatzer Wald\\ 
    17 &  &  & A7 & Ausfahrt Gramschatzer Wald & Ausfahrt Würzburg-Estenfeld\\ 
    17 &  &  & B19 & B19 Ausfahrt Estenfeld Ost & Ausfahrt Würzburg-Estenfeld\\ 
    17 &  &  & B19 & B19/WÜ8 & B19 Ausfahrt Estenfeld Ost\\ 
    17 &  &  & B19 & Würzburg B19 Ikea & B19/WÜ8\\ 
    17 &  &  & B19 & Würzburg B19 Lengfeld & Würzburg B19 Ikea\\ 
    17 &  &  & B19 & Würzburg B8/B19 (Grainbergknoten) & Würzburg B19 Lengfeld\\ 
    \hline
    18 & Gochsheim OT Weyer & Bamberg, Haßfurt & St2277 & Gochsheim OT Weyer & Ausfahrt Schonungen\\ 
    18 &  &  & A70 & Ausfahrt Schonungen & Ausfahrt Haßfurt\\ 
    18 &  &  & A70 & Ausfahrt Haßfurt & Ausfahrt Knetzgau\\ 
    18 &  &  & A70 & Ausfahrt Knetzgau & Ausfahrt Eltmann\\ 
    18 &  &  & A70 & Ausfahrt Eltmann & Ausfahrt Viereth-Thunstadt\\ 
    18 &  &  & A70 & Ausfahrt Viereth-Thunstadt & Ausfahrt Bamberg-Hafen\\ 
    \hline
    19 & Gochsheim OT Weyer & Bad Kissingen & St2277 & Gochsheim OT Weyer & Ausfahrt Schonungen\\ 
    19 &  &  & A70 & Ausfahrt Gochsheim & Ausfahrt Schonungen\\ 
    19 &  &  & A70 & Ausfahrt Schweinfurt-Zentrum & Ausfahrt Gochsheim\\ 
    19 &  &  & A70 & Ausfahrt Schweinfurt-Hafen & Ausfahrt Schweinfurt-Zentrum\\ 
    19 &  &  & A70 & Ausfahrt Schweinfurt-Bergrheinfeld & Ausfahrt Schweinfurt-Hafen\\ 
    19 &  &  & A70 & Autobahnkreuz Werntal & Ausfahrt Schweinfurt-Bergrheinfeld\\ 
    19 &  &  & A70 & Ausfahrt Werneck & Autobahnkreuz Werntal\\ 
    19 &  &  & A71 & Autobahnkreuz Werntal & Ausfahrt Schweinfurt-West\\ 
    19 &  &  & A71 & Ausfahrt Schweinfurt-West & Ausfahrt Poppenhausen\\ 
    19 &  &  & A71 & Ausfahrt Poppenhausen & Ausfahrt Bad Kissingen/Oerlenbach\\ 
    19 &  &  & B19 & Ausfahrt Bad Kissingen/Oerlenbach & B286/B19 (bei Oerlenbach)\\ 
    19 &  &  & B286 & B286/B19 (bei Oerlenbach) & Oerlenbach\\ 
    19 &  &  & B286 & Oerlenbach & B286/KG46\\ 
    19 &  &  & B286 & B286/KG46 & Arnshausen\\ 
    19 &  &  & B286 & Arnshausen & Bad Kissingen\\ 
    \hline
    20 & Gochsheim OT Weyer & Schweinfurt & St2277 & Gochsheim OT Weyer & Ausfahrt Schonungen\\ 
    20 &  &  & A70 & Ausfahrt Gochsheim & Ausfahrt Schonungen\\ 
    20 &  &  & A70 & Ausfahrt Schweinfurt-Zentrum & Ausfahrt Gochsheim\\ 
    20 &  &  & B286 & Ausfahrt Schweinfurt-Zentrum & B286 Schweinfurt Abfahrt Hans-Böckler-Straße\\ 
    20 &  &  & B286 & B286 Schweinfurt Abfahrt Hans-Böckler-Straße & B286/B26 (Schweinfurt Nähe Mainbrücke)\\ 
    20 &  &  & B286 & B286/B26 (Schweinfurt Nähe Mainbrücke) & B286/B303\\ 
    \hline
    21 & Gochsheim OT Weyer & Gerolzhofen & St2277 & St2272/St2277 (Kreisel südlich Gochsheim) & Gochsheim OT Weyer\\ 
    21 &  &  & St2272 & St2272/St2277 (Kreisel südlich Gochsheim) & Grettstatt\\ 
    21 &  &  & St2272 & Grettstatt & Sulzheim\\ 
    21 &  &  & St2272 & Sulzheim & Alitzheim\\ 
    21 &  &  & B286 & B286/St2275 (Gerolzhofen bei Rügshofen) & B286/St2272 (bei Alitzheim)\\ 
    21 &  &  & St2275 & Gerolzhofen & B286/St2275 (Gerolzhofen bei Rügshofen)\\ 
    \hline
    22 & Gochsheim OT Weyer & Gochsheim & St2277 & St2272/St2277 (Kreisel südlich Gochsheim) & Gochsheim OT Weyer\\         
    \hline
\end{longtabu}

\begin{listing}[htbp]
\begin{minted}{sql}
    SELECT 
	`potentials`.`id` AS `id`,
	`from_places`.`name` AS `Quelle`, 
	`to_places`.`name` AS `Ziel`,
	`streets`.`street` AS `Straße`,
	`from_street_places`.`name` AS `Straßenbeginn`,
   `to_street_places`.`name` AS `Straßenende`
FROM `potentials`
LEFT JOIN `places` AS `from_places` ON `potentials`.`from_id` = `from_places`.`id`
LEFT JOIN `places` AS `to_places` ON `potentials`.`to_id` = `to_places`.`id`
LEFT JOIN `routes` ON `routes`.`potential_id` = `potentials`.`id`
LEFT JOIN `streets` ON `streets`.`id` = `routes`.`street_id`
LEFT JOIN `places` AS `from_street_places` ON `streets`.`from_id` = `from_street_places`.`id`
LEFT JOIN `places` AS `to_street_places` ON `streets`.`to_id` = `to_street_places`.`id`
WHERE `from_places`.`name` = 'Gochsheim OT Weyer'
ORDER BY `potentials`.`id`, `routes`.`number_on_route`;
\end{minted}
\caption{SQL-Abfrage der zugeordneten Straßen mit der Quelle Gochsheim OT Weyer}\label{lst-rt-weyer}
\end{listing}


Länge, Fahrzeiten und Google Maps:
\newline
\begin{longtabu}{| l | *5{X[l]|}}
    \hline
    id & Quelle & Ziel & Fahrtstrecke [m] & Fahrtdauer [min] & Google-Maps Link\\ 
    \hline
    17 & Gochsheim OT Weyer & Würzburg, Rottendorf & 51100 & 38 & \url{https://www.google.com/maps/dir/50.0225798,10.3146385/49.7931,9.9280108}\\ 
    \hline
    18 & Gochsheim OT Weyer & Bamberg, Haßfurt & 49400 & 34 & \url{https://www.google.com/maps/dir/50.0225798,10.3146385/49.8912678,10.8865984}\\ 
    \hline
    19 & Gochsheim OT Weyer & Bad Kissingen & 42000 & 35 & \url{https://www.google.com/maps/dir/50.0225798,10.3146385/50.1990369,10.0762182}\\ 
    \hline
    20 & Gochsheim OT Weyer & Schweinfurt & 11300 & 13 & \url{https://www.google.com/maps/dir/50.0225798,10.3146385/50.0439484,10.2257843}\\ 
    \hline
    21 & Gochsheim OT Weyer & Gerolzhofen & 17500 & 18 & \url{https://www.google.com/maps/dir/50.0225798,10.3146385/49.9010511,10.3489622}\\ 
    \hline
    22 & Gochsheim OT Weyer & Gochsheim & 2600 & 4 & \url{https://www.google.com/maps/dir/50.0225798,10.3146385/50.019526,10.2822383}\\ 
    \hline
\end{longtabu}

\begin{listing}[htbp]
    \begin{minted}{sql}
        SELECT 
        `potentials`.`id` AS `id`, 
        `from_places`.`name` AS `Quelle`,
        `to_places`.`name` AS `Ziel`, 
        `potentials`.`length` AS `Fahrtstrecke [m]`, 
        `potentials`.`miv-duration` AS `Fahrtdauer [min]`,
        CONCAT('https://www.google.com/maps/dir/', `from_places`.`LAT`, ",", `from_places`.`LONG`, '/', `to_places`.`LAT`, ',', `to_places`.`LONG`) AS `Google-Maps Link`
    FROM `potentials`
    LEFT JOIN `places` AS `from_places` ON `potentials`.`from_id` = `from_places`.`id`
    LEFT JOIN `places` AS `to_places` ON `potentials`.`to_id` = `to_places`.`id`
    WHERE `from_places`.`name` = 'Gochsheim OT Weyer'
    ORDER BY `potentials`.`id`;
    \end{minted}
    \caption{SQL-Abfrage der Fahrtstrecke, Fahrtdauer und des Google-Maps-Link mit der Quelle Gochsheim OT Weyer}\label{lst-f-weyer}
\end{listing}
                
                \subsection{Schwebheim}
                Zugeordnete Routen:
\newline
\newline
\begin{longtabu}{|l|l|l|l|*2{X[l]|}}
    \hline
    id & Quelle & Ziel & Straße & Straßenbeginn & Straßenende\\ 
    \hline
    23 & Schwebheim & Schweinfurt & St2277 & B286/St2277 (bei Schwebheim) & Schwebheim\\ 
    23 &  &  & B286 & B286/St2277 (bei Schwebheim) & B286/SW3/St2271 (nördlich Schwebheim)\\ 
    23 &  &  & B286 & B286/SW3/St2271 (nördlich Schwebheim) & Ausfahrt Schweinfurt-Zentrum\\ 
    23 &  &  & B286 & Ausfahrt Schweinfurt-Zentrum & B286 Schweinfurt Abfahrt Hans-Böckler-Straße\\ 
    23 &  &  & B286 & B286 Schweinfurt Abfahrt Hans-Böckler-Straße & B286/B26 (Schweinfurt Nähe Mainbrücke)\\ 
    23 &  &  & B286 & B286/B26 (Schweinfurt Nähe Mainbrücke) & B286/B303\\ 
    \hline
    24 & Schwebheim & Gochsheim & St2272 & St2272/SW3 Gochsheim Kreisel Industriestraße & St2272/St2277 (Kreisel südlich Gochsheim)\\ 
    24 &  &  & SW3 & St2272/SW3 Gochsheim & St2272/SW3 Gochsheim Kreisel Industriestraße\\ 
    \hline
    25 & Schwebheim & Gerolzhofen & St2277 & B286/St2277 (bei Schwebheim) & Schwebheim\\ 
    25 &  &  & B286 & B286/St2271 (bei Unterspießheim) & B286/St2277 (bei Schwebheim)\\ 
    25 &  &  & B286 & B286/St2272 (bei Alitzheim) & B286/St2271 (bei Unterspießheim)\\ 
    25 &  &  & B286 & B286/St2275 (Gerolzhofen bei Rügshofen) & B286/St2272 (bei Alitzheim)\\ 
    25 &  &  & St2275 & Gerolzhofen & B286/St2275 (Gerolzhofen bei Rügshofen)\\ 
    \hline
    26 & Schwebheim & Grettstatt & SW28 & Schwebheim & Grettstatt\\ 
    \hline
\end{longtabu}

\begin{listing}[htbp]
\begin{minted}{sql}
    SELECT 
	`potentials`.`id` AS `id`,
	`from_places`.`name` AS `Quelle`, 
	`to_places`.`name` AS `Ziel`,
	`streets`.`street` AS `Straße`,
	`from_street_places`.`name` AS `Straßenbeginn`,
   `to_street_places`.`name` AS `Straßenende`
FROM `potentials`
LEFT JOIN `places` AS `from_places` ON `potentials`.`from_id` = `from_places`.`id`
LEFT JOIN `places` AS `to_places` ON `potentials`.`to_id` = `to_places`.`id`
LEFT JOIN `routes` ON `routes`.`potential_id` = `potentials`.`id`
LEFT JOIN `streets` ON `streets`.`id` = `routes`.`street_id`
LEFT JOIN `places` AS `from_street_places` ON `streets`.`from_id` = `from_street_places`.`id`
LEFT JOIN `places` AS `to_street_places` ON `streets`.`to_id` = `to_street_places`.`id`
WHERE `from_places`.`name` = 'Schwebheim'
ORDER BY `potentials`.`id`, `routes`.`number_on_route`;
\end{minted}
\caption{SQL-Abfrage der zugeordneten Straßen mit der Quelle Schwebheim}\label{lst-rt-schwebheim}
\end{listing}


Länge, Fahrzeiten und Google Maps:
\newline
\begin{longtabu}{| l | *5{X[l]|}}
    \hline
    id & Quelle & Ziel & Fahrtstrecke [m] & Fahrtdauer [min] & Google-Maps Link\\ 
    \hline
    23 & Schwebheim & Schweinfurt & 8900 & 11 & \url{https://www.google.com/maps/dir/49.9952839,10.2457311/50.0439484,10.2257843}\\ 
    \hline
    24 & Schwebheim & Gochsheim & 5400 & 7 & \url{https://www.google.com/maps/dir/49.9952839,10.2457311/50.019526,10.2822383}\\ 
    \hline
    25 & Schwebheim & Gerolzhofen & 15000 & 14 & \url{https://www.google.com/maps/dir/49.9952839,10.2457311/49.9010511,10.3489622}\\ 
    \hline
    26 & Schwebheim & Grettstatt & 6000 & 6 & \url{https://www.google.com/maps/dir/49.9952839,10.2457311/49.9847108,10.3121683}\\ 
    \hline
\end{longtabu}

\begin{listing}[htbp]
    \begin{minted}{sql}
        SELECT 
        `potentials`.`id` AS `id`, 
        `from_places`.`name` AS `Quelle`,
        `to_places`.`name` AS `Ziel`, 
        `potentials`.`length` AS `Fahrtstrecke [m]`, 
        `potentials`.`miv-duration` AS `Fahrtdauer [min]`,
        CONCAT('https://www.google.com/maps/dir/', `from_places`.`LAT`, ",", `from_places`.`LONG`, '/', `to_places`.`LAT`, ',', `to_places`.`LONG`) AS `Google-Maps Link`
    FROM `potentials`
    LEFT JOIN `places` AS `from_places` ON `potentials`.`from_id` = `from_places`.`id`
    LEFT JOIN `places` AS `to_places` ON `potentials`.`to_id` = `to_places`.`id`
    WHERE `from_places`.`name` = 'Schwebheim'
    ORDER BY `potentials`.`id`;
    \end{minted}
    \caption{SQL-Abfrage der Fahrtstrecke, Fahrtdauer und des Google-Maps-Link mit der Quelle Schwebheim}\label{lst-f-schwebheim}
\end{listing}
                
                \subsection{Grettstatt}
                Zugeordnete Routen:
\newline
\newline
\begin{longtabu}{|l|l|l|l|*2{X[l]|}}
    \hline
    id & Quelle & Ziel & Straße & Straßenbeginn & Straßenende\\ 
    \hline
    27 & Grettstatt & Würzburg, Rottendorf & St2272 & St2272/St2277 (Kreisel südlich Gochsheim) & Grettstatt\\ 
    27 &  &  & St2272 & St2272/SW3 Gochsheim Kreisel Industriestraße & St2272/St2277 (Kreisel südlich Gochsheim)\\ 
    27 &  &  & SW3 & St2272/SW3 Gochsheim & St2272/SW3 Gochsheim Kreisel Industriestraße\\ 
    27 &  &  & St2272 & Ausfahrt Gochsheim & St2272/SW3 Gochsheim\\ 
    27 &  &  & A70 & Ausfahrt Schweinfurt-Zentrum & Ausfahrt Gochsheim\\ 
    27 &  &  & A70 & Ausfahrt Schweinfurt-Hafen & Ausfahrt Schweinfurt-Zentrum\\ 
    27 &  &  & A70 & Ausfahrt Schweinfurt-Bergrheinfeld & Ausfahrt Schweinfurt-Hafen\\ 
    27 &  &  & A70 & Autobahnkreuz Werntal & Ausfahrt Schweinfurt-Bergrheinfeld\\ 
    27 &  &  & A70 & Ausfahrt Werneck & Autobahnkreuz Werntal\\ 
    27 &  &  & A70 & Autobahnkreuz Scheinfurt-Werneck & Ausfahrt Werneck\\ 
    27 &  &  & A7 & Autobahnkreuz Scheinfurt-Werneck & Ausfahrt Gramschatzer Wald\\ 
    27 &  &  & A7 & Ausfahrt Gramschatzer Wald & Ausfahrt Würzburg-Estenfeld\\ 
    27 &  &  & B19 & B19 Ausfahrt Estenfeld Ost & Ausfahrt Würzburg-Estenfeld\\ 
    27 &  &  & B19 & B19/WÜ8 & B19 Ausfahrt Estenfeld Ost\\ 
    27 &  &  & B19 & Würzburg B19 Ikea & B19/WÜ8\\ 
    27 &  &  & B19 & Würzburg B19 Lengfeld & Würzburg B19 Ikea\\ 
    27 &  &  & B19 & Würzburg B8/B19 (Grainbergknoten) & Würzburg B19 Lengfeld\\ 
    \hline
    28 & Grettstatt & Bamberg, Haßfurt & St2272 & St2272/St2277 (Kreisel südlich Gochsheim) & Grettstatt\\ 
    28 &  &  & St2277 & St2272/St2277 (Kreisel südlich Gochsheim) & Gochsheim OT Weyer\\ 
    28 &  &  & St2277 & Gochsheim OT Weyer & Ausfahrt Schonungen\\ 
    28 &  &  & A70 & Ausfahrt Schonungen & Ausfahrt Haßfurt\\ 
    28 &  &  & A70 & Ausfahrt Haßfurt & Ausfahrt Knetzgau\\ 
    28 &  &  & A70 & Ausfahrt Knetzgau & Ausfahrt Eltmann\\ 
    28 &  &  & A70 & Ausfahrt Eltmann & Ausfahrt Viereth-Thunstadt\\ 
    28 &  &  & A70 & Ausfahrt Viereth-Thunstadt & Ausfahrt Bamberg-Hafen\\ 
    \hline
    29 & Grettstatt & Schweinfurt & St2272 & St2272/St2277 (Kreisel südlich Gochsheim) & Grettstatt\\ 
    29 &  &  & St2272 & St2272/SW3 Gochsheim Kreisel Industriestraße & St2272/St2277 (Kreisel südlich Gochsheim)\\ 
    29 &  &  & SW3 & St2272/SW3 Gochsheim Kreisel Industriestraße & B286/SW3/St2271 (nördlich Schwebheim)\\ 
    29 &  &  & B286 & B286/SW3/St2271 (nördlich Schwebheim) & Ausfahrt Schweinfurt-Zentrum\\ 
    29 &  &  & B286 & Ausfahrt Schweinfurt-Zentrum & B286 Schweinfurt Abfahrt Hans-Böckler-Straße\\ 
    29 &  &  & B286 & B286 Schweinfurt Abfahrt Hans-Böckler-Straße & B286/B26 (Schweinfurt Nähe Mainbrücke)\\ 
    29 &  &  & B286 & B286/B26 (Schweinfurt Nähe Mainbrücke) & B286/B303\\ 
    \hline
    30 & Grettstatt & Sennfeld & St2272 & St2272/St2277 (Kreisel südlich Gochsheim) & Grettstatt\\ 
    30 &  &  & St2272 & St2272/SW3 Gochsheim Kreisel Industriestraße & St2272/St2277 (Kreisel südlich Gochsheim)\\ 
    30 &  &  & SW3 & St2272/SW3 Gochsheim & St2272/SW3 Gochsheim Kreisel Industriestraße\\ 
    30 &  &  & St2272 & Ausfahrt Gochsheim & St2272/SW3 Gochsheim\\ 
    30 &  &  & St2272 & St2271/St2272 (bei Sennfeld) & Ausfahrt Gochsheim\\ 
    \hline
    31 & Grettstatt & Gochsheim & St2272 & St2272/St2277 (Kreisel südlich Gochsheim) & Grettstatt\\ 
    \hline
    32 & Grettstatt & Gerolzhofen & St2272 & Grettstatt & Sulzheim\\ 
    32 &  &  & St2272 & Sulzheim & Alitzheim\\ 
    32 &  &  & St2272 & Alitzheim & B286/St2272 (bei Alitzheim)\\ 
    32 &  &  & B286 & B286/St2275 (Gerolzhofen bei Rügshofen) & B286/St2272 (bei Alitzheim)\\ 
    32 &  &  & St2275 & Gerolzhofen & B286/St2275 (Gerolzhofen bei Rügshofen)\\ 
    \hline
\end{longtabu}

\begin{listing}[htbp]
\begin{minted}{sql}
    SELECT 
	`potentials`.`id` AS `id`,
	`from_places`.`name` AS `Quelle`, 
	`to_places`.`name` AS `Ziel`,
	`streets`.`street` AS `Straße`,
	`from_street_places`.`name` AS `Straßenbeginn`,
   `to_street_places`.`name` AS `Straßenende`
FROM `potentials`
LEFT JOIN `places` AS `from_places` ON `potentials`.`from_id` = `from_places`.`id`
LEFT JOIN `places` AS `to_places` ON `potentials`.`to_id` = `to_places`.`id`
LEFT JOIN `routes` ON `routes`.`potential_id` = `potentials`.`id`
LEFT JOIN `streets` ON `streets`.`id` = `routes`.`street_id`
LEFT JOIN `places` AS `from_street_places` ON `streets`.`from_id` = `from_street_places`.`id`
LEFT JOIN `places` AS `to_street_places` ON `streets`.`to_id` = `to_street_places`.`id`
WHERE `from_places`.`name` = 'Grettstatt'
ORDER BY `potentials`.`id`, `routes`.`number_on_route`;
\end{minted}
\caption{SQL-Abfrage der zugeordneten Straßen mit der Quelle Grettstatt}\label{lst-rt-grettstatt}
\end{listing}


Länge, Fahrzeiten und Google Maps:
\newline
\begin{longtabu}{| l | *5{X[l]|}}
    \hline
    id & Quelle & Ziel & Fahrtstrecke [m] & Fahrtdauer [min] & Google-Maps Link\\ 
    \hline
    27 & Grettstatt & Würzburg, Rottendorf & 51900 & 41 & \url{https://www.google.com/maps/dir/49.9847108,10.3121683/49.7931,9.9280108}\\ 
    \hline
    28 & Grettstatt & Bamberg, Haßfurt & 55300 & 39 & \url{https://www.google.com/maps/dir/49.9847108,10.3121683/49.8912678,10.8865984}\\ 
    \hline
    29 & Grettstatt & Schweinfurt & 13200 & 15 & \url{https://www.google.com/maps/dir/49.9847108,10.3121683/50.0439484,10.2257843}\\ 
    \hline
    30 & Grettstatt & Sennfeld & 8700 & 11 & \url{https://www.google.com/maps/dir/49.9847108,10.3121683/50.0422146,10.2609081}\\ 
    \hline
    31 & Grettstatt & Gochsheim & 5100 & 7 & \url{https://www.google.com/maps/dir/49.9847108,10.3121683/50.019526,10.2822383}\\ 
    \hline
    32 & Grettstatt & Gerolzhofen & 11200 & 12 & \url{https://www.google.com/maps/dir/49.9847108,10.3121683/49.9010511,10.3489622}\\ 
    \hline
\end{longtabu}

\begin{listing}[htbp]
    \begin{minted}{sql}
        SELECT 
        `potentials`.`id` AS `id`, 
        `from_places`.`name` AS `Quelle`,
        `to_places`.`name` AS `Ziel`, 
        `potentials`.`length` AS `Fahrtstrecke [m]`, 
        `potentials`.`miv-duration` AS `Fahrtdauer [min]`,
        CONCAT('https://www.google.com/maps/dir/', `from_places`.`LAT`, ",", `from_places`.`LONG`, '/', `to_places`.`LAT`, ',', `to_places`.`LONG`) AS `Google-Maps Link`
    FROM `potentials`
    LEFT JOIN `places` AS `from_places` ON `potentials`.`from_id` = `from_places`.`id`
    LEFT JOIN `places` AS `to_places` ON `potentials`.`to_id` = `to_places`.`id`
    WHERE `from_places`.`name` = 'Grettstatt'
    ORDER BY `potentials`.`id`;
    \end{minted}
    \caption{SQL-Abfrage der Fahrtstrecke, Fahrtdauer und des Google-Maps-Link mit der Quelle Grettstatt}\label{lst-f-grettstatt}
\end{listing}
                
                \subsection{Grettstatt OT Dürrfeld}
                Zugeordnete Routen:
\newline
\newline
\begin{longtabu}{|l|l|l|l|*2{X[l]|}}
    \hline
    id & Quelle & Ziel & Straße & Straßenbeginn & Straßenende\\ 
    \hline
    33 & Grettstatt OT Dürrfeld & Würzburg, Rottendorf & SW54 & Grettstatt OT Dürrfeld & Obereuerheim\\ 
    33 &  &  & St2277 & Untereuerheim & Obereuerheim\\ 
    33 &  &  & St2277 & Ausfahrt Schonungen & Untereuerheim\\ 
    33 &  &  & A70 & Ausfahrt Gochsheim & Ausfahrt Schonungen\\ 
    33 &  &  & A70 & Ausfahrt Schweinfurt-Zentrum & Ausfahrt Gochsheim\\ 
    33 &  &  & A70 & Ausfahrt Schweinfurt-Hafen & Ausfahrt Schweinfurt-Zentrum\\ 
    33 &  &  & A70 & Ausfahrt Schweinfurt-Bergrheinfeld & Ausfahrt Schweinfurt-Hafen\\ 
    33 &  &  & A70 & Autobahnkreuz Werntal & Ausfahrt Schweinfurt-Bergrheinfeld\\ 
    33 &  &  & A70 & Ausfahrt Werneck & Autobahnkreuz Werntal\\ 
    33 &  &  & A70 & Autobahnkreuz Scheinfurt-Werneck & Ausfahrt Werneck\\ 
    33 &  &  & A7 & Autobahnkreuz Scheinfurt-Werneck & Ausfahrt Gramschatzer Wald\\ 
    33 &  &  & A7 & Ausfahrt Gramschatzer Wald & Ausfahrt Würzburg-Estenfeld\\ 
    33 &  &  & B19 & B19 Ausfahrt Estenfeld Ost & Ausfahrt Würzburg-Estenfeld\\ 
    33 &  &  & B19 & B19/WÜ8 & B19 Ausfahrt Estenfeld Ost\\ 
    33 &  &  & B19 & Würzburg B19 Ikea & B19/WÜ8\\ 
    33 &  &  & B19 & Würzburg B19 Lengfeld & Würzburg B19 Ikea\\ 
    33 &  &  & B19 & Würzburg B8/B19 (Grainbergknoten) & Würzburg B19 Lengfeld\\ 
    \hline
    35 & Grettstatt OT Dürrfeld & Schweinfurt & SW54 & Grettstatt OT Dürrfeld & Obereuerheim\\ 
    35 &  &  & St2277 & Untereuerheim & Obereuerheim\\ 
    35 &  &  & St2277 & Ausfahrt Schonungen & Untereuerheim\\ 
    35 &  &  & A70 & Ausfahrt Gochsheim & Ausfahrt Schonungen\\ 
    35 &  &  & A70 & Ausfahrt Schweinfurt-Zentrum & Ausfahrt Gochsheim\\ 
    35 &  &  & B286 & Ausfahrt Schweinfurt-Zentrum & B286 Schweinfurt Abfahrt Hans-Böckler-Straße\\ 
    35 &  &  & B286 & B286 Schweinfurt Abfahrt Hans-Böckler-Straße & B286/B26 (Schweinfurt Nähe Mainbrücke)\\ 
    35 &  &  & B286 & B286/B26 (Schweinfurt Nähe Mainbrücke) & B286/B303\\ 
    \hline
    37 & Grettstatt OT Dürrfeld & Gochsheim & SW28 & Grettstatt OT Dürrfeld & Grettstatt\\ 
    37 &  &  & St2272 & St2272/St2277 (Kreisel südlich Gochsheim) & Grettstatt\\ 
    \hline
    38 & Grettstatt OT Dürrfeld & Gerolzhofen & SW54 & Grettstatt OT Dürrfeld & Kleinrheinfeld\\ 
    38 &  &  & SW54 & Kleinrheinfeld & ST2275/SW54\\ 
    38 &  &  & St2275 & St2275/SW40 & ST2275/SW54\\ 
    38 &  &  & St2275 & Mönchstockheim & St2275/SW40\\ 
    38 &  &  & St2275 & B286/St2275 (Gerolzhofen bei Rügshofen) & Mönchstockheim\\ 
    38 &  &  & St2275 & Gerolzhofen & B286/St2275 (Gerolzhofen bei Rügshofen)\\ 
    \hline
    39 & Grettstatt OT Dürrfeld & Grettstatt & SW28 & Grettstatt OT Dürrfeld & Grettstatt\\ 
    \hline
\end{longtabu}

\begin{listing}[htbp]
\begin{minted}{sql}
    SELECT 
	`potentials`.`id` AS `id`,
	`from_places`.`name` AS `Quelle`, 
	`to_places`.`name` AS `Ziel`,
	`streets`.`street` AS `Straße`,
	`from_street_places`.`name` AS `Straßenbeginn`,
   `to_street_places`.`name` AS `Straßenende`
FROM `potentials`
LEFT JOIN `places` AS `from_places` ON `potentials`.`from_id` = `from_places`.`id`
LEFT JOIN `places` AS `to_places` ON `potentials`.`to_id` = `to_places`.`id`
LEFT JOIN `routes` ON `routes`.`potential_id` = `potentials`.`id`
LEFT JOIN `streets` ON `streets`.`id` = `routes`.`street_id`
LEFT JOIN `places` AS `from_street_places` ON `streets`.`from_id` = `from_street_places`.`id`
LEFT JOIN `places` AS `to_street_places` ON `streets`.`to_id` = `to_street_places`.`id`
WHERE `from_places`.`name` = 'Grettstatt OT Dürrfeld'
ORDER BY `potentials`.`id`, `routes`.`number_on_route`;
\end{minted}
\caption{SQL-Abfrage der zugeordneten Straßen mit der Quelle Grettstatt OT Dürrfeld}\label{lst-rt-duerrfeld}
\end{listing}


Länge, Fahrzeiten und Google Maps:
\newline
\begin{longtabu}{| l | *5{X[l]|}}
    \hline
    id & Quelle & Ziel & Fahrtstrecke [m] & Fahrtdauer [min] & Google-Maps Link\\ 
    \hline
    33 & Grettstatt OT Dürrfeld & Würzburg, Rottendorf & 56600 & 42 & \url{https://www.google.com/maps/dir/49.9801682,10.3634918/49.7931,9.9280108}\\ 
    \hline
    35 & Grettstatt OT Dürrfeld & Schweinfurt & 17200 & 17 & \url{https://www.google.com/maps/dir/49.9801682,10.3634918/50.0439484,10.2257843}\\ 
    \hline
    37 & Grettstatt OT Dürrfeld & Gochsheim & 8900 & 10 & \url{https://www.google.com/maps/dir/49.9801682,10.3634918/50.019526,10.2822383}\\ 
    \hline
    38 & Grettstatt OT Dürrfeld & Gerolzhofen & 11200 & 12 & \url{https://www.google.com/maps/dir/49.9801682,10.3634918/49.9010511,10.3489622}\\ 
    \hline
    39 & Grettstatt OT Dürrfeld & Grettstatt & 4100 & 4 & \url{https://www.google.com/maps/dir/49.9801682,10.3634918/49.9847108,10.3121683}\\ 
    \hline
\end{longtabu}

\begin{listing}[htbp]
    \begin{minted}{sql}
        SELECT 
        `potentials`.`id` AS `id`, 
        `from_places`.`name` AS `Quelle`,
        `to_places`.`name` AS `Ziel`, 
        `potentials`.`length` AS `Fahrtstrecke [m]`, 
        `potentials`.`miv-duration` AS `Fahrtdauer [min]`,
        CONCAT('https://www.google.com/maps/dir/', `from_places`.`LAT`, ",", `from_places`.`LONG`, '/', `to_places`.`LAT`, ',', `to_places`.`LONG`) AS `Google-Maps Link`
    FROM `potentials`
    LEFT JOIN `places` AS `from_places` ON `potentials`.`from_id` = `from_places`.`id`
    LEFT JOIN `places` AS `to_places` ON `potentials`.`to_id` = `to_places`.`id`
    WHERE `from_places`.`name` = 'Grettstatt OT Dürrfeld'
    ORDER BY `potentials`.`id`;
    \end{minted}
    \caption{SQL-Abfrage der Fahrtstrecke, Fahrtdauer und des Google-Maps-Link mit der Quelle Grettstatt OT Dürrfeld}\label{lst-f-duerrfeld}
\end{listing}
                
                \subsection{Donnersdorf}
                Zugeordnete Routen:
\newline
\newline
\begin{longtabu}{|l|l|l|l|*2{X[l]|}}
    \hline
    id & Quelle & Ziel & Straße & Straßenbeginn & Straßenende\\ 
    \hline
    40 & Donnersdorf & Schweinfurt & St2275 & Donnersdorf & Dampfach\\ 
    40 &  &  & St2275 & Dampfach & St2275/St2426\\ 
    40 &  &  & St2426 & St2275/St2426 & Ausfahrt Haßfurt\\ 
    40 &  &  & A70 & Ausfahrt Schonungen & Ausfahrt Haßfurt\\ 
    40 &  &  & A70 & Ausfahrt Gochsheim & Ausfahrt Schonungen\\ 
    40 &  &  & A70 & Ausfahrt Schweinfurt-Zentrum & Ausfahrt Gochsheim\\ 
    40 &  &  & B286 & Ausfahrt Schweinfurt-Zentrum & B286 Schweinfurt Abfahrt Hans-Böckler-Straße\\ 
    40 &  &  & B286 & B286 Schweinfurt Abfahrt Hans-Böckler-Straße & B286/B26 (Schweinfurt Nähe Mainbrücke)\\ 
    40 &  &  & B286 & B286/B26 (Schweinfurt Nähe Mainbrücke) & B286/B303\\ 
    \hline
    42 & Donnersdorf & Grettstatt & St2277 & St2277/SW28 (Abzweig Ri. Dampfach) & Donnersdorf\\ 
    42 &  &  & St2277 & Pusselsheim & St2277/SW28 (Abzweig Ri. Dampfach)\\ 
    42 &  &  & SW28 & Pusselsheim & Grettstatt OT Dürrfeld\\ 
    42 &  &  & SW28 & Grettstatt OT Dürrfeld & Grettstatt\\ 
    \hline
\end{longtabu}

\begin{listing}[htbp]
\begin{minted}{sql}
    SELECT 
	`potentials`.`id` AS `id`,
	`from_places`.`name` AS `Quelle`, 
	`to_places`.`name` AS `Ziel`,
	`streets`.`street` AS `Straße`,
	`from_street_places`.`name` AS `Straßenbeginn`,
   `to_street_places`.`name` AS `Straßenende`
FROM `potentials`
LEFT JOIN `places` AS `from_places` ON `potentials`.`from_id` = `from_places`.`id`
LEFT JOIN `places` AS `to_places` ON `potentials`.`to_id` = `to_places`.`id`
LEFT JOIN `routes` ON `routes`.`potential_id` = `potentials`.`id`
LEFT JOIN `streets` ON `streets`.`id` = `routes`.`street_id`
LEFT JOIN `places` AS `from_street_places` ON `streets`.`from_id` = `from_street_places`.`id`
LEFT JOIN `places` AS `to_street_places` ON `streets`.`to_id` = `to_street_places`.`id`
WHERE `from_places`.`name` = 'Donnersdorf'
ORDER BY `potentials`.`id`, `routes`.`number_on_route`;
\end{minted}
\caption{SQL-Abfrage der zugeordneten Straßen mit der Quelle Donnersdorf}\label{lst-rt-donnersdorf}
\end{listing}


Länge, Fahrzeiten und Google Maps:
\newline
\begin{longtabu}{| l | *5{X[l]|}}
    \hline
    id & Quelle & Ziel & Fahrtstrecke [m] & Fahrtdauer [min] & Google-Maps Link\\ 
    \hline
    40 & Donnersdorf & Schweinfurt & 23000 & 20 & \url{https://www.google.com/maps/dir/49.9708256,10.4176558/50.0439484,10.2257843}\\ 
    \hline
    42 & Donnersdorf & Grettstatt & 9100 & 9 & \url{https://www.google.com/maps/dir/49.9708256,10.4176558/49.9847108,10.3121683}\\ 
    \hline
\end{longtabu}

\begin{listing}[htbp]
    \begin{minted}{sql}
        SELECT 
        `potentials`.`id` AS `id`, 
        `from_places`.`name` AS `Quelle`,
        `to_places`.`name` AS `Ziel`, 
        `potentials`.`length` AS `Fahrtstrecke [m]`, 
        `potentials`.`miv-duration` AS `Fahrtdauer [min]`,
        CONCAT('https://www.google.com/maps/dir/', `from_places`.`LAT`, ",", `from_places`.`LONG`, '/', `to_places`.`LAT`, ',', `to_places`.`LONG`) AS `Google-Maps Link`
    FROM `potentials`
    LEFT JOIN `places` AS `from_places` ON `potentials`.`from_id` = `from_places`.`id`
    LEFT JOIN `places` AS `to_places` ON `potentials`.`to_id` = `to_places`.`id`
    WHERE `from_places`.`name` = 'Donnersdorf'
    ORDER BY `potentials`.`id`;
    \end{minted}
    \caption{SQL-Abfrage der Fahrtstrecke, Fahrtdauer und des Google-Maps-Link mit der Quelle Donnersdorf}\label{lst-f-donnersdorf}
\end{listing}
                
                \subsection{Sulzheim}
                Zugeordnete Routen:
\newline
\newline
\begin{longtabu}{|l|l|l|l|*2{X[l]|}}
    \hline
    id & Quelle & Ziel & Straße & Straßenbeginn & Straßenende\\ 
    \hline
    43 & Sulzheim & Schweinfurt & St2272 & Sulzheim & Alitzheim\\ 
    43 &  &  & St2272 & Alitzheim & B286/St2272 (bei Alitzheim)\\ 
    43 &  &  & B286 & B286/St2272 (bei Alitzheim) & B286/St2271 (bei Unterspießheim)\\ 
    43 &  &  & B286 & B286/St2271 (bei Unterspießheim) & B286/St2277 (bei Schwebheim)\\ 
    43 &  &  & B286 & B286/St2277 (bei Schwebheim) & B286/SW3/St2271 (nördlich Schwebheim)\\ 
    43 &  &  & B286 & B286/SW3/St2271 (nördlich Schwebheim) & Ausfahrt Schweinfurt-Zentrum\\ 
    43 &  &  & B286 & Ausfahrt Schweinfurt-Zentrum & B286 Schweinfurt Abfahrt Hans-Böckler-Straße\\ 
    43 &  &  & B286 & B286 Schweinfurt Abfahrt Hans-Böckler-Straße & B286/B26 (Schweinfurt Nähe Mainbrücke)\\ 
    43 &  &  & B286 & B286/B26 (Schweinfurt Nähe Mainbrücke) & B286/B303\\ 
    \hline
    44 & Sulzheim & Gochsheim & St2272 & Grettstatt & Sulzheim\\ 
    44 &  &  & St2272 & St2272/St2277 (Kreisel südlich Gochsheim) & Grettstatt\\ 
    \hline
    45 & Sulzheim & Sennfeld & St2272 & Grettstatt & Sulzheim\\ 
    45 &  &  & St2272 & St2272/St2277 (Kreisel südlich Gochsheim) & Grettstatt\\ 
    45 &  &  & St2272 & St2272/SW3 Gochsheim Kreisel Industriestraße & St2272/St2277 (Kreisel südlich Gochsheim)\\ 
    45 &  &  & SW3 & St2272/SW3 Gochsheim & St2272/SW3 Gochsheim Kreisel Industriestraße\\ 
    45 &  &  & St2272 & Ausfahrt Gochsheim & St2272/SW3 Gochsheim\\ 
    45 &  &  & St2272 & St2271/St2272 (bei Sennfeld) & Ausfahrt Gochsheim\\ 
    \hline
    46 & Sulzheim & Gerolzhofen & St2272 & Sulzheim & Alitzheim\\ 
    46 &  &  & St2272 & Alitzheim & B286/St2272 (bei Alitzheim)\\ 
    46 &  &  & B286 & B286/St2275 (Gerolzhofen bei Rügshofen) & B286/St2272 (bei Alitzheim)\\ 
    46 &  &  & St2275 & Gerolzhofen & B286/St2275 (Gerolzhofen bei Rügshofen)\\ 
    \hline
    47 & Sulzheim & Kitzingen & St2272 & Sulzheim & Alitzheim\\ 
    47 &  &  & SW40 & Herlheim & Alitzheim\\ 
    47 &  &  & SW40 & SW39/SW40 (zw. Herlheim u Zeilitzheim) & Herlheim\\ 
    47 &  &  & SW39 & SW39/SW40 (zw. Herlheim u Zeilitzheim) & Kolitzheim\\ 
    47 &  &  & St2271 & Kolitzheim & Gaibach\\ 
    47 &  &  & St2271 & Gaibach & Volkach\\ 
    47 &  &  & St2271 & Volkach & St2271/KT57\\ 
    47 &  &  & St2271 & St2271/KT57 & Gerlachshausen\\ 
    47 &  &  & St2271 & Gerlachshausen & B22/St2271 (bei Stadtschwarzach)\\ 
    47 &  &  & St2271 & B22/St2271 (bei Stadtschwarzach) & Hörblach\\ 
    47 &  &  & St2271 & Hörblach & Ausfahrt Kitzingen-Schwarzach\\ 
    47 &  &  & St2271 & Ausfahrt Kitzingen-Schwarzach & St2271/St2272 (bei Kitzingen-Etwashausen)\\ 
    47 &  &  & St2271 & St2271/St2272 (bei Kitzingen-Etwashausen) & B8/St2271 (Kitzingen bei e-center)\\ 
    47 &  &  & B8 & Kitzingen & B8/St2271 (Kitzingen bei e-center)\\ 
    \hline
    48 & Sulzheim & Alitzheim & St2272 & Sulzheim & Alitzheim\\ 
    \hline
\end{longtabu}

\begin{listing}[htbp]
\begin{minted}{sql}
    SELECT 
	`potentials`.`id` AS `id`,
	`from_places`.`name` AS `Quelle`, 
	`to_places`.`name` AS `Ziel`,
	`streets`.`street` AS `Straße`,
	`from_street_places`.`name` AS `Straßenbeginn`,
   `to_street_places`.`name` AS `Straßenende`
FROM `potentials`
LEFT JOIN `places` AS `from_places` ON `potentials`.`from_id` = `from_places`.`id`
LEFT JOIN `places` AS `to_places` ON `potentials`.`to_id` = `to_places`.`id`
LEFT JOIN `routes` ON `routes`.`potential_id` = `potentials`.`id`
LEFT JOIN `streets` ON `streets`.`id` = `routes`.`street_id`
LEFT JOIN `places` AS `from_street_places` ON `streets`.`from_id` = `from_street_places`.`id`
LEFT JOIN `places` AS `to_street_places` ON `streets`.`to_id` = `to_street_places`.`id`
WHERE `from_places`.`name` = 'Sulzheim'
ORDER BY `potentials`.`id`, `routes`.`number_on_route`;
\end{minted}
\caption{SQL-Abfrage der zugeordneten Straßen mit der Quelle Sulzheim}\label{lst-rt-sulzheim}
\end{listing}


Länge, Fahrzeiten und Google Maps:
\newline
\begin{longtabu}{| l | *5{X[l]|}}
    \hline
    id & Quelle & Ziel & Fahrtstrecke [m] & Fahrtdauer [min] & Google-Maps Link\\ 
    \hline
    43 & Sulzheim & Schweinfurt & 19300 & 18 & \url{https://www.google.com/maps/dir/49.9469107,10.339796/50.0439484,10.2257843}\\ 
    \hline
    44 & Sulzheim & Gochsheim & 9900 & 11 & \url{https://www.google.com/maps/dir/49.9469107,10.339796/50.019526,10.2822383}\\ 
    \hline
    45 & Sulzheim & Sennfeld & 13900 & 16 & \url{https://www.google.com/maps/dir/49.9469107,10.339796/50.0422146,10.2609081}\\ 
    \hline
    46 & Sulzheim & Gerolzhofen & 6500 & 9 & \url{https://www.google.com/maps/dir/49.9469107,10.339796/49.9010511,10.3489622}\\ 
    \hline
    47 & Sulzheim & Kitzingen & 31900 & 31 & \url{https://www.google.com/maps/dir/49.9469107,10.339796/49.7355709,10.1617438}\\ 
    \hline
    48 & Sulzheim & Alitzheim & 1700 & 2 & \url{https://www.google.com/maps/dir/49.9469107,10.339796/49.9354067,10.3266909}\\ 
    \hline
\end{longtabu}

\begin{listing}[htbp]
    \begin{minted}{sql}
        SELECT 
        `potentials`.`id` AS `id`, 
        `from_places`.`name` AS `Quelle`,
        `to_places`.`name` AS `Ziel`, 
        `potentials`.`length` AS `Fahrtstrecke [m]`, 
        `potentials`.`miv-duration` AS `Fahrtdauer [min]`,
        CONCAT('https://www.google.com/maps/dir/', `from_places`.`LAT`, ",", `from_places`.`LONG`, '/', `to_places`.`LAT`, ',', `to_places`.`LONG`) AS `Google-Maps Link`
    FROM `potentials`
    LEFT JOIN `places` AS `from_places` ON `potentials`.`from_id` = `from_places`.`id`
    LEFT JOIN `places` AS `to_places` ON `potentials`.`to_id` = `to_places`.`id`
    WHERE `from_places`.`name` = 'Sulzheim'
    ORDER BY `potentials`.`id`;
    \end{minted}
    \caption{SQL-Abfrage der Fahrtstrecke, Fahrtdauer und des Google-Maps-Link mit der Quelle Sulzheim}\label{lst-f-sulzheim}
\end{listing}
                
                \subsection{Alitzheim}
                Zugeordnete Routen:
\newline
\newline
\begin{longtabu}{|l|l|l|l|*2{X[l]|}}
    \hline
    id & Quelle & Ziel & Straße & Straßenbeginn & Straßenende\\ 
    \hline
    49 & Alitzheim & Schweinfurt & St2272 & Alitzheim & B286/St2272 (bei Alitzheim)\\ 
    49 &  &  & B286 & B286/St2272 (bei Alitzheim) & B286/St2271 (bei Unterspießheim)\\ 
    49 &  &  & B286 & B286/St2271 (bei Unterspießheim) & B286/St2277 (bei Schwebheim)\\ 
    49 &  &  & B286 & B286/St2277 (bei Schwebheim) & B286/SW3/St2271 (nördlich Schwebheim)\\ 
    49 &  &  & B286 & B286/SW3/St2271 (nördlich Schwebheim) & Ausfahrt Schweinfurt-Zentrum\\ 
    49 &  &  & B286 & Ausfahrt Schweinfurt-Zentrum & B286 Schweinfurt Abfahrt Hans-Böckler-Straße\\ 
    49 &  &  & B286 & B286 Schweinfurt Abfahrt Hans-Böckler-Straße & B286/B26 (Schweinfurt Nähe Mainbrücke)\\ 
    49 &  &  & B286 & B286/B26 (Schweinfurt Nähe Mainbrücke) & B286/B303\\ 
    \hline
    50 & Alitzheim & Gochsheim & St2272 & Sulzheim & Alitzheim\\ 
    50 &  &  & St2272 & Grettstatt & Sulzheim\\ 
    50 &  &  & St2272 & St2272/St2277 (Kreisel südlich Gochsheim) & Grettstatt\\ 
    \hline
    51 & Alitzheim & Sennfeld & St2272 & Sulzheim & Alitzheim\\ 
    51 &  &  & St2272 & Grettstatt & Sulzheim\\ 
    51 &  &  & St2272 & St2272/St2277 (Kreisel südlich Gochsheim) & Grettstatt\\ 
    51 &  &  & St2272 & St2272/SW3 Gochsheim Kreisel Industriestraße & St2272/St2277 (Kreisel südlich Gochsheim)\\ 
    51 &  &  & SW3 & St2272/SW3 Gochsheim & St2272/SW3 Gochsheim Kreisel Industriestraße\\ 
    51 &  &  & St2272 & Ausfahrt Gochsheim & St2272/SW3 Gochsheim\\ 
    51 &  &  & St2272 & St2271/St2272 (bei Sennfeld) & Ausfahrt Gochsheim\\ 
    \hline
    52 & Alitzheim & Gerolzhofen & St2272 & Alitzheim & B286/St2272 (bei Alitzheim)\\ 
    52 &  &  & St2272 & B286/St2272 (bei Alitzheim) & Gerolzhofen\\ 
    \hline
    53 & Alitzheim & Kitzingen & SW40 & Herlheim & Alitzheim\\ 
    53 &  & Kitzingen & SW40 & SW39/SW40 (zw. Herlheim u Zeilitzheim) & Herlheim\\ 
    53 &  &  & SW39 & SW39/SW40 (zw. Herlheim u Zeilitzheim) & Kolitzheim\\ 
    53 &  &  & St2271 & Kolitzheim & Gaibach\\ 
    53 &  &  & St2271 & Gaibach & Volkach\\ 
    53 &  &  & St2271 & Volkach & St2271/KT57\\ 
    53 &  &  & St2271 & St2271/KT57 & Gerlachshausen\\ 
    53 &  &  & St2271 & Gerlachshausen & B22/St2271 (bei Stadtschwarzach)\\ 
    53 &  &  & St2271 & B22/St2271 (bei Stadtschwarzach) & Hörblach\\ 
    53 &  &  & St2271 & Hörblach & Ausfahrt Kitzingen-Schwarzach\\ 
    53 &  &  & St2271 & Ausfahrt Kitzingen-Schwarzach & St2271/St2272 (bei Kitzingen-Etwashausen)\\ 
    53 &  &  & St2271 & St2271/St2272 (bei Kitzingen-Etwashausen) & B8/St2271 (Kitzingen bei e-center)\\ 
    53 &  &  & B8 & Kitzingen & B8/St2271 (Kitzingen bei e-center)\\ 
    \hline
\end{longtabu}

\begin{listing}[htbp]
\begin{minted}{sql}
    SELECT 
	`potentials`.`id` AS `id`,
	`from_places`.`name` AS `Quelle`, 
	`to_places`.`name` AS `Ziel`,
	`streets`.`street` AS `Straße`,
	`from_street_places`.`name` AS `Straßenbeginn`,
   `to_street_places`.`name` AS `Straßenende`
FROM `potentials`
LEFT JOIN `places` AS `from_places` ON `potentials`.`from_id` = `from_places`.`id`
LEFT JOIN `places` AS `to_places` ON `potentials`.`to_id` = `to_places`.`id`
LEFT JOIN `routes` ON `routes`.`potential_id` = `potentials`.`id`
LEFT JOIN `streets` ON `streets`.`id` = `routes`.`street_id`
LEFT JOIN `places` AS `from_street_places` ON `streets`.`from_id` = `from_street_places`.`id`
LEFT JOIN `places` AS `to_street_places` ON `streets`.`to_id` = `to_street_places`.`id`
WHERE `from_places`.`name` = 'Alitzheim'
ORDER BY `potentials`.`id`, `routes`.`number_on_route`;
\end{minted}
\caption{SQL-Abfrage der zugeordneten Straßen mit der Quelle Alitzheim}\label{lst-rt-alitzheim}
\end{listing}


Länge, Fahrzeiten und Google Maps:
\newline
\begin{longtabu}{| l | *5{X[l]|}}
    \hline
    id & Quelle & Ziel & Fahrtstrecke [m] & Fahrtdauer [min] & Google-Maps Link\\ 
    \hline
    49 & Alitzheim & Schweinfurt & 18600 & 17 & \url{https://www.google.com/maps/dir/49.9354067,10.3266909/50.0439484,10.2257843}\\ 
    \hline
    50 & Alitzheim & Gochsheim & 11500 & 13 & \url{https://www.google.com/maps/dir/49.9354067,10.3266909/50.019526,10.2822383}\\ 
    \hline
    51 & Alitzheim & Sennfeld & 17600 & 16 & \url{https://www.google.com/maps/dir/49.9354067,10.3266909/50.0422146,10.2609081}\\ 
    \hline
    52 & Alitzheim & Gerolzhofen & 5700 & 7 & \url{https://www.google.com/maps/dir/49.9354067,10.3266909/49.9010511,10.3489622}\\ 
    \hline
    53 & Alitzheim & Kitzingen & 30200 & 29 & \url{https://www.google.com/maps/dir/49.9354067,10.3266909/49.7355709,10.1617438}\\ 
    \hline
\end{longtabu}

\begin{listing}[htbp]
    \begin{minted}{sql}
        SELECT 
        `potentials`.`id` AS `id`, 
        `from_places`.`name` AS `Quelle`,
        `to_places`.`name` AS `Ziel`, 
        `potentials`.`length` AS `Fahrtstrecke [m]`, 
        `potentials`.`miv-duration` AS `Fahrtdauer [min]`,
        CONCAT('https://www.google.com/maps/dir/', `from_places`.`LAT`, ",", `from_places`.`LONG`, '/', `to_places`.`LAT`, ',', `to_places`.`LONG`) AS `Google-Maps Link`
    FROM `potentials`
    LEFT JOIN `places` AS `from_places` ON `potentials`.`from_id` = `from_places`.`id`
    LEFT JOIN `places` AS `to_places` ON `potentials`.`to_id` = `to_places`.`id`
    WHERE `from_places`.`name` = 'Alitzheim'
    ORDER BY `potentials`.`id`;
    \end{minted}
    \caption{SQL-Abfrage der Fahrtstrecke, Fahrtdauer und des Google-Maps-Link mit der Quelle Alitzheim}\label{lst-f-alitzheim}
\end{listing}
                
                \subsection{Mönchstockheim}
                Zugeordnete Routen:
\newline
\newline
\begin{longtabu}{|l|l|l|l|*2{X[l]|}}
    \hline
    id & Quelle & Ziel & Straße & Straßenbeginn & Straßenende\\ 
    \hline
    54 & Mönchstockheim & Schweinfurt & SW53 & Mönchstockheim & B286/St2272 (bei Alitzheim)\\ 
    54 &  &  & B286 & B286/St2272 (bei Alitzheim) & B286/St2271 (bei Unterspießheim)\\ 
    54 &  &  & B286 & B286/St2271 (bei Unterspießheim) & B286/St2277 (bei Schwebheim)\\ 
    54 &  &  & B286 & B286/St2277 (bei Schwebheim) & B286/SW3/St2271 (nördlich Schwebheim)\\ 
    54 &  &  & B286 & B286/SW3/St2271 (nördlich Schwebheim) & Ausfahrt Schweinfurt-Zentrum\\ 
    54 &  &  & B286 & Ausfahrt Schweinfurt-Zentrum & B286 Schweinfurt Abfahrt Hans-Böckler-Straße\\ 
    54 &  &  & B286 & B286 Schweinfurt Abfahrt Hans-Böckler-Straße & B286/B26 (Schweinfurt Nähe Mainbrücke)\\ 
    54 &  &  & B286 & B286/B26 (Schweinfurt Nähe Mainbrücke) & B286/B303\\ 
    \hline
    55 & Mönchstockheim & Gochsheim & SW53 & Mönchstockheim & B286/St2272 (bei Alitzheim)\\ 
    55 &  &  & St2272 & Sulzheim & Alitzheim\\ 
    55 &  &  & St2272 & Grettstatt & Sulzheim\\ 
    55 &  &  & St2272 & St2272/St2277 (Kreisel südlich Gochsheim) & Grettstatt\\ 
    55 &  &  & St2272 & St2272/SW3 Gochsheim Kreisel Industriestraße & St2272/St2277 (Kreisel südlich Gochsheim)\\ 
    \hline
    56 & Mönchstockheim & Alitzheim & SW53 & Mönchstockheim & B286/St2272 (bei Alitzheim)\\ 
    56 &  &  & St2272 & Alitzheim & B286/St2272 (bei Alitzheim)\\ 
    \hline
\end{longtabu}

\begin{listing}[htbp]
\begin{minted}{sql}
    SELECT 
	`potentials`.`id` AS `id`,
	`from_places`.`name` AS `Quelle`, 
	`to_places`.`name` AS `Ziel`,
	`streets`.`street` AS `Straße`,
	`from_street_places`.`name` AS `Straßenbeginn`,
   `to_street_places`.`name` AS `Straßenende`
FROM `potentials`
LEFT JOIN `places` AS `from_places` ON `potentials`.`from_id` = `from_places`.`id`
LEFT JOIN `places` AS `to_places` ON `potentials`.`to_id` = `to_places`.`id`
LEFT JOIN `routes` ON `routes`.`potential_id` = `potentials`.`id`
LEFT JOIN `streets` ON `streets`.`id` = `routes`.`street_id`
LEFT JOIN `places` AS `from_street_places` ON `streets`.`from_id` = `from_street_places`.`id`
LEFT JOIN `places` AS `to_street_places` ON `streets`.`to_id` = `to_street_places`.`id`
WHERE `from_places`.`name` = 'Mönchstockheim'
ORDER BY `potentials`.`id`, `routes`.`number_on_route`;
\end{minted}
\caption{SQL-Abfrage der zugeordneten Straßen mit der Quelle Mönchstockheim}\label{lst-rt-moenchstockheim}
\end{listing}


Länge, Fahrzeiten und Google Maps:
\newline
\begin{longtabu}{| l | *5{X[l]|}}
    \hline
    id & Quelle & Ziel & Fahrtstrecke [m] & Fahrtdauer [min] & Google-Maps Link\\ 
    \hline
    54 & Mönchstockheim & Schweinfurt & 19600 & 18 & \url{https://www.google.com/maps/dir/49.9326596,10.3648351/50.0439484,10.2257843}\\ 
    \hline
    55 & Mönchstockheim & Gochsheim & 14000 & 15 & \url{https://www.google.com/maps/dir/49.9326596,10.3648351/50.019526,10.2822383}\\ 
    \hline
    56 & Mönchstockheim & Alitzheim & 3800 & 4 & \url{https://www.google.com/maps/dir/49.9326596,10.3648351/49.9354067,10.3266909}\\ 
    \hline
\end{longtabu}

\begin{listing}[htbp]
    \begin{minted}{sql}
        SELECT 
        `potentials`.`id` AS `id`, 
        `from_places`.`name` AS `Quelle`,
        `to_places`.`name` AS `Ziel`, 
        `potentials`.`length` AS `Fahrtstrecke [m]`, 
        `potentials`.`miv-duration` AS `Fahrtdauer [min]`,
        CONCAT('https://www.google.com/maps/dir/', `from_places`.`LAT`, ",", `from_places`.`LONG`, '/', `to_places`.`LAT`, ',', `to_places`.`LONG`) AS `Google-Maps Link`
    FROM `potentials`
    LEFT JOIN `places` AS `from_places` ON `potentials`.`from_id` = `from_places`.`id`
    LEFT JOIN `places` AS `to_places` ON `potentials`.`to_id` = `to_places`.`id`
    WHERE `from_places`.`name` = 'Mönchstockheim'
    ORDER BY `potentials`.`id`;
    \end{minted}
    \caption{SQL-Abfrage der Fahrtstrecke, Fahrtdauer und des Google-Maps-Link mit der Quelle Mönchstockheim}\label{lst-f-moenchstockheim}
\end{listing}
                
                \subsection{Vögnitz}
                Zugeordnete Routen:
\newline
\newline
\begin{longtabu}{|l|l|l|l|*2{X[l]|}}
    \hline
    id & Quelle & Ziel & Straße & Straßenbeginn & Straßenende\\ 
    \hline
    57 & Vögnitz & Schweinfurt & SW53 & Mönchstockheim & Vögnitz\\ 
    57 &  &  & SW53 & Mönchstockheim & B286/St2272 (bei Alitzheim)\\ 
    57 &  &  & B286 & B286/St2272 (bei Alitzheim) & B286/St2271 (bei Unterspießheim)\\ 
    57 &  &  & B286 & B286/St2271 (bei Unterspießheim) & B286/St2277 (bei Schwebheim)\\ 
    57 &  &  & B286 & B286/St2277 (bei Schwebheim) & B286/SW3/St2271 (nördlich Schwebheim)\\ 
    57 &  &  & B286 & B286/SW3/St2271 (nördlich Schwebheim) & Ausfahrt Schweinfurt-Zentrum\\ 
    57 &  &  & B286 & Ausfahrt Schweinfurt-Zentrum & B286 Schweinfurt Abfahrt Hans-Böckler-Straße\\ 
    57 &  &  & B286 & B286 Schweinfurt Abfahrt Hans-Böckler-Straße & B286/B26 (Schweinfurt Nähe Mainbrücke)\\ 
    57 &  &  & B286 & B286/B26 (Schweinfurt Nähe Mainbrücke) & B286/B303\\ 
    \hline
    58 & Vögnitz & Alitzheim & SW53 & Mönchstockheim & Vögnitz\\ 
    58 &  &  & SW53 & Mönchstockheim & B286/St2272 (bei Alitzheim)\\ 
    58 &  &  & St2272 & Alitzheim & B286/St2272 (bei Alitzheim)\\ 
    \hline
\end{longtabu}

\begin{listing}[htbp]
\begin{minted}{sql}
    SELECT 
	`potentials`.`id` AS `id`,
	`from_places`.`name` AS `Quelle`, 
	`to_places`.`name` AS `Ziel`,
	`streets`.`street` AS `Straße`,
	`from_street_places`.`name` AS `Straßenbeginn`,
   `to_street_places`.`name` AS `Straßenende`
FROM `potentials`
LEFT JOIN `places` AS `from_places` ON `potentials`.`from_id` = `from_places`.`id`
LEFT JOIN `places` AS `to_places` ON `potentials`.`to_id` = `to_places`.`id`
LEFT JOIN `routes` ON `routes`.`potential_id` = `potentials`.`id`
LEFT JOIN `streets` ON `streets`.`id` = `routes`.`street_id`
LEFT JOIN `places` AS `from_street_places` ON `streets`.`from_id` = `from_street_places`.`id`
LEFT JOIN `places` AS `to_street_places` ON `streets`.`to_id` = `to_street_places`.`id`
WHERE `from_places`.`name` = 'Vögnitz'
ORDER BY `potentials`.`id`, `routes`.`number_on_route`;
\end{minted}
\caption{SQL-Abfrage der zugeordneten Straßen mit der Quelle Vögnitz}\label{lst-rt-voegnitz}
\end{listing}


Länge, Fahrzeiten und Google Maps:
\newline
\begin{longtabu}{| l | *5{X[l]|}}
    \hline
    id & Quelle & Ziel & Fahrtstrecke [m] & Fahrtdauer [min] & Google-Maps Link\\ 
    \hline
    57 & Vögnitz & Schweinfurt & 21800 & 21 & \url{https://www.google.com/maps/dir/49.9308299,10.3902545/50.0439484,10.2257843}\\ 
    \hline
    58 & Vögnitz & Alitzheim & 5900 & 7 & \url{https://www.google.com/maps/dir/49.9308299,10.3902545/49.9354067,10.3266909}\\ 
    \hline
\end{longtabu}

\begin{listing}[htbp]
    \begin{minted}{sql}
        SELECT 
        `potentials`.`id` AS `id`, 
        `from_places`.`name` AS `Quelle`,
        `to_places`.`name` AS `Ziel`, 
        `potentials`.`length` AS `Fahrtstrecke [m]`, 
        `potentials`.`miv-duration` AS `Fahrtdauer [min]`,
        CONCAT('https://www.google.com/maps/dir/', `from_places`.`LAT`, ",", `from_places`.`LONG`, '/', `to_places`.`LAT`, ',', `to_places`.`LONG`) AS `Google-Maps Link`
    FROM `potentials`
    LEFT JOIN `places` AS `from_places` ON `potentials`.`from_id` = `from_places`.`id`
    LEFT JOIN `places` AS `to_places` ON `potentials`.`to_id` = `to_places`.`id`
    WHERE `from_places`.`name` = 'Vögnitz'
    ORDER BY `potentials`.`id`;
    \end{minted}
    \caption{SQL-Abfrage der Fahrtstrecke, Fahrtdauer und des Google-Maps-Link mit der Quelle Vögnitz}\label{lst-f-voegnitz}
\end{listing}
                
                \subsection{Kolitzheim}
                Berücksichtigt wurden nur die Ortsteile der Gemeinde, für die eine Benutzung der Schiene plausibel ist:\newline
                - Herlheim\newline
                - Oberspießheim\newline
                - Unterspießheim\newline
                - Zeilitzheim
                
                \subsection{Kolitzheim OT Herlheim}
                Zugeordnete Routen:
\newline
\newline
\begin{longtabu}{|l|l|l|l|*2{X[l]|}}
    \hline
    id & Quelle & Ziel & Straße & Straßenbeginn & Straßenende\\ 
    \hline
    59 & Herlheim & Schweinfurt & SW42 & Herlheim & Oberspießheim\\ 
    59 &  &  & SW42 & Unterspießheim & Oberspießheim\\ 
    59 &  &  & St2271 & B286/St2271 (bei Unterspießheim) & Unterspießheim\\ 
    59 &  &  & B286 & B286/St2271 (bei Unterspießheim) & B286/St2277 (bei Schwebheim)\\ 
    59 &  &  & B286 & B286/St2277 (bei Schwebheim) & B286/SW3/St2271 (nördlich Schwebheim)\\ 
    59 &  &  & B286 & B286/SW3/St2271 (nördlich Schwebheim) & Ausfahrt Schweinfurt-Zentrum\\ 
    59 &  &  & B286 & Ausfahrt Schweinfurt-Zentrum & B286 Schweinfurt Abfahrt Hans-Böckler-Straße\\ 
    59 &  &  & B286 & B286 Schweinfurt Abfahrt Hans-Böckler-Straße & B286/B26 (Schweinfurt Nähe Mainbrücke)\\ 
    59 &  &  & B286 & B286/B26 (Schweinfurt Nähe Mainbrücke) & B286/B303\\ 
    \hline
    60 & Herlheim & Gochsheim & SW40 & Herlheim & Alitzheim\\ 
    60 &  &  & St2272 & Sulzheim & Alitzheim\\ 
    60 &  &  & St2272 & Grettstatt & Sulzheim\\ 
    60 &  &  & St2272 & St2272/St2277 (Kreisel südlich Gochsheim) & Grettstatt\\ 
    \hline
    61 & Herlheim & Sennfeld & SW42 & Herlheim & Oberspießheim\\ 
    61 &  &  & SW42 & Unterspießheim & Oberspießheim\\ 
    61 &  &  & St2271 & B286/St2271 (bei Unterspießheim) & Unterspießheim\\ 
    61 &  &  & B286 & B286/St2271 (bei Unterspießheim) & B286/St2277 (bei Schwebheim)\\ 
    61 &  &  & B286 & B286/St2277 (bei Schwebheim) & B286/SW3/St2271 (nördlich Schwebheim)\\ 
    61 &  &  & St2271 & St2271/St2272 (bei Sennfeld) & B286/SW3/St2271 (nördlich Schwebheim)\\ 
    \hline
    62 & Herlheim & Alitzheim & SW40 & Herlheim & Alitzheim\\ 
    \hline
\end{longtabu}

\begin{listing}[htbp]
\begin{minted}{sql}
    SELECT 
	`potentials`.`id` AS `id`,
	`from_places`.`name` AS `Quelle`, 
	`to_places`.`name` AS `Ziel`,
	`streets`.`street` AS `Straße`,
	`from_street_places`.`name` AS `Straßenbeginn`,
   `to_street_places`.`name` AS `Straßenende`
FROM `potentials`
LEFT JOIN `places` AS `from_places` ON `potentials`.`from_id` = `from_places`.`id`
LEFT JOIN `places` AS `to_places` ON `potentials`.`to_id` = `to_places`.`id`
LEFT JOIN `routes` ON `routes`.`potential_id` = `potentials`.`id`
LEFT JOIN `streets` ON `streets`.`id` = `routes`.`street_id`
LEFT JOIN `places` AS `from_street_places` ON `streets`.`from_id` = `from_street_places`.`id`
LEFT JOIN `places` AS `to_street_places` ON `streets`.`to_id` = `to_street_places`.`id`
WHERE `from_places`.`name` = 'Herlheim'
ORDER BY `potentials`.`id`, `routes`.`number_on_route`;
\end{minted}
\caption{SQL-Abfrage der zugeordneten Straßen mit der Quelle Herlheim}\label{lst-rt-herlheim}
\end{listing}


Länge, Fahrzeiten und Google Maps:
\newline
\begin{longtabu}{| l | *5{X[l]|}}
    \hline
    id & Quelle & Ziel & Fahrtstrecke [m] & Fahrtdauer [min] & Google-Maps Link\\ 
    \hline
    59 & Herlheim & Schweinfurt & 17500 & 18 & \url{https://www.google.com/maps/dir/49.923343,10.2853985/50.0439484,10.2257843}\\ 
    \hline
    60 & Herlheim & Gochsheim & 14900 & 16 & \url{https://www.google.com/maps/dir/49.923343,10.2853985/50.019526,10.2822383}\\ 
    \hline
    61 & Herlheim & Sennfeld & 15200 & 16 & \url{https://www.google.com/maps/dir/49.923343,10.2853985/50.0422146,10.2609081}\\ 
    \hline
    62 & Herlheim & Alitzheim & 3300 & 3 & \url{https://www.google.com/maps/dir/49.923343,10.2853985/49.9354067,10.3266909}\\ 
    \hline
\end{longtabu}

\begin{listing}[htbp]
    \begin{minted}{sql}
        SELECT 
        `potentials`.`id` AS `id`, 
        `from_places`.`name` AS `Quelle`,
        `to_places`.`name` AS `Ziel`, 
        `potentials`.`length` AS `Fahrtstrecke [m]`, 
        `potentials`.`miv-duration` AS `Fahrtdauer [min]`,
        CONCAT('https://www.google.com/maps/dir/', `from_places`.`LAT`, ",", `from_places`.`LONG`, '/', `to_places`.`LAT`, ',', `to_places`.`LONG`) AS `Google-Maps Link`
    FROM `potentials`
    LEFT JOIN `places` AS `from_places` ON `potentials`.`from_id` = `from_places`.`id`
    LEFT JOIN `places` AS `to_places` ON `potentials`.`to_id` = `to_places`.`id`
    WHERE `from_places`.`name` = 'Herlheim'
    ORDER BY `potentials`.`id`;
    \end{minted}
    \caption{SQL-Abfrage der Fahrtstrecke, Fahrtdauer und des Google-Maps-Link mit der Quelle Herlheim}\label{lst-f-herlheim}
\end{listing}
                
                \subsection{Kolitzheim OT Oberspießheim}
                Zugeordnete Routen:
\newline
\newline
\begin{longtabu}{|l|l|l|l|*2{X[l]|}}
    \hline
    id & Quelle & Ziel & Straße & Straßenbeginn & Straßenende\\ 
    \hline
    63 & Oberspießheim & Kitzingen & SW42 & Unterspießheim & Oberspießheim\\ 
    63 &  &  & St2271 & Unterspießheim & Kolitzheim\\ 
    63 &  &  & St2271 & Kolitzheim & Gaibach\\ 
    63 &  &  & St2271 & Gaibach & Volkach\\ 
    63 &  &  & St2271 & Volkach & St2271/KT57\\ 
    63 &  &  & St2271 & St2271/KT57 & Gerlachshausen\\ 
    63 &  &  & St2271 & Gerlachshausen & B22/St2271 (bei Stadtschwarzach)\\ 
    63 &  &  & St2271 & B22/St2271 (bei Stadtschwarzach) & Hörblach\\ 
    63 &  &  & St2271 & Hörblach & Ausfahrt Kitzingen-Schwarzach\\ 
    63 &  &  & St2271 & Ausfahrt Kitzingen-Schwarzach & St2271/St2272 (bei Kitzingen-Etwashausen)\\ 
    63 &  &  & St2271 & St2271/St2272 (bei Kitzingen-Etwashausen) & B8/St2271 (Kitzingen bei e-center)\\ 
    63 &  &  & B8 & Kitzingen & B8/St2271 (Kitzingen bei e-center)\\ 
    \hline
    64 & Oberspießheim & Alitzheim & SW42 & Herlheim & Oberspießheim\\ 
    64 &  &  & SW40 & Herlheim & Alitzheim\\ 
    \hline
\end{longtabu}

\begin{listing}[htbp]
\begin{minted}{sql}
    SELECT 
	`potentials`.`id` AS `id`,
	`from_places`.`name` AS `Quelle`, 
	`to_places`.`name` AS `Ziel`,
	`streets`.`street` AS `Straße`,
	`from_street_places`.`name` AS `Straßenbeginn`,
   `to_street_places`.`name` AS `Straßenende`
FROM `potentials`
LEFT JOIN `places` AS `from_places` ON `potentials`.`from_id` = `from_places`.`id`
LEFT JOIN `places` AS `to_places` ON `potentials`.`to_id` = `to_places`.`id`
LEFT JOIN `routes` ON `routes`.`potential_id` = `potentials`.`id`
LEFT JOIN `streets` ON `streets`.`id` = `routes`.`street_id`
LEFT JOIN `places` AS `from_street_places` ON `streets`.`from_id` = `from_street_places`.`id`
LEFT JOIN `places` AS `to_street_places` ON `streets`.`to_id` = `to_street_places`.`id`
WHERE `from_places`.`name` = 'Oberspießheim'
ORDER BY `potentials`.`id`, `routes`.`number_on_route`;
\end{minted}
\caption{SQL-Abfrage der zugeordneten Straßen mit der Quelle Oberspießheim}\label{lst-rt-oberspiessheim}
\end{listing}


Länge, Fahrzeiten und Google Maps:
\newline
\begin{longtabu}{| l | *5{X[l]|}}
    \hline
    id & Quelle & Ziel & Fahrtstrecke [m] & Fahrtdauer [min] & Google-Maps Link\\ 
    \hline
    63 & Oberspießheim & Kitzingen & 29200 & 29 & \url{https://www.google.com/maps/dir/49.9472774,10.2758058/49.7355709,10.1617438}\\ 
    \hline
    64 & Oberspießheim & Alitzheim & 6300 & 6 & \url{https://www.google.com/maps/dir/49.9472774,10.2758058/49.9354067,10.3266909}\\ 
    \hline
\end{longtabu}

\begin{listing}[htbp]
    \begin{minted}{sql}
        SELECT 
        `potentials`.`id` AS `id`, 
        `from_places`.`name` AS `Quelle`,
        `to_places`.`name` AS `Ziel`, 
        `potentials`.`length` AS `Fahrtstrecke [m]`, 
        `potentials`.`miv-duration` AS `Fahrtdauer [min]`,
        CONCAT('https://www.google.com/maps/dir/', `from_places`.`LAT`, ",", `from_places`.`LONG`, '/', `to_places`.`LAT`, ',', `to_places`.`LONG`) AS `Google-Maps Link`
    FROM `potentials`
    LEFT JOIN `places` AS `from_places` ON `potentials`.`from_id` = `from_places`.`id`
    LEFT JOIN `places` AS `to_places` ON `potentials`.`to_id` = `to_places`.`id`
    WHERE `from_places`.`name` = 'Oberspießheim'
    ORDER BY `potentials`.`id`;
    \end{minted}
    \caption{SQL-Abfrage der Fahrtstrecke, Fahrtdauer und des Google-Maps-Link mit der Quelle Oberspießheim}\label{lst-f-oberspiessheim}
\end{listing}
                
                \subsection{Kolitzheim OT Unterspießheim}
                Zugeordnete Routen:
\newline
\newline
\begin{longtabu}{|l|l|l|l|*2{X[l]|}}
    \hline
    id & Quelle & Ziel & Straße & Straßenbeginn & Straßenende\\ 
    \hline
    65 & Unterspießheim & Lülsfeld & SW42 & Unterspießheim & Oberspießheim\\ 
    65 &  &  & SW42 & Herlheim & Oberspießheim\\ 
    65 &  &  & SW42 & Herlheim & Brünnstadt\\ 
    65 &  &  & SW42 & Brünnstadt & Frankenwinheim\\ 
    65 &  &  & SW44 & Lülsfeld & Frankenwinheim\\ 
    \hline
    66 & Unterspießheim & Wiesentheid & St2271 & B286/St2271 (bei Unterspießheim) & Unterspießheim\\ 
    66 &  &  & B286 & B286/St2272 (bei Alitzheim) & B286/St2271 (bei Unterspießheim)\\ 
    66 &  &  & B286 & B286/St2275 (Gerolzhofen bei Rügshofen) & B286/St2272 (bei Alitzheim)\\ 
    66 &  &  & B286 & B286/St2274 (Gerolzhofen bei Geomaris) & B286/St2275 (Gerolzhofen bei Rügshofen)\\ 
    66 &  &  & B286 & Prichsenstadt OT Neuses & B286/St2274 (Gerolzhofen bei Geomaris)\\ 
    66 &  &  & B286 & B286/St2272 (Wiesentheid bei Blutbank) & Prichsenstadt OT Neuses\\ 
    66 &  &  & St2272 & B286/St2272 (Wiesentheid bei Blutbank) & Wiesentheid\\ 
    \hline
    67 & Unterspießheim & Kitzingen & St2271 & Unterspießheim & Kolitzheim\\ 
    67 &  &  & St2271 & Kolitzheim & Gaibach\\ 
    67 &  &  & St2271 & Gaibach & Volkach\\ 
    67 &  &  & St2271 & Volkach & St2271/KT57\\ 
    67 &  &  & St2271 & St2271/KT57 & Gerlachshausen\\ 
    67 &  &  & St2271 & Gerlachshausen & B22/St2271 (bei Stadtschwarzach)\\ 
    67 &  &  & St2271 & B22/St2271 (bei Stadtschwarzach) & Hörblach\\ 
    67 &  &  & St2271 & Hörblach & Ausfahrt Kitzingen-Schwarzach\\ 
    67 &  &  & St2271 & Ausfahrt Kitzingen-Schwarzach & St2271/St2272 (bei Kitzingen-Etwashausen)\\ 
    67 &  &  & St2271 & St2271/St2272 (bei Kitzingen-Etwashausen) & B8/St2271 (Kitzingen bei e-center)\\ 
    67 &  &  & B8 & Kitzingen & B8/St2271 (Kitzingen bei e-center)\\ 
    \hline
    68 & Unterspießheim & Alitzheim & St2271 & B286/St2271 (bei Unterspießheim) & Unterspießheim\\ 
    68 &  &  & B286 & B286/St2272 (bei Alitzheim) & B286/St2271 (bei Unterspießheim)\\ 
    68 &  &  & St2272 & Alitzheim & B286/St2272 (bei Alitzheim)\\ 
    \hline
\end{longtabu}

\begin{listing}[htbp]
\begin{minted}{sql}
    SELECT 
	`potentials`.`id` AS `id`,
	`from_places`.`name` AS `Quelle`, 
	`to_places`.`name` AS `Ziel`,
	`streets`.`street` AS `Straße`,
	`from_street_places`.`name` AS `Straßenbeginn`,
   `to_street_places`.`name` AS `Straßenende`
FROM `potentials`
LEFT JOIN `places` AS `from_places` ON `potentials`.`from_id` = `from_places`.`id`
LEFT JOIN `places` AS `to_places` ON `potentials`.`to_id` = `to_places`.`id`
LEFT JOIN `routes` ON `routes`.`potential_id` = `potentials`.`id`
LEFT JOIN `streets` ON `streets`.`id` = `routes`.`street_id`
LEFT JOIN `places` AS `from_street_places` ON `streets`.`from_id` = `from_street_places`.`id`
LEFT JOIN `places` AS `to_street_places` ON `streets`.`to_id` = `to_street_places`.`id`
WHERE `from_places`.`name` = 'Unterspießheim'
ORDER BY `potentials`.`id`, `routes`.`number_on_route`;
\end{minted}
\caption{SQL-Abfrage der zugeordneten Straßen mit der Quelle Unterspießheim}\label{lst-rt-unterspiessheim}
\end{listing}


Länge, Fahrzeiten und Google Maps:
\newline
\begin{longtabu}{| l | *5{X[l]|}}
    \hline
    id & Quelle & Ziel & Fahrtstrecke [m] & Fahrtdauer [min] & Google-Maps Link\\ 
    \hline
    65 & Unterspießheim & Lülsfeld & 11600 & 12 & \url{https://www.google.com/maps/dir/49.9540768,10.2585303/49.8677403,10.3199678}\\ 
    \hline
    66 & Unterspießheim & Wiesentheid & 24100 & 21 & \url{https://www.google.com/maps/dir/49.9540768,10.2585303/49.7942401,10.3426344}\\ 
    \hline
    67 & Unterspießheim & Kitzingen & 28700 & 28 & \url{https://www.google.com/maps/dir/49.9540768,10.2585303/49.7355709,10.1617438}\\ 
    \hline
    68 & Unterspießheim & Alitzheim & 8800 & 8 & \url{https://www.google.com/maps/dir/49.9540768,10.2585303/49.9354067,10.3266909}\\ 
    \hline
\end{longtabu}

\begin{listing}[htbp]
    \begin{minted}{sql}
        SELECT 
        `potentials`.`id` AS `id`, 
        `from_places`.`name` AS `Quelle`,
        `to_places`.`name` AS `Ziel`, 
        `potentials`.`length` AS `Fahrtstrecke [m]`, 
        `potentials`.`miv-duration` AS `Fahrtdauer [min]`,
        CONCAT('https://www.google.com/maps/dir/', `from_places`.`LAT`, ",", `from_places`.`LONG`, '/', `to_places`.`LAT`, ',', `to_places`.`LONG`) AS `Google-Maps Link`
    FROM `potentials`
    LEFT JOIN `places` AS `from_places` ON `potentials`.`from_id` = `from_places`.`id`
    LEFT JOIN `places` AS `to_places` ON `potentials`.`to_id` = `to_places`.`id`
    WHERE `from_places`.`name` = 'Unterspießheim'
    ORDER BY `potentials`.`id`;
    \end{minted}
    \caption{SQL-Abfrage der Fahrtstrecke, Fahrtdauer und des Google-Maps-Link mit der Quelle Unterspießheim}\label{lst-f-unterspiessheim}
\end{listing}
                
                \subsection{Kolitzheim OT Zeilitzheim}
                Zugeordnete Routen:
\newline
\newline
\begin{longtabu}{|l|l|l|l|*2{X[l]|}}
    \hline
    id & Quelle & Ziel & Straße & Straßenbeginn & Straßenende\\ 
    \hline
    69 & Zeilitzheim & Wiesentheid & KT37 & Zeilitzheim & Krautheim\\ 
    69 &  &  & KT37 & Krautheim & Rimbach\\ 
    69 &  &  & KT37 & Rimbach & Eichfeld\\ 
    69 &  &  & St2260 & Eichfeld & Prichsenstadt OT Laub\\ 
    69 &  &  & KT45 & KT10/KT45 (südlich Laub) & Prichsenstadt OT Laub\\ 
    69 &  &  & KT10 & Wiesentheid & KT10/KT45 (südlich Laub)\\ 
    \hline
    70 & Zeilitzheim & Kitzingen & SW37 & Gaibach & Zeilitzheim\\ 
    70 &  &  & St2271 & Gaibach & Volkach\\ 
    70 &  &  & St2271 & Volkach & St2271/KT57\\ 
    70 &  &  & St2271 & St2271/KT57 & Gerlachshausen\\ 
    70 &  &  & St2271 & Gerlachshausen & B22/St2271 (bei Stadtschwarzach)\\ 
    70 &  &  & St2271 & B22/St2271 (bei Stadtschwarzach) & Hörblach\\ 
    70 &  &  & St2271 & Hörblach & Ausfahrt Kitzingen-Schwarzach\\ 
    70 &  &  & St2271 & Ausfahrt Kitzingen-Schwarzach & St2271/St2272 (bei Kitzingen-Etwashausen)\\ 
    70 &  &  & St2271 & St2271/St2272 (bei Kitzingen-Etwashausen) & B8/St2271 (Kitzingen bei e-center)\\ 
    70 &  &  & B8 & Kitzingen & B8/St2271 (Kitzingen bei e-center)\\ 
    \hline
    71 & Zeilitzheim & Alitzheim & SW40 & Zeilitzheim & SW39/SW40 (zw. Herlheim u Zeilitzheim)\\ 
    71 &  &  & SW40 & SW39/SW40 (zw. Herlheim u Zeilitzheim) & Herlheim\\ 
    71 &  &  & SW40 & Herlheim & Alitzheim\\        
    \hline
\end{longtabu}

\begin{listing}[htbp]
\begin{minted}{sql}
    SELECT 
	`potentials`.`id` AS `id`,
	`from_places`.`name` AS `Quelle`, 
	`to_places`.`name` AS `Ziel`,
	`streets`.`street` AS `Straße`,
	`from_street_places`.`name` AS `Straßenbeginn`,
   `to_street_places`.`name` AS `Straßenende`
FROM `potentials`
LEFT JOIN `places` AS `from_places` ON `potentials`.`from_id` = `from_places`.`id`
LEFT JOIN `places` AS `to_places` ON `potentials`.`to_id` = `to_places`.`id`
LEFT JOIN `routes` ON `routes`.`potential_id` = `potentials`.`id`
LEFT JOIN `streets` ON `streets`.`id` = `routes`.`street_id`
LEFT JOIN `places` AS `from_street_places` ON `streets`.`from_id` = `from_street_places`.`id`
LEFT JOIN `places` AS `to_street_places` ON `streets`.`to_id` = `to_street_places`.`id`
WHERE `from_places`.`name` = 'Zeilitzheim'
ORDER BY `potentials`.`id`, `routes`.`number_on_route`;
\end{minted}
\caption{SQL-Abfrage der zugeordneten Straßen mit der Quelle Zeilitzheim}\label{lst-rt-zeilitzheim}
\end{listing}


Länge, Fahrzeiten und Google Maps:
\newline
\begin{longtabu}{| l | *5{X[l]|}}
    \hline
    id & Quelle & Ziel & Fahrtstrecke [m] & Fahrtdauer [min] & Google-Maps Link\\ 
    \hline
    69 & Zeilitzheim & Wiesentheid & 14500 & 16 & \url{https://www.google.com/maps/dir/49.8980076,10.2680938/49.7942401,10.3426344}\\ 
    \hline
    70 & Zeilitzheim & Kitzingen & 23400 & 23 & \url{https://www.google.com/maps/dir/49.8980076,10.2680938/49.7355709,10.1617438}\\ 
    \hline
    71 & Zeilitzheim & Alitzheim & 6700 & 6 & \url{https://www.google.com/maps/dir/49.8980076,10.2680938/49.9354067,10.3266909}\\ 
    \hline
\end{longtabu}

\begin{listing}[htbp]
    \begin{minted}{sql}
        SELECT 
        `potentials`.`id` AS `id`, 
        `from_places`.`name` AS `Quelle`,
        `to_places`.`name` AS `Ziel`, 
        `potentials`.`length` AS `Fahrtstrecke [m]`, 
        `potentials`.`miv-duration` AS `Fahrtdauer [min]`,
        CONCAT('https://www.google.com/maps/dir/', `from_places`.`LAT`, ",", `from_places`.`LONG`, '/', `to_places`.`LAT`, ',', `to_places`.`LONG`) AS `Google-Maps Link`
    FROM `potentials`
    LEFT JOIN `places` AS `from_places` ON `potentials`.`from_id` = `from_places`.`id`
    LEFT JOIN `places` AS `to_places` ON `potentials`.`to_id` = `to_places`.`id`
    WHERE `from_places`.`name` = 'Zeilitzheim'
    ORDER BY `potentials`.`id`;
    \end{minted}
    \caption{SQL-Abfrage der Fahrtstrecke, Fahrtdauer und des Google-Maps-Link mit der Quelle Zeilitzheim}\label{lst-f-zeilitzheim}
\end{listing}
                
                \subsection{Gerolzhofen}
                Zugeordnete Routen:
\newline
\newline
\begin{longtabu}{|l|l|l|l|*2{X[l]|}}
    \hline
    id & Quelle & Ziel & Straße & Straßenbeginn & Straßenende\\ 
    \hline
    72 & Gerolzhofen & Rottendorf & St2274 & Gerolzhofen & B286/St2274 (Gerolzhofen bei Geomaris)\\ 
    72 &  &  & B286 & Prichsenstadt OT Neuses & B286/St2274 (Gerolzhofen bei Geomaris)\\ 
    72 &  &  & B286 & B286/St2272 (Wiesentheid bei Blutbank) & Prichsenstadt OT Neuses\\ 
    72 &  &  & B286 & Wiesentheid Kreuzung Gewerbegebiet Althölzl & B286/St2272 (Wiesentheid bei Blutbank)\\ 
    72 &  &  & B286 & Ausfahrt Wiesentheid & Wiesentheid Kreuzung Gewerbegebiet Althölzl\\ 
    72 &  &  & A3 & Ausfahrt Kitzingen-Schwarzach & Ausfahrt Wiesentheid\\ 
    72 &  &  & A3 & Autobahnkreuz Biebelried & Ausfahrt Kitzingen-Schwarzach\\ 
    72 &  &  & A3 & Ausfahrt Würzburg-Biebelried & Autobahnkreuz Biebelried\\ 
    72 &  &  & B8 & Ausfahrt Würzburg-Biebelried & Ausfahrt Biebelried-West\\ 
    72 &  &  & B8 & B8/B22 (am Mainfrankenpark) & Ausfahrt Würzburg-Biebelried\\ 
    72 &  &  & B8 & Ausfahrt Rottendorf & B8/B22 (am Mainfrankenpark)\\ 
    72 &  &  & B8 & Ausfahrt Wöllriederhof & Ausfahrt Rottendorf\\ 
    72 &  &  & B8 & Ausfahrt Würzburg/Nürnberger Straße & Ausfahrt Wöllriederhof\\ 
    72 &  &  & B8 & Würzburg B8/B19 (Grainbergknoten) & Ausfahrt Würzburg/Nürnberger Straße\\ 
    \hline
    73 & Gerolzhofen & Würzburg & St2274 & St2274/SW37 & Gerolzhofen\\ 
    73 &  &  & St2274 & Frankenwinheim & St2274/SW37\\ 
    73 &  &  & St2274 & Krautheim & Frankenwinheim\\ 
    73 &  &  & St2274 & Obervolkach & Krautheim\\ 
    73 &  &  & St2274 & Volkach & Obervolkach\\ 
    73 &  &  & St2260 & Astheim & Volkach\\ 
    73 &  &  & St2260 & St2260/KT30 & Astheim\\ 
    73 &  &  & St2260 & St2260/WÜ4 & St2260/KT30\\ 
    73 &  &  & St2260 & Prosselsheim & St2260/WÜ4\\ 
    73 &  &  & St2260 & Seligenstadt & Prosselsheim\\ 
    73 &  &  & St2260 & St2260/WÜ5 & Seligenstadt\\ 
    73 &  &  & St2260 & St2260/WÜ2 (östlich von Kürnach) & St2260/WÜ5\\ 
    73 &  &  & St2260 & St2260/WÜ26 & St2260/WÜ2 (östlich von Kürnach)\\ 
    73 &  &  & St2260 & B19/St2260 & St2260/WÜ26\\ 
    73 &  &  & B19 & B19/St2260 & Unterpleichfeld\\ 
    73 &  &  & B19 & Ausfahrt Würzburg-Estenfeld & B19/St2260\\ 
    73 &  &  & B19 & B19 Ausfahrt Estenfeld Ost & Ausfahrt Würzburg-Estenfeld\\ 
    73 &  &  & B19 & B19/WÜ8 & B19 Ausfahrt Estenfeld Ost\\ 
    73 &  &  & B19 & Würzburg B19 Ikea & B19/WÜ8\\ 
    73 &  &  & B19 & Würzburg B19 Lengfeld & Würzburg B19 Ikea\\ 
    73 &  &  & B19 & Würzburg B8/B19 (Grainbergknoten) & Würzburg B19 Lengfeld\\ 
    \hline
    74 & Gerolzhofen & Haßfurt & St2275 & Gerolzhofen & B286/St2275 (Gerolzhofen bei Rügshofen)\\ 
    74 &  &  & St2275 & B286/St2275 (Gerolzhofen bei Rügshofen) & Mönchstockheim\\ 
    74 &  &  & St2275 & Mönchstockheim & St2275/SW40\\ 
    74 &  &  & St2275 & St2275/SW40 & ST2275/SW54\\ 
    74 &  &  & St2275 & ST2275/SW54 & Donnersdorf\\ 
    74 &  &  & St2275 & Donnersdorf & Dampfach\\ 
    74 &  &  & St2275 & Dampfach & St2275/St2426\\ 
    74 &  &  & St2275 & St2275/St2426 & Seinsfeld\\ 
    74 &  &  & St2275 & Steinsfeld & Wonfurt\\ 
    74 &  &  & St2275 & Wonfurt & St2275/St2276\\ 
    74 &  &  & St2276 & St2275/St2276 & Haßfurt\\ 
    \hline
    75 & Gerolzhofen & Bad Kissingen & St2275 & Gerolzhofen & B286/St2275 (Gerolzhofen bei Rügshofen)\\ 
    75 &  &  & B286 & B286/St2275 (Gerolzhofen bei Rügshofen) & B286/St2272 (bei Alitzheim)\\ 
    75 &  &  & B286 & B286/St2272 (bei Alitzheim) & B286/St2271 (bei Unterspießheim)\\ 
    75 &  &  & B286 & B286/St2271 (bei Unterspießheim) & B286/St2277 (bei Schwebheim)\\ 
    75 &  &  & B286 & B286/St2277 (bei Schwebheim) & B286/SW3/St2271 (nördlich Schwebheim)\\ 
    75 &  &  & B286 & B286/SW3/St2271 (nördlich Schwebheim) & Ausfahrt Schweinfurt-Zentrum\\ 
    75 &  &  & A70 & Ausfahrt Schweinfurt-Hafen & Ausfahrt Schweinfurt-Zentrum\\ 
    75 &  &  & A70 & Ausfahrt Schweinfurt-Bergrheinfeld & Ausfahrt Schweinfurt-Hafen\\ 
    75 &  &  & A70 & Autobahnkreuz Werntal & Ausfahrt Schweinfurt-Bergrheinfeld\\ 
    75 &  &  & A70 & Ausfahrt Werneck & Autobahnkreuz Werntal\\ 
    75 &  &  & A71 & Autobahnkreuz Werntal & Ausfahrt Schweinfurt-West\\ 
    75 &  &  & A71 & Ausfahrt Schweinfurt-West & Ausfahrt Poppenhausen\\ 
    75 &  &  & A71 & Ausfahrt Poppenhausen & Ausfahrt Bad Kissingen/Oerlenbach\\ 
    75 &  &  & B19 & Ausfahrt Bad Kissingen/Oerlenbach & B286/B19 (bei Oerlenbach)\\ 
    75 &  &  & B286 & B286/B19 (bei Oerlenbach) & Oerlenbach\\ 
    75 &  &  & B286 & Oerlenbach & B286/KG46\\ 
    75 &  &  & B286 & B286/KG46 & Arnshausen\\ 
    75 &  &  & B286 & Arnshausen & Bad Kissingen\\ 
    \hline
    76 & Gerolzhofen & Schweinfurt & St2275 & Gerolzhofen & B286/St2275 (Gerolzhofen bei Rügshofen)\\ 
    76 &  &  & B286 & B286/St2275 (Gerolzhofen bei Rügshofen) & B286/St2272 (bei Alitzheim)\\ 
    76 &  &  & B286 & B286/St2272 (bei Alitzheim) & B286/St2271 (bei Unterspießheim)\\ 
    76 &  &  & B286 & B286/St2271 (bei Unterspießheim) & B286/St2277 (bei Schwebheim)\\ 
    76 &  &  & B286 & B286/St2277 (bei Schwebheim) & B286/SW3/St2271 (nördlich Schwebheim)\\ 
    76 &  &  & B286 & B286/SW3/St2271 (nördlich Schwebheim) & Ausfahrt Schweinfurt-Zentrum\\ 
    76 &  &  & B286 & Ausfahrt Schweinfurt-Zentrum & B286 Schweinfurt Abfahrt Hans-Böckler-Straße\\ 
    76 &  &  & B286 & B286 Schweinfurt Abfahrt Hans-Böckler-Straße & B286/B26 (Schweinfurt Nähe Mainbrücke)\\ 
    76 &  &  & B286 & B286/B26 (Schweinfurt Nähe Mainbrücke) & B286/B303\\ 
    \hline
    77 & Gerolzhofen & Sennfeld & St2275 & Gerolzhofen & B286/St2275 (Gerolzhofen bei Rügshofen)\\ 
    77 &  &  & B286 & B286/St2275 (Gerolzhofen bei Rügshofen) & B286/St2272 (bei Alitzheim)\\ 
    77 &  &  & B286 & B286/St2272 (bei Alitzheim) & B286/St2271 (bei Unterspießheim)\\ 
    77 &  &  & B286 & B286/St2271 (bei Unterspießheim) & B286/St2277 (bei Schwebheim)\\ 
    77 &  &  & B286 & B286/St2277 (bei Schwebheim) & B286/SW3/St2271 (nördlich Schwebheim)\\ 
    77 &  &  & St2271 & St2271/St2272 (bei Sennfeld) & B286/SW3/St2271 (nördlich Schwebheim)\\ 
    \hline
    78 & Gerolzhofen & Gochsheim & St2272 & B286/St2272 (bei Alitzheim) & Gerolzhofen\\ 
    78 &  &  & St2272 & Alitzheim & B286/St2272 (bei Alitzheim)\\ 
    78 &  &  & St2272 & Sulzheim & Alitzheim\\ 
    78 &  &  & St2272 & Grettstatt & Sulzheim\\ 
    78 &  &  & St2272 & St2272/St2277 (Kreisel südlich Gochsheim) & Grettstatt\\ 
    \hline
    79 & Gerolzhofen & Lülsfeld & St2274 & St2274/SW37 & Gerolzhofen\\ 
    79 &  &  & St2274 & Frankenwinheim & St2274/SW37\\ 
    79 &  &  & SW44 & Lülsfeld & Frankenwinheim\\ 
    \hline
    80 & Gerolzhofen & Prichsenstadt & St2274 & Gerolzhofen & B286/St2274 (Gerolzhofen bei Geomaris)\\ 
    80 &  &  & B286 & Prichsenstadt OT Neuses & B286/St2274 (Gerolzhofen bei Geomaris)\\ 
    80 &  &  & St2420 & Prichsenstadt & Prichsenstadt OT Neuses\\ 
    \hline
    81 & Gerolzhofen & Wiesentheid & St2274 & Gerolzhofen & B286/St2274 (Gerolzhofen bei Geomaris)\\ 
    81 &  &  & B286 & Prichsenstadt OT Neuses & B286/St2274 (Gerolzhofen bei Geomaris)\\ 
    81 &  &  & B286 & B286/St2272 (Wiesentheid bei Blutbank) & Prichsenstadt OT Neuses\\ 
    81 &  &  & St2272 & B286/St2272 (Wiesentheid bei Blutbank) & Wiesentheid\\ 
    \hline
    82 & Gerolzhofen & Kitzingen & St2274 & Gerolzhofen & B286/St2274 (Gerolzhofen bei Geomaris)\\ 
    82 &  &  & B286 & Prichsenstadt OT Neuses & B286/St2274 (Gerolzhofen bei Geomaris)\\ 
    82 &  &  & B22 & Stadelschwarzach & Prichsenstadt OT Neuses\\ 
    82 &  &  & B22 & Prichsenstadt OT Laub & Stadelschwarzach\\ 
    82 &  &  & B22 & Wiesentheid OT Reupelsdorf & Prichsenstadt OT Laub\\ 
    82 &  &  & B22 & Düllstadt & Wiesentheid OT Reupelsdorf\\ 
    82 &  &  & B22 & B22/KT11 & Düllstadt\\ 
    82 &  &  & B22 & B22/St2271 (bei Stadtschwarzach) & B22/KT11\\ 
    82 &  &  & St2271 & B22/St2271 (bei Stadtschwarzach) & Hörblach\\ 
    82 &  &  & St2271 & Hörblach & Ausfahrt Kitzingen-Schwarzach\\ 
    82 &  &  & St2271 & Ausfahrt Kitzingen-Schwarzach & St2271/St2272 (bei Kitzingen-Etwashausen)\\ 
    82 &  &  & St2271 & St2271/St2272 (bei Kitzingen-Etwashausen) & B8/St2271 (Kitzingen bei e-center)\\ 
    82 &  &  & St2271 & B8/St2271 (Kitzingen bei e-center) & Hohenfeld\\ 
    \hline
\end{longtabu}

\begin{listing}[htbp]
\begin{minted}{sql}
    SELECT 
	`potentials`.`id` AS `id`,
	`from_places`.`name` AS `Quelle`, 
	`to_places`.`name` AS `Ziel`,
	`streets`.`street` AS `Straße`,
	`from_street_places`.`name` AS `Straßenbeginn`,
   `to_street_places`.`name` AS `Straßenende`
FROM `potentials`
LEFT JOIN `places` AS `from_places` ON `potentials`.`from_id` = `from_places`.`id`
LEFT JOIN `places` AS `to_places` ON `potentials`.`to_id` = `to_places`.`id`
LEFT JOIN `routes` ON `routes`.`potential_id` = `potentials`.`id`
LEFT JOIN `streets` ON `streets`.`id` = `routes`.`street_id`
LEFT JOIN `places` AS `from_street_places` ON `streets`.`from_id` = `from_street_places`.`id`
LEFT JOIN `places` AS `to_street_places` ON `streets`.`to_id` = `to_street_places`.`id`
WHERE `from_places`.`name` = 'Gerolzhofen'
ORDER BY `potentials`.`id`, `routes`.`number_on_route`;
\end{minted}
\caption{SQL-Abfrage der zugeordneten Straßen mit der Quelle Gerolzhofen}\label{lst-rt-gerolzhofen}
\end{listing}


Länge, Fahrzeiten und Google Maps:
\newline
\begin{longtabu}{| l | *5{X[l]|}}
    \hline
    id & Quelle & Ziel & Fahrtstrecke [m] & Fahrtdauer [min] & Google-Maps Link\\ 
    \hline
    72 & Gerolzhofen & Rottendorf & 41700 & 34 & \url{https://www.google.com/maps/dir/49.9010511,10.3489622/49.790427,10.0258189}\\ 
    \hline
    73 & Gerolzhofen & Würzburg & 40000 & 41 & \url{https://www.google.com/maps/dir/49.9010511,10.3489622/49.7931,9.9280108}\\ 
    \hline
    74 & Gerolzhofen & Haßfurt & 20700 & 23 & \url{https://www.google.com/maps/dir/49.9010511,10.3489622/50.0313932,10.5068495}\\ 
    \hline
    75 & Gerolzhofen & Bad Kissingen & 50900 & 45 & \url{https://www.google.com/maps/dir/49.9010511,10.3489622/50.1990369,10.0762182}\\ 
    \hline
    76 & Gerolzhofen & Schweinfurt & 21700 & 21 & \url{https://www.google.com/maps/dir/49.9010511,10.3489622/50.0439484,10.2257843}\\ 
    \hline
    77 & Gerolzhofen & Sennfeld & 20700 & 20 & \url{https://www.google.com/maps/dir/49.9010511,10.3489622/50.0422146,10.2609081}\\ 
    \hline
    78 & Gerolzhofen & Gochsheim & 15700 & 18 & \url{https://www.google.com/maps/dir/49.9010511,10.3489622/50.019526,10.2822383}\\ 
    \hline
    79 & Gerolzhofen & Lülsfeld & 6500 & 9 & \url{https://www.google.com/maps/dir/49.9010511,10.3489622/49.8677403,10.3199678}\\ 
    \hline
    80 & Gerolzhofen & Prichsenstadt & 10300 & 13 & \url{https://www.google.com/maps/dir/49.9010511,10.3489622/49.8176258,10.3528515}\\ 
    \hline
    81 & Gerolzhofen & Wiesentheid & 14600 & 16 & \url{https://www.google.com/maps/dir/49.9010511,10.3489622/49.7942401,10.3426344}\\ 
    \hline
    82 & Gerolzhofen & Kitzingen & 29500 & 29 & \url{https://www.google.com/maps/dir/49.9010511,10.3489622/49.7355709,10.1617438}\\ 
    \hline
\end{longtabu}

\begin{listing}[htbp]
    \begin{minted}{sql}
        SELECT 
        `potentials`.`id` AS `id`, 
        `from_places`.`name` AS `Quelle`,
        `to_places`.`name` AS `Ziel`, 
        `potentials`.`length` AS `Fahrtstrecke [m]`, 
        `potentials`.`miv-duration` AS `Fahrtdauer [min]`,
        CONCAT('https://www.google.com/maps/dir/', `from_places`.`LAT`, ",", `from_places`.`LONG`, '/', `to_places`.`LAT`, ',', `to_places`.`LONG`) AS `Google-Maps Link`
    FROM `potentials`
    LEFT JOIN `places` AS `from_places` ON `potentials`.`from_id` = `from_places`.`id`
    LEFT JOIN `places` AS `to_places` ON `potentials`.`to_id` = `to_places`.`id`
    WHERE `from_places`.`name` = 'Gerolzhofen'
    ORDER BY `potentials`.`id`;
    \end{minted}
    \caption{SQL-Abfrage der Fahrtstrecke, Fahrtdauer und des Google-Maps-Link mit der Quelle Gerolzhofen}\label{lst-f-gerolzhofen}
\end{listing}
                
                \subsection{Dingolshausen}
                Zugeordnete Routen:
\newline
\newline
\begin{longtabu}{|l|l|l|l|*2{X[l]|}}
    \hline
    id & Quelle & Ziel & Straße & Straßenbeginn & Straßenende\\ 
    \hline
    83 & Dingolshausen &  & St2274 & B286/St2274 (Gerolzhofen bei Geomaris) & Dingolshausen\\ 
    83 &  &  & B286 & Prichsenstadt OT Neuses & B286/St2274 (Gerolzhofen bei Geomaris)\\ 
    83 &  &  & B286 & B286/St2272 (Wiesentheid bei Blutbank) & Prichsenstadt OT Neuses\\ 
    83 &  &  & B286 & Wiesentheid Kreuzung Gewerbegebiet Althölzl & B286/St2272 (Wiesentheid bei Blutbank)\\ 
    83 &  &  & B286 & Ausfahrt Wiesentheid & Wiesentheid Kreuzung Gewerbegebiet Althölzl\\ 
    83 &  &  & A3 & Ausfahrt Kitzingen-Schwarzach & Ausfahrt Wiesentheid\\ 
    83 &  &  & A3 & Autobahnkreuz Biebelried & Ausfahrt Kitzingen-Schwarzach\\ 
    83 &  &  & A3 & Ausfahrt Würzburg-Biebelried & Autobahnkreuz Biebelried\\ 
    83 &  &  & B8 & B8/B22 (am Mainfrankenpark) & Ausfahrt Würzburg-Biebelried\\ 
    83 &  &  & B8 & Ausfahrt Rottendorf & B8/B22 (am Mainfrankenpark)\\ 
    83 &  &  & B8 & Ausfahrt Wöllriederhof & Ausfahrt Rottendorf\\ 
    83 &  &  & B8 & Ausfahrt Würzburg/Nürnberger Straße & Ausfahrt Wöllriederhof\\ 
    83 &  &  & B8 & Würzburg B8/B19 (Grainbergknoten) & Ausfahrt Würzburg/Nürnberger Straße\\ 
    \hline
    84 & Dingolshausen & Schweinfurt & St2274 & B286/St2274 (Gerolzhofen bei Geomaris) & Dingolshausen\\ 
    84 &  &  & B286 & B286/St2274 (Gerolzhofen bei Geomaris) & B286/St2275 (Gerolzhofen bei Rügshofen)\\ 
    84 &  &  & B286 & B286/St2275 (Gerolzhofen bei Rügshofen) & B286/St2272 (bei Alitzheim)\\ 
    84 &  &  & B286 & B286/St2272 (bei Alitzheim) & B286/St2271 (bei Unterspießheim)\\ 
    84 &  &  & B286 & B286/St2271 (bei Unterspießheim) & B286/St2277 (bei Schwebheim)\\ 
    84 &  &  & B286 & B286/St2277 (bei Schwebheim) & B286/SW3/St2271 (nördlich Schwebheim)\\ 
    84 &  &  & B286 & B286/SW3/St2271 (nördlich Schwebheim) & Ausfahrt Schweinfurt-Zentrum\\ 
    84 &  &  & B286 & Ausfahrt Schweinfurt-Zentrum & B286 Schweinfurt Abfahrt Hans-Böckler-Straße\\ 
    84 &  &  & B286 & B286 Schweinfurt Abfahrt Hans-Böckler-Straße & B286/B26 (Schweinfurt Nähe Mainbrücke)\\ 
    84 &  &  & B286 & B286/B26 (Schweinfurt Nähe Mainbrücke) & B286/B303\\ 
    \hline
    85 & Dingolshausen & Lülsfeld & St2274 & B286/St2274 (Gerolzhofen bei Geomaris) & Dingolshausen\\ 
    85 &  &  & St2274 & Gerolzhofen & B286/St2274 (Gerolzhofen bei Geomaris)\\ 
    85 &  &  & St2274 & St2274/SW37 & Gerolzhofen\\ 
    85 &  &  & St2274 & Frankenwinheim & St2274/SW37\\ 
    85 &  &  & SW44 & Lülsfeld & Frankenwinheim\\ 
    \hline
    86 & Dingolshausen & Kitzingen & St2274 & B286/St2274 (Gerolzhofen bei Geomaris) & Dingolshausen\\ 
    86 &  &  & B286 & Prichsenstadt OT Neuses & B286/St2274 (Gerolzhofen bei Geomaris)\\ 
    86 &  &  & B22 & Stadelschwarzach & Prichsenstadt OT Neuses\\ 
    86 &  &  & B22 & Prichsenstadt OT Laub & Stadelschwarzach\\ 
    86 &  &  & B22 & Wiesentheid OT Reupelsdorf & Prichsenstadt OT Laub\\ 
    86 &  &  & B22 & Düllstadt & Wiesentheid OT Reupelsdorf\\ 
    86 &  &  & B22 & B22/KT11 & Düllstadt\\ 
    86 &  &  & B22 & B22/St2271 (bei Stadtschwarzach) & B22/KT11\\ 
    86 &  &  & St2271 & B22/St2271 (bei Stadtschwarzach) & Hörblach\\ 
    86 &  &  & St2271 & Hörblach & Ausfahrt Kitzingen-Schwarzach\\ 
    86 &  &  & St2271 & Ausfahrt Kitzingen-Schwarzach & St2271/St2272 (bei Kitzingen-Etwashausen)\\ 
    86 &  &  & St2271 & St2271/St2272 (bei Kitzingen-Etwashausen) & B8/St2271 (Kitzingen bei e-center)\\ 
    86 &  &  & B8 & Kitzingen & B8/St2271 (Kitzingen bei e-center)\\ 
    \hline
    87 & Dingolshausen & Gerolzhofen & St2274 & B286/St2274 (Gerolzhofen bei Geomaris) & Dingolshausen\\ 
    87 &  &  & St2274 & Gerolzhofen & B286/St2274 (Gerolzhofen bei Geomaris)\\ 
    \hline
\end{longtabu}

\begin{listing}[htbp]
\begin{minted}{sql}
    SELECT 
	`potentials`.`id` AS `id`,
	`from_places`.`name` AS `Quelle`, 
	`to_places`.`name` AS `Ziel`,
	`streets`.`street` AS `Straße`,
	`from_street_places`.`name` AS `Straßenbeginn`,
   `to_street_places`.`name` AS `Straßenende`
FROM `potentials`
LEFT JOIN `places` AS `from_places` ON `potentials`.`from_id` = `from_places`.`id`
LEFT JOIN `places` AS `to_places` ON `potentials`.`to_id` = `to_places`.`id`
LEFT JOIN `routes` ON `routes`.`potential_id` = `potentials`.`id`
LEFT JOIN `streets` ON `streets`.`id` = `routes`.`street_id`
LEFT JOIN `places` AS `from_street_places` ON `streets`.`from_id` = `from_street_places`.`id`
LEFT JOIN `places` AS `to_street_places` ON `streets`.`to_id` = `to_street_places`.`id`
WHERE `from_places`.`name` = 'Dingolshausen'
ORDER BY `potentials`.`id`, `routes`.`number_on_route`;
\end{minted}
\caption{SQL-Abfrage der zugeordneten Straßen mit der Quelle Dingolshausen}\label{lst-rt-dingolshausen}
\end{listing}


Länge, Fahrzeiten und Google Maps:
\newline
\begin{longtabu}{| l | *5{X[l]|}}
    \hline
    id & Quelle & Ziel & Fahrtstrecke [m] & Fahrtdauer [min] & Google-Maps Link\\ 
    \hline
    83 & Dingolshausen & Würzburg & 46500 & 42 & \url{https://www.google.com/maps/dir/49.9061195,10.3878922/49.7931,9.9280108}\\ 
    \hline
    84 & Dingolshausen & Schweinfurt & 23400 & 22 & \url{https://www.google.com/maps/dir/49.9061195,10.3878922/50.0439484,10.2257843}\\ 
    \hline
    85 & Dingolshausen & Lülsfeld & 8300 & 10 & \url{https://www.google.com/maps/dir/49.9061195,10.3878922/49.8677403,10.3199678}\\ 
    \hline
    86 & Dingolshausen & Kitzingen & 30100 & 29 & \url{https://www.google.com/maps/dir/49.9061195,10.3878922/49.7355709,10.1617438}\\ 
    \hline
    87 & Dingolshausen & Gerolzhofen & 3100 & 5 & \url{https://www.google.com/maps/dir/49.9061195,10.3878922/49.9010511,10.3489622}\\ 
    \hline
\end{longtabu}

\begin{listing}[htbp]
    \begin{minted}{sql}
        SELECT 
        `potentials`.`id` AS `id`, 
        `from_places`.`name` AS `Quelle`,
        `to_places`.`name` AS `Ziel`, 
        `potentials`.`length` AS `Fahrtstrecke [m]`, 
        `potentials`.`miv-duration` AS `Fahrtdauer [min]`,
        CONCAT('https://www.google.com/maps/dir/', `from_places`.`LAT`, ",", `from_places`.`LONG`, '/', `to_places`.`LAT`, ',', `to_places`.`LONG`) AS `Google-Maps Link`
    FROM `potentials`
    LEFT JOIN `places` AS `from_places` ON `potentials`.`from_id` = `from_places`.`id`
    LEFT JOIN `places` AS `to_places` ON `potentials`.`to_id` = `to_places`.`id`
    WHERE `from_places`.`name` = 'Dingolshausen'
    ORDER BY `potentials`.`id`;
    \end{minted}
    \caption{SQL-Abfrage der Fahrtstrecke, Fahrtdauer und des Google-Maps-Link mit der Quelle Dingolshausen}\label{lst-f-dingolshausen}
\end{listing}
                
                \subsection{Dingolshausen OT Bischwind}
                Zugeordnete Routen:
\newline
\newline
\begin{longtabu}{|l|l|l|l|*2{X[l]|}}
    \hline
    id & Quelle & Ziel & Straße & Straßenbeginn & Straßenende\\ 
    \hline
    88 & Dingolshausen OT Bischwind & Würzburg & SW52 & Dingolshausen & Dingolshausen OT Bischwind\\ 
    88 &  &  & St2274 & B286/St2274 (Gerolzhofen bei Geomaris) & Dingolshausen\\ 
    88 &  &  & B286 & Prichsenstadt OT Neuses & B286/St2274 (Gerolzhofen bei Geomaris)\\ 
    88 &  &  & B286 & B286/St2272 (Wiesentheid bei Blutbank) & Prichsenstadt OT Neuses\\ 
    88 &  &  & B286 & Wiesentheid Kreuzung Gewerbegebiet Althölzl & B286/St2272 (Wiesentheid bei Blutbank)\\ 
    88 &  &  & B286 & Ausfahrt Wiesentheid & Wiesentheid Kreuzung Gewerbegebiet Althölzl\\ 
    88 &  &  & A3 & Ausfahrt Kitzingen-Schwarzach & Ausfahrt Wiesentheid\\ 
    88 &  &  & A3 & Autobahnkreuz Biebelried & Ausfahrt Kitzingen-Schwarzach\\ 
    88 &  &  & A3 & Ausfahrt Würzburg-Biebelried & Autobahnkreuz Biebelried\\ 
    88 &  &  & B8 & B8/B22 (am Mainfrankenpark) & Ausfahrt Würzburg-Biebelried\\ 
    88 &  &  & B8 & Ausfahrt Rottendorf & B8/B22 (am Mainfrankenpark)\\ 
    88 &  &  & B8 & Ausfahrt Wöllriederhof & Ausfahrt Rottendorf\\ 
    88 &  &  & B8 & Ausfahrt Würzburg/Nürnberger Straße & Ausfahrt Wöllriederhof\\ 
    88 &  &  & B8 & Würzburg B8/B19 (Grainbergknoten) & Ausfahrt Würzburg/Nürnberger Straße\\ 
    \hline
    89 & Dingolshausen OT Bischwind & Schweinfurt & SW53 & Vögnitz & Dingolshausen OT Bischwind\\ 
    89 &  &  & SW53 & Mönchstockheim & Vögnitz\\ 
    89 &  &  & SW53 & Mönchstockheim & B286/St2272 (bei Alitzheim)\\ 
    89 &  &  & B286 & B286/St2272 (bei Alitzheim) & B286/St2271 (bei Unterspießheim)\\ 
    89 &  &  & B286 & B286/St2271 (bei Unterspießheim) & B286/St2277 (bei Schwebheim)\\ 
    89 &  &  & B286 & B286/St2277 (bei Schwebheim) & B286/SW3/St2271 (nördlich Schwebheim)\\ 
    89 &  &  & B286 & B286/SW3/St2271 (nördlich Schwebheim) & Ausfahrt Schweinfurt-Zentrum\\ 
    89 &  &  & B286 & Ausfahrt Schweinfurt-Zentrum & B286 Schweinfurt Abfahrt Hans-Böckler-Straße\\ 
    89 &  &  & B286 & B286 Schweinfurt Abfahrt Hans-Böckler-Straße & B286/B26 (Schweinfurt Nähe Mainbrücke)\\ 
    89 &  &  & B286 & B286/B26 (Schweinfurt Nähe Mainbrücke) & B286/B303\\ 
    \hline
    91 & Dingolshausen OT Bischwind & Kitzingen & SW52 & Dingolshausen & Dingolshausen OT Bischwind\\ 
    91 &  &  & St2274 & B286/St2274 (Gerolzhofen bei Geomaris) & Dingolshausen\\ 
    91 &  &  & B286 & Prichsenstadt OT Neuses & B286/St2274 (Gerolzhofen bei Geomaris)\\ 
    91 &  &  & B22 & Stadelschwarzach & Prichsenstadt OT Neuses\\ 
    91 &  &  & B22 & Prichsenstadt OT Laub & Stadelschwarzach\\ 
    91 &  &  & B22 & Wiesentheid OT Reupelsdorf & Prichsenstadt OT Laub\\ 
    91 &  &  & B22 & Düllstadt & Wiesentheid OT Reupelsdorf\\ 
    91 &  &  & B22 & B22/KT11 & Düllstadt\\ 
    91 &  &  & B22 & B22/St2271 (bei Stadtschwarzach) & B22/KT11\\ 
    91 &  &  & St2271 & B22/St2271 (bei Stadtschwarzach) & Hörblach\\ 
    91 &  &  & St2271 & Hörblach & Ausfahrt Kitzingen-Schwarzach\\ 
    91 &  &  & St2271 & Ausfahrt Kitzingen-Schwarzach & St2271/St2272 (bei Kitzingen-Etwashausen)\\ 
    91 &  &  & St2271 & St2271/St2272 (bei Kitzingen-Etwashausen) & B8/St2271 (Kitzingen bei e-center)\\ 
    91 &  &  & B8 & Kitzingen & B8/St2271 (Kitzingen bei e-center)\\ 
    \hline
    92 & Dingolshausen OT Bischwind & Gerolzhofen & SW52 & Dingolshausen & Dingolshausen OT Bischwind\\ 
    92 &  &  & St2274 & B286/St2274 (Gerolzhofen bei Geomaris) & Dingolshausen\\ 
    \hline
\end{longtabu}

\begin{listing}[htbp]
\begin{minted}{sql}
    SELECT 
	`potentials`.`id` AS `id`,
	`from_places`.`name` AS `Quelle`, 
	`to_places`.`name` AS `Ziel`,
	`streets`.`street` AS `Straße`,
	`from_street_places`.`name` AS `Straßenbeginn`,
   `to_street_places`.`name` AS `Straßenende`
FROM `potentials`
LEFT JOIN `places` AS `from_places` ON `potentials`.`from_id` = `from_places`.`id`
LEFT JOIN `places` AS `to_places` ON `potentials`.`to_id` = `to_places`.`id`
LEFT JOIN `routes` ON `routes`.`potential_id` = `potentials`.`id`
LEFT JOIN `streets` ON `streets`.`id` = `routes`.`street_id`
LEFT JOIN `places` AS `from_street_places` ON `streets`.`from_id` = `from_street_places`.`id`
LEFT JOIN `places` AS `to_street_places` ON `streets`.`to_id` = `to_street_places`.`id`
WHERE `from_places`.`name` = 'Dingolshausen OT Bischwind'
ORDER BY `potentials`.`id`, `routes`.`number_on_route`;
\end{minted}
\caption{SQL-Abfrage der zugeordneten Straßen mit der Quelle Dingolshausen OT Bischwind}\label{lst-rt-bischwind}
\end{listing}


Länge, Fahrzeiten und Google Maps:
\newline
\begin{longtabu}{| l | *5{X[l]|}}
    \hline
    id & Quelle & Ziel & Fahrtstrecke [m] & Fahrtdauer [min] & Google-Maps Link\\ 
    \hline
    88 & Dingolshausen OT Bischwind & Würzburg & 48700 & 43 & \url{https://www.google.com/maps/dir/49.9238518,10.3970751/49.7931,9.9280108}\\ 
    \hline
    89 & Dingolshausen OT Bischwind & Schweinfurt & 22400 & 21 & \url{https://www.google.com/maps/dir/49.9238518,10.3970751/50.0439484,10.2257843}\\ 
    \hline
    91 & Dingolshausen OT Bischwind & Kitzingen & 32200 & 30 & \url{https://www.google.com/maps/dir/49.9238518,10.3970751/49.7355709,10.1617438}\\ 
    \hline
    92 & Dingolshausen OT Bischwind & Gerolzhofen & 5200 & 6 & \url{https://www.google.com/maps/dir/49.9238518,10.3970751/49.9010511,10.3489622}\\ 
    \hline
\end{longtabu}

\begin{listing}[htbp]
    \begin{minted}{sql}
        SELECT 
        `potentials`.`id` AS `id`, 
        `from_places`.`name` AS `Quelle`,
        `to_places`.`name` AS `Ziel`, 
        `potentials`.`length` AS `Fahrtstrecke [m]`, 
        `potentials`.`miv-duration` AS `Fahrtdauer [min]`,
        CONCAT('https://www.google.com/maps/dir/', `from_places`.`LAT`, ",", `from_places`.`LONG`, '/', `to_places`.`LAT`, ',', `to_places`.`LONG`) AS `Google-Maps Link`
    FROM `potentials`
    LEFT JOIN `places` AS `from_places` ON `potentials`.`from_id` = `from_places`.`id`
    LEFT JOIN `places` AS `to_places` ON `potentials`.`to_id` = `to_places`.`id`
    WHERE `from_places`.`name` = 'Dingolshausen OT Bischwind'
    ORDER BY `potentials`.`id`;
    \end{minted}
    \caption{SQL-Abfrage der Fahrtstrecke, Fahrtdauer und des Google-Maps-Link mit der Quelle Dingolshausen OT Bischwind}\label{lst-f-bischwind}
\end{listing}
                
                \subsection{Michelau}
                Zugeordnete Routen:
\newline
\newline
\begin{longtabu}{|l|l|l|l|*2{X[l]|}}
    \hline
    id & Quelle & Ziel & Straße & Straßenbeginn & Straßenende\\ 
    \hline
    93 & Michelau & Schweinfurt & St2274 & Dingolshausen & Michelau\\ 
    93 &  &  & St2274 & B286/St2274 (Gerolzhofen bei Geomaris) & Dingolshausen\\ 
    93 &  &  & B286 & B286/St2274 (Gerolzhofen bei Geomaris) & B286/St2275 (Gerolzhofen bei Rügshofen)\\ 
    93 &  &  & B286 & B286/St2275 (Gerolzhofen bei Rügshofen) & B286/St2272 (bei Alitzheim)\\ 
    93 &  &  & B286 & B286/St2272 (bei Alitzheim) & B286/St2271 (bei Unterspießheim)\\ 
    93 &  &  & B286 & B286/St2271 (bei Unterspießheim) & B286/St2277 (bei Schwebheim)\\ 
    93 &  &  & B286 & B286/St2277 (bei Schwebheim) & B286/SW3/St2271 (nördlich Schwebheim)\\ 
    93 &  &  & B286 & B286/SW3/St2271 (nördlich Schwebheim) & Ausfahrt Schweinfurt-Zentrum\\ 
    93 &  &  & B286 & Ausfahrt Schweinfurt-Zentrum & B286 Schweinfurt Abfahrt Hans-Böckler-Straße\\ 
    93 &  &  & B286 & B286 Schweinfurt Abfahrt Hans-Böckler-Straße & B286/B26 (Schweinfurt Nähe Mainbrücke)\\ 
    93 &  &  & B286 & B286/B26 (Schweinfurt Nähe Mainbrücke) & B286/B303\\ 
    \hline
    94 & Michelau & Gerolzhofen & St2274 & Dingolshausen & Michelau\\ 
    94 &  &  & St2274 & B286/St2274 (Gerolzhofen bei Geomaris) & Dingolshausen\\ 
    \hline
\end{longtabu}

\begin{listing}[htbp]
\begin{minted}{sql}
    SELECT 
	`potentials`.`id` AS `id`,
	`from_places`.`name` AS `Quelle`, 
	`to_places`.`name` AS `Ziel`,
	`streets`.`street` AS `Straße`,
	`from_street_places`.`name` AS `Straßenbeginn`,
   `to_street_places`.`name` AS `Straßenende`
FROM `potentials`
LEFT JOIN `places` AS `from_places` ON `potentials`.`from_id` = `from_places`.`id`
LEFT JOIN `places` AS `to_places` ON `potentials`.`to_id` = `to_places`.`id`
LEFT JOIN `routes` ON `routes`.`potential_id` = `potentials`.`id`
LEFT JOIN `streets` ON `streets`.`id` = `routes`.`street_id`
LEFT JOIN `places` AS `from_street_places` ON `streets`.`from_id` = `from_street_places`.`id`
LEFT JOIN `places` AS `to_street_places` ON `streets`.`to_id` = `to_street_places`.`id`
WHERE `from_places`.`name` = 'Michelau'
ORDER BY `potentials`.`id`, `routes`.`number_on_route`;
\end{minted}
\caption{SQL-Abfrage der zugeordneten Straßen mit der Quelle Michelau}\label{lst-rt-michelau}
\end{listing}


Länge, Fahrzeiten und Google Maps:
\newline
\begin{longtabu}{| l | *5{X[l]|}}
    \hline
    id & Quelle & Ziel & Fahrtstrecke [m] & Fahrtdauer [min] & Google-Maps Link\\ 
    \hline
    93 & Michelau & Schweinfurt & 26500 & 25 & \url{https://www.google.com/maps/dir/49.9064415,10.4300791/50.0439484,10.2257843}\\ 
    \hline
    94 & Michelau & Gerolzhofen & 6200 & 8 & \url{https://www.google.com/maps/dir/49.9064415,10.4300791/49.9010511,10.3489622}\\ 
    \hline
\end{longtabu}

\begin{listing}[htbp]
    \begin{minted}{sql}
        SELECT 
        `potentials`.`id` AS `id`, 
        `from_places`.`name` AS `Quelle`,
        `to_places`.`name` AS `Ziel`, 
        `potentials`.`length` AS `Fahrtstrecke [m]`, 
        `potentials`.`miv-duration` AS `Fahrtdauer [min]`,
        CONCAT('https://www.google.com/maps/dir/', `from_places`.`LAT`, ",", `from_places`.`LONG`, '/', `to_places`.`LAT`, ',', `to_places`.`LONG`) AS `Google-Maps Link`
    FROM `potentials`
    LEFT JOIN `places` AS `from_places` ON `potentials`.`from_id` = `from_places`.`id`
    LEFT JOIN `places` AS `to_places` ON `potentials`.`to_id` = `to_places`.`id`
    WHERE `from_places`.`name` = 'Michelau'
    ORDER BY `potentials`.`id`;
    \end{minted}
    \caption{SQL-Abfrage der Fahrtstrecke, Fahrtdauer und des Google-Maps-Link mit der Quelle Michelau}\label{lst-f-michelau}
\end{listing}
                
                \subsection{Frankenwinheim}
                Zugeordnete Routen:
\newline
\newline
\begin{longtabu}{|l|l|l|l|*2{X[l]|}}
    \hline
    id & Quelle & Ziel & Straße & Straßenbeginn & Straßenende\\ 
    \hline
    95 & Frankenwinheim & Würzburg & St2274 & Krautheim & Frankenwinheim\\ 
    95 &  &  & St2274 & Obervolkach & Krautheim\\ 
    95 &  &  & St2274 & Volkach & Obervolkach\\ 
    95 &  &  & St2271 & Volkach & St2271/KT57\\ 
    95 &  &  & St2271 & St2271/KT57 & Gerlachshausen\\ 
    95 &  &  & St2271 & Gerlachshausen & B22/St2271 (bei Stadtschwarzach)\\ 
    95 &  &  & St2271 & B22/St2271 (bei Stadtschwarzach) & Hörblach\\ 
    95 &  &  & St2271 & Hörblach & Ausfahrt Kitzingen-Schwarzach\\ 
    95 &  &  & A3 & Autobahnkreuz Biebelried & Ausfahrt Kitzingen-Schwarzach\\ 
    95 &  &  & A3 & Ausfahrt Würzburg-Biebelried & Autobahnkreuz Biebelried\\ 
    95 &  &  & B8 & B8/B22 (am Mainfrankenpark) & Ausfahrt Würzburg-Biebelried\\ 
    95 &  &  & B8 & Ausfahrt Rottendorf & B8/B22 (am Mainfrankenpark)\\ 
    95 &  &  & B8 & Ausfahrt Wöllriederhof & Ausfahrt Rottendorf\\ 
    95 &  &  & B8 & Ausfahrt Würzburg/Nürnberger Straße & Ausfahrt Wöllriederhof\\ 
    95 &  &  & B8 & Würzburg B8/B19 (Grainbergknoten) & Ausfahrt Würzburg/Nürnberger Straße\\ 
    \hline
    96 & Frankenwinheim & Schweinfurt & St2274 & Frankenwinheim & St2274/SW37\\ 
    96 &  &  & St2274 & St2274/SW37 & Gerolzhofen\\ 
    96 &  &  & WÜ4 & St2260/WÜ4 & Kaltenhausen\\ 
    96 &  &  & B286 & B286/St2275 (Gerolzhofen bei Rügshofen) & B286/St2272 (bei Alitzheim)\\ 
    96 &  &  & B286 & B286/St2272 (bei Alitzheim) & B286/St2271 (bei Unterspießheim)\\ 
    96 &  &  & B286 & B286/St2271 (bei Unterspießheim) & B286/St2277 (bei Schwebheim)\\ 
    96 &  &  & B286 & B286/St2277 (bei Schwebheim) & B286/SW3/St2271 (nördlich Schwebheim)\\ 
    96 &  &  & B286 & B286/SW3/St2271 (nördlich Schwebheim) & Ausfahrt Schweinfurt-Zentrum\\ 
    96 &  &  & B286 & Ausfahrt Schweinfurt-Zentrum & B286 Schweinfurt Abfahrt Hans-Böckler-Straße\\ 
    96 &  &  & B286 & B286 Schweinfurt Abfahrt Hans-Böckler-Straße & B286/B26 (Schweinfurt Nähe Mainbrücke)\\ 
    96 &  &  & B286 & B286/B26 (Schweinfurt Nähe Mainbrücke) & B286/B303\\ 
    \hline
    97 & Frankenwinheim & Kitzingen & St2274 & Krautheim & Frankenwinheim\\ 
    97 &  &  & St2274 & Obervolkach & Krautheim\\ 
    97 &  &  & St2274 & Volkach & Obervolkach\\ 
    97 &  &  & St2271 & Volkach & St2271/KT57\\ 
    97 &  &  & St2271 & St2271/KT57 & Gerlachshausen\\ 
    97 &  &  & St2271 & Gerlachshausen & B22/St2271 (bei Stadtschwarzach)\\ 
    97 &  &  & St2271 & B22/St2271 (bei Stadtschwarzach) & Hörblach\\ 
    97 &  &  & St2271 & Hörblach & Ausfahrt Kitzingen-Schwarzach\\ 
    97 &  &  & St2271 & Ausfahrt Kitzingen-Schwarzach & St2271/St2272 (bei Kitzingen-Etwashausen)\\ 
    97 &  &  & St2271 & St2271/St2272 (bei Kitzingen-Etwashausen) & B8/St2271 (Kitzingen bei e-center)\\ 
    97 &  &  & B8 & Kitzingen & B8/St2271 (Kitzingen bei e-center)\\ 
    \hline
    98 & Frankenwinheim & Gerolzhofen & St2274 & Frankenwinheim & St2274/SW37\\ 
    98 &  &  & St2274 & St2274/SW37 & Gerolzhofen\\ 
    \hline
    99 & Frankenwinheim & Lülsfeld & SW44 & Lülsfeld & Frankenwinheim\\ 
    \hline
\end{longtabu}

\begin{listing}[htbp]
\begin{minted}{sql}
    SELECT 
	`potentials`.`id` AS `id`,
	`from_places`.`name` AS `Quelle`, 
	`to_places`.`name` AS `Ziel`,
	`streets`.`street` AS `Straße`,
	`from_street_places`.`name` AS `Straßenbeginn`,
   `to_street_places`.`name` AS `Straßenende`
FROM `potentials`
LEFT JOIN `places` AS `from_places` ON `potentials`.`from_id` = `from_places`.`id`
LEFT JOIN `places` AS `to_places` ON `potentials`.`to_id` = `to_places`.`id`
LEFT JOIN `routes` ON `routes`.`potential_id` = `potentials`.`id`
LEFT JOIN `streets` ON `streets`.`id` = `routes`.`street_id`
LEFT JOIN `places` AS `from_street_places` ON `streets`.`from_id` = `from_street_places`.`id`
LEFT JOIN `places` AS `to_street_places` ON `streets`.`to_id` = `to_street_places`.`id`
WHERE `from_places`.`name` = 'Frankenwinheim'
ORDER BY `potentials`.`id`, `routes`.`number_on_route`;
\end{minted}
\caption{SQL-Abfrage der zugeordneten Straßen mit der Quelle Frankenwinheim}\label{lst-rt-frankenwinheim}
\end{listing}


Länge, Fahrzeiten und Google Maps:
\newline
\begin{longtabu}{| l | *5{X[l]|}}
    \hline
    id & Quelle & Ziel & Fahrtstrecke [m] & Fahrtdauer [min] & Google-Maps Link\\ 
    \hline
    95 & Frankenwinheim & Würzburg & 38700 & 39 & \url{https://www.google.com/maps/dir/49.8869303,10.3139214/49.7931,9.9280108}\\ 
    \hline
    96 & Frankenwinheim & Schweinfurt & 24500 & 24 & \url{https://www.google.com/maps/dir/49.8869303,10.3139214/50.0439484,10.2257843}\\ 
    \hline
    97 & Frankenwinheim & Kitzingen & 26000 & 26 & \url{https://www.google.com/maps/dir/49.8869303,10.3139214/49.7355709,10.1617438}\\ 
    \hline
    98 & Frankenwinheim & Gerolzhofen & 3700 & 6 & \url{https://www.google.com/maps/dir/49.8869303,10.3139214/49.9010511,10.3489622}\\ 
    \hline
    99 & Frankenwinheim & Lülsfeld & 2400 & 3 & \url{https://www.google.com/maps/dir/49.8869303,10.3139214/49.8677403,10.3199678}\\ 
    \hline
\end{longtabu}

\begin{listing}[htbp]
    \begin{minted}{sql}
        SELECT 
        `potentials`.`id` AS `id`, 
        `from_places`.`name` AS `Quelle`,
        `to_places`.`name` AS `Ziel`, 
        `potentials`.`length` AS `Fahrtstrecke [m]`, 
        `potentials`.`miv-duration` AS `Fahrtdauer [min]`,
        CONCAT('https://www.google.com/maps/dir/', `from_places`.`LAT`, ",", `from_places`.`LONG`, '/', `to_places`.`LAT`, ',', `to_places`.`LONG`) AS `Google-Maps Link`
    FROM `potentials`
    LEFT JOIN `places` AS `from_places` ON `potentials`.`from_id` = `from_places`.`id`
    LEFT JOIN `places` AS `to_places` ON `potentials`.`to_id` = `to_places`.`id`
    WHERE `from_places`.`name` = 'Frankenwinheim'
    ORDER BY `potentials`.`id`;
    \end{minted}
    \caption{SQL-Abfrage der Fahrtstrecke, Fahrtdauer und des Google-Maps-Link mit der Quelle Frankenwinheim}\label{lst-f-frankenwinheim}
\end{listing}
                
                \subsection{Oberschwarzach}
                Zugeordnete Routen:
\newline
\newline
\begin{longtabu}{|l|l|l|l|*2{X[l]|}}
    \hline
    id & Quelle & Ziel & Straße & Straßenbeginn & Straßenende\\ 
    \hline
    100 & Oberschwarzach & Schweinfurt & St2272 & Mutzenroth & Oberschwarzach\\ 
    100 &  &  & St2272 & Wiebelsberg & Mutzenroth\\ 
    100 &  &  & St2272 & Gerolzhofen & Wiebelsberg\\ 
    100 &  &  & St2274 & Gerolzhofen & B286/St2274 (Gerolzhofen bei Geomaris)\\ 
    100 &  &  & B286 & B286/St2274 (Gerolzhofen bei Geomaris) & B286/St2275 (Gerolzhofen bei Rügshofen)\\ 
    100 &  &  & B286 & B286/St2275 (Gerolzhofen bei Rügshofen) & B286/St2272 (bei Alitzheim)\\ 
    100 &  &  & B286 & B286/St2272 (bei Alitzheim) & B286/St2271 (bei Unterspießheim)\\ 
    100 &  &  & B286 & B286/St2271 (bei Unterspießheim) & B286/St2277 (bei Schwebheim)\\ 
    100 &  &  & B286 & B286/St2277 (bei Schwebheim) & B286/SW3/St2271 (nördlich Schwebheim)\\ 
    100 &  &  & B286 & B286/SW3/St2271 (nördlich Schwebheim) & Ausfahrt Schweinfurt-Zentrum\\ 
    100 &  &  & B286 & Ausfahrt Schweinfurt-Zentrum & B286 Schweinfurt Abfahrt Hans-Böckler-Straße\\ 
    100 &  &  & B286 & B286 Schweinfurt Abfahrt Hans-Böckler-Straße & B286/B26 (Schweinfurt Nähe Mainbrücke)\\ 
    100 &  &  & B286 & B286/B26 (Schweinfurt Nähe Mainbrücke) & B286/B303\\ 
    \hline
    101 & Oberschwarzach & Lülsfeld & SW47 & Oberschwarzach & Prichsenstadt OT Bimbach\\ 
    101 &  &  & KT39 & Prichsenstadt OT Brünnau & Järkendorf\\ 
    101 &  &  & SW44 & Järkendorf & Lülsfeld\\ 
    \hline
    102 & Oberschwarzach & Wiesentheid & SW48 & B22/St2272/SW48 & Oberschwarzach\\ 
    102 &  &  & B22 & B22/KT42 (bei Neudorf) & B22/St2272/SW48\\ 
    102 &  &  & B22 & Prichsenstadt OT Neuses & B22/KT42 (bei Neudorf)\\ 
    102 &  &  & B286 & B286/St2272 (Wiesentheid bei Blutbank) & Prichsenstadt OT Neuses\\ 
    102 &  &  & St2272 & B286/St2272 (Wiesentheid bei Blutbank) & Wiesentheid\\ 
    \hline
    103 & Oberschwarzach & Kitzingen & SW48 & B22/St2272/SW48 & Oberschwarzach\\ 
    103 &  &  & B22 & B22/KT42 (bei Neudorf) & B22/St2272/SW48\\ 
    103 &  &  & B22 & Prichsenstadt OT Neuses & B22/KT42 (bei Neudorf)\\ 
    103 &  &  & B22 & Stadelschwarzach & Prichsenstadt OT Neuses\\ 
    103 &  &  & B22 & Prichsenstadt OT Laub & Stadelschwarzach\\ 
    103 &  &  & B22 & Wiesentheid OT Reupelsdorf & Prichsenstadt OT Laub\\ 
    103 &  &  & B22 & Düllstadt & Wiesentheid OT Reupelsdorf\\ 
    103 &  &  & B22 & B22/KT11 & Düllstadt\\ 
    103 &  &  & B22 & B22/St2271 (bei Stadtschwarzach) & B22/KT11\\ 
    103 &  &  & St2271 & B22/St2271 (bei Stadtschwarzach) & Hörblach\\ 
    103 &  &  & St2271 & Hörblach & Ausfahrt Kitzingen-Schwarzach\\ 
    103 &  &  & St2271 & Ausfahrt Kitzingen-Schwarzach & St2271/St2272 (bei Kitzingen-Etwashausen)\\ 
    103 &  &  & St2271 & St2271/St2272 (bei Kitzingen-Etwashausen) & B8/St2271 (Kitzingen bei e-center)\\ 
    103 &  &  & B8 & Kitzingen & B8/St2271 (Kitzingen bei e-center)\\ 
    \hline
    104 & Oberschwarzach & Järkendorf & SW47 & Oberschwarzach & Prichsenstadt OT Bimbach\\ 
    104 &  &  & KT39 & Prichsenstadt OT Brünnau & Järkendorf\\ 
    \hline
    105 & Oberschwarzach & Gerolzhofen & St2272 & Mutzenroth & Oberschwarzach\\ 
    105 &  &  & St2272 & Wiebelsberg & Mutzenroth\\ 
    105 &  &  & St2272 & Gerolzhofen & Wiebelsberg\\ 
    \hline
\end{longtabu}

\begin{listing}[htbp]
\begin{minted}{sql}
    SELECT 
	`potentials`.`id` AS `id`,
	`from_places`.`name` AS `Quelle`, 
	`to_places`.`name` AS `Ziel`,
	`streets`.`street` AS `Straße`,
	`from_street_places`.`name` AS `Straßenbeginn`,
   `to_street_places`.`name` AS `Straßenende`
FROM `potentials`
LEFT JOIN `places` AS `from_places` ON `potentials`.`from_id` = `from_places`.`id`
LEFT JOIN `places` AS `to_places` ON `potentials`.`to_id` = `to_places`.`id`
LEFT JOIN `routes` ON `routes`.`potential_id` = `potentials`.`id`
LEFT JOIN `streets` ON `streets`.`id` = `routes`.`street_id`
LEFT JOIN `places` AS `from_street_places` ON `streets`.`from_id` = `from_street_places`.`id`
LEFT JOIN `places` AS `to_street_places` ON `streets`.`to_id` = `to_street_places`.`id`
WHERE `from_places`.`name` = 'Oberschwarzach'
ORDER BY `potentials`.`id`, `routes`.`number_on_route`;
\end{minted}
\caption{SQL-Abfrage der zugeordneten Straßen mit der Quelle Oberschwarzach}\label{lst-rt-oberschwarzach}
\end{listing}


Länge, Fahrzeiten und Google Maps:
\newline
\begin{longtabu}{| l | *5{X[l]|}}
    \hline
    id & Quelle & Ziel & Fahrtstrecke [m] & Fahrtdauer [min] & Google-Maps Link\\ 
    \hline
    100 & Oberschwarzach & Schweinfurt & 28400 & 26 & \url{https://www.google.com/maps/dir/49.8604457,10.4089168/50.0439484,10.2257843}\\ 
    \hline
    101 & Oberschwarzach & Lülsfeld & 8400 & 10 & \url{https://www.google.com/maps/dir/49.8604457,10.4089168/49.8677403,10.3199678}\\ 
    \hline
    102 & Oberschwarzach & Wiesentheid & 11200 & 11 & \url{https://www.google.com/maps/dir/49.8604457,10.4089168/49.7942401,10.3426344}\\ 
    \hline
    103 & Oberschwarzach & Kitzingen & 26500 & 25 & \url{https://www.google.com/maps/dir/49.8604457,10.4089168/49.7355709,10.1617438}\\ 
    \hline
    104 & Oberschwarzach & Järkendorf & 6400 & 7 & \url{https://www.google.com/maps/dir/49.8604457,10.4089168/49.8522178,10.3290937}\\ 
    \hline
    105 & Oberschwarzach & Gerolzhofen & 7300 & 9 & \url{https://www.google.com/maps/dir/49.8604457,10.4089168/49.9010511,10.3489622}\\ 
    \hline
\end{longtabu}

\begin{listing}[htbp]
    \begin{minted}{sql}
        SELECT 
        `potentials`.`id` AS `id`, 
        `from_places`.`name` AS `Quelle`,
        `to_places`.`name` AS `Ziel`, 
        `potentials`.`length` AS `Fahrtstrecke [m]`, 
        `potentials`.`miv-duration` AS `Fahrtdauer [min]`,
        CONCAT('https://www.google.com/maps/dir/', `from_places`.`LAT`, ",", `from_places`.`LONG`, '/', `to_places`.`LAT`, ',', `to_places`.`LONG`) AS `Google-Maps Link`
    FROM `potentials`
    LEFT JOIN `places` AS `from_places` ON `potentials`.`from_id` = `from_places`.`id`
    LEFT JOIN `places` AS `to_places` ON `potentials`.`to_id` = `to_places`.`id`
    WHERE `from_places`.`name` = 'Oberschwarzach'
    ORDER BY `potentials`.`id`;
    \end{minted}
    \caption{SQL-Abfrage der Fahrtstrecke, Fahrtdauer und des Google-Maps-Link mit der Quelle Oberschwarzach}\label{lst-f-oberschwarzach}
\end{listing}
                
                \subsection{Volkach}
                Zugeordnete Routen:
\newline
\newline
\begin{longtabu}{|l|l|l|l|*2{X[l]|}}
    \hline
    id & Quelle & Ziel & Straße & Straßenbeginn & Straßenende\\ 
    \hline
    106 & Volkach & Schweinfurt & St2271 & Gaibach & Volkach\\ 
    106 &  &  & St2271 & Kolitzheim & Gaibach\\ 
    106 &  &  & St2271 & Unterspießheim & Kolitzheim\\ 
    106 &  &  & St2271 & B286/St2271 (bei Unterspießheim) & Unterspießheim\\ 
    106 &  &  & B286 & B286/St2271 (bei Unterspießheim) & B286/St2277 (bei Schwebheim)\\ 
    106 &  &  & B286 & B286/St2277 (bei Schwebheim) & B286/SW3/St2271 (nördlich Schwebheim)\\ 
    106 &  &  & B286 & B286/SW3/St2271 (nördlich Schwebheim) & Ausfahrt Schweinfurt-Zentrum\\ 
    106 &  &  & B286 & Ausfahrt Schweinfurt-Zentrum & B286 Schweinfurt Abfahrt Hans-Böckler-Straße\\ 
    106 &  &  & B286 & B286 Schweinfurt Abfahrt Hans-Böckler-Straße & B286/B26 (Schweinfurt Nähe Mainbrücke)\\ 
    106 &  &  & B286 & B286/B26 (Schweinfurt Nähe Mainbrücke) & B286/B303\\ 
    \hline
    107 & Volkach & Kitzingen & St2271 & Volkach & St2271/KT57\\ 
    107 &  &  & St2271 & St2271/KT57 & Gerlachshausen\\ 
    107 &  &  & St2271 & Gerlachshausen & B22/St2271 (bei Stadtschwarzach)\\ 
    107 &  &  & St2271 & B22/St2271 (bei Stadtschwarzach) & Hörblach\\ 
    107 &  &  & St2271 & Hörblach & Ausfahrt Kitzingen-Schwarzach\\ 
    107 &  &  & St2271 & Ausfahrt Kitzingen-Schwarzach & St2271/St2272 (bei Kitzingen-Etwashausen)\\ 
    107 &  &  & St2271 & St2271/St2272 (bei Kitzingen-Etwashausen) & B8/St2271 (Kitzingen bei e-center)\\ 
    107 &  &  & St2271 & B8/St2271 (Kitzingen bei e-center) & Hohenfeld\\ 
    107 &  &  & B8 & Kitzingen & B8/St2271 (Kitzingen bei e-center)\\ 
    \hline
    108 & Volkach & Lülsfeld & KT36 & Volkach & Rimbach\\ 
    108 &  &  & SW43 & Rimbach & Lülsfeld\\ 
    \hline
    109 & Volkach & Gerolzhofen & St2274 & Volkach & Obervolkach\\ 
    109 &  &  & St2274 & Obervolkach & Krautheim\\ 
    109 &  &  & St2274 & Krautheim & Frankenwinheim\\ 
    109 &  &  & St2274 & Frankenwinheim & St2274/SW37\\ 
    109 &  &  & St2274 & St2274/SW37 & Gerolzhofen\\ 
    \hline
\end{longtabu}

\begin{listing}[htbp]
\begin{minted}{sql}
    SELECT 
	`potentials`.`id` AS `id`,
	`from_places`.`name` AS `Quelle`, 
	`to_places`.`name` AS `Ziel`,
	`streets`.`street` AS `Straße`,
	`from_street_places`.`name` AS `Straßenbeginn`,
   `to_street_places`.`name` AS `Straßenende`
FROM `potentials`
LEFT JOIN `places` AS `from_places` ON `potentials`.`from_id` = `from_places`.`id`
LEFT JOIN `places` AS `to_places` ON `potentials`.`to_id` = `to_places`.`id`
LEFT JOIN `routes` ON `routes`.`potential_id` = `potentials`.`id`
LEFT JOIN `streets` ON `streets`.`id` = `routes`.`street_id`
LEFT JOIN `places` AS `from_street_places` ON `streets`.`from_id` = `from_street_places`.`id`
LEFT JOIN `places` AS `to_street_places` ON `streets`.`to_id` = `to_street_places`.`id`
WHERE `from_places`.`name` = 'Volkach'
ORDER BY `potentials`.`id`, `routes`.`number_on_route`;
\end{minted}
\caption{SQL-Abfrage der zugeordneten Straßen mit der Quelle Volkach}\label{lst-rt-volkach}
\end{listing}


Länge, Fahrzeiten und Google Maps:
\newline
\begin{longtabu}{| l | *5{X[l]|}}
    \hline
    id & Quelle & Ziel & Fahrtstrecke [m] & Fahrtdauer [min] & Google-Maps Link\\ 
    \hline
    106 & Volkach & Schweinfurt & 23700 & 24 & \url{https://www.google.com/maps/dir/49.8658315,10.226397/50.0439484,10.2257843}\\ 
    \hline
    107 & Volkach & Kitzingen & 17700 & 18 & \url{https://www.google.com/maps/dir/49.8658315,10.226397/49.7355709,10.1617438}\\ 
    \hline
    108 & Volkach & Lülsfeld & 8900 & 9 & \url{https://www.google.com/maps/dir/49.8658315,10.226397/49.8677403,10.3199678}\\ 
    \hline
    109 & Volkach & Gerolzhofen & 12600 & 15 & \url{https://www.google.com/maps/dir/49.8658315,10.226397/49.9010511,10.3489622}\\ 
    \hline
\end{longtabu}

\begin{listing}[htbp]
    \begin{minted}{sql}
        SELECT 
        `potentials`.`id` AS `id`, 
        `from_places`.`name` AS `Quelle`,
        `to_places`.`name` AS `Ziel`, 
        `potentials`.`length` AS `Fahrtstrecke [m]`, 
        `potentials`.`miv-duration` AS `Fahrtdauer [min]`,
        CONCAT('https://www.google.com/maps/dir/', `from_places`.`LAT`, ",", `from_places`.`LONG`, '/', `to_places`.`LAT`, ',', `to_places`.`LONG`) AS `Google-Maps Link`
    FROM `potentials`
    LEFT JOIN `places` AS `from_places` ON `potentials`.`from_id` = `from_places`.`id`
    LEFT JOIN `places` AS `to_places` ON `potentials`.`to_id` = `to_places`.`id`
    WHERE `from_places`.`name` = 'Volkach'
    ORDER BY `potentials`.`id`;
    \end{minted}
    \caption{SQL-Abfrage der Fahrtstrecke, Fahrtdauer und des Google-Maps-Link mit der Quelle Volkach}\label{lst-f-volkach}
\end{listing}
                
                \subsection{Lülsfeld}
                Zugeordnete Routen:
\newline
\newline
\begin{longtabu}{|l|l|l|l|*2{X[l]|}}
    \hline
    id & Quelle & Ziel & Straße & Straßenbeginn & Straßenende\\ 
    \hline
    110 & Lülsfeld & Schweinfurt & SW44 & Lülsfeld & Frankenwinheim\\ 
    110 &  &  & St2274 & Frankenwinheim & St2274/SW37\\ 
    110 &  &  & St2274 & St2274/SW37 & Gerolzhofen\\ 
    110 &  &  & WÜ4 & St2260/WÜ4 & Kaltenhausen\\ 
    110 &  &  & B286 & B286/St2275 (Gerolzhofen bei Rügshofen) & B286/St2272 (bei Alitzheim)\\ 
    110 &  &  & B286 & B286/St2272 (bei Alitzheim) & B286/St2271 (bei Unterspießheim)\\ 
    110 &  &  & B286 & B286/St2271 (bei Unterspießheim) & B286/St2277 (bei Schwebheim)\\ 
    110 &  &  & B286 & B286/St2277 (bei Schwebheim) & B286/SW3/St2271 (nördlich Schwebheim)\\ 
    110 &  &  & B286 & B286/SW3/St2271 (nördlich Schwebheim) & Ausfahrt Schweinfurt-Zentrum\\ 
    110 &  &  & B286 & Ausfahrt Schweinfurt-Zentrum & B286 Schweinfurt Abfahrt Hans-Böckler-Straße\\ 
    110 &  &  & B286 & B286 Schweinfurt Abfahrt Hans-Böckler-Straße & B286/B26 (Schweinfurt Nähe Mainbrücke)\\ 
    110 &  &  & B286 & B286/B26 (Schweinfurt Nähe Mainbrücke) & B286/B303\\ 
    \hline
    111 & Lülsfeld & Gerolzhofen & SW44 & Lülsfeld & Frankenwinheim\\ 
    111 &  &  & St2274 & Frankenwinheim & St2274/SW37\\ 
    111 &  &  & St2274 & St2274/SW37 & Gerolzhofen\\ 
    \hline
    112 & Lülsfeld & Wiesentheid & SW43 & Rimbach & Lülsfeld\\ 
    112 &  &  & KT37 & Rimbach & Eichfeld\\ 
    112 &  &  & St2260 & Eichfeld & Prichsenstadt OT Laub\\ 
    112 &  &  & KT45 & KT10/KT45 (südlich Laub) & Prichsenstadt OT Laub\\ 
    112 &  &  & KT10 & Wiesentheid & KT10/KT45 (südlich Laub)\\ 
    \hline
    113 & Lülsfeld & Kitzingen & SW43 & Rimbach & Lülsfeld\\ 
    113 &  &  & KT36 & Volkach & Rimbach\\ 
    113 &  &  & St2271 & Volkach & St2271/KT57\\ 
    113 &  &  & St2271 & St2271/KT57 & Gerlachshausen\\ 
    113 &  &  & St2271 & Gerlachshausen & B22/St2271 (bei Stadtschwarzach)\\ 
    113 &  &  & St2271 & B22/St2271 (bei Stadtschwarzach) & Hörblach\\ 
    113 &  &  & St2271 & Hörblach & Ausfahrt Kitzingen-Schwarzach\\ 
    113 &  &  & St2271 & Ausfahrt Kitzingen-Schwarzach & St2271/St2272 (bei Kitzingen-Etwashausen)\\ 
    113 &  &  & St2271 & St2271/St2272 (bei Kitzingen-Etwashausen) & B8/St2271 (Kitzingen bei e-center)\\ 
    113 &  &  & B8 & Kitzingen & B8/St2271 (Kitzingen bei e-center)\\ 
    \hline
\end{longtabu}

\begin{listing}[htbp]
\begin{minted}{sql}
    SELECT 
	`potentials`.`id` AS `id`,
	`from_places`.`name` AS `Quelle`, 
	`to_places`.`name` AS `Ziel`,
	`streets`.`street` AS `Straße`,
	`from_street_places`.`name` AS `Straßenbeginn`,
   `to_street_places`.`name` AS `Straßenende`
FROM `potentials`
LEFT JOIN `places` AS `from_places` ON `potentials`.`from_id` = `from_places`.`id`
LEFT JOIN `places` AS `to_places` ON `potentials`.`to_id` = `to_places`.`id`
LEFT JOIN `routes` ON `routes`.`potential_id` = `potentials`.`id`
LEFT JOIN `streets` ON `streets`.`id` = `routes`.`street_id`
LEFT JOIN `places` AS `from_street_places` ON `streets`.`from_id` = `from_street_places`.`id`
LEFT JOIN `places` AS `to_street_places` ON `streets`.`to_id` = `to_street_places`.`id`
WHERE `from_places`.`name` = 'Lülsfeld'
ORDER BY `potentials`.`id`, `routes`.`number_on_route`;
\end{minted}
\caption{SQL-Abfrage der zugeordneten Straßen mit der Quelle Lülsfeld}\label{lst-rt-luelsfeld}
\end{listing}


Länge, Fahrzeiten und Google Maps:
\newline
\begin{longtabu}{| l | *5{X[l]|}}
    \hline
    id & Quelle & Ziel & Fahrtstrecke [m] & Fahrtdauer [min] & Google-Maps Link\\ 
    \hline
    110 & Lülsfeld & Schweinfurt & 26800 & 25 & \url{https://www.google.com/maps/dir/49.8677403,10.3199678/50.0439484,10.2257843}\\ 
    \hline
    111 & Lülsfeld & Gerolzhofen & 6200 & 8 & \url{https://www.google.com/maps/dir/49.8677403,10.3199678/49.9010511,10.3489622}\\ 
    \hline
    112 & Lülsfeld & Wiesentheid & 11200 & 11 & \url{https://www.google.com/maps/dir/49.8677403,10.3199678/49.7942401,10.3426344}\\ 
    \hline
    113 & Lülsfeld & Kitzingen & 23600 & 23 & \url{https://www.google.com/maps/dir/49.8677403,10.3199678/49.7355709,10.1617438}\\ 
    \hline
\end{longtabu}

\begin{listing}[htbp]
    \begin{minted}{sql}
        SELECT 
        `potentials`.`id` AS `id`, 
        `from_places`.`name` AS `Quelle`,
        `to_places`.`name` AS `Ziel`, 
        `potentials`.`length` AS `Fahrtstrecke [m]`, 
        `potentials`.`miv-duration` AS `Fahrtdauer [min]`,
        CONCAT('https://www.google.com/maps/dir/', `from_places`.`LAT`, ",", `from_places`.`LONG`, '/', `to_places`.`LAT`, ',', `to_places`.`LONG`) AS `Google-Maps Link`
    FROM `potentials`
    LEFT JOIN `places` AS `from_places` ON `potentials`.`from_id` = `from_places`.`id`
    LEFT JOIN `places` AS `to_places` ON `potentials`.`to_id` = `to_places`.`id`
    WHERE `from_places`.`name` = 'Lülsfeld'
    ORDER BY `potentials`.`id`;
    \end{minted}
    \caption{SQL-Abfrage der Fahrtstrecke, Fahrtdauer und des Google-Maps-Link mit der Quelle Lülsfeld}\label{lst-f-luelsfeld}
\end{listing}
                
                \subsection{Schallfeld}
                Zugeordnete Routen:
\newline
\newline
\begin{longtabu}{|l|l|l|l|*2{X[l]|}}
    \hline
    id & Quelle & Ziel & Straße & Straßenbeginn & Straßenende\\ 
    \hline
    114 & Schallfeld & Schweinfurt & SW45 & Gerolzhofen & Schallfeld\\ 
    114 &  &  & St2274 & Gerolzhofen & B286/St2274 (Gerolzhofen bei Geomaris)\\ 
    114 &  &  & B286 & B286/St2274 (Gerolzhofen bei Geomaris) & B286/St2275 (Gerolzhofen bei Rügshofen)\\ 
    114 &  &  & B286 & B286/St2275 (Gerolzhofen bei Rügshofen) & B286/St2272 (bei Alitzheim)\\ 
    114 &  &  & B286 & B286/St2272 (bei Alitzheim) & B286/St2271 (bei Unterspießheim)\\ 
    114 &  &  & B286 & B286/St2271 (bei Unterspießheim) & B286/St2277 (bei Schwebheim)\\ 
    114 &  &  & B286 & B286/St2277 (bei Schwebheim) & B286/SW3/St2271 (nördlich Schwebheim)\\ 
    114 &  &  & B286 & B286/SW3/St2271 (nördlich Schwebheim) & Ausfahrt Schweinfurt-Zentrum\\ 
    114 &  &  & B286 & Ausfahrt Schweinfurt-Zentrum & B286 Schweinfurt Abfahrt Hans-Böckler-Straße\\ 
    114 &  &  & B286 & B286 Schweinfurt Abfahrt Hans-Böckler-Straße & B286/B26 (Schweinfurt Nähe Mainbrücke)\\ 
    114 &  &  & B286 & B286/B26 (Schweinfurt Nähe Mainbrücke) & B286/B303\\ 
    \hline
    115 & Schallfeld & Wiesentheid & SW45 & Schallfeld & Prichsenstadt OT Brünnau\\ 
    115 &  &  & KT40 & Prichsenstadt OT Neuses & Prichsenstadt OT Brünnau\\ 
    115 &  &  & B286 & B286/St2272 (Wiesentheid bei Blutbank) & Prichsenstadt OT Neuses\\ 
    115 &  &  & St2272 & B286/St2272 (Wiesentheid bei Blutbank) & Wiesentheid\\ 
    \hline
    116 & Schallfeld & Kitzingen & SW45 & Schallfeld & Prichsenstadt OT Brünnau\\ 
    116 &  &  & KT40 & Prichsenstadt OT Neuses & Prichsenstadt OT Brünnau\\ 
    116 &  &  & B22 & Stadelschwarzach & Prichsenstadt OT Neuses\\ 
    116 &  &  & B22 & Prichsenstadt OT Laub & Stadelschwarzach\\ 
    116 &  &  & B22 & Wiesentheid OT Reupelsdorf & Prichsenstadt OT Laub\\ 
    116 &  &  & B22 & Düllstadt & Wiesentheid OT Reupelsdorf\\ 
    116 &  &  & B22 & B22/KT11 & Düllstadt\\ 
    116 &  &  & B22 & B22/St2271 (bei Stadtschwarzach) & B22/KT11\\ 
    116 &  &  & St2271 & B22/St2271 (bei Stadtschwarzach) & Hörblach\\ 
    116 &  &  & St2271 & Hörblach & Ausfahrt Kitzingen-Schwarzach\\ 
    116 &  &  & St2271 & Ausfahrt Kitzingen-Schwarzach & St2271/St2272 (bei Kitzingen-Etwashausen)\\ 
    116 &  &  & St2271 & St2271/St2272 (bei Kitzingen-Etwashausen) & B8/St2271 (Kitzingen bei e-center)\\ 
    116 &  &  & B8 & Kitzingen & B8/St2271 (Kitzingen bei e-center)\\ 
    \hline
    117 & Schallfeld & Lülsfeld & SW43 & Lülsfeld & Schallfeld\\ 
    \hline
\end{longtabu}

\begin{listing}[htbp]
\begin{minted}{sql}
    SELECT 
	`potentials`.`id` AS `id`,
	`from_places`.`name` AS `Quelle`, 
	`to_places`.`name` AS `Ziel`,
	`streets`.`street` AS `Straße`,
	`from_street_places`.`name` AS `Straßenbeginn`,
   `to_street_places`.`name` AS `Straßenende`
FROM `potentials`
LEFT JOIN `places` AS `from_places` ON `potentials`.`from_id` = `from_places`.`id`
LEFT JOIN `places` AS `to_places` ON `potentials`.`to_id` = `to_places`.`id`
LEFT JOIN `routes` ON `routes`.`potential_id` = `potentials`.`id`
LEFT JOIN `streets` ON `streets`.`id` = `routes`.`street_id`
LEFT JOIN `places` AS `from_street_places` ON `streets`.`from_id` = `from_street_places`.`id`
LEFT JOIN `places` AS `to_street_places` ON `streets`.`to_id` = `to_street_places`.`id`
WHERE `from_places`.`name` = 'Schallfeld'
ORDER BY `potentials`.`id`, `routes`.`number_on_route`;
\end{minted}
\caption{SQL-Abfrage der zugeordneten Straßen mit der Quelle Schallfeld}\label{lst-rt-schallfeld}
\end{listing}


Länge, Fahrzeiten und Google Maps:
\newline
\begin{longtabu}{| l | *5{X[l]|}}
    \hline
    id & Quelle & Ziel & Fahrtstrecke [m] & Fahrtdauer [min] & Google-Maps Link\\ 
    \hline
    114 & Schallfeld & Schweinfurt & 25000 & 23 & \url{https://www.google.com/maps/dir/49.8734493,10.3574113/50.0439484,10.2257843}\\ 
    \hline
    115 & Schallfeld & Wiesentheid & 9300 & 10 & \url{https://www.google.com/maps/dir/49.8734493,10.3574113/49.7942401,10.3426344}\\ 
    \hline
    116 & Schallfeld & Kitzingen & 24600 & 25 & \url{https://www.google.com/maps/dir/49.8734493,10.3574113/49.7355709,10.1617438}\\ 
    \hline
    117 & Schallfeld & Lülsfeld & 3100 & 4 & \url{https://www.google.com/maps/dir/49.8734493,10.3574113/49.8677403,10.3199678}\\ 
    \hline
\end{longtabu}

\begin{listing}[htbp]
    \begin{minted}{sql}
        SELECT 
        `potentials`.`id` AS `id`, 
        `from_places`.`name` AS `Quelle`,
        `to_places`.`name` AS `Ziel`, 
        `potentials`.`length` AS `Fahrtstrecke [m]`, 
        `potentials`.`miv-duration` AS `Fahrtdauer [min]`,
        CONCAT('https://www.google.com/maps/dir/', `from_places`.`LAT`, ",", `from_places`.`LONG`, '/', `to_places`.`LAT`, ',', `to_places`.`LONG`) AS `Google-Maps Link`
    FROM `potentials`
    LEFT JOIN `places` AS `from_places` ON `potentials`.`from_id` = `from_places`.`id`
    LEFT JOIN `places` AS `to_places` ON `potentials`.`to_id` = `to_places`.`id`
    WHERE `from_places`.`name` = 'Schallfeld'
    ORDER BY `potentials`.`id`;
    \end{minted}
    \caption{SQL-Abfrage der Fahrtstrecke, Fahrtdauer und des Google-Maps-Link mit der Quelle Schallfeld}\label{lst-f-schallfeld}
\end{listing}
                
                \subsection{Prichsenstadt}
                Zugeordnete Routen:
\newline
\newline
\begin{longtabu}{|l|l|l|l|*2{X[l]|}}
    \hline
    id & Quelle & Ziel & Straße & Straßenbeginn & Straßenende\\ 
    \hline
    118 & Prichsenstadt & Bamberg, Haßfurt & St2420 & Prichsenstadt & Prichsenstadt OT Neuses\\ 
    118 &  &  & B22 & Prichsenstadt OT Neuses & B22/KT42 (bei Neudorf)\\ 
    118 &  &  & B22 & B22/KT42 (bei Neudorf) & B22/St2272/SW48\\ 
    118 &  &  & B22 & B22/St2272/SW48 & Breitbach\\ 
    118 &  &  & B22 & Breitbach & Ebrach\\ 
    \hline
    119 & Prichsenstadt & Schweinfurt & St2274 & Gerolzhofen & B286/St2274 (Gerolzhofen bei Geomaris)\\ 
    119 &  &  & B286 & Prichsenstadt OT Neuses & B286/St2274 (Gerolzhofen bei Geomaris)\\ 
    119 &  &  & B286 & B286/St2274 (Gerolzhofen bei Geomaris) & B286/St2275 (Gerolzhofen bei Rügshofen)\\ 
    119 &  &  & B286 & B286/St2275 (Gerolzhofen bei Rügshofen) & B286/St2272 (bei Alitzheim)\\ 
    119 &  &  & B286 & B286/St2272 (bei Alitzheim) & B286/St2271 (bei Unterspießheim)\\ 
    119 &  &  & B286 & B286/St2271 (bei Unterspießheim) & B286/St2277 (bei Schwebheim)\\ 
    119 &  &  & B286 & B286/St2277 (bei Schwebheim) & B286/SW3/St2271 (nördlich Schwebheim)\\ 
    119 &  &  & B286 & B286/SW3/St2271 (nördlich Schwebheim) & Ausfahrt Schweinfurt-Zentrum\\ 
    119 &  &  & B286 & Ausfahrt Schweinfurt-Zentrum & B286 Schweinfurt Abfahrt Hans-Böckler-Straße\\ 
    119 &  &  & B286 & B286 Schweinfurt Abfahrt Hans-Böckler-Straße & B286/B26 (Schweinfurt Nähe Mainbrücke)\\ 
    119 &  &  & B286 & B286/B26 (Schweinfurt Nähe Mainbrücke) & B286/B303\\ 
    \hline
    120 & Prichsenstadt & Gerolzhofen & St2420 & Prichsenstadt & Prichsenstadt OT Neuses\\ 
    120 &  &  & B286 & Prichsenstadt OT Neuses & B286/St2274 (Gerolzhofen bei Geomaris)\\ 
    120 &  &  & St2274 & Gerolzhofen & B286/St2274 (Gerolzhofen bei Geomaris)\\ 
    \hline
    121 & Prichsenstadt & Lülsfeld & St2420 & Prichsenstadt & Prichsenstadt OT Neuses\\ 
    121 &  &  & KT40 & Prichsenstadt OT Neuses & Prichsenstadt OT Brünnau\\ 
    121 &  &  & KT39 & Prichsenstadt OT Brünnau & Järkendorf\\ 
    121 &  &  & SW44 & Järkendorf & Lülsfeld\\ 
    \hline
    122 & Prichsenstadt & Wiesentheid & St2420 & Wiesentheid & Prichsenstadt\\ 
    \hline
    123 & Prichsenstadt & Kitzingen & St2260 & Prichsenstadt OT Laub & Prichsenstadt\\ 
    123 &  &  & B22 & Prichsenstadt OT Laub & Stadelschwarzach\\ 
    123 &  &  & B22 & Wiesentheid OT Reupelsdorf & Prichsenstadt OT Laub\\ 
    123 &  &  & B22 & Düllstadt & Wiesentheid OT Reupelsdorf\\ 
    123 &  &  & B22 & B22/KT11 & Düllstadt\\ 
    123 &  &  & B22 & B22/St2271 (bei Stadtschwarzach) & B22/KT11\\ 
    123 &  &  & St2271 & B22/St2271 (bei Stadtschwarzach) & Hörblach\\ 
    123 &  &  & St2271 & Hörblach & Ausfahrt Kitzingen-Schwarzach\\ 
    123 &  &  & St2271 & Ausfahrt Kitzingen-Schwarzach & St2271/St2272 (bei Kitzingen-Etwashausen)\\ 
    123 &  &  & St2271 & St2271/St2272 (bei Kitzingen-Etwashausen) & B8/St2271 (Kitzingen bei e-center)\\ 
    123 &  &  & B8 & Kitzingen & B8/St2271 (Kitzingen bei e-center)\\ 
    \hline
    124 & Prichsenstadt & Würzburg, Rottendorf & St2420 & Wiesentheid & Prichsenstadt\\ 
    124 &  &  & St2272 & B286/St2272 (Wiesentheid bei Blutbank) & Wiesentheid\\ 
    124 &  &  & B286 & Wiesentheid Kreuzung Gewerbegebiet Althölzl & B286/St2272 (Wiesentheid bei Blutbank)\\ 
    124 &  &  & B286 & Ausfahrt Wiesentheid & Wiesentheid Kreuzung Gewerbegebiet Althölzl\\ 
    124 &  &  & A3 & Ausfahrt Kitzingen-Schwarzach & Ausfahrt Wiesentheid\\ 
    124 &  &  & A3 & Autobahnkreuz Biebelried & Ausfahrt Kitzingen-Schwarzach\\ 
    124 &  &  & A3 & Ausfahrt Würzburg-Biebelried & Autobahnkreuz Biebelried\\ 
    124 &  &  & B8 & B8/B22 (am Mainfrankenpark) & Ausfahrt Würzburg-Biebelried\\ 
    124 &  &  & B8 & Ausfahrt Rottendorf & B8/B22 (am Mainfrankenpark)\\ 
    124 &  &  & B8 & Ausfahrt Wöllriederhof & Ausfahrt Rottendorf\\ 
    124 &  &  & B8 & Ausfahrt Würzburg/Nürnberger Straße & Ausfahrt Wöllriederhof\\ 
    124 &  &  & B8 & Würzburg B8/B19 (Grainbergknoten) & Ausfahrt Würzburg/Nürnberger Straße\\ 
    \hline
    125 & Prichsenstadt & Nürnberg, Erlangen & St2420 & Wiesentheid & Prichsenstadt\\ 
    125 &  &  & St2272 & B286/St2272 (Wiesentheid bei Blutbank) & Wiesentheid\\ 
    125 &  &  & B286 & Wiesentheid Kreuzung Gewerbegebiet Althölzl & B286/St2272 (Wiesentheid bei Blutbank)\\ 
    125 &  &  & B286 & Ausfahrt Wiesentheid & Wiesentheid Kreuzung Gewerbegebiet Althölzl\\ 
    125 &  &  & A3 & Ausfahrt Wiesentheid & Ausfahrt Geiselwind\\ 
    125 &  &  & A3 & Ausfahrt Geiselwind & Ausfahrt Schlüsselfeld\\ 
    125 &  &  & A3 & Ausfahrt Schlüsselfeld & Ausfahrt Höchstadt-Nord\\ 
    125 &  &  & A3 & Ausfahrt Höchstadt-Nord & Ausfahrt Pommersfelden\\ 
    125 &  &  & A3 & Ausfahrt Pommersfelden & Ausfahrt Höchstadt-Ost\\ 
    125 &  &  & A3 & Ausfahrt Höchstadt-Ost & Ausfahrt Erlangen-West\\ 
    125 &  &  & A3 & Ausfahrt Erlangen-West & Ausfahrt Erlangen-Frauenaurach\\ 
    125 &  &  & A3 & Ausfahrt Erlangen-Frauenaurach & Autobahnkreuz Fürth-Erlangen\\ 
    \hline
\end{longtabu}

\begin{listing}[htbp]
\begin{minted}{sql}
    SELECT 
	`potentials`.`id` AS `id`,
	`from_places`.`name` AS `Quelle`, 
	`to_places`.`name` AS `Ziel`,
	`streets`.`street` AS `Straße`,
	`from_street_places`.`name` AS `Straßenbeginn`,
   `to_street_places`.`name` AS `Straßenende`
FROM `potentials`
LEFT JOIN `places` AS `from_places` ON `potentials`.`from_id` = `from_places`.`id`
LEFT JOIN `places` AS `to_places` ON `potentials`.`to_id` = `to_places`.`id`
LEFT JOIN `routes` ON `routes`.`potential_id` = `potentials`.`id`
LEFT JOIN `streets` ON `streets`.`id` = `routes`.`street_id`
LEFT JOIN `places` AS `from_street_places` ON `streets`.`from_id` = `from_street_places`.`id`
LEFT JOIN `places` AS `to_street_places` ON `streets`.`to_id` = `to_street_places`.`id`
WHERE `from_places`.`name` = 'Prichsenstadt'
ORDER BY `potentials`.`id`, `routes`.`number_on_route`;
\end{minted}
\caption{SQL-Abfrage der zugeordneten Straßen mit der Quelle Prichsenstadt}\label{lst-rt-prichsenstadt}
\end{listing}


Länge, Fahrzeiten und Google Maps:
\newline
\begin{longtabu}{| l | *5{X[l]|}}
    \hline
    id & Quelle & Ziel & Fahrtstrecke [m] & Fahrtdauer [min] & Google-Maps Link\\ 
    \hline
    118 & Prichsenstadt & Bamberg, Haßfurt & 47600 & 45 & \url{https://www.google.com/maps/dir/49.8176258,10.3528515/49.8912678,10.8865984}\\ 
    \hline
    119 & Prichsenstadt & Schweinfurt & 31400 & 27 & \url{https://www.google.com/maps/dir/49.8176258,10.3528515/50.0439484,10.2257843}\\ 
    \hline
    120 & Prichsenstadt & Gerolzhofen & 11500 & 12 & \url{https://www.google.com/maps/dir/49.8176258,10.3528515/49.9010511,10.3489622}\\ 
    \hline
    121 & Prichsenstadt & Lülsfeld & 8400 & 10 & \url{https://www.google.com/maps/dir/49.8176258,10.3528515/49.8677403,10.3199678}\\ 
    \hline
    122 & Prichsenstadt & Wiesentheid & 3000 & 4 & \url{https://www.google.com/maps/dir/49.8176258,10.3528515/49.7942401,10.3426344}\\ 
    \hline
    123 & Prichsenstadt & Kitzingen & 19300 & 21 & \url{https://www.google.com/maps/dir/49.8176258,10.3528515/49.7355709,10.1617438}\\ 
    \hline
    124 & Prichsenstadt & Würzburg, Rottendorf & 39600 & 36 & \url{https://www.google.com/maps/dir/49.8176258,10.3528515/49.7931,9.9280108}\\ 
    \hline
    125 & Prichsenstadt & Nürnberg, Erlangen & 62700 & 42 & \url{https://www.google.com/maps/dir/49.8176258,10.3528515/49.5598096,10.9916482}\\ 
    \hline
\end{longtabu}

\begin{listing}[htbp]
    \begin{minted}{sql}
        SELECT 
        `potentials`.`id` AS `id`, 
        `from_places`.`name` AS `Quelle`,
        `to_places`.`name` AS `Ziel`, 
        `potentials`.`length` AS `Fahrtstrecke [m]`, 
        `potentials`.`miv-duration` AS `Fahrtdauer [min]`,
        CONCAT('https://www.google.com/maps/dir/', `from_places`.`LAT`, ",", `from_places`.`LONG`, '/', `to_places`.`LAT`, ',', `to_places`.`LONG`) AS `Google-Maps Link`
    FROM `potentials`
    LEFT JOIN `places` AS `from_places` ON `potentials`.`from_id` = `from_places`.`id`
    LEFT JOIN `places` AS `to_places` ON `potentials`.`to_id` = `to_places`.`id`
    WHERE `from_places`.`name` = 'Prichsenstadt'
    ORDER BY `potentials`.`id`;
    \end{minted}
    \caption{SQL-Abfrage der Fahrtstrecke, Fahrtdauer und des Google-Maps-Link mit der Quelle Prichsenstadt}\label{lst-f-prichsenstadt}
\end{listing}
                
                \subsection{Prichsenstadt OT Altenschönbach}
                Zugeordnete Routen:
\newline
\newline
\begin{longtabu}{|l|l|l|l|*2{X[l]|}}
    \hline
    id & Quelle & Ziel & Straße & Straßenbeginn & Straßenende\\ 
    \hline
    126 & Prichsenstadt OT Altenschönbach & Bamberg, Haßfurt & St2272 & Siegendorf & Prichsenstadt OT Altenschönbach\\ 
    126 &  &  & St2272 & B22/St2272/SW48 & Siegendorf\\ 
    126 &  &  & B22 & B22/St2272/SW48 & Breitbach\\ 
    126 &  &  & B22 & Breitbach & Ebrach\\ 
    \hline
    127 & Prichsenstadt OT Altenschönbach & Schweinfurt & St2272 & Siegendorf & Prichsenstadt OT Altenschönbach\\ 
    127 &  &  & St2272 & B22/St2272/SW48 & Siegendorf\\ 
    127 &  &  & B22 & B22/KT42 (bei Neudorf) & B22/St2272/SW48\\ 
    127 &  &  & B22 & Prichsenstadt OT Neuses & B22/KT42 (bei Neudorf)\\ 
    127 &  &  & B286 & Prichsenstadt OT Neuses & B286/St2274 (Gerolzhofen bei Geomaris)\\ 
    127 &  &  & B286 & B286/St2274 (Gerolzhofen bei Geomaris) & B286/St2275 (Gerolzhofen bei Rügshofen)\\ 
    127 &  &  & B286 & B286/St2275 (Gerolzhofen bei Rügshofen) & B286/St2272 (bei Alitzheim)\\ 
    127 &  &  & B286 & B286/St2272 (bei Alitzheim) & B286/St2271 (bei Unterspießheim)\\ 
    127 &  &  & B286 & B286/St2271 (bei Unterspießheim) & B286/St2277 (bei Schwebheim)\\ 
    127 &  &  & B286 & B286/St2277 (bei Schwebheim) & B286/SW3/St2271 (nördlich Schwebheim)\\ 
    127 &  &  & B286 & B286/SW3/St2271 (nördlich Schwebheim) & Ausfahrt Schweinfurt-Zentrum\\ 
    127 &  &  & B286 & Ausfahrt Schweinfurt-Zentrum & B286 Schweinfurt Abfahrt Hans-Böckler-Straße\\ 
    127 &  &  & B286 & B286 Schweinfurt Abfahrt Hans-Böckler-Straße & B286/B26 (Schweinfurt Nähe Mainbrücke)\\ 
    127 &  &  & B286 & B286/B26 (Schweinfurt Nähe Mainbrücke) & B286/B303\\ 
    \hline
    128 & Prichsenstadt OT Altenschönbach & Gerolzhofen & St2272 & Siegendorf & Prichsenstadt OT Altenschönbach\\ 
    128 &  &  & St2272 & B22/St2272/SW48 & Siegendorf\\ 
    128 &  &  & SW48 & B22/St2272/SW48 & Oberschwarzach\\ 
    128 &  &  & St2272 & Mutzenroth & Oberschwarzach\\ 
    128 &  &  & St2272 & Wiebelsberg & Mutzenroth\\ 
    128 &  &  & St2272 & Gerolzhofen & Wiebelsberg\\ 
    \hline
    129 & Prichsenstadt OT Altenschönbach & Lülsfeld & St2272 & Siegendorf & Prichsenstadt OT Altenschönbach\\ 
    129 &  &  & St2272 & B22/St2272/SW48 & Siegendorf\\ 
    129 &  &  & B22 & B22/KT42 (bei Neudorf) & B22/St2272/SW48\\ 
    129 &  &  & B22 & Prichsenstadt OT Neuses & B22/KT42 (bei Neudorf)\\ 
    129 &  &  & B22 & Stadelschwarzach & Prichsenstadt OT Neuses\\ 
    129 &  &  & KT38 & Järkendorf & Stadelschwarzach\\ 
    129 &  &  & SW44 & Järkendorf & Lülsfeld\\ 
    \hline
    130 & Prichsenstadt OT Altenschönbach & Kitzingen & St2272 & Prichsenstadt OT Altenschönbach & Prichsenstadt OT Kirchschönbach\\ 
    130 &  &  & St2272 & Prichsenstadt OT Kirchschönbach & Wiesentheid OT Geesdorf\\ 
    130 &  &  & St2272 & Wiesentheid OT Geesdorf & B286/St2272 (Wiesentheid bei Blutbank)\\ 
    130 &  &  & St2272 & B286/St2272 (Wiesentheid bei Blutbank) & Wiesentheid\\ 
    130 &  &  & St2272 & Wiesentheid & Wiesentheid OT Feuerbach\\ 
    130 &  &  & St2272 & Wiesentheid OT Feuerbach & Kleinlangheim\\ 
    130 &  &  & St2272 & Kleinlangheim & Großlangheim\\ 
    130 &  &  & St2272 & Großlangheim & St2271/St2272 (bei Kitzingen-Etwashausen)\\ 
    130 &  &  & St2272 & St2271/St2272 (bei Kitzingen-Etwashausen) & Kitzingen\\ 
    \hline
    131 & Prichsenstadt OT Altenschönbach & Würzburg, Rottendorf & St2272 & Prichsenstadt OT Altenschönbach & Prichsenstadt OT Kirchschönbach\\ 
    131 &  &  & St2272 & Prichsenstadt OT Kirchschönbach & Wiesentheid OT Geesdorf\\ 
    131 &  &  & St2272 & Wiesentheid OT Geesdorf & B286/St2272 (Wiesentheid bei Blutbank)\\ 
    131 &  &  & B286 & Wiesentheid Kreuzung Gewerbegebiet Althölzl & B286/St2272 (Wiesentheid bei Blutbank)\\ 
    131 &  &  & B286 & Ausfahrt Wiesentheid & Wiesentheid Kreuzung Gewerbegebiet Althölzl\\ 
    131 &  &  & A3 & Ausfahrt Kitzingen-Schwarzach & Ausfahrt Wiesentheid\\ 
    131 &  &  & A3 & Autobahnkreuz Biebelried & Ausfahrt Kitzingen-Schwarzach\\ 
    131 &  &  & A3 & Ausfahrt Würzburg-Biebelried & Autobahnkreuz Biebelried\\ 
    131 &  &  & B8 & B8/B22 (am Mainfrankenpark) & Ausfahrt Würzburg-Biebelried\\ 
    131 &  &  & B8 & Ausfahrt Rottendorf & B8/B22 (am Mainfrankenpark)\\ 
    131 &  &  & B8 & Ausfahrt Wöllriederhof & Ausfahrt Rottendorf\\ 
    131 &  &  & B8 & Ausfahrt Würzburg/Nürnberger Straße & Ausfahrt Wöllriederhof\\ 
    131 &  &  & B8 & Würzburg B8/B19 (Grainbergknoten) & Ausfahrt Würzburg/Nürnberger Straße\\ 
    \hline
    132 & Prichsenstadt OT Altenschönbach & Nürnberg, Erlangen & St2272 & Prichsenstadt OT Altenschönbach & Prichsenstadt OT Kirchschönbach\\ 
    132 &  &  & St2260 & Wiesentheid OT Geesdorf & Gräfenneuses\\ 
    132 &  &  & St2260 & Gräfenneuses & Geiselwind\\ 
    132 &  &  & St2257 & Geiselwind & Ausfahrt Geiselwind\\ 
    132 &  &  & A3 & Ausfahrt Geiselwind & Ausfahrt Schlüsselfeld\\ 
    132 &  &  & A3 & Ausfahrt Schlüsselfeld & Ausfahrt Höchstadt-Nord\\ 
    132 &  &  & A3 & Ausfahrt Höchstadt-Nord & Ausfahrt Pommersfelden\\ 
    132 &  &  & A3 & Ausfahrt Pommersfelden & Ausfahrt Höchstadt-Ost\\ 
    132 &  &  & A3 & Ausfahrt Höchstadt-Ost & Ausfahrt Erlangen-West\\ 
    132 &  &  & A3 & Ausfahrt Erlangen-Frauenaurach & Autobahnkreuz Fürth-Erlangen\\ 
    \hline
    133 & Prichsenstadt OT Altenschönbach & Prichsenstadt & St2272 & Prichsenstadt OT Altenschönbach & Prichsenstadt OT Kirchschönbach\\ 
    133 &  &  & KT46 & Prichsenstadt OT Kirchschönbach & Prichsenstadt\\ 
    \hline
\end{longtabu}

\begin{listing}[htbp]
\begin{minted}{sql}
    SELECT 
	`potentials`.`id` AS `id`,
	`from_places`.`name` AS `Quelle`, 
	`to_places`.`name` AS `Ziel`,
	`streets`.`street` AS `Straße`,
	`from_street_places`.`name` AS `Straßenbeginn`,
   `to_street_places`.`name` AS `Straßenende`
FROM `potentials`
LEFT JOIN `places` AS `from_places` ON `potentials`.`from_id` = `from_places`.`id`
LEFT JOIN `places` AS `to_places` ON `potentials`.`to_id` = `to_places`.`id`
LEFT JOIN `routes` ON `routes`.`potential_id` = `potentials`.`id`
LEFT JOIN `streets` ON `streets`.`id` = `routes`.`street_id`
LEFT JOIN `places` AS `from_street_places` ON `streets`.`from_id` = `from_street_places`.`id`
LEFT JOIN `places` AS `to_street_places` ON `streets`.`to_id` = `to_street_places`.`id`
WHERE `from_places`.`name` = 'Prichsenstadt OT Altenschönbach'
ORDER BY `potentials`.`id`, `routes`.`number_on_route`;
\end{minted}
\caption{SQL-Abfrage der zugeordneten Straßen mit der Quelle Prichsenstadt OT Altenschönbach}\label{lst-rt-altenschoenbach}
\end{listing}


Länge, Fahrzeiten und Google Maps:
\newline
\begin{longtabu}{| l | *5{X[l]|}}
    \hline
    id & Quelle & Ziel & Fahrtstrecke [m] & Fahrtdauer [min] & Google-Maps Link\\ 
    \hline
    126 & Prichsenstadt OT Altenschönbach & Bamberg, Haßfurt & 43600 & 42 & \url{https://www.google.com/maps/dir/49.8235388,10.3974385/49.8912678,10.8865984}\\ 
    \hline
    127 & Prichsenstadt OT Altenschönbach & Schweinfurt & 35000 & 30 & \url{https://www.google.com/maps/dir/49.8235388,10.3974385/50.0439484,10.2257843}\\ 
    \hline
    128 & Prichsenstadt OT Altenschönbach & Gerolzhofen & 11700 & 14 & \url{https://www.google.com/maps/dir/49.8235388,10.3974385/49.9010511,10.3489622}\\ 
    \hline
    129 & Prichsenstadt OT Altenschönbach & Lülsfeld & 11900 & 12 & \url{https://www.google.com/maps/dir/49.8235388,10.3974385/49.8677403,10.3199678}\\ 
    \hline
    130 & Prichsenstadt OT Altenschönbach & Kitzingen & 23500 & 24 & \url{https://www.google.com/maps/dir/49.8235388,10.3974385/49.7355709,10.1617438}\\ 
    \hline
    131 & Prichsenstadt OT Altenschönbach & Würzburg, Rottendorf & 43800 & 39 & \url{https://www.google.com/maps/dir/49.8235388,10.3974385/49.7931,9.9280108}\\ 
    \hline
    132 & Prichsenstadt OT Altenschönbach & Nürnberg, Erlangen & 61500 & 42 & \url{https://www.google.com/maps/dir/49.8235388,10.3974385/49.5598096,10.9916482}\\ 
    \hline
    133 & Prichsenstadt OT Altenschönbach & Prichsenstadt & 4000 & 5 & \url{https://www.google.com/maps/dir/49.8235388,10.3974385/49.8176258,10.3528515}\\ 
    \hline
\end{longtabu}

\begin{listing}[htbp]
    \begin{minted}{sql}
        SELECT 
        `potentials`.`id` AS `id`, 
        `from_places`.`name` AS `Quelle`,
        `to_places`.`name` AS `Ziel`, 
        `potentials`.`length` AS `Fahrtstrecke [m]`, 
        `potentials`.`miv-duration` AS `Fahrtdauer [min]`,
        CONCAT('https://www.google.com/maps/dir/', `from_places`.`LAT`, ",", `from_places`.`LONG`, '/', `to_places`.`LAT`, ',', `to_places`.`LONG`) AS `Google-Maps Link`
    FROM `potentials`
    LEFT JOIN `places` AS `from_places` ON `potentials`.`from_id` = `from_places`.`id`
    LEFT JOIN `places` AS `to_places` ON `potentials`.`to_id` = `to_places`.`id`
    WHERE `from_places`.`name` = 'Prichsenstadt OT Altenschönbach'
    ORDER BY `potentials`.`id`;
    \end{minted}
    \caption{SQL-Abfrage der Fahrtstrecke, Fahrtdauer und des Google-Maps-Link mit der Quelle Prichsenstadt OT Altenschönbach}\label{lst-f-altenschoenbach}
\end{listing}
                
                \subsection{Prichsenstadt OT Bimbach}
                Zugeordnete Routen:
\newline
\newline
\begin{longtabu}{|l|l|l|l|*2{X[l]|}}
    \hline
    id & Quelle & Ziel & Straße & Straßenbeginn & Straßenende\\ 
    \hline
    134 & Prichsenstadt OT Bimbach & Bamberg, Haßfurt & KT42 & B22/KT42 (bei Neudorf) & Prichsenstadt OT Bimbach\\ 
    134 &  &  & B22 & B22/KT42 (bei Neudorf) & B22/St2272/SW48\\ 
    134 &  &  & B22 & B22/St2272/SW48 & Breitbach\\ 
    134 &  &  & B22 & Breitbach & Ebrach\\ 
    \hline
    135 & Prichsenstadt OT Bimbach & Schweinfurt & SW42 & Schallfeld & Prichsenstadt OT Bimbach\\ 
    135 &  &  & SW45 & Gerolzhofen & Schallfeld\\ 
    135 &  &  & St2274 & Gerolzhofen & B286/St2274 (Gerolzhofen bei Geomaris)\\ 
    135 &  &  & B286 & B286/St2274 (Gerolzhofen bei Geomaris) & B286/St2275 (Gerolzhofen bei Rügshofen)\\ 
    135 &  &  & B286 & B286/St2275 (Gerolzhofen bei Rügshofen) & B286/St2272 (bei Alitzheim)\\ 
    135 &  &  & B286 & B286/St2272 (bei Alitzheim) & B286/St2271 (bei Unterspießheim)\\ 
    135 &  &  & B286 & B286/St2271 (bei Unterspießheim) & B286/St2277 (bei Schwebheim)\\ 
    135 &  &  & B286 & B286/St2277 (bei Schwebheim) & B286/SW3/St2271 (nördlich Schwebheim)\\ 
    135 &  &  & B286 & B286/SW3/St2271 (nördlich Schwebheim) & Ausfahrt Schweinfurt-Zentrum\\ 
    135 &  &  & B286 & Ausfahrt Schweinfurt-Zentrum & B286 Schweinfurt Abfahrt Hans-Böckler-Straße\\ 
    135 &  &  & B286 & B286 Schweinfurt Abfahrt Hans-Böckler-Straße & B286/B26 (Schweinfurt Nähe Mainbrücke)\\ 
    135 &  &  & B286 & B286/B26 (Schweinfurt Nähe Mainbrücke) & B286/B303\\ 
    \hline
    136 & Prichsenstadt OT Bimbach & Gerolzhofen & SW42 & Schallfeld & Prichsenstadt OT Bimbach\\ 
    136 &  &  & SW45 & Gerolzhofen & Schallfeld\\ 
    \hline
    137 & Prichsenstadt OT Bimbach & Lülsfeld & SW42 & Schallfeld & Prichsenstadt OT Bimbach\\ 
    137 &  &  & SW43 & Lülsfeld & Schallfeld\\ 
    \hline
    138 & Prichsenstadt OT Bimbach & Wiesentheid & KT42 & B22/KT42 (bei Neudorf) & Prichsenstadt OT Bimbach\\ 
    138 &  &  & B22 & Prichsenstadt OT Neuses & B22/KT42 (bei Neudorf)\\ 
    138 &  &  & B286 & B286/St2272 (Wiesentheid bei Blutbank) & Prichsenstadt OT Neuses\\ 
    138 &  &  & St2272 & B286/St2272 (Wiesentheid bei Blutbank) & Wiesentheid\\ 
    \hline
    139 & Prichsenstadt OT Bimbach & Kitzingen & KT42 & B22/KT42 (bei Neudorf) & Prichsenstadt OT Bimbach\\ 
    139 &  &  & B22 & Prichsenstadt OT Neuses & B22/KT42 (bei Neudorf)\\ 
    139 &  &  & B22 & Stadelschwarzach & Prichsenstadt OT Neuses\\ 
    139 &  &  & B22 & Prichsenstadt OT Laub & Stadelschwarzach\\ 
    139 &  &  & B22 & Wiesentheid OT Reupelsdorf & Prichsenstadt OT Laub\\ 
    139 &  &  & B22 & Düllstadt & Wiesentheid OT Reupelsdorf\\ 
    139 &  &  & B22 & B22/KT11 & Düllstadt\\ 
    139 &  &  & B22 & B22/St2271 (bei Stadtschwarzach) & B22/KT11\\ 
    139 &  &  & St2271 & B22/St2271 (bei Stadtschwarzach) & Hörblach\\ 
    139 &  &  & St2271 & Hörblach & Ausfahrt Kitzingen-Schwarzach\\ 
    139 &  &  & St2271 & Ausfahrt Kitzingen-Schwarzach & St2271/St2272 (bei Kitzingen-Etwashausen)\\ 
    139 &  &  & St2271 & St2271/St2272 (bei Kitzingen-Etwashausen) & B8/St2271 (Kitzingen bei e-center)\\ 
    139 &  &  & B8 & Kitzingen & B8/St2271 (Kitzingen bei e-center)\\ 
    \hline
    140 & Prichsenstadt OT Bimbach & Würzburg, Rottendorf & B22 & Prichsenstadt OT Neuses & B22/KT42 (bei Neudorf)\\ 
    140 &  &  & B286 & Prichsenstadt OT Neuses & B286/St2274 (Gerolzhofen bei Geomaris)\\ 
    140 &  &  & B286 & B286/St2272 (Wiesentheid bei Blutbank) & Prichsenstadt OT Neuses\\ 
    140 &  &  & B286 & Wiesentheid Kreuzung Gewerbegebiet Althölzl & B286/St2272 (Wiesentheid bei Blutbank)\\ 
    140 &  &  & B286 & Ausfahrt Wiesentheid & Wiesentheid Kreuzung Gewerbegebiet Althölzl\\ 
    140 &  &  & A3 & Ausfahrt Kitzingen-Schwarzach & Ausfahrt Wiesentheid\\ 
    140 &  &  & A3 & Autobahnkreuz Biebelried & Ausfahrt Kitzingen-Schwarzach\\ 
    140 &  &  & A3 & Ausfahrt Würzburg-Biebelried & Autobahnkreuz Biebelried\\ 
    140 &  &  & B8 & B8/B22 (am Mainfrankenpark) & Ausfahrt Würzburg-Biebelried\\ 
    140 &  &  & B8 & Ausfahrt Rottendorf & B8/B22 (am Mainfrankenpark)\\ 
    140 &  &  & B8 & Ausfahrt Wöllriederhof & Ausfahrt Rottendorf\\ 
    140 &  &  & B8 & Ausfahrt Würzburg/Nürnberger Straße & Ausfahrt Wöllriederhof\\ 
    140 &  &  & B8 & Würzburg B8/B19 (Grainbergknoten) & Ausfahrt Würzburg/Nürnberger Straße\\ 
    \hline
    141 & Prichsenstadt OT Bimbach & Nürnberg, Erlangen & B22 & Prichsenstadt OT Neuses & B22/KT42 (bei Neudorf)\\ 
    141 &  &  & B286 & B286/St2272 (Wiesentheid bei Blutbank) & Prichsenstadt OT Neuses\\ 
    141 &  &  & B286 & Wiesentheid Kreuzung Gewerbegebiet Althölzl & B286/St2272 (Wiesentheid bei Blutbank)\\ 
    141 &  &  & B286 & Ausfahrt Wiesentheid & Wiesentheid Kreuzung Gewerbegebiet Althölzl\\ 
    141 &  &  & A3 & Ausfahrt Wiesentheid & Ausfahrt Geiselwind\\ 
    141 &  &  & A3 & Ausfahrt Geiselwind & Ausfahrt Schlüsselfeld\\ 
    141 &  &  & A3 & Ausfahrt Schlüsselfeld & Ausfahrt Höchstadt-Nord\\ 
    141 &  &  & A3 & Ausfahrt Höchstadt-Nord & Ausfahrt Pommersfelden\\ 
    141 &  &  & A3 & Ausfahrt Pommersfelden & Ausfahrt Höchstadt-Ost\\ 
    141 &  &  & A3 & Ausfahrt Höchstadt-Ost & Ausfahrt Erlangen-West\\ 
    141 &  &  & A3 & Ausfahrt Erlangen-West & Ausfahrt Erlangen-Frauenaurach\\ 
    141 &  &  & A3 & Ausfahrt Erlangen-Frauenaurach & Autobahnkreuz Fürth-Erlangen\\ 
    \hline
    142 & Prichsenstadt OT Bimbach & Järkendorf & KT39 & Prichsenstadt OT Brünnau & Järkendorf\\ 
    \hline
\end{longtabu}

\begin{listing}[htbp]
\begin{minted}{sql}
    SELECT 
	`potentials`.`id` AS `id`,
	`from_places`.`name` AS `Quelle`, 
	`to_places`.`name` AS `Ziel`,
	`streets`.`street` AS `Straße`,
	`from_street_places`.`name` AS `Straßenbeginn`,
   `to_street_places`.`name` AS `Straßenende`
FROM `potentials`
LEFT JOIN `places` AS `from_places` ON `potentials`.`from_id` = `from_places`.`id`
LEFT JOIN `places` AS `to_places` ON `potentials`.`to_id` = `to_places`.`id`
LEFT JOIN `routes` ON `routes`.`potential_id` = `potentials`.`id`
LEFT JOIN `streets` ON `streets`.`id` = `routes`.`street_id`
LEFT JOIN `places` AS `from_street_places` ON `streets`.`from_id` = `from_street_places`.`id`
LEFT JOIN `places` AS `to_street_places` ON `streets`.`to_id` = `to_street_places`.`id`
WHERE `from_places`.`name` = 'Prichsenstadt OT Bimbach'
ORDER BY `potentials`.`id`, `routes`.`number_on_route`;
\end{minted}
\caption{SQL-Abfrage der zugeordneten Straßen mit der Quelle Prichsenstadt OT Bimbach}\label{lst-rt-bimbach}
\end{listing}


Länge, Fahrzeiten und Google Maps:
\newline
\begin{longtabu}{| l | *5{X[l]|}}
    \hline
    id & Quelle & Ziel & Fahrtstrecke [m] & Fahrtdauer [min] & Google-Maps Link\\ 
    \hline
    134 & Prichsenstadt OT Bimbach & Bamberg, Haßfurt & 43500 & 42 & \url{https://www.google.com/maps/dir/49.8614801,10.3794984/49.8912678,10.8865984}\\ 
    \hline
    135 & Prichsenstadt OT Bimbach & Schweinfurt & 27200 & 26 & \url{https://www.google.com/maps/dir/49.8614801,10.3794984/50.0439484,10.2257843}\\ 
    \hline
    136 & Prichsenstadt OT Bimbach & Gerolzhofen & 5900 & 9 & \url{https://www.google.com/maps/dir/49.8614801,10.3794984/49.9010511,10.3489622}\\ 
    \hline
    137 & Prichsenstadt OT Bimbach & Lülsfeld & 5400 & 7 & \url{https://www.google.com/maps/dir/49.8614801,10.3794984/49.8677403,10.3199678}\\ 
    \hline
    138 & Prichsenstadt OT Bimbach & Wiesentheid & 10200 & 11 & \url{https://www.google.com/maps/dir/49.8614801,10.3794984/49.7942401,10.3426344}\\ 
    \hline
    139 & Prichsenstadt OT Bimbach & Kitzingen & 25400 & 25 & \url{https://www.google.com/maps/dir/49.8614801,10.3794984/49.7355709,10.1617438}\\ 
    \hline
    140 & Prichsenstadt OT Bimbach & Würzburg, Rottendorf & 46400 & 40 & \url{https://www.google.com/maps/dir/49.8614801,10.3794984/49.7931,9.9280108}\\ 
    \hline
    141 & Prichsenstadt OT Bimbach & Nürnberg, Erlangen & 74000 & 49 & \url{https://www.google.com/maps/dir/49.8614801,10.3794984/49.5598096,10.9916482}\\ 
    \hline
    142 & Prichsenstadt OT Bimbach & Järkendorf & 4200 & 6 & \url{https://www.google.com/maps/dir/49.8614801,10.3794984/49.8522178,10.3290937}\\ 
    \hline
\end{longtabu}

\begin{listing}[htbp]
    \begin{minted}{sql}
        SELECT 
        `potentials`.`id` AS `id`, 
        `from_places`.`name` AS `Quelle`,
        `to_places`.`name` AS `Ziel`, 
        `potentials`.`length` AS `Fahrtstrecke [m]`, 
        `potentials`.`miv-duration` AS `Fahrtdauer [min]`,
        CONCAT('https://www.google.com/maps/dir/', `from_places`.`LAT`, ",", `from_places`.`LONG`, '/', `to_places`.`LAT`, ',', `to_places`.`LONG`) AS `Google-Maps Link`
    FROM `potentials`
    LEFT JOIN `places` AS `from_places` ON `potentials`.`from_id` = `from_places`.`id`
    LEFT JOIN `places` AS `to_places` ON `potentials`.`to_id` = `to_places`.`id`
    WHERE `from_places`.`name` = 'Prichsenstadt OT Bimbach'
    ORDER BY `potentials`.`id`;
    \end{minted}
    \caption{SQL-Abfrage der Fahrtstrecke, Fahrtdauer und des Google-Maps-Link mit der Quelle Prichsenstadt OT Bimbach}\label{lst-f-bimbach}
\end{listing}
                
                \subsection{Prichsenstadt OT Brünnau}
                Zugeordnete Routen:
\newline
\newline
\begin{longtabu}{|l|l|l|l|*2{X[l]|}}
    \hline
    id & Quelle & Ziel & Straße & Straßenbeginn & Straßenende\\ 
    \hline
    143 & Prichsenstadt OT Brünnau & Schweinfurt & KT40 & Prichsenstadt OT Neuses & Prichsenstadt OT Brünnau\\ 
    143 &  &  & B286 & Prichsenstadt OT Neuses & B286/St2274 (Gerolzhofen bei Geomaris)\\ 
    143 &  &  & B286 & B286/St2274 (Gerolzhofen bei Geomaris) & B286/St2275 (Gerolzhofen bei Rügshofen)\\ 
    143 &  &  & B286 & B286/St2275 (Gerolzhofen bei Rügshofen) & B286/St2272 (bei Alitzheim)\\ 
    143 &  &  & B286 & B286/St2272 (bei Alitzheim) & B286/St2271 (bei Unterspießheim)\\ 
    143 &  &  & B286 & B286/St2271 (bei Unterspießheim) & B286/St2277 (bei Schwebheim)\\ 
    143 &  &  & B286 & B286/St2277 (bei Schwebheim) & B286/SW3/St2271 (nördlich Schwebheim)\\ 
    143 &  &  & B286 & B286/SW3/St2271 (nördlich Schwebheim) & Ausfahrt Schweinfurt-Zentrum\\ 
    143 &  &  & B286 & Ausfahrt Schweinfurt-Zentrum & B286 Schweinfurt Abfahrt Hans-Böckler-Straße\\ 
    143 &  &  & B286 & B286 Schweinfurt Abfahrt Hans-Böckler-Straße & B286/B26 (Schweinfurt Nähe Mainbrücke)\\ 
    143 &  &  & B286 & B286/B26 (Schweinfurt Nähe Mainbrücke) & B286/B303\\ 
    \hline
    144 & Prichsenstadt OT Brünnau & Gerolzhofen & SW45 & Schallfeld & Prichsenstadt OT Brünnau\\ 
    144 &  &  & SW45 & Gerolzhofen & Schallfeld\\ 
    \hline
    145 & Prichsenstadt OT Brünnau & Lülsfeld & KT39 & Prichsenstadt OT Brünnau & Järkendorf\\ 
    145 &  &  & SW44 & Järkendorf & Lülsfeld\\ 
    \hline
    146 & Prichsenstadt OT Brünnau & Wiesentheid & KT40 & Prichsenstadt OT Neuses & Prichsenstadt OT Brünnau\\ 
    146 &  &  & B286 & B286/St2272 (Wiesentheid bei Blutbank) & Prichsenstadt OT Neuses\\ 
    146 &  &  & St2272 & B286/St2272 (Wiesentheid bei Blutbank) & Wiesentheid\\ 
    \hline
    147 & Prichsenstadt OT Brünnau & Kitzingen & KT40 & Prichsenstadt OT Neuses & Prichsenstadt OT Brünnau\\ 
    147 &  &  & B22 & Stadelschwarzach & Prichsenstadt OT Neuses\\ 
    147 &  &  & B22 & Prichsenstadt OT Laub & Stadelschwarzach\\ 
    147 &  &  & B22 & Wiesentheid OT Reupelsdorf & Prichsenstadt OT Laub\\ 
    147 &  &  & B22 & Düllstadt & Wiesentheid OT Reupelsdorf\\ 
    147 &  &  & B22 & B22/KT11 & Düllstadt\\ 
    147 &  &  & B22 & B22/St2271 (bei Stadtschwarzach) & B22/KT11\\ 
    147 &  &  & St2271 & B22/St2271 (bei Stadtschwarzach) & Hörblach\\ 
    147 &  &  & St2271 & Hörblach & Ausfahrt Kitzingen-Schwarzach\\ 
    147 &  &  & St2271 & Ausfahrt Kitzingen-Schwarzach & St2271/St2272 (bei Kitzingen-Etwashausen)\\ 
    147 &  &  & St2271 & St2271/St2272 (bei Kitzingen-Etwashausen) & B8/St2271 (Kitzingen bei e-center)\\ 
    147 &  &  & B8 & Kitzingen & B8/St2271 (Kitzingen bei e-center)\\ 
    \hline
    148 & Prichsenstadt OT Brünnau & Würzburg, Rottendorf & KT40 & Prichsenstadt OT Neuses & Prichsenstadt OT Brünnau\\ 
    148 &  &  & B286 & Prichsenstadt OT Neuses & B286/St2274 (Gerolzhofen bei Geomaris)\\ 
    148 &  &  & B286 & B286/St2272 (Wiesentheid bei Blutbank) & Prichsenstadt OT Neuses\\ 
    148 &  &  & B286 & Wiesentheid Kreuzung Gewerbegebiet Althölzl & B286/St2272 (Wiesentheid bei Blutbank)\\ 
    148 &  &  & B286 & Ausfahrt Wiesentheid & Wiesentheid Kreuzung Gewerbegebiet Althölzl\\ 
    148 &  &  & A3 & Ausfahrt Kitzingen-Schwarzach & Ausfahrt Wiesentheid\\ 
    148 &  &  & A3 & Autobahnkreuz Biebelried & Ausfahrt Kitzingen-Schwarzach\\ 
    148 &  &  & A3 & Ausfahrt Würzburg-Biebelried & Autobahnkreuz Biebelried\\ 
    148 &  &  & B8 & B8/B22 (am Mainfrankenpark) & Ausfahrt Würzburg-Biebelried\\ 
    148 &  &  & B8 & Ausfahrt Rottendorf & B8/B22 (am Mainfrankenpark)\\ 
    148 &  &  & B8 & Ausfahrt Wöllriederhof & Ausfahrt Rottendorf\\ 
    148 &  &  & B8 & Ausfahrt Würzburg/Nürnberger Straße & Ausfahrt Wöllriederhof\\ 
    148 &  &  & B8 & Würzburg B8/B19 (Grainbergknoten) & Ausfahrt Würzburg/Nürnberger Straße\\ 
    \hline
    149 & Prichsenstadt OT Brünnau & Järkendorf & KT39 & Prichsenstadt OT Brünnau & Järkendorf\\ 
    \hline
\end{longtabu}

\begin{listing}[htbp]
\begin{minted}{sql}
    SELECT 
	`potentials`.`id` AS `id`,
	`from_places`.`name` AS `Quelle`, 
	`to_places`.`name` AS `Ziel`,
	`streets`.`street` AS `Straße`,
	`from_street_places`.`name` AS `Straßenbeginn`,
   `to_street_places`.`name` AS `Straßenende`
FROM `potentials`
LEFT JOIN `places` AS `from_places` ON `potentials`.`from_id` = `from_places`.`id`
LEFT JOIN `places` AS `to_places` ON `potentials`.`to_id` = `to_places`.`id`
LEFT JOIN `routes` ON `routes`.`potential_id` = `potentials`.`id`
LEFT JOIN `streets` ON `streets`.`id` = `routes`.`street_id`
LEFT JOIN `places` AS `from_street_places` ON `streets`.`from_id` = `from_street_places`.`id`
LEFT JOIN `places` AS `to_street_places` ON `streets`.`to_id` = `to_street_places`.`id`
WHERE `from_places`.`name` = 'Prichsenstadt OT Brünnau'
ORDER BY `potentials`.`id`, `routes`.`number_on_route`;
\end{minted}
\caption{SQL-Abfrage der zugeordneten Straßen mit der Quelle Prichsenstadt OT Brünnau}\label{lst-rt-bruennau}
\end{listing}


Länge, Fahrzeiten und Google Maps:
\newline
\begin{longtabu}{| l | *5{X[l]|}}
    \hline
    id & Quelle & Ziel & Fahrtstrecke [m] & Fahrtdauer [min] & Google-Maps Link\\ 
    \hline
    143 & Prichsenstadt OT Brünnau & Schweinfurt & 29700 & 27 & \url{https://www.google.com/maps/dir/49.8573989,10.353504/50.0439484,10.2257843}\\ 
    \hline
    144 & Prichsenstadt OT Brünnau & Gerolzhofen & 5500 & 8 & \url{https://www.google.com/maps/dir/49.8573989,10.353504/49.9010511,10.3489622}\\ 
    \hline
    145 & Prichsenstadt OT Brünnau & Lülsfeld & 3900 & 6 & \url{https://www.google.com/maps/dir/49.8573989,10.353504/49.8677403,10.3199678}\\ 
    \hline
    146 & Prichsenstadt OT Brünnau & Wiesentheid & 7400 & 9 & \url{https://www.google.com/maps/dir/49.8573989,10.353504/49.7942401,10.3426344}\\ 
    \hline
    147 & Prichsenstadt OT Brünnau & Kitzingen & 22700 & 23 & \url{https://www.google.com/maps/dir/49.8573989,10.353504/49.7355709,10.1617438}\\ 
    \hline
    148 & Prichsenstadt OT Brünnau & Würzburg, Rottendorf & 39100 & 37 & \url{https://www.google.com/maps/dir/49.8573989,10.353504/49.7931,9.9280108}\\ 
    \hline
    149 & Prichsenstadt OT Brünnau & Järkendorf & 1900 & 3 & \url{https://www.google.com/maps/dir/49.8573989,10.353504/49.8522178,10.3290937}\\ 
    \hline
\end{longtabu}

\begin{listing}[htbp]
    \begin{minted}{sql}
        SELECT 
        `potentials`.`id` AS `id`, 
        `from_places`.`name` AS `Quelle`,
        `to_places`.`name` AS `Ziel`, 
        `potentials`.`length` AS `Fahrtstrecke [m]`, 
        `potentials`.`miv-duration` AS `Fahrtdauer [min]`,
        CONCAT('https://www.google.com/maps/dir/', `from_places`.`LAT`, ",", `from_places`.`LONG`, '/', `to_places`.`LAT`, ',', `to_places`.`LONG`) AS `Google-Maps Link`
    FROM `potentials`
    LEFT JOIN `places` AS `from_places` ON `potentials`.`from_id` = `from_places`.`id`
    LEFT JOIN `places` AS `to_places` ON `potentials`.`to_id` = `to_places`.`id`
    WHERE `from_places`.`name` = 'Prichsenstadt OT Brünnau'
    ORDER BY `potentials`.`id`;
    \end{minted}
    \caption{SQL-Abfrage der Fahrtstrecke, Fahrtdauer und des Google-Maps-Link mit der Quelle Prichsenstadt OT Brünnau}\label{lst-f-bruennau}
\end{listing}
                
                \subsection{Järkendorf}
                Zugeordnete Routen:
\newline
\newline
\begin{longtabu}{|l|l|l|l|*2{X[l]|}}
    \hline
    id & Quelle & Ziel & Straße & Straßenbeginn & Straßenende\\ 
    \hline
    150 & Järkendorf & Bamberg, Haßfurt & KT39 & Prichsenstadt OT Brünnau & Järkendorf\\ 
    150 &  &  & KT40 & Prichsenstadt OT Neuses & Prichsenstadt OT Brünnau\\ 
    150 &  &  & B22 & Prichsenstadt OT Neuses & B22/KT42 (bei Neudorf)\\ 
    150 &  &  & B22 & B22/KT42 (bei Neudorf) & B22/St2272/SW48\\ 
    150 &  &  & B22 & B22/St2272/SW48 & Breitbach\\ 
    150 &  &  & B22 & Breitbach & Ebrach\\ 
    \hline
    151 & Järkendorf & Schweinfurt & KT39 & Prichsenstadt OT Brünnau & Järkendorf\\ 
    151 &  &  & KT40 & Prichsenstadt OT Neuses & Prichsenstadt OT Brünnau\\ 
    151 &  &  & B286 & Prichsenstadt OT Neuses & B286/St2274 (Gerolzhofen bei Geomaris)\\ 
    151 &  &  & B286 & B286/St2274 (Gerolzhofen bei Geomaris) & B286/St2275 (Gerolzhofen bei Rügshofen)\\ 
    151 &  &  & B286 & B286/St2275 (Gerolzhofen bei Rügshofen) & B286/St2272 (bei Alitzheim)\\ 
    151 &  &  & B286 & B286/St2272 (bei Alitzheim) & B286/St2271 (bei Unterspießheim)\\ 
    151 &  &  & B286 & B286/St2271 (bei Unterspießheim) & B286/St2277 (bei Schwebheim)\\ 
    151 &  &  & B286 & B286/St2277 (bei Schwebheim) & B286/SW3/St2271 (nördlich Schwebheim)\\ 
    151 &  &  & B286 & B286/SW3/St2271 (nördlich Schwebheim) & Ausfahrt Schweinfurt-Zentrum\\ 
    151 &  &  & B286 & Ausfahrt Schweinfurt-Zentrum & B286 Schweinfurt Abfahrt Hans-Böckler-Straße\\ 
    151 &  &  & B286 & B286 Schweinfurt Abfahrt Hans-Böckler-Straße & B286/B26 (Schweinfurt Nähe Mainbrücke)\\ 
    151 &  &  & B286 & B286/B26 (Schweinfurt Nähe Mainbrücke) & B286/B303\\ 
    \hline
    152 & Järkendorf & Gerolzhofen & SW44 & Järkendorf & Lülsfeld\\ 
    152 &  &  & SW44 & Lülsfeld & Frankenwinheim\\ 
    152 &  &  & St2274 & Frankenwinheim & St2274/SW37\\ 
    152 &  &  & St2274 & St2274/SW37 & Gerolzhofen\\ 
    \hline
    153 & Järkendorf & Lülsfeld & SW44 & Järkendorf & Lülsfeld\\ 
    \hline
    154 & Järkendorf & Wiesentheid & KT39 & Prichsenstadt OT Brünnau & Järkendorf\\ 
    154 &  &  & KT40 & Prichsenstadt OT Neuses & Prichsenstadt OT Brünnau\\ 
    154 &  &  & B286 & B286/St2272 (Wiesentheid bei Blutbank) & Prichsenstadt OT Neuses\\ 
    154 &  &  & St2272 & B286/St2272 (Wiesentheid bei Blutbank) & Wiesentheid\\ 
    \hline
    155 & Järkendorf & Kitzingen & KT38 & Järkendorf & Stadelschwarzach\\ 
    155 &  &  & B22 & Prichsenstadt OT Laub & Stadelschwarzach\\ 
    155 &  &  & B22 & Wiesentheid OT Reupelsdorf & Prichsenstadt OT Laub\\ 
    155 &  &  & B22 & Düllstadt & Wiesentheid OT Reupelsdorf\\ 
    155 &  &  & B22 & B22/KT11 & Düllstadt\\ 
    155 &  &  & B22 & B22/St2271 (bei Stadtschwarzach) & B22/KT11\\ 
    155 &  &  & St2271 & B22/St2271 (bei Stadtschwarzach) & Hörblach\\ 
    155 &  &  & St2271 & Hörblach & Ausfahrt Kitzingen-Schwarzach\\ 
    155 &  &  & St2271 & Ausfahrt Kitzingen-Schwarzach & St2271/St2272 (bei Kitzingen-Etwashausen)\\ 
    155 &  &  & St2271 & St2271/St2272 (bei Kitzingen-Etwashausen) & B8/St2271 (Kitzingen bei e-center)\\ 
    155 &  &  & B8 & Kitzingen & B8/St2271 (Kitzingen bei e-center)\\ 
    \hline
    156 & Järkendorf & Würzburg, Rottendorf & KT38 & Järkendorf & Stadelschwarzach\\ 
    156 &  &  & B22 & Stadelschwarzach & Prichsenstadt OT Neuses\\ 
    156 &  &  & B22 & Prichsenstadt OT Laub & Stadelschwarzach\\ 
    156 &  &  & B22 & Wiesentheid OT Reupelsdorf & Prichsenstadt OT Laub\\ 
    156 &  &  & B22 & Düllstadt & Wiesentheid OT Reupelsdorf\\ 
    156 &  &  & B22 & B22/KT11 & Düllstadt\\ 
    156 &  &  & B22 & B22/St2271 (bei Stadtschwarzach) & B22/KT11\\ 
    156 &  &  & St2271 & B22/St2271 (bei Stadtschwarzach) & Hörblach\\ 
    156 &  &  & St2271 & Hörblach & Ausfahrt Kitzingen-Schwarzach\\ 
    156 &  &  & A3 & Autobahnkreuz Biebelried & Ausfahrt Kitzingen-Schwarzach\\ 
    156 &  &  & A3 & Ausfahrt Würzburg-Biebelried & Autobahnkreuz Biebelried\\ 
    156 &  &  & B8 & B8/B22 (am Mainfrankenpark) & Ausfahrt Würzburg-Biebelried\\ 
    156 &  &  & B8 & Ausfahrt Rottendorf & B8/B22 (am Mainfrankenpark)\\ 
    156 &  &  & B8 & Ausfahrt Wöllriederhof & Ausfahrt Rottendorf\\ 
    156 &  &  & B8 & Ausfahrt Würzburg/Nürnberger Straße & Ausfahrt Wöllriederhof\\ 
    156 &  &  & B8 & Würzburg B8/B19 (Grainbergknoten) & Ausfahrt Würzburg/Nürnberger Straße\\ 
    \hline
    157 & Järkendorf & Nürnberg, Erlangen & KT39 & Prichsenstadt OT Brünnau & Järkendorf\\ 
    157 &  &  & KT40 & Prichsenstadt OT Neuses & Prichsenstadt OT Brünnau\\ 
    157 &  &  & B286 & B286/St2272 (Wiesentheid bei Blutbank) & Prichsenstadt OT Neuses\\ 
    157 &  &  & B286 & Wiesentheid Kreuzung Gewerbegebiet Althölzl & B286/St2272 (Wiesentheid bei Blutbank)\\ 
    157 &  &  & B286 & Ausfahrt Wiesentheid & Wiesentheid Kreuzung Gewerbegebiet Althölzl\\ 
    157 &  &  & A3 & Ausfahrt Wiesentheid & Ausfahrt Geiselwind\\ 
    157 &  &  & A3 & Ausfahrt Geiselwind & Ausfahrt Schlüsselfeld\\ 
    157 &  &  & A3 & Ausfahrt Schlüsselfeld & Ausfahrt Höchstadt-Nord\\ 
    157 &  &  & A3 & Ausfahrt Höchstadt-Nord & Ausfahrt Pommersfelden\\ 
    157 &  &  & A3 & Ausfahrt Pommersfelden & Ausfahrt Höchstadt-Ost\\ 
    157 &  &  & A3 & Ausfahrt Höchstadt-Ost & Ausfahrt Erlangen-West\\ 
    157 &  &  & A3 & Ausfahrt Erlangen-West & Ausfahrt Erlangen-Frauenaurach\\ 
    157 &  &  & A3 & Ausfahrt Erlangen-Frauenaurach & Autobahnkreuz Fürth-Erlangen\\ 
    \hline
\end{longtabu}

\begin{listing}[htbp]
\begin{minted}{sql}
    SELECT 
	`potentials`.`id` AS `id`,
	`from_places`.`name` AS `Quelle`, 
	`to_places`.`name` AS `Ziel`,
	`streets`.`street` AS `Straße`,
	`from_street_places`.`name` AS `Straßenbeginn`,
   `to_street_places`.`name` AS `Straßenende`
FROM `potentials`
LEFT JOIN `places` AS `from_places` ON `potentials`.`from_id` = `from_places`.`id`
LEFT JOIN `places` AS `to_places` ON `potentials`.`to_id` = `to_places`.`id`
LEFT JOIN `routes` ON `routes`.`potential_id` = `potentials`.`id`
LEFT JOIN `streets` ON `streets`.`id` = `routes`.`street_id`
LEFT JOIN `places` AS `from_street_places` ON `streets`.`from_id` = `from_street_places`.`id`
LEFT JOIN `places` AS `to_street_places` ON `streets`.`to_id` = `to_street_places`.`id`
WHERE `from_places`.`name` = 'Järkendorf'
ORDER BY `potentials`.`id`, `routes`.`number_on_route`;
\end{minted}
\caption{SQL-Abfrage der zugeordneten Straßen mit der Quelle Järkendorf}\label{lst-rt-jaerkendorf}
\end{listing}


Länge, Fahrzeiten und Google Maps:
\newline
\begin{longtabu}{| l | *5{X[l]|}}
    \hline
    id & Quelle & Ziel & Fahrtstrecke [m] & Fahrtdauer [min] & Google-Maps Link\\ 
    \hline
    150 & Järkendorf & Bamberg, Haßfurt & 47500 & 45 & \url{https://www.google.com/maps/dir/49.8522178,10.3290937/49.8912678,10.8865984}\\ 
    \hline
    151 & Järkendorf & Schweinfurt & 31300 & 28 & \url{https://www.google.com/maps/dir/49.8522178,10.3290937/50.0439484,10.2257843}\\ 
    \hline
    152 & Järkendorf & Gerolzhofen & 7400 & 10 & \url{https://www.google.com/maps/dir/49.8522178,10.3290937/49.9010511,10.3489622}\\ 
    \hline
    153 & Järkendorf & Lülsfeld & 2000 & 3 & \url{https://www.google.com/maps/dir/49.8522178,10.3290937/49.8677403,10.3199678}\\ 
    \hline
    154 & Järkendorf & Wiesentheid & 8400 & 9 & \url{https://www.google.com/maps/dir/49.8522178,10.3290937/49.7942401,10.3426344}\\ 
    \hline
    155 & Järkendorf & Kitzingen & 21100 & 21 & \url{https://www.google.com/maps/dir/49.8522178,10.3290937/49.7355709,10.1617438}\\ 
    \hline
    156 & Järkendorf & Würzburg, Rottendorf & 37600 & 36 & \url{https://www.google.com/maps/dir/49.8522178,10.3290937/49.7931,9.9280108}\\ 
    \hline
    157 & Järkendorf & Nürnberg, Erlangen & 72900 & 48 & \url{https://www.google.com/maps/dir/49.8522178,10.3290937/49.5598096,10.9916482}\\ 
    \hline
\end{longtabu}

\begin{listing}[htbp]
    \begin{minted}{sql}
        SELECT 
        `potentials`.`id` AS `id`, 
        `from_places`.`name` AS `Quelle`,
        `to_places`.`name` AS `Ziel`, 
        `potentials`.`length` AS `Fahrtstrecke [m]`, 
        `potentials`.`miv-duration` AS `Fahrtdauer [min]`,
        CONCAT('https://www.google.com/maps/dir/', `from_places`.`LAT`, ",", `from_places`.`LONG`, '/', `to_places`.`LAT`, ',', `to_places`.`LONG`) AS `Google-Maps Link`
    FROM `potentials`
    LEFT JOIN `places` AS `from_places` ON `potentials`.`from_id` = `from_places`.`id`
    LEFT JOIN `places` AS `to_places` ON `potentials`.`to_id` = `to_places`.`id`
    WHERE `from_places`.`name` = 'Järkendorf'
    ORDER BY `potentials`.`id`;
    \end{minted}
    \caption{SQL-Abfrage der Fahrtstrecke, Fahrtdauer und des Google-Maps-Link mit der Quelle Järkendorf}\label{lst-f-jaerkendorf}
\end{listing}
                
                \subsection{Prichsenstadt OT Kirchschönbach}
                \begin{tabularx}{\textwidth}{*5{X}}
Quelle & Ziel & NettoPotenzial & MIV-Veränderung & Potenzial-ID\\ 
Prichsenstadt OT Kirchschönbach & Bamberg, Haßfurt & 1 & -1 & 158\\ 
Prichsenstadt OT Kirchschönbach & Schweinfurt & 2 & -3 & 159\\ 
Prichsenstadt OT Kirchschönbach & Gerolzhofen & 2 & -3 & 160\\ 
Prichsenstadt OT Kirchschönbach & Lülsfeld & 3 & -4 & 161\\ 
Prichsenstadt OT Kirchschönbach & Kitzingen & 10 & -16 & 162\\ 
Prichsenstadt OT Kirchschönbach & Würzburg, Rottendorf & 5 & -8 & 163\\ 
Prichsenstadt OT Kirchschönbach & Nürnberg, Erlangen & 1 & -1 & 164\\ 
Prichsenstadt OT Kirchschönbach & Prichsenstadt & * & 60 & 165\\ 
\end{tabularx}
\newline
\newline
* Neue Verkehre um den Bahnhof zu erreichen.
\newline
\begin{listing}[htbp]
\begin{minted}{sql}
SELECT
`from_places`.`name` AS `Quelle`, 
`to_places`.`name` AS `Ziel`, 
`potentials`.`netto` AS `NettoPotenzial`, 
`potentials`.`miv-change` AS `MIV-Veränderung`, 
`potentials`.`id` AS `Potenzial-ID`
FROM `potentials`
LEFT JOIN `places` `from_places` ON `from_places`.`id` = `potentials`.`from_id`
LEFT JOIN `places` `to_places` ON `to_places`.`id` = `potentials`.`to_id`
WHERE `from_places`.`name` = "Prichsenstadt OT Kirchschönbach";
\end{minted}
\caption{SQL-Abfrage der Netto-Potenziale und MIV-Veränderung mit der Quelle Kirchschönbach}\label{lst-fz-kirchschoenbach}
\end{listing}
                
                \subsection{Prichsenstadt OT Laub}
                \begin{tabularx}{\textwidth}{*5{X}}
Quelle & Ziel & NettoPotenzial & MIV-Veränderung & Potenzial-ID\\ 
Prichsenstadt OT Laub & Bamberg, Haßfurt & 1 & -1 & 166\\ 
Prichsenstadt OT Laub & Schweinfurt & 1 & -1 & 167\\ 
Prichsenstadt OT Laub & Gerolzhofen & 1 & -1 & 168\\ 
Prichsenstadt OT Laub & Lülsfeld & 2 & -3 & 169\\ 
Prichsenstadt OT Laub & Kitzingen & 7 & -11 & 170\\ 
Prichsenstadt OT Laub & Würzburg, Rottendorf & 3 & -4 & 171\\ 
Prichsenstadt OT Laub & Nürnberg, Erlangen & 1 & -1 & 172\\ 
Prichsenstadt OT Laub & Prichsenstadt & * & 28 & 173\\ 
Prichsenstadt OT Laub & Stadelschwarzach & * & 13 & 174\\ 
\end{tabularx}    
\newline
\newline
* Neue Verkehre um den Bahnhof zu erreichen.
\newline
\begin{listing}[htbp]
\begin{minted}{sql}
SELECT
`from_places`.`name` AS `Quelle`, 
`to_places`.`name` AS `Ziel`, 
`potentials`.`netto` AS `NettoPotenzial`, 
`potentials`.`miv-change` AS `MIV-Veränderung`, 
`potentials`.`id` AS `Potenzial-ID`
FROM `potentials`
LEFT JOIN `places` `from_places` ON `from_places`.`id` = `potentials`.`from_id`
LEFT JOIN `places` `to_places` ON `to_places`.`id` = `potentials`.`to_id`
WHERE `from_places`.`name` = "Prichsenstadt OT Laub";
\end{minted}
\caption{SQL-Abfrage der Netto-Potenziale und MIV-Veränderung mit der Quelle Laub}\label{lst-fz-laub}
\end{listing}
                
                \subsection{Prichsenstadt OT Neudorf}
                \begin{tabular}{ l  l  l  l  l }
Quelle & Ziel & NettoPotenzial & MIV-Veränderung & Potenzial-ID\\ 
Prichsenstadt OT Neudorf & Schweinfurt & 1 & -1 & 175\\ 
Prichsenstadt OT Neudorf & Gerolzhofen & 1 & -1 & 176\\ 
Prichsenstadt OT Neudorf & Lülsfeld & 1 & -1 & 177\\ 
Prichsenstadt OT Neudorf & Wiesentheid & 5 & -8 & 178\\ 
Prichsenstadt OT Neudorf & Kitzingen & 3 & -4 & 179\\ 
Prichsenstadt OT Neudorf & Würzburg, Rottendorf & 1 & -1 & 180\\ 
Prichsenstadt OT Neudorf & Stadelschwarzach & * & 30 & 181\\ 
\end{tabular}
\newline
\newline
* Neue Verkehre um den Bahnhof zu erreichen.
\newline
\begin{listing}[htbp]
\begin{minted}{sql}
SELECT
`from_places`.`name` AS `Quelle`, 
`to_places`.`name` AS `Ziel`, 
`potentials`.`netto` AS `NettoPotenzial`, 
`potentials`.`miv-change` AS `MIV-Veränderung`, 
`potentials`.`id` AS `Potenzial-ID`
FROM `potentials`
LEFT JOIN `places` `from_places` ON `from_places`.`id` = `potentials`.`from_id`
LEFT JOIN `places` `to_places` ON `to_places`.`id` = `potentials`.`to_id`
WHERE `from_places`.`name` = "Prichsenstadt OT Neudorf";
\end{minted}
\caption{SQL-Abfrage der Netto-Potenziale und MIV-Veränderung mit der Quelle Neudorf}\label{lst-fz-neudorf}
\end{listing}
                
                \subsection{Prichsenstadt OT Neuses}
                \begin{tabularx}{\textwidth}{*5{X}}
Quelle & Ziel & NettoPotenzial & MIV-Veränderung & Potenzial-ID\\ 
Prichsenstadt OT Neuses & Lülsfeld & 1 & -1 & 182\\ 
Prichsenstadt OT Neuses & Wiesentheid & 5 & -8 & 183\\ 
Prichsenstadt OT Neuses & Kitzingen & 3 & -4 & 184\\ 
Prichsenstadt OT Neuses & Würzburg, Rottendorf & 1 & -1 & 185\\ 
Prichsenstadt OT Neuses & Stadelschwarzach & * & 25 & 186\\ 
\end{tabularx}
\newline
\newline
* Neue Verkehre um den Bahnhof zu erreichen.
\newline
\begin{listing}[htbp]
\begin{minted}{sql}
SELECT
`from_places`.`name` AS `Quelle`, 
`to_places`.`name` AS `Ziel`, 
`potentials`.`netto` AS `NettoPotenzial`, 
`potentials`.`miv-change` AS `MIV-Veränderung`, 
`potentials`.`id` AS `Potenzial-ID`
FROM `potentials`
LEFT JOIN `places` `from_places` ON `from_places`.`id` = `potentials`.`from_id`
LEFT JOIN `places` `to_places` ON `to_places`.`id` = `potentials`.`to_id`
WHERE `from_places`.`name` = "Prichsenstadt OT Neuses";
\end{minted}
\caption{SQL-Abfrage der Netto-Potenziale und MIV-Veränderung mit der Quelle Neuses}\label{lst-fz-neuses}
\end{listing}
                
                \subsection{Prichsenstadt OT Stadelschwarzach}
                \begin{tabularx}{\textwidth}{*5{X}}
Quelle & Ziel & NettoPotenzial & MIV-Veränderung & Potenzial-ID\\ 
Stadelschwarzach & Bamberg, Haßfurt & 2 & -3 & 187\\ 
Stadelschwarzach & Schweinfurt & 5 & -8 & 188\\ 
Stadelschwarzach & Gerolzhofen & 5 & -8 & 189\\ 
Stadelschwarzach & Lülsfeld & 7 & -11 & 190\\ 
Stadelschwarzach & Wiesentheid & 41 & -65 & 191\\ 
Stadelschwarzach & Kitzingen & 26 & -41 & 192\\ 
Stadelschwarzach & Würzburg, Rottendorf & 6 & -9 & 193\\ 
Stadelschwarzach & Nürnberg, Erlangen & 2 & -3 & 194\\ 
\end{tabularx}    
\newline
\newline
\begin{listing}[htbp]
\begin{minted}{sql}
SELECT
`from_places`.`name` AS `Quelle`, 
`to_places`.`name` AS `Ziel`, 
`potentials`.`netto` AS `NettoPotenzial`, 
`potentials`.`miv-change` AS `MIV-Veränderung`, 
`potentials`.`id` AS `Potenzial-ID`
FROM `potentials`
LEFT JOIN `places` `from_places` ON `from_places`.`id` = `potentials`.`from_id`
LEFT JOIN `places` `to_places` ON `to_places`.`id` = `potentials`.`to_id`
WHERE `from_places`.`name` = "Stadelschwarzach";
\end{minted}
\caption{SQL-Abfrage der Netto-Potenziale und MIV-Veränderung mit der Quelle Stadelschwarzach}\label{lst-fz-stadelschwarzach}
\end{listing}
                
                \subsection{Wiesentheid}
                \begin{tabular}{ l  l  l  l  l }
Quelle & Ziel & NettoPotenzial & MIV-Veränderung & Potenzial-ID\\ 
Wiesentheid & Schweinfurt & 22 & -35 & 195\\ 
Wiesentheid & Gerolzhofen & 19 & -30 & 196\\ 
Wiesentheid & Prichsenstadt & 35 & -56 & 197\\ 
Wiesentheid & Kleinlangheim & 7 & -11 & 198\\ 
Wiesentheid & Kitzingen & 172 & -275 & 199\\ 
Wiesentheid & Würzburg, Rottendorf & 52 & -83 & 200\\ 
Wiesentheid & Nürnberg, Erlangen & 3 & -4 & 201\\ 
\end{tabular}    
\newline
\newline
\begin{listing}[htbp]
\begin{minted}{sql}
SELECT
`from_places`.`name` AS `Quelle`, 
`to_places`.`name` AS `Ziel`, 
`potentials`.`netto` AS `NettoPotenzial`, 
`potentials`.`miv-change` AS `MIV-Veränderung`, 
`potentials`.`id` AS `Potenzial-ID`
FROM `potentials`
LEFT JOIN `places` `from_places` ON `from_places`.`id` = `potentials`.`from_id`
LEFT JOIN `places` `to_places` ON `to_places`.`id` = `potentials`.`to_id`
WHERE `from_places`.`name` = "Wiesentheid";
\end{minted}
\caption{SQL-Abfrage der Netto-Potenziale und MIV-Veränderung mit der Quelle Wiesentheid}\label{lst-fz-wiesentheid}
\end{listing}
                
                \subsection{Wiesentheid OT Feuerbach}
                \begin{tabular}{ l  l  l  l  l }
Quelle & Ziel & NettoPotenzial & MIV-Veränderung & Potenzial-ID\\ 
Wiesentheid OT Feuerbach & Schweinfurt & 2 & -3 & 202\\ 
Wiesentheid OT Feuerbach & Gerolzhofen & 2 & -3 & 203\\ 
Wiesentheid OT Feuerbach & Prichsenstadt & 3 & -4 & 204\\ 
Wiesentheid OT Feuerbach & Kleinlangheim & 1 & -1 & 205\\ 
Wiesentheid OT Feuerbach & Kitzingen & 13 & -20 & 206\\ 
Wiesentheid OT Feuerbach & Würzburg, Rottendorf & 4 & -6 & 207\\ 
Wiesentheid OT Feuerbach & Nürnberg, Erlangen & 1 & -1 & 208\\ 
\end{tabular}
\newline
\newline
\begin{listing}[htbp]
\begin{minted}{sql}
SELECT
`from_places`.`name` AS `Quelle`, 
`to_places`.`name` AS `Ziel`, 
`potentials`.`netto` AS `NettoPotenzial`, 
`potentials`.`miv-change` AS `MIV-Veränderung`, 
`potentials`.`id` AS `Potenzial-ID`
FROM `potentials`
LEFT JOIN `places` `from_places` ON `from_places`.`id` = `potentials`.`from_id`
LEFT JOIN `places` `to_places` ON `to_places`.`id` = `potentials`.`to_id`
WHERE `from_places`.`name` = "Wiesentheid OT Feuerbach";
\end{minted}
\caption{SQL-Abfrage der Netto-Potenziale und MIV-Veränderung mit der Quelle Feuerbach}\label{lst-fz-feuerbach}
\end{listing}
                
                \subsection{Wiesentheid OT Geesdorf}
                \begin{tabular}{ l  l  l  l  l }
Quelle & Ziel & NettoPotenzial & MIV-Veränderung & Potenzial-ID\\ 
Wiesentheid OT Geesdorf & Schweinfurt & 1 & -1 & 209\\ 
Wiesentheid OT Geesdorf & Gerolzhofen & 1 & -1 & 210\\ 
Wiesentheid OT Geesdorf & Kitzingen & 1 & -1 & 211\\ 
Wiesentheid OT Geesdorf & Würzburg, Rottendorf & 6 & -9 & 212\\ 
Wiesentheid OT Geesdorf & Wiesentheid & * & 43 & 213\\ 
\end{tabular}    
\newline
\newline
* Neue Verkehre um den Bahnhof zu erreichen.
\newline
\begin{listing}[htbp]
\begin{minted}{sql}
SELECT
`from_places`.`name` AS `Quelle`, 
`to_places`.`name` AS `Ziel`, 
`potentials`.`netto` AS `NettoPotenzial`, 
`potentials`.`miv-change` AS `MIV-Veränderung`, 
`potentials`.`id` AS `Potenzial-ID`
FROM `potentials`
LEFT JOIN `places` `from_places` ON `from_places`.`id` = `potentials`.`from_id`
LEFT JOIN `places` `to_places` ON `to_places`.`id` = `potentials`.`to_id`
WHERE `from_places`.`name` = "Wiesentheid OT Geesdorf";
\end{minted}
\caption{SQL-Abfrage der Netto-Potenziale und MIV-Veränderung mit der Quelle Geesdorf}\label{lst-fz-geesdorf}
\end{listing}
                
                \subsection{Wiesentheid OT Reupelsdorf}
                \begin{tabularx}{\textwidth}{*5{X}}
Quelle & Ziel & NettoPotenzial & MIV-Veränderung & Potenzial-ID\\ 
Wiesentheid OT Reupelsdorf & Schweinfurt & 1 & -1 & 214\\ 
Wiesentheid OT Reupelsdorf & Gerolzhofen & 1 & -1 & 215\\ 
Wiesentheid OT Reupelsdorf & Kleinlangheim & 1 & -1 & 216\\ 
Wiesentheid OT Reupelsdorf & Kitzingen & 9 & -14 & 217\\ 
Wiesentheid OT Reupelsdorf & Würzburg, Rottendorf & 5 & -8 & 218\\ 
Wiesentheid OT Reupelsdorf & Nürnberg, Erlangen & 1 & -1 & 219\\ 
Wiesentheid OT Reupelsdorf & Stadelschwarzach & * & 5 & 220\\ 
Wiesentheid OT Reupelsdorf & Wiesentheid & * & 40 & 221\\ 
\end{tabularx}
\newline
\newline
* Neue Verkehre um den Bahnhof zu erreichen.
\newline
\begin{listing}[htbp]
\begin{minted}{sql}
SELECT
`from_places`.`name` AS `Quelle`, 
`to_places`.`name` AS `Ziel`, 
`potentials`.`netto` AS `NettoPotenzial`, 
`potentials`.`miv-change` AS `MIV-Veränderung`, 
`potentials`.`id` AS `Potenzial-ID`
FROM `potentials`
LEFT JOIN `places` `from_places` ON `from_places`.`id` = `potentials`.`from_id`
LEFT JOIN `places` `to_places` ON `to_places`.`id` = `potentials`.`to_id`
WHERE `from_places`.`name` = "Wiesentheid OT Reupelsdorf";
\end{minted}
\caption{SQL-Abfrage der Netto-Potenziale und MIV-Veränderung mit der Quelle Reupelsdorf}\label{lst-fz-reupelsdorf}
\end{listing}
                
                \subsection{Wiesentheid OT Untersambach}
                \begin{tabular}{ l  l  l  l  l }
Quelle & Ziel & NettoPotenzial & MIV-Veränderung & Potenzial-ID\\ 
Wiesentheid OT Untersambach & Schweinfurt & 1 & -1 & 222\\ 
Wiesentheid OT Untersambach & Gerolzhofen & 1 & -1 & 223\\ 
Wiesentheid OT Untersambach & Kitzingen & 7 & -11 & 224\\ 
Wiesentheid OT Untersambach & Würzburg, Rottendorf & 4 & -6 & 225\\ 
Wiesentheid OT Untersambach & Wiesentheid & * & 33 & 226\\ 
\end{tabular}
\newline
\newline
* Neue Verkehre um den Bahnhof zu erreichen.
\newline
\begin{listing}[htbp]
\begin{minted}{sql}
SELECT
`from_places`.`name` AS `Quelle`, 
`to_places`.`name` AS `Ziel`, 
`potentials`.`netto` AS `NettoPotenzial`, 
`potentials`.`miv-change` AS `MIV-Veränderung`, 
`potentials`.`id` AS `Potenzial-ID`
FROM `potentials`
LEFT JOIN `places` `from_places` ON `from_places`.`id` = `potentials`.`from_id`
LEFT JOIN `places` `to_places` ON `to_places`.`id` = `potentials`.`to_id`
WHERE `from_places`.`name` = "Wiesentheid OT Untersambach";
\end{minted}
\caption{SQL-Abfrage der Netto-Potenziale und MIV-Veränderung mit der Quelle Untersambach}\label{lst-fz-untersambach}
\end{listing}
                
                \subsection{Rüdenhausen}
                \begin{tabular}{ l  l  l  l  l }
Quelle & Ziel & NettoPotenzial & MIV-Veränderung & Potenzial-ID\\ 
Rüdenhausen & Schweinfurt & 3 & -4 & 227\\ 
Rüdenhausen & Kitzingen & 21 & -57 & 228\\ 
Rüdenhausen & Würzburg, Rottendorf & 12 & -19 & 229\\ 
Rüdenhausen & Wiesentheid OT Feuerbach & * & 83 & 230\\ 
Rüdenhausen & Wiesentheid & * & 8 & 231\\ 
\end{tabular}
\newline
\newline
* Neue Verkehre um den Bahnhof zu erreichen.
\newline
\begin{listing}[htbp]
\begin{minted}{sql}
SELECT
`from_places`.`name` AS `Quelle`, 
`to_places`.`name` AS `Ziel`, 
`potentials`.`netto` AS `NettoPotenzial`, 
`potentials`.`miv-change` AS `MIV-Veränderung`, 
`potentials`.`id` AS `Potenzial-ID`
FROM `potentials`
LEFT JOIN `places` `from_places` ON `from_places`.`id` = `potentials`.`from_id`
LEFT JOIN `places` `to_places` ON `to_places`.`id` = `potentials`.`to_id`
WHERE `from_places`.`name` = "Rüdenhausen";
\end{minted}
\caption{SQL-Abfrage der Netto-Potenziale und MIV-Veränderung mit der Quelle Rüdenhausen}\label{lst-fz-ruedenhausen}
\end{listing}
                
                \subsection{Abtswind}
                \input{Routen/abtswind.tex}
                
                \subsection{Kleinlangheim}
                \begin{tabular}{ l  l  l  l  l }
Quelle & Ziel & NettoPotenzial & MIV-Veränderung & Potenzial-ID\\ 
Kleinlangheim & Schweinfurt & 4 & -6 & 235\\ 
Kleinlangheim & Gerolzhofen & 5 & -8 & 236\\ 
Kleinlangheim & Wiesentheid & 25 & -40 & 237\\ 
Kleinlangheim & Kitzingen & 321 & -513 & 238\\ 
Kleinlangheim & Würzburg, Rottendorf & 5 & -8 & 239\\ 
\end{tabular}    
\newline
\newline
\begin{listing}[htbp]
\begin{minted}{sql}
SELECT
`from_places`.`name` AS `Quelle`, 
`to_places`.`name` AS `Ziel`, 
`potentials`.`netto` AS `NettoPotenzial`, 
`potentials`.`miv-change` AS `MIV-Veränderung`, 
`potentials`.`id` AS `Potenzial-ID`
FROM `potentials`
LEFT JOIN `places` `from_places` ON `from_places`.`id` = `potentials`.`from_id`
LEFT JOIN `places` `to_places` ON `to_places`.`id` = `potentials`.`to_id`
WHERE `from_places`.`name` = "Kleinlangheim";
\end{minted}
\caption{SQL-Abfrage der Netto-Potenziale und MIV-Veränderung mit der Quelle Kleinlangheim}\label{lst-fz-kleinlangheim}
\end{listing}
                
                \subsection{Wiesenbronn}
                \begin{tabular}{ l  l  l  l  l }
Quelle & Ziel & NettoPotenzial & MIV-Veränderung & Potenzial-ID\\ 
Wiesenbronn & Schweinfurt & 3 & -4 & 240\\ 
Wiesenbronn & Gerolzhofen & 2 & -3 & 241\\ 
Wiesenbronn & Würzburg, Rottendorf & 8 & -12 & 242\\ 
Wiesenbronn & Kleinlangheim & * & 13 & 243\\ 
Wiesenbronn & Großlangheim & * & 20 & 244\\ 
\end{tabular}
\newline
\newline
* Neue Verkehre um den Bahnhof zu erreichen.
\newline
\begin{listing}[htbp]
\begin{minted}{sql}
SELECT
`from_places`.`name` AS `Quelle`, 
`to_places`.`name` AS `Ziel`, 
`potentials`.`netto` AS `NettoPotenzial`, 
`potentials`.`miv-change` AS `MIV-Veränderung`, 
`potentials`.`id` AS `Potenzial-ID`
FROM `potentials`
LEFT JOIN `places` `from_places` ON `from_places`.`id` = `potentials`.`from_id`
LEFT JOIN `places` `to_places` ON `to_places`.`id` = `potentials`.`to_id`
WHERE `from_places`.`name` = "Wiesenbronn";
\end{minted}
\caption{SQL-Abfrage der Netto-Potenziale und MIV-Veränderung mit der Quelle Wiesenbronn}\label{lst-fz-wiesenbronn}
\end{listing}
                
                \subsection{Großlangheim}
                \begin{tabular}{ l  l  l  l  l }
Quelle & Ziel & NettoPotenzial & MIV-Veränderung & Potenzial-ID\\ 
Großlangheim & Schweinfurt & 5 & -8 & 245\\ 
Großlangheim & Wiesentheid & 7 & -11 & 246\\ 
Großlangheim & Kitzingen & 313 & -500 & 247\\ 
Großlangheim & Würzburg, Rottendorf & 25 & -40 & 248\\ 
\end{tabular}    
\newline
\newline
\begin{listing}[htbp]
\begin{minted}{sql}
SELECT
`from_places`.`name` AS `Quelle`, 
`to_places`.`name` AS `Ziel`, 
`potentials`.`netto` AS `NettoPotenzial`, 
`potentials`.`miv-change` AS `MIV-Veränderung`, 
`potentials`.`id` AS `Potenzial-ID`
FROM `potentials`
LEFT JOIN `places` `from_places` ON `from_places`.`id` = `potentials`.`from_id`
LEFT JOIN `places` `to_places` ON `to_places`.`id` = `potentials`.`to_id`
WHERE `from_places`.`name` = "Großlangheim";
\end{minted}
\caption{SQL-Abfrage der Netto-Potenziale und MIV-Veränderung mit der Quelle Großlangheim}\label{lst-fz-grosslangheim}
\end{listing}
                
                \subsection{Kitzingen}
                \begin{tabular}{ l  l  l  l  l }
Quelle & Ziel & NettoPotenzial & MIV-Veränderung & Potenzial-ID\\ 
Kitzingen & Schweinfurt & 64 & -102 & 249\\ 
Kitzingen & Gochsheim & 9 & -14 & 250\\ 
Kitzingen & Gerolzhofen & 11 & -17 & 251\\ 
Kitzingen & Prichsenstadt & 18 & -28 & 252\\ 
Kitzingen & Wiesentheid & 49 & -78 & 253\\ 
Kitzingen & Kleinlangheim & 10 & -16 & 254\\ 
Kitzingen & Großlangheim & 37 & -59 & 255\\ 
\end{tabular} 
\newline
\newline
\begin{listing}[htbp]
\begin{minted}{sql}
SELECT
`from_places`.`name` AS `Quelle`, 
`to_places`.`name` AS `Ziel`, 
`potentials`.`netto` AS `NettoPotenzial`, 
`potentials`.`miv-change` AS `MIV-Veränderung`, 
`potentials`.`id` AS `Potenzial-ID`
FROM `potentials`
LEFT JOIN `places` `from_places` ON `from_places`.`id` = `potentials`.`from_id`
LEFT JOIN `places` `to_places` ON `to_places`.`id` = `potentials`.`to_id`
WHERE `from_places`.`name` = "Kitzingen";
\end{minted}
\caption{SQL-Abfrage der Netto-Potenziale und MIV-Veränderung mit der Quelle Kitzingen}\label{lst-fz-kitzingen}
\end{listing}
                
                \subsection{Würzburg}
                \begin{tabular}{ l  l  l  l  l }
Quelle & Ziel & NettoPotenzial & MIV-Veränderung & Potenzial-ID\\ 
Würzburg & Sennfeld & 24 & -38 & 256\\ 
Würzburg & Gochsheim & 60 & -96 & 257\\ 
Würzburg & Gerolzhofen & 37 & -59 & 258\\ 
Würzburg & Prichsenstadt & 24 & -38 & 259\\ 
Würzburg & Wiesentheid & 46 & -73 & 260\\ 
\end{tabular}
\newline
\newline
\begin{listing}[htbp]
\begin{minted}{sql}
SELECT
`from_places`.`name` AS `Quelle`, 
`to_places`.`name` AS `Ziel`, 
`potentials`.`netto` AS `NettoPotenzial`, 
`potentials`.`miv-change` AS `MIV-Veränderung`, 
`potentials`.`id` AS `Potenzial-ID`
FROM `potentials`
LEFT JOIN `places` `from_places` ON `from_places`.`id` = `potentials`.`from_id`
LEFT JOIN `places` `to_places` ON `to_places`.`id` = `potentials`.`to_id`
WHERE `from_places`.`name` = "Würzburg";
\end{minted}
\caption{SQL-Abfrage der Netto-Potenziale und MIV-Veränderung mit der Quelle Würzburg}\label{lst-fz-wuerzburg}
\end{listing}

    \chapter{Auswertung}
        \section{vermiedener Gesamtverkehr und lokale Emissionen}


                \paragraph{Vermeidung}
Die Reaktivierung würde werktäglich eine Straßenverkersleistung von ca. 136.797 PKW-Kilometer aus dem Straßennetz herraus nehmen.

\begin{listing}[htbp]
\begin{minted}{sql}
SELECT SUM(t1.gesamtfahrleistung)  
FROM
(SELECT (potentials.`miv-change` * potentials.`length` * 0.001) AS gesamtfahrleistung
FROM potentials
WHERE potentials.`miv-change` < 0
) t1
\end{minted}
\caption{SQL-Abfrage der vermiedenen werktäglichen Straßenverkehrsleistung}\label{lst-verm-werktaeglich}
\end{listing}

\paragraph{Neuinduktionen}
Es entsteht eine Neubelastung von ca. 9.831 PKW-Kilometern täglich.

Diese enststeht vor allem aus den Hol- und Bringverkehr zum nächsten Bahnhof für Ortschaften, welche nicht selbst direkt an der Strecke liegen.

\begin{listing}[htbp]
\begin{minted}{sql}
SELECT SUM(t1.gesamtfahrleistung)  
FROM
(SELECT (potentials.`miv-change` * potentials.`length` * 0.001) AS gesamtfahrleistung
FROM potentials
WHERE potentials.`miv-change` > 0) t1
\end{minted}
\caption{SQL-Abfrage der neu entstehenden werktäglichen Straßenverkehrsleistung}\label{lst-neu-werktaeglich}
\end{listing}

\paragraph{Verkehrssaldo}

Im Saldo bedeutet dies eine werktägliche Verkehrsentlastung von ca. 126.966 PKW-Kilometern auf den Straßen zwischen Bamberg, Nürnberg, Schweinfurt und Würzburg.

\begin{listing}[htbp]
\begin{minted}{sql}
SELECT SUM(t1.gesamtfahrleistung)  
FROM
(SELECT (potentials.`miv-change` * potentials.`length` * 0.001 AS gesamtfahrleistung
FROM potentials) t1
\end{minted}
\caption{SQL-Abfrage des Saldos der werktäglichen Straßenverkehrsleistung}\label{lst-neu-werktaeglich}
\end{listing}

\subsection{vermiedene Verkehrsemissionen}

Aus den vermiedenen Verkehren ergeben sich folgende Emissionensvermeidungen.

\subsubsection{Kohlenstoffdioxid-Emissionen}

Kohlenstoffdioxid ist das haupsächlich bei der motorischen Verbrennung anfallende Gas. Es ist für sich genommen nicht gefährlich, jedoch akkumuliert sich das CO2 aus der Verbrennung fossiler Brennstoffe in der Athmosphäre und trägt damit zum menschengemachten Klimawandel erheblich bei.
Das Schweinfurter Becken und das Steigerwaldvorland sind bereits jetzt vom Klimawandel und ausbleibenden Niederschlägen getroffen, wie man an den sich nicht mehr auffüllenden Grundwasserreserven und dem Kitzinger Doppel-Temperaturrekord von 40,3°C am 5 Juli und 7. August 2015 sehen kann.
Daher haben CO2-Emissionen auch einen direkten Bezug zu der Region und deren Lebensgrundlagen, wie zum Beispiel dem Weinbau.
\newline
\newline
Seit 2020 ist ein Grenzwert von 95gr CO2 / km für alle neu zugelassenen Pkw in Kraft. Das Durchschnittsalter der Fahrzeuge beträgt gemäß Kraftfahr-Bundesamt 9,6 Jahre. Da mit einer Reaktivierung der Strecke nicht in unter 5-10 Jahren zu rechnen ist, dürfte dieser Grenzwert dann \enquote{durchschnittlich} sein. Auch wenn valide Zweifel an der Einhaltung des Grenzwertes in den letzten Jahren durch \enquote{Defeat Devices} in den Neu-Fahrzeugen angebracht erscheinen, verwende ich diesen wert, um die Berechnung fachlich nicht unnötig angreifbar zu machen.
Aus diesem Grund wird mit dem heute neuestem Grenzwert für PKW-Co2-Emissionen die vermiedenen Emissionen berechnet. Das gleiche gilt auch für neu entstehenden Verkehr (zum Beispiel auf dem Weg zu Bahnhöfen).
\newline
\newline
Die Schliephake-Studie geht von Verkehren an normalen Werktagen aus. Die Werktage in Bayern sind kommunal unterschiedlich, zum Beispiel öffnen am 15. August in Kitzingen die Geschäfte und in anderen Gemeinden bleiben diese Geschlossen. Es wird folglich mit 249 \enquote{normalen} Werktagen gerechnet. Für die restlichen Tage wird nur die Hälfte des Verkehrs und somit auch die Hälfte der Entlastung angenommen, auch wenn dieser sehr grobe Ansatz den touristischen Angeboten weder hinsichtlich Tagestouristen noch hinsichtlich Ferientourismus fachlich der Schliephake-Studie in ihrer Feingleidrigkeit annähernd gerecht wird.
\newline
\newline
Die folgende Tabelle Zeigt die vermiedenen Emissionen an Werktagen für das gesamte Jahr, die vermiedenen Emissionen an Nicht-Werktagen für das gesamte Jahr und deren Summe für das ganze Jahr.
\newline
\newline
\begin{tabular}{|l|l|l|}
\hline
CO2 Werktags [t] & CO2 Nicht-Werktags [t] & CO2 Ganzjährig [t]\\ 
\hline
-3.003 & -699 & -3702\\ 
\hline
\end{tabular}        

\begin{listing}[htbp]
\begin{minted}{sql}
SELECT 
ROUND(SUM(t1.gesamtfahrleistung) * 0.000095 * 249) AS `CO2 Werktags [t]`, 
ROUND(SUM(t1.gesamtfahrleistung) * 0.000095 * 0.5 * 116) AS `CO2 Nicht-Werktags [t]`, 
ROUND(
        (SUM(t1.gesamtfahrleistung) * 0.000095 * 249) 
        + (SUM(t1.gesamtfahrleistung) * 0.000095 * 0.5 * 116)
        ) AS `CO2 Ganzjährig [t]`
FROM (
        SELECT potentials.`miv-change` * potentials.`length` * 0.001 AS gesamtfahrleistung
        FROM potentials
) t1
\end{minted}
\caption{SQL-Abfrage der Veränderung der CO2-Emissionen}\label{lst-emmissionen-co2}
\end{listing}

\subsubsection{Emissionen an Kohlenwasserstoffen (HC)}
Kohlenwasserstoffe sind eine Stoffgruppe, aus Kohlenstoffatomen und Wasserstoffatomen zusammen gesetzt ist. Sie entsteht bei der motorischen Verbrennung, da die reale motorische im unterschied zur idealen motorischen Verbrennung nie vollständig verläuft. Die meisten dieser unvollständig verbrannten Restprodukte verbrennen im Katalysator nach dem Motor, daher sind diese Emissionen heutzutage nicht mehr das größte Problem des Straßenverkehrs. Dies betrifft vor allem PKWs mit Fremdzünder / Ottomotor. Die Kohlenwasserstoffe gelten je nach einzelnem Stoff als Krebs-eregent, unweltschädlich und als starke Klimagase. 
\newline
\newline
Ebenfalls wie den CO2-Emissionen verwenden wir die aktuellste Schadstoffregulierung für Neuwagen um die eingesparten Emissionen zu errechnen. Die begründung ist hier analog zu den Ausführungen in der Sektion \enquote{Kohlenstoffdioxid-Emissionen}.
\newline
\newline
Für Euro6-PKW sind 100mg pro gefahrenen Kilometer zulässig. 
\newline
\newline
Jedes Jahr könnte also die Freisetzung von knapp 4 Tonnen Kohlenwasserstoffen im Steigerwaldvorland vermieden werden.
\newline
\newline
Die folgende Tabelle Zeigt die vermiedenen Emissionen an Werktagen für das gesamte Jahr, die vermiedenen Emissionen an Nicht-Werktagen für das gesamte Jahr und deren Summe für das ganze Jahr.
\newline
\newline
\begin{tabular}{|l|l|l|}
\hline
HC Werktags [kg] & HC Nicht-Werktags [kg] & HC Ganzjährig [kg]\\ 
\hline
-3161 & -736 & -3898\\ 
\hline
\end{tabular}

\begin{listing}[htbp]
\begin{minted}{sql}
SELECT 
ROUND(SUM(t1.gesamtfahrleistung) * 0.0001 * 249) AS `HC Werktags [kg]`, 
ROUND(SUM(t1.gesamtfahrleistung) * 0.0001 * 0.5 * 116) AS `HC Nicht-Werktags [kg]`, 
ROUND(
        (SUM(t1.gesamtfahrleistung) * 0.0001 * 249) 
        + (SUM(t1.gesamtfahrleistung) * 0.0001 * 0.5 * 116)
) AS `HC Ganzjährig [kg]`
FROM
(
        SELECT potentials.`miv-change` * potentials.`length` * 0.001 AS gesamtfahrleistung
        FROM potentials
) t1
\end{minted}
\caption{SQL-Abfrage der Veränderung der HC-Emissionen}\label{lst-emmissionen-hc}
\end{listing}

\subsubsection{Stickstoffoxid-Emissionen (NOx)}

Stickstoffoxid ist eine Sammelbezeichnung für verschiedene gasförmige Verbindungen, die aus den Atomen Stickstoff (N) und Sauerstoff (O) aufgebaut sind.
\newline
\newline
Stickstoffoxide sind ein anhaltendes Problem, da nach aktuellen Regulierungen die Grenzwerte in der Außenluft mit 40 µg/m³ nur an 18 Tagen je Jahr überschritten werden dürfen. Diese Grenzwertüberschreitungen treten also vor allem dort auf, wo der Straßenverkehr sehr gebündelt verläuft. Die Deutsche Umwelthilfe hat wegen lokaler Grenzwertüberschreitungen etliche Städte auf \enquote{Luftreinhaltung} verklagt und regelmäßig damit vor Gericht Erfolg. Die verringerung des Verkehrs durch die Reaktivierung der Steigerwaldbahn kann auch zur Verringerung der NOx-Emissionen beitragen und somit für Schweinfurt und Würzburg ein effektiver Baustein in einem \enquote{Luftreinhalteplan} sein und das Risiko auf \enquote{Luftreinhaltung} verklagt zu werden, abmildern.
\newline
\newline
Stickstoffoxiden wird vor allem eine langsame Schädigung der Lunge als gesundheitliche Folge aus langer, häufiger und grenzwert-überschreitender Exponation zugeschrieben.
\newline
\newline
Die Festlegung eines Wertes zur Berechnung vermiedenen Emissionen erfolgt analog zu den Ausführungen bei den CO2-Emissionen. Neu zugelassene Fahrzeuge dürfen nach Euro6-Norm lediglich 80 mg pro Gefahrenen Kilometer bei einem Selbstzünder-Motor und 60 mg pro gefahrenen Kilometer bei einem Fremdzünder-Motor emittieren. Daraus errechnet sich mit dem aktuellen Verhältnis der Neuzulassungen von Fremd- und Selbstzündern ein Schnitt von 73 mg pro gefahrenem Kilometer.
\newline
\newline
Die folgende Tabelle Zeigt die vermiedenen Emissionen an Werktagen für das gesamte Jahr, die vermiedenen Emissionen an Nicht-Werktagen für das gesamte Jahr und deren Summe für das ganze Jahr.
\newline
\newline
\begin{tabular}{|l | l | l |}
\hline
NOx Werktags [kg] & NOx Nicht-Werktags [kg] & NOx Ganzjährig [kg]\\ 
\hline
-2308 & -538 & -2845\\ 
\hline
\end{tabular}

\begin{listing}[htbp]
\begin{minted}{sql}
SELECT 
ROUND(SUM(t1.gesamtfahrleistung) * 0.000073 * 249) AS `NOx Werktags [kg]`, 
ROUND(SUM(t1.gesamtfahrleistung) * 0.000073 * 0.5 * 116) AS `NOx Nicht-Werktags [kg]`, 
ROUND(
        (SUM(t1.gesamtfahrleistung) * 0.000073 * 249) 
        + (SUM(t1.gesamtfahrleistung) * 0.000073 * 0.5 * 116)
) AS `NOx Ganzjährig [kg]`
FROM
(
        SELECT potentials.`miv-change` * potentials.`length` * 0.001 AS gesamtfahrleistung
        FROM potentials
) t1
\end{minted}
\caption{SQL-Abfrage der Veränderung der NOx-Emissionen}\label{lst-emmissionen-nox}
\end{listing}

\subsubsection{Emissionen von Feinstaub-Partikeln}

Deren Emissionen lassen sich zwar berechnen, werden aber gerade im ländlichen Kontext mit der weiten Verbreitung von Holzöfen von deren Emissionen saisonal stark überlagert. Eine mess- oder sogar spürbare Veränderung durch die Reaktivierung der Steigerwaldbahn ist in dieser Schadstoffklasse bestenfalls an der Mainbrücke der B286 in Schweinfurt an warmen Sommertagen nachweisbar. Auf den 300m über den Main wird jedoch nur ein kleiner Anteil der Verkehrsleistung der hier errechneten Gesamtverkehrsleistung erbracht. Eine Gesamtberechnung der Emissionen in dieser Schadstoffklasse erübrigt sich daher.

\subsubsection{Reifenabrieb / Microplastik-Eintrag entlang von Straßen}
Beim Fahren von PKWs verschleißen die Reifen wie auch die Bremsen. Das abgefahrene Gummi und der Bremsstaub lagern sich entlang von Straßen an, zum Beispiel in Absetzbecken und Gräben. Gelangt dieser Abrieb in Fließgewässer oder sickert in das Grundwasser ein, ist es möglich, dass aus diesen Stoffen Schwermetalle und Microplastik dauerhaft und irreversibel in die Umwelt gelangen. Eine Verminderung des Straßenverkehrs bedeutet ebenso, dass die Schadstoffbelastung durch abgefahrenen Gummi und Bremsabrieb in die Gräben und die ersten Meter der Ackerflächen neben den Straßen, welche zum Teil zur Lebensmittel- und Futtermittelproduktion genutzt werden, sich ebenfalls verringert. Zwar ist durch Lebensmittelkontrollen sichergestellt, dass die Mengen, die dadurch in die Nahrungskette zu uns als Menschen zurück kommen, sehr gering sind, ist vollkommen ungeklärt, was dieses unterschätzte Problem für die Umwelt bedeutet. (https://www.springerprofessional.de/fahrwerk/schadstoffe/unterschaetzte-umweltgefahr-reifenabrieb-/15490524)
\newline
\newline
Hersteller geben an, dass normale PKW-Reifen für eine Fahrleistung je nach Fahrstil von 40.000 bis 50.000 Kilometern ausgelegt sind. Die Gewichtsangaben aus Datenblättern verschiedener Hersteller, die ein Gewicht des Reifens spezifizieren, legen nahe, dass ein PKW innerhalb dieser Fahrstrecke ca. 3 kg Reifenabrieb entlang der Straßen und Wege verteilt. Dies korrespondiert auch grob mit der Berichterstattung zu Microplastik und Reifenabrieb, zum Beispiel in der ARD (12000km / 1,3kg): https://www.daserste.de/information/wissen-kultur/w-wie-wissen/reifenabrieb-100.html
\newline
\newline
Es liegt daher Nahe mit eine Schätzwert von 75 gr je 1000 gefahrenen Kilometern den vermiedenen Eintrag dieser Stoffe in die Umwelt zu beziffern. Die Annahmen zu Werk- und Feiertagen sind bereits in der Sektion CO2-Emissionen erläutert.
\newline
\newline
Jedes Jahr könnten im Steigerwaldvorland kanpp 3 Tonnen Reifenabrieb weniger in die Umwelt gelangen.
\newline
\newline
Die folgende Tabelle Zeigt die vermiedenen Emissionen an Werktagen für das gesamte Jahr, die vermiedenen Emissionen an Nicht-Werktagen für das gesamte Jahr und deren Summe für das ganze Jahr.
\newline
\newline
\begin{tabular}{|l | l | l |}
        \hline
        Reifenabrieb an Werktagen [kg] & Reifenabrieb an Nicht-Werktagen [kg] & Reifenabrieb Ganzjährig [kg]\\ 
        \hline
        -2371 & -552 & -2923\\ 
        \hline
\end{tabular}

\begin{listing}[htbp]
\begin{minted}{sql}
SELECT 
ROUND(SUM(t1.gesamtfahrleistung) * 0.000075 * 249) AS `Reifenabrieb an Werktagen [kg]`, 
ROUND(SUM(t1.gesamtfahrleistung) * 0.000075 * 0.5 * 116) AS `Reifenabrieb an Nicht-Werktagen [kg]`, 
ROUND(
        (SUM(t1.gesamtfahrleistung) * 0.000075 * 249) 
        + (SUM(t1.gesamtfahrleistung) * 0.000075 * 0.5 * 116)
) AS `Reifenabrieb Ganzjährig [kg]`
FROM
(
        SELECT potentials.`miv-change` * potentials.`length` * 0.001 AS gesamtfahrleistung
        FROM potentials
) t1
\end{minted}
\caption{SQL-Abfrage der Veränderung des Eintrags von Reifenabrieb in die Umwelt}\label{lst-emmissionen-reifenabrieb}
\end{listing}

\subsection{vermiedene Verkehrsunfälle und Folgeschäden}

Gemäß der Veröffentlichung \enquote{Verkehr in Zahlen 2018} des BMVI hatte Deutschland im Jahr 2017 3180 tötlich verletzte Straßenverkehrsteilnehmer.
66500 haben sich im Jahr 2017 schwer und 323800 leicht bei der Teilnahme am Straßenverkehr verletzt. 90100 Unfälle mit schweren Sachschäden gab es 2017.
Die Veröffentlichung gibt je Milliarde Farzeugkilometer auf Straßen (Autobahnen sind bei den meisten Verkehrsbeziehungen untergeordnet relevant) 4,2 Getötete, 400 Unfälle mit Personenschäden und 516 Verletzte an.
Download: \url{https://www.bmvi.de/SharedDocs/DE/Publikationen/G/verkehr-in-zahlen_2018-pdf.pdf?__blob=publicationFile}
\newline
\newline
Die Reaktivierung der Steigerwaldbahn verlagert werktäglich 126.966 PKW-Kilometer. Die Schliephake-Studie betrachtet ausschließlich den werktäglichen Verkehr. Daher wird für die Nicht-Werktage lediglich die Hälfte dieses Verkehres angenommen, auch wenn dies nicht annähernd an die Präzision der Schliephake-Studie herranreicht. Weiterhin wird mit 249 Werktagen im Jahr gerechnet, auch wenn diese nicht in allen Kommunen an der Strecke gleich sind.
\newline
\newline
Im Jahr summiert sich damit vermiedene Leistung im Straßenverkehr auf knapp 40 Mio km.
\newline
\newline
\begin{tabular}{|l | l | l |}
        \hline
        Fahrleistung an Werktagen [km] & Fahrleistung an Nicht-Werktagen [km] & Fahrleistung Ganzjährig [km]\\ 
        \hline
        -31614558.900 & -7364033.8000 & -38978592.7000\\ 
        \hline
\end{tabular}
\newline
\newline
\begin{listing}[htbp]
\begin{minted}{sql}
SELECT
SUM(t1.gesamtfahrleistung) * 249 AS `Fahrleistung an Werktagen [km]`, 
SUM(t1.gesamtfahrleistung) * 0.5 * 116 AS `Fahrleistung an Nicht-Werktagen [km]`, 
(SUM(t1.gesamtfahrleistung) * 249) + (SUM(t1.gesamtfahrleistung) * 0.5 * 116)  AS `Fahrleistung Ganzjährig [km]`
FROM
(
        SELECT potentials.`miv-change` * potentials.`length` * 0.001 AS gesamtfahrleistung
        FROM potentials
) t1
\end{minted}
\caption{SQL-Abfrage der jährlichen Gesamtfahrleistung}\label{lst-gesamtfahrleistung-jaehrlich}
\end{listing}

Daraus ergeben sich folgende statistische Zahlen im Mittel:\newline
\newline
- \textbf{vermiedene Getötete: \(\approx\) 0.168 / Jahr} (oder als Kehrbruch ausgelegt: wenn man die Steigerwaldbahn nicht reaktiviert, akzeptiert man, dass \textbf{ca.\ alle 6 Jahre eine Person vermeidbar im Straßenverkehr im Steigerwaldvorland umkommt})\newline
- \textbf{vermiedene Unfälle mit Personenschäden: \(\approx\) 16 / Jahr}\newline
- \textbf{vermiedene Verletzte: \(\approx\) 21 / Jahr}\newline
\newline
Natürlich könnten diese Zahlen deutlich höher ausfallen, wenn man die Betrachtungsweise dahingehend verändert, dass man die Veränderung des Verkehrsfluss durch einzelne Unfallschwerpunkte und deren charakteristisches Unfallbild und deren charackteristische Unfallschwere einzeln betrachtet. Dies führt jedoch im Rahmen dieser Berechnung zu weit.

\subsection{vermiedene Betriebskosten für PKWs}
Aus der vermiedenen Fahrleistung lässt sich natürlich errechnen, wie hoch die Einsparungen von Betriebskosten von PKWs gesamtheitlich ausfallen dürften. Dies ist eine wichtige Kennzahl, denn durch das billigere Pendeln mit VGN-Verbundfahrkarten wird für den PKW-Betrieb gebundenes Einkommen frei, welches anderweitig ausgegeben werden kann. Es stellt sich die Vermutung an, dass Aufwendungen für den PKW Betrieb nur zu ganz kleinen Teilen in der Region verbleiben und mehrheitlich aus der Region abfließen. Die Verringerung der Benutzung des Automobils verringert also auch einen Abfluss der Kaufkraft und kann damit die Region wirtschaftlich stärken, da diese Kaufkraft lokal in den Wirtschafts-Kreislauf gelangt.
\newline
\newline
Der ADAC reportiert die \enquote{wahren Kosten} eines PKWs in \enquote{pro Monat} und \enquote{Cent pro Kilometer} für Neuwagen in dem Bericht \enquote{ADAC AutokostenHerbst/Winter 2019/2020}. Dieser Bericht listet diese Kosten für Neufahrzeuge, die, sobald die Reaktivierung abgeschlossen sein wird, im Durchschnittsalter des Flottenmixes zugelassener Fahrzeuge sein werden.\newline
Besondere Aussagekraft hat dieses Dokument hinsichtlich des Umstandes, dass es keinen Neuwagen gibt, welcher mit den rund 30 Cent/km, welcher gerne aus der Finanzamt-basierten Kostenerstattung herangezogen wird, hinkommt. Jeder Neuwagen, den man aktuell kaufen kann, liegt deutlich darüber. Massenfahrzeuge liegen oft bei 60 bis 70 Cent/km.\newline
Download: \url{https://www.adac.de/_mmm/pdf/autokostenuebersicht_47085.pdf}\newline
\newline
Aus diesem Grund werde ich mit 55 cent / Kilometer rechnen.\newline
\newline
Im Jahr summiert sich damit vermiedene Leistung im Straßenverkehr auf knapp 40 Mio km.
\newline
\newline
\begin{tabular}{| l | l | l |}
        \hline
        Fahrleistung an Werktagen [km] & Fahrleistung an Nicht-Werktagen [km] & Fahrleistung Ganzjährig [km]\\ 
        \hline
        -31614558.900 & -7364033.8000 & -38978592.7000\\ 
        \hline
\end{tabular}
\newline
\newline
\begin{listing}[htbp]
\begin{minted}{sql}
SELECT
SUM(t1.gesamtfahrleistung) * 249 AS `Fahrleistung an Werktagen [km]`, 
SUM(t1.gesamtfahrleistung) * 0.5 * 116 AS `Fahrleistung an Nicht-Werktagen [km]`, 
(SUM(t1.gesamtfahrleistung) * 249) + (SUM(t1.gesamtfahrleistung) * 0.5 * 116)  AS `Fahrleistung Ganzjährig [km]`
FROM
(
        SELECT potentials.`miv-change` * potentials.`length` * 0.001 AS gesamtfahrleistung
        FROM potentials
) t1
\end{minted}
\caption{SQL-Abfrage der jährlichen Gesamtfahrleistung}\label{lst-gesamtfahrleistung-jaehrlich}
\end{listing}

Somit ergeben sich folgende Vermiedene Betriebskosten:
\newline
\newline
\begin{tabular}{ | l |}
        \hline
        vermiedene Betriebskosten  Ganzjährig [T€]\\ 
        \hline
        -21438\\
        \hline
\end{tabular}

\begin{listing}[htbp]
\begin{minted}{sql}
SELECT 
ROUND(
        (SUM(t1.gesamtfahrleistung) * 249 * 0.00055) 
        + (SUM(t1.gesamtfahrleistung) * 0.5 * 116 * 0.00055)
)  AS `vermiedene Betriebskosten  Ganzjährig [T€]`
FROM
(
        SELECT potentials.`miv-change` * potentials.`length` * 0.001 AS gesamtfahrleistung
        FROM potentials
) t1
\end{minted}
\caption{SQL-Abfrage der jährlich vermiedenen Betriebskosten}\label{lst-betriebskosten-jaehrlich}
\end{listing}

Abzüglich der Kosten für die Fahrkarten, welche hier noch nicht gegengerechnet sind, stehen den Haushalten im Steigerwaldvorland also jedes Jahr ca. 21,4 Mio € zur Verfügung, welche wahrscheinlich vorwiegend lokal ausgegeben werden; zum Beispiel in der Gastronomie, für die Ausbildung der Kinder oder für Bau und/oder Unterhalt einer eigenen Immobilie (-> Handwerk). Dies entspricht einem kleinem Konjunkturprogramm für die Landkreise Kitzingen und Schweinfurt sowie die Kreisfreie Stadt Schweinfurt. Bei den Hauptkosten für einen PKW verbleibt die ausgegebene Kaufkraft hingegen im größeren Anteil nicht in den Landkreisen Schweinfurt und Kitzingen und kann daher als dem lokalen Wirtschaftskreislauf als größtenteils \enquote{entzogen} angesehen werden.

\section{Veränderung des Straßenverkehrs auf einzelnen Straßen}
                \subsection{Gesamtliste}
                \subsection{hervorgehobene Neuralgische Punkte im Straßennetz}
        \section{Verlagerung der Einzelorte}
                \subsection{Schweinfurt}
                \subsection{Sennfeld}
                \subsection{Gochsheim}
                \subsection{Gochsheim OT Weyer}
                \subsection{Schwebheim}
                \subsection{Grettstatt}
                \subsection{Grettstatt OT Dürrfeld}
                \subsection{Donnersdorf}
                \subsection{Sulzheim}
                \subsection{Alitzheim}
                \subsection{Mönchstockheim}
                \subsection{Vögnitz}
                \subsection{Kolitzheim}
                \subsection{Gerolzhofen}
                \subsection{Dingolshausen}
                \subsection{Michelau}
                \subsection{Frankenwinheim}
                \subsection{Oberschwarzach}
                \subsection{Volkach}
                \subsection{Lülsfeld}
                \subsection{Schallfeld}
                \subsection{Prichsenstadt}
                \subsection{Prichsenstadt OT Altenschönbach}
                \subsection{Prichsenstadt OT Bimbach} 
                \subsection{Prichsenstadt OT Brünnau}
                \subsection{Järkendorf}
                \subsection{Prichsenstadt OT Kirchschönbach}
                \subsection{Prichsenstadt OT Laub}
                \subsection{Prichsenstadt OT Neudorf}
                \subsection{Prichsenstadt OT Neuses}
                \subsection{Prichsenstadt OT Stadelschwarzach}
                \subsection{Wiesentheid}
                \subsection{Wiesentheid OT Feuerbach}
                \subsection{Wiesentheid OT Geesdorf}
                \subsection{Wiesentheid OT Reupelsdorf}
                \subsection{Wiesentheid OT Untersambach}
                \subsection{Rüdenhausen}
                \subsection{Abtswind}
                \subsection{Kleinlangheim}
                \subsection{Wiesenbronn}
                \subsection{Großlangheim}
                \subsection{Kitzingen}
                \subsection{Würzburg}
        
        
        \section{gewonnene Produktivität}
Aufgrund der durchgeführten Erhebung der PKW-Fahrzeiten lässt sich grob ein Zugewinn an Produktivität 

        \section{Zugewinn an Umsteigern}


        \chapter{Listings}
        \input{listings.tex}
    
\end{document}

