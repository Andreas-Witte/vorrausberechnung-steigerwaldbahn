\begin{tabularx}{\textwidth}{*5{X}}
Quelle & Ziel & NettoPotenzial & MIV-Veränderung & Potenzial-ID\\ 
Prichsenstadt OT Altenschönbach & Bamberg, Haßfurt & 1 & -1 & 126\\ 
Prichsenstadt OT Altenschönbach & Schweinfurt & 2 & -3 & 127\\ 
Prichsenstadt OT Altenschönbach & Gerolzhofen & 2 & -3 & 128\\ 
Prichsenstadt OT Altenschönbach & Lülsfeld & 3 & -4 & 129\\ 
Prichsenstadt OT Altenschönbach & Kitzingen & 10 & -16 & 130\\ 
Prichsenstadt OT Altenschönbach & Würzburg, Rottendorf & 4 & -6 & 131\\ 
Prichsenstadt OT Altenschönbach & Nürnberg, Erlangen & 1 & -1 & 132\\ 
Prichsenstadt OT Altenschönbach & Prichsenstadt & * & 57 & 133\\ 
\end{tabularx}    
\newline
\newline
* Neue Verkehre um den Bahnhof zu erreichen.
\newline
\begin{listing}[htbp]
\begin{minted}{sql}
SELECT
`from_places`.`name` AS `Quelle`, 
`to_places`.`name` AS `Ziel`, 
`potentials`.`netto` AS `NettoPotenzial`, 
`potentials`.`miv-change` AS `MIV-Veränderung`, 
`potentials`.`id` AS `Potenzial-ID` 
FROM `potentials`
LEFT JOIN `places` `from_places` ON `from_places`.`id` = `potentials`.`from_id`
LEFT JOIN `places` `to_places` ON `to_places`.`id` = `potentials`.`to_id`
WHERE `from_places`.`name` = "Prichsenstadt OT Altenschönbach";
\end{minted}
\caption{SQL-Abfrage der Netto-Potenziale und MIV-Veränderung mit der Quelle Altenschönbach}\label{lst-fz-altenschoenbach}
\end{listing}