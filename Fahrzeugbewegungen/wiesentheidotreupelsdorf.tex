\begin{tabular}{ l  l  l  l  l }
Quelle & Ziel & NettoPotenzial & MIV-Veränderung & Potenzial-ID\\ 
Wiesentheid OT Reupelsdorf & Schweinfurt & 1 & -1 & 214\\ 
Wiesentheid OT Reupelsdorf & Gerolzhofen & 1 & -1 & 215\\ 
Wiesentheid OT Reupelsdorf & Kleinlangheim & 1 & -1 & 216\\ 
Wiesentheid OT Reupelsdorf & Kitzingen & 9 & -14 & 217\\ 
Wiesentheid OT Reupelsdorf & Würzburg, Rottendorf & 5 & -8 & 218\\ 
Wiesentheid OT Reupelsdorf & Nürnberg, Erlangen & 1 & -1 & 219\\ 
Wiesentheid OT Reupelsdorf & Stadelschwarzach & * & 5 & 220\\ 
Wiesentheid OT Reupelsdorf & Wiesentheid & * & 40 & 221\\ 
\end{tabular}
\newline
\newline
* Neue Verkehre um den Bahnhof zu erreichen.
\newline
\begin{listing}[htbp]
\begin{minted}{sql}
SELECT
`from_places`.`name` AS `Quelle`, 
`to_places`.`name` AS `Ziel`, 
`potentials`.`netto` AS `NettoPotenzial`, 
`potentials`.`miv-change` AS `MIV-Veränderung`, 
`potentials`.`id` AS `Potenzial-ID`
FROM `potentials`
LEFT JOIN `places` `from_places` ON `from_places`.`id` = `potentials`.`from_id`
LEFT JOIN `places` `to_places` ON `to_places`.`id` = `potentials`.`to_id`
WHERE `from_places`.`name` = "Wiesentheid OT Reupelsdorf";
\end{minted}
\caption{SQL-Abfrage der Netto-Potenziale und MIV-Veränderung mit der Quelle Reupelsdorf}\label{lst-fz-reupelsdorf}
\end{listing}