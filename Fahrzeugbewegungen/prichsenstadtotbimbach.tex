\begin{tabularx}{\textwidth}{*5{X}}
Quelle & Ziel & NettoPotenzial & MIV-Veränderung & Potenzial-ID\\ 
Prichsenstadt OT Bimbach & Bamberg, Haßfurt & 1 & -1 & 134\\ 
Prichsenstadt OT Bimbach & Schweinfurt & 1 & -1 & 135\\ 
Prichsenstadt OT Bimbach & Gerolzhofen & 1 & -1 & 136\\ 
Prichsenstadt OT Bimbach & Lülsfeld & 1 & -1 & 137\\ 
Prichsenstadt OT Bimbach & Wiesentheid & 6 & -9 & 138\\ 
Prichsenstadt OT Bimbach & Kitzingen & 4 & -6 & 139\\ 
Prichsenstadt OT Bimbach & Würzburg, Rottendorf & 2 & -3 & 140\\ 
Prichsenstadt OT Bimbach & Nürnberg, Erlangen & 1 & -1 & 141\\ 
Prichsenstadt OT Bimbach & Järkendorf & * & 43 & 142\\ 
\end{tabularx}
\newline
\newline
* Neue Verkehre um den Bahnhof zu erreichen.
\newline
\begin{listing}[htbp]
\begin{minted}{sql}
SELECT
`from_places`.`name` AS `Quelle`, 
`to_places`.`name` AS `Ziel`, 
`potentials`.`netto` AS `NettoPotenzial`, 
`potentials`.`miv-change` AS `MIV-Veränderung`, 
`potentials`.`id` AS `Potenzial-ID`
FROM `potentials`
LEFT JOIN `places` `from_places` ON `from_places`.`id` = `potentials`.`from_id`
LEFT JOIN `places` `to_places` ON `to_places`.`id` = `potentials`.`to_id`
WHERE `from_places`.`name` = "Prichsenstadt OT Bimbach";
\end{minted}
\caption{SQL-Abfrage der Netto-Potenziale und MIV-Veränderung mit der Quelle Bimbach}\label{lst-fz-bimbach}
\end{listing}