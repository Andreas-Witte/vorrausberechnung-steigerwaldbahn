\begin{tabular}{ l  l  l  l  l }
Quelle & Ziel & NettoPotenzial & MIV-Veränderung & Potenzial-ID\\ 
Wiesenbronn & Schweinfurt & 3 & -4 & 240\\ 
Wiesenbronn & Gerolzhofen & 2 & -3 & 241\\ 
Wiesenbronn & Würzburg, Rottendorf & 8 & -12 & 242\\ 
Wiesenbronn & Kleinlangheim & * & 13 & 243\\ 
Wiesenbronn & Großlangheim & * & 20 & 244\\ 
\end{tabular}
\newline
\newline
* Neue Verkehre um den Bahnhof zu erreichen.
\newline
\begin{listing}[htbp]
\begin{minted}{sql}
SELECT
`from_places`.`name` AS `Quelle`, 
`to_places`.`name` AS `Ziel`, 
`potentials`.`netto` AS `NettoPotenzial`, 
`potentials`.`miv-change` AS `MIV-Veränderung`, 
`potentials`.`id` AS `Potenzial-ID`
FROM `potentials`
LEFT JOIN `places` `from_places` ON `from_places`.`id` = `potentials`.`from_id`
LEFT JOIN `places` `to_places` ON `to_places`.`id` = `potentials`.`to_id`
WHERE `from_places`.`name` = "Wiesenbronn";
\end{minted}
\caption{SQL-Abfrage der Netto-Potenziale und MIV-Veränderung mit der Quelle Wiesenbronn}\label{lst-fz-wiesenbronn}
\end{listing}