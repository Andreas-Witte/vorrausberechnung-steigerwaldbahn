\begin{tabular}{ l  l  l  l  l }
Quelle & Ziel & NettoPotenzial & MIV-Veränderung & Potenzial-ID\\ 
Prichsenstadt OT Kirchschönbach & Bamberg, Haßfurt & 1 & -1 & 158\\ 
Prichsenstadt OT Kirchschönbach & Schweinfurt & 2 & -3 & 159\\ 
Prichsenstadt OT Kirchschönbach & Gerolzhofen & 2 & -3 & 160\\ 
Prichsenstadt OT Kirchschönbach & Lülsfeld & 3 & -4 & 161\\ 
Prichsenstadt OT Kirchschönbach & Kitzingen & 10 & -16 & 162\\ 
Prichsenstadt OT Kirchschönbach & Würzburg, Rottendorf & 5 & -8 & 163\\ 
Prichsenstadt OT Kirchschönbach & Nürnberg, Erlangen & 1 & -1 & 164\\ 
Prichsenstadt OT Kirchschönbach & Prichsenstadt & * & 60 & 165\\ 
\end{tabular}
\newline
\newline
* Neue Verkehre um den Bahnhof zu erreichen.
\newline
\begin{listing}[htbp]
\begin{minted}{sql}
SELECT
`from_places`.`name` AS `Quelle`, 
`to_places`.`name` AS `Ziel`, 
`potentials`.`netto` AS `NettoPotenzial`, 
`potentials`.`miv-change` AS `MIV-Veränderung`, 
`potentials`.`id` AS `Potenzial-ID`
FROM `potentials`
LEFT JOIN `places` `from_places` ON `from_places`.`id` = `potentials`.`from_id`
LEFT JOIN `places` `to_places` ON `to_places`.`id` = `potentials`.`to_id`
WHERE `from_places`.`name` = "Prichsenstadt OT Kirchschönbach";
\end{minted}
\caption{SQL-Abfrage der Netto-Potenziale und MIV-Veränderung mit der Quelle Kirchschönbach}\label{lst-fz-kirchschoenbach}
\end{listing}