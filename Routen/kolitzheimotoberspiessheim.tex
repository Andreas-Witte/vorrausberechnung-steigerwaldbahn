Zugeordnete Routen:
\newline
\newline
\begin{longtabu}{|l|l|l|l|*2{X[l]|}}
    \hline
    id & Quelle & Ziel & Straße & Straßenbeginn & Straßenende\\ 
    \hline
    63 & Oberspießheim & Kitzingen & SW42 & Unterspießheim & Oberspießheim\\ 
    63 &  &  & St2271 & Unterspießheim & Kolitzheim\\ 
    63 &  &  & St2271 & Kolitzheim & Gaibach\\ 
    63 &  &  & St2271 & Gaibach & Volkach\\ 
    63 &  &  & St2271 & Volkach & St2271/KT57\\ 
    63 &  &  & St2271 & St2271/KT57 & Gerlachshausen\\ 
    63 &  &  & St2271 & Gerlachshausen & B22/St2271 (bei Stadtschwarzach)\\ 
    63 &  &  & St2271 & B22/St2271 (bei Stadtschwarzach) & Hörblach\\ 
    63 &  &  & St2271 & Hörblach & Ausfahrt Kitzingen-Schwarzach\\ 
    63 &  &  & St2271 & Ausfahrt Kitzingen-Schwarzach & St2271/St2272 (bei Kitzingen-Etwashausen)\\ 
    63 &  &  & St2271 & St2271/St2272 (bei Kitzingen-Etwashausen) & B8/St2271 (Kitzingen bei e-center)\\ 
    63 &  &  & B8 & Kitzingen & B8/St2271 (Kitzingen bei e-center)\\ 
    \hline
    64 & Oberspießheim & Alitzheim & SW42 & Herlheim & Oberspießheim\\ 
    64 &  &  & SW40 & Herlheim & Alitzheim\\ 
    \hline
\end{longtabu}

\begin{listing}[htbp]
\begin{minted}{sql}
    SELECT 
	`potentials`.`id` AS `id`,
	`from_places`.`name` AS `Quelle`, 
	`to_places`.`name` AS `Ziel`,
	`streets`.`street` AS `Straße`,
	`from_street_places`.`name` AS `Straßenbeginn`,
   `to_street_places`.`name` AS `Straßenende`
FROM `potentials`
LEFT JOIN `places` AS `from_places` ON `potentials`.`from_id` = `from_places`.`id`
LEFT JOIN `places` AS `to_places` ON `potentials`.`to_id` = `to_places`.`id`
LEFT JOIN `routes` ON `routes`.`potential_id` = `potentials`.`id`
LEFT JOIN `streets` ON `streets`.`id` = `routes`.`street_id`
LEFT JOIN `places` AS `from_street_places` ON `streets`.`from_id` = `from_street_places`.`id`
LEFT JOIN `places` AS `to_street_places` ON `streets`.`to_id` = `to_street_places`.`id`
WHERE `from_places`.`name` = 'Oberspießheim'
ORDER BY `potentials`.`id`, `routes`.`number_on_route`;
\end{minted}
\caption{SQL-Abfrage der zugeordneten Straßen mit der Quelle Oberspießheim}\label{lst-rt-oberspiessheim}
\end{listing}


Länge, Fahrzeiten und Google Maps:
\newline
\begin{longtabu}{| l | *5{X[l]|}}
    \hline
    id & Quelle & Ziel & Fahrtstrecke [m] & Fahrtdauer [min] & Google-Maps Link\\ 
    \hline
    63 & Oberspießheim & Kitzingen & 29200 & 29 & \url{https://www.google.com/maps/dir/49.9472774,10.2758058/49.7355709,10.1617438}\\ 
    \hline
    64 & Oberspießheim & Alitzheim & 6300 & 6 & \url{https://www.google.com/maps/dir/49.9472774,10.2758058/49.9354067,10.3266909}\\ 
    \hline
\end{longtabu}

\begin{listing}[htbp]
    \begin{minted}{sql}
        SELECT 
        `potentials`.`id` AS `id`, 
        `from_places`.`name` AS `Quelle`,
        `to_places`.`name` AS `Ziel`, 
        `potentials`.`length` AS `Fahrtstrecke [m]`, 
        `potentials`.`miv-duration` AS `Fahrtdauer [min]`,
        CONCAT('https://www.google.com/maps/dir/', `from_places`.`LAT`, ",", `from_places`.`LONG`, '/', `to_places`.`LAT`, ',', `to_places`.`LONG`) AS `Google-Maps Link`
    FROM `potentials`
    LEFT JOIN `places` AS `from_places` ON `potentials`.`from_id` = `from_places`.`id`
    LEFT JOIN `places` AS `to_places` ON `potentials`.`to_id` = `to_places`.`id`
    WHERE `from_places`.`name` = 'Oberspießheim'
    ORDER BY `potentials`.`id`;
    \end{minted}
    \caption{SQL-Abfrage der Fahrtstrecke, Fahrtdauer und des Google-Maps-Link mit der Quelle Oberspießheim}\label{lst-f-oberspiessheim}
\end{listing}